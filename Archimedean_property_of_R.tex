\documentclass[12pt]{article}
\usepackage[utf8]{inputenc}
\usepackage{amsmath}
\usepackage{amsthm}
\usepackage{geometry}
\usepackage{amsfonts}
\usepackage{bm}
\usepackage{hyperref}
\usepackage{xcolor}
\geometry{
headheight=15pt,
left=60pt,
right=60pt
}
\usepackage{fancyhdr}
\pagestyle{fancy}
\fancyhf{}
\lhead{}
\chead{Archimedean property of \texorpdfstring{\( \mathbb{R} \)}{}}
\rhead{}

\pagenumbering{gobble}
\setlength{\parindent}{0pt}
\hypersetup{
    colorlinks=true,
    linkcolor=blue
}

\newcommand{\newp}{\vspace{5mm}}

\theoremstyle{definition}
\newtheorem{theorem}{Theorem}
\newtheorem{lemma}{Lemma}

\newtheorem*{remark}{Remark}

\begin{document}

The following is paraphrased from pages 8-9 of \hyperlink{pma}{[PMA]}.

\section{Archimedean property of \texorpdfstring{\( \mathbb{R} \)}{}}

\begin{theorem}[Theorem 1.19, p. 8, \hyperlink{pma}{[PMA]}]

There exists an ordered field \( \mathbb{R} \) which has the least-upper-bound property. Moreover, \( \mathbb{R} \) contains \( \mathbb{Q} \) as a subfield.

\end{theorem}

A consequence of Theorem 1 is:

\begin{theorem}[Archimedean property of \( \mathbb{R} \)]

Let \( x > 0 \) and \( y \) be real numbers. Then there exists a positive integer \( n \) such that \( nx > y \).

\end{theorem}

\begin{proof}

Suppose to the contrary that for all positive integers \( n \) we have \( nx \leq y \). Then the set \( A = \{ nx : n \in \mathbb{N} \} \) is non-empty and bounded above, so by the least-upper-bound property of \( \mathbb{R} \) the supremum \( \alpha = \sup A \) exists in \( \mathbb{R} \). Since \( x > 0 \), we have \( \alpha - x < \alpha \) so that \( \alpha - x \) is not an upper bound for \( A \). Hence there exists a positive integer \( m \) such that \( \alpha - x < mx \), which gives \( \alpha < (m+1)x \); but this contradicts the fact that \( \alpha \) is the supremum of \( A \).
\end{proof}

\section{Density of \texorpdfstring{\( \mathbb{Q} \)}{} in \texorpdfstring{\( \mathbb{R} \)}{}}

\begin{lemma}

Any real number lies between two consecutive integers. That is, for any \( x \in \mathbb{R} \) there exists an \( m \in \mathbb{Z} \) such that \( m - 1 \leq x < m \).

\end{lemma}

\begin{proof}

By the Archimedean property, there exist positive integers \( m_1, m_2 \) such that \( m_1 > x \) and \( m_2 > - x \), which gives \( -m_2 < x < m_1 \). This implies that the set \( A = \{ n \in \mathbb{Z} : x < n \} \) is non-empty (\( m_1 \in A \)) and bounded below (by \( -m_2 \)). Then by the well-ordering principle, \( A \) has a least element; call it \( m \). Since this is the least element of \( A \), we must have \( m - 1 \not\in A \) and so \( m - 1 \leq x < m \).
\end{proof}

\begin{theorem}

Between any two real numbers there exists a rational number. That is, for any \( x, y \in \mathbb{R} \) with \( x < y \) there exists a \( p \in \mathbb{Q} \) such that \( x < p < y \).

\end{theorem}

\begin{proof}

By the Archimedean property, there exists a positive integer \( n \) such that \( n(y - x) > 1 \). By Lemma 1, there exists an integer \( m \) such that \( m - 1 \leq nx < m \). Combining these inequalities gives \( nx < m \leq 1 + nx < ny \), which implies \( x < \frac{m}{n} < y \). So the desired rational is \( p = \frac{m}{n} \).
\end{proof}

\hrulefill

\hypertarget{pma}{\textcolor{blue}{[PMA]} Rudin, W. (1976) \textit{Principles of Mathematical Analysis.} 3rd edn.}

\end{document}
