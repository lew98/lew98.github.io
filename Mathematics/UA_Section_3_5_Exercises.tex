\documentclass[12pt]{article}
\usepackage[utf8]{inputenc}
\usepackage[utf8]{inputenc}
\usepackage{amsmath}
\usepackage{amsthm}
\usepackage{geometry}
\usepackage{amsfonts}
\usepackage{mathrsfs}
\usepackage{bm}
\usepackage{hyperref}
\usepackage[dvipsnames]{xcolor}
\usepackage[inline]{enumitem}
\usepackage{mathtools}
\usepackage{changepage}
\usepackage{lipsum}
\usepackage{tikz}
\usetikzlibrary{matrix, patterns, decorations.pathreplacing, calligraphy}
\usepackage{tikz-cd}
\usepackage[nameinlink]{cleveref}
\geometry{
headheight=15pt,
left=60pt,
right=60pt
}
\setlength{\emergencystretch}{20pt}
\usepackage{fancyhdr}
\pagestyle{fancy}
\fancyhf{}
\lhead{}
\chead{Section 3.5 Exercises}
\rhead{\thepage}
\hypersetup{
    colorlinks=true,
    linkcolor=blue,
    urlcolor=blue
}

\theoremstyle{definition}
\newtheorem*{remark}{Remark}

\newtheoremstyle{exercise}
    {}
    {}
    {}
    {}
    {\bfseries}
    {.}
    { }
    {\thmname{#1}\thmnumber{#2}\thmnote{ (#3)}}
\theoremstyle{exercise}
\newtheorem{exercise}{Exercise 3.5.}

\newtheoremstyle{solution}
    {}
    {}
    {}
    {}
    {\itshape\color{magenta}}
    {.}
    { }
    {\thmname{#1}\thmnote{ #3}}
\theoremstyle{solution}
\newtheorem*{solution}{Solution}

\Crefformat{exercise}{#2Exercise 3.5.#1#3}

\newcommand{\interior}[1]{%
  {\kern0pt#1}^{\mathrm{o}}%
}
\newcommand{\ts}{\textsuperscript}
\newcommand{\setcomp}[1]{#1^{\mathsf{c}}}
\newcommand{\quand}{\quad \text{and} \quad}
\newcommand{\N}{\mathbf{N}}
\newcommand{\Z}{\mathbf{Z}}
\newcommand{\Q}{\mathbf{Q}}
\newcommand{\I}{\mathbf{I}}
\newcommand{\R}{\mathbf{R}}
\newcommand{\C}{\mathbf{C}}

\DeclarePairedDelimiter\abs{\lvert}{\rvert}
% Swap the definition of \abs* and \norm*, so that \abs
% and \norm resizes the size of the brackets, and the 
% starred version does not.
\makeatletter
\let\oldabs\abs
\def\abs{\@ifstar{\oldabs}{\oldabs*}}
%
\let\oldnorm\norm
\def\norm{\@ifstar{\oldnorm}{\oldnorm*}}
\makeatother

\setlist[enumerate,1]{label={(\alph*)}}

\begin{document}

\section{Section 3.5 Exercises}

Exercises with solutions from Section 3.5 of \hyperlink{ua}{[UA]}.

\begin{exercise}
\label{ex:1}
    Argue that a set \( A \) is a \( G_{\delta} \) set if and only if its complement is an \( F_{\sigma} \) set.
\end{exercise}

\begin{solution}
    This is immediate from De Morgan's Laws (see \href{https://lew98.github.io/Mathematics/UA_Section_3_2_Exercises.pdf}{Exercise 3.2.9}).
\end{solution}

\begin{exercise}
\label{ex:2}
    Replace each \rule{1cm}{0.15mm} with the word \textit{finite} or \textit{countable} depending on which is more appropriate.
    \begin{enumerate}
        \item The \rule{1cm}{0.15mm} union of \( F_{\sigma} \) sets is an \( F_{\sigma} \) set.

        \item The \rule{1cm}{0.15mm} intersection of \( F_{\sigma} \) sets is an \( F_{\sigma} \) set.

        \item The \rule{1cm}{0.15mm} union of \( G_{\delta} \) sets is a \( G_{\delta} \) set.

        \item The \rule{1cm}{0.15mm} intersection of \( G_{\delta} \) sets is a \( G_{\delta} \) set.
    \end{enumerate}
\end{exercise}

\begin{solution}
    \begin{enumerate}
        \item The countable union of \( F_{\sigma} \) sets is an \( F_{\sigma} \) set. Suppose we have a countable collection \( \{ A_m : m \in \N \} \) of \( F_{\sigma} \) sets, i.e.\ for each \( m \in \N \) there is a countable collection \( \{ B_{m,n} : n \in \N \} \) of closed sets such that \( A_m = \bigcup_{n=1}^{\infty} B_{m,n} \). Then
        \[
            \bigcup_{m=1}^{\infty} A_m = \bigcup_{m=1}^{\infty} \bigcup_{n=1}^{\infty} B_{m,n} = \bigcup_{(m, n) \in \N^2} B_{m,n}.
        \]
        \( \N^2 \) is countable by Theorem 1.5.8 (ii), so we have expressed \( \bigcup_{m=1}^{\infty} A_m \) as a countable union of closed sets and hence \( \bigcup_{m=1}^{\infty} A_m \) is an \( F_{\sigma} \) set.

        \item The finite intersection of \( F_{\sigma} \) sets is an \( F_{\sigma} \) set. To see this, it will suffice to show that if \( A \) and \( B \) are \( F_{\sigma} \) sets, then \( A \cap B \) is an \( F_{\sigma} \) set; the general case will follow from an induction argument. Suppose therefore that \( A = \bigcup_{m=1}^{\infty} A_m \) and \( B = \bigcup_{n=1}^{\infty} B_n \), where \( \{ A_m : m \in \N \} \) and \( \{ B_n : n \in \N \} \) are countable collections of closed sets. We claim that
        \[
            A \cap B = \left( \bigcup_{m=1}^{\infty} A_m \right) \cap \left( \bigcup_{n=1}^{\infty} B_n \right) = \bigcup_{(m, n) \in \N^2} (A_m \cap B_n).
        \]
        Indeed,
        \begin{align*}
            \textstyle x \in \left( \bigcup_{m=1}^{\infty} A_m \right) \cap \left( \bigcup_{n=1}^{\infty} B_n \right) &\iff \textstyle \left( x \in \bigcup_{m=1}^{\infty} A_m \right) \text{ and } \left( x \in \bigcup_{n=1}^{\infty} B_n \right) \\
            &\iff (\exists m \in \N : x \in A_m) \text{ and } (\exists n \in \N : x \in B_n) \\
            &\iff (\exists (m, n) \in \N^2 : x \in A_m \text{ and } x \in B_n) \\
            &\iff (\exists (m, n) \in \N^2 : x \in A_m \cap B_n) \\
            &\iff \textstyle x \in \bigcup_{(m, n) \in \N^2} (A_m \cap B_n).
        \end{align*}
        For any \( (m, n) \in \N^2 \), the intersection \( A_m \cap B_n \) is closed since both \( A_m \) and \( B_n \) are closed. Thus we have expressed \( A \cap B \) as a countable union of closed sets and hence \( A \cap B \) is an \( F_{\sigma} \) set.

        The countable intersection of \( F_{\sigma} \) sets need not be an \( F_{\sigma} \) set. For a counterexample, let \( \{ r_1, r_2, \ldots \} \) be an enumeration of \( \Q \) and for positive integers \( m \) and \( n \), set
        \[
            B_{m, n} := \left( -\infty, r_m - \tfrac{1}{n} \right] \cup \left[ r_m + \tfrac{1}{n}, \infty \right).
        \]
        Each \( B_{m, n} \) is a closed set, so if we let \( A_m := \bigcup_{n=1}^{\infty} B_{m, n} \) for each \( m \in \N \), then each \( A_m \) is an \( F_{\sigma} \) set. We claim that \( \bigcap_{m=1}^{\infty} A_m = \I \), the set of irrational numbers. To see this, we will show that \( \setcomp{(\bigcap_{m=1}^{\infty} A_m)} = \Q \). By De Morgan's Laws, we have
        \begin{align*}
            \textstyle \setcomp{\left( \bigcap_{m=1}^{\infty} A_m \right)} &= \textstyle \bigcup_{m=1}^{\infty} \setcomp{A_m} \\[2mm]
            &= \textstyle \bigcup_{m=1}^{\infty} \setcomp{\left( \bigcup_{n=1}^{\infty} B_{m, n} \right)} \\[2mm]
            &= \textstyle \bigcup_{m=1}^{\infty} \bigcap_{n=1}^{\infty} \setcomp{B_{m, n}} \\[2mm]
            &= \textstyle \bigcup_{m=1}^{\infty} \bigcap_{n=1}^{\infty} \left( r_m - \tfrac{1}{n}, r_m + \tfrac{1}{n} \right) \\[2mm]
            &= \textstyle \bigcup_{m=1}^{\infty} \{ r_m \} \\[2mm]
            &= \Q.
        \end{align*}
        Thus \( \bigcap_{m=1}^{\infty} A_m = \I \). As we will show in \Cref{ex:6}, \( \I \) is not an \( F_{\sigma} \) set.

        \item The finite union of \( G_{\delta} \) sets is a \( G_{\delta} \) set, but the countable union of \( G_{\delta} \) sets need not be a \( G_{\delta} \) set; these statements follow from part (b) of this exercise, \Cref{ex:1}, and De Morgan's Laws.

        \item The countable intersection of \( G_{\delta} \) sets is a \( G_{\delta} \) set. Again, this follows from part (a) of this exercise, \Cref{ex:1}, and De Morgan's Laws.
    \end{enumerate}
\end{solution}

\begin{exercise}
\label{ex:3}
    (This exercise has already appeared as \href{https://lew98.github.io/Mathematics/UA_Section_3_2_Exercises.pdf}{Exercise 3.2.15}.)
    \begin{enumerate}
        \item Show that a closed interval \( [a, b] \) is a \( G_{\delta} \) set.
        
        \item Show that the half-open interval \( (a, b] \) is both a \( G_{\delta} \) and an \( F_{\sigma} \) set.

        \item Show that \( \Q \) is an \( F_{\sigma} \) set, and the set of irrationals \( \I \) forms a \( G_{\delta} \) set.
    \end{enumerate}
\end{exercise}

\begin{solution}
    See \href{https://lew98.github.io/Mathematics/UA_Section_3_2_Exercises.pdf}{Exercise 3.2.15}.
\end{solution}

\begin{exercise}
\label{ex:4}
    Starting with \( n = 1 \), inductively construct a nested sequence of \textit{closed} intervals \( I_1 \supseteq I_2 \supseteq I_3 \supseteq \cdots \) satisfying \( I_n \subseteq G_n \). Give special attention to the issue of the endpoints of each \( I_n \). Show how this leads to a proof of the theorem.
\end{exercise}

\begin{solution}
    Since \( G_1 \) is dense, it must be non-empty, i.e.\ there exists some \( x_1 \in G_1 \). Since \( G_1 \) is open, there exists an \( \epsilon_1 > 0 \) such that \( (x_1 - \epsilon_1, x_1 + \epsilon_1) \subseteq G_1 \). Set
    \[
        a_1 := x_1 - \tfrac{\epsilon_1}{2}, \quad b_1 := x_1 + \tfrac{\epsilon_1}{2}, \quand I_1 = [a_1, b_1].
    \]
    Then \( I_1 \subseteq (x_1 - \epsilon_1, x_1 + \epsilon_1) \subseteq G_1 \). This handles the base case. Now suppose that after \( n \) steps we have chosen nested, closed intervals \( I_1 = [a_1, b_1] \supseteq \cdots \supseteq I_n = [a_n, b_n] \) such that \( I_1 \subseteq G_1, \ldots, I_n \subseteq G_n \) and \( a_1 < b_1, \ldots, a_n < b_n \). Since \( G_{n+1} \) is dense, there exists some \( x_{n+1} \in G_{n+1} \) such that \( a_n < x_{n+1} < b_n \), and since \( G_{n+1} \) is open, there exists some \( \epsilon_{n+1} > 0 \) such that \( (x_{n+1} - \epsilon_{n+1}, x_{n+1} + \epsilon_{n+1}) \subseteq G_{n+1} \). Let \( \delta = \min \left\{ \tfrac{\epsilon_{n+1}}{2}, x_{n+1} - a_n, b_n - x_{n+1} \right\} \), and set
    \[
        a_{n+1} := x_{n+1} - \delta, \quad b_{n+1} := x_{n+1} + \delta, \quand I_{n+1} = [a_{n+1}, b_{n+1}].
    \]
    Then \( a_{n+1} < b_{n+1} \), and since \( \delta \leq x_{n+1} - a_n \) and \( \delta \leq b_n - x_{n+1} \), we have \( I_{n+1} \subseteq I_n \). Moreover, because \( \delta \leq \tfrac{\epsilon_{n+1}}{2} \), we also have \( I_{n+1} \subseteq (x_{n+1} - \epsilon_{n+1}, x_{n+1} + \epsilon_{n+1}) \subseteq G_{n+1} \). This completes the induction step.

    Thus we obtain a nested sequence of closed intervals \( (I_n) \) such that \( I_n \subseteq G_n \) for each \( n \in \N \). We may now appeal to the Nested Interval Property to obtain some \( x \in \bigcap_{n=1}^{\infty} I_n \), which must also belong to \( \bigcap_{n=1}^{\infty} G_n \).
\end{solution}

\begin{exercise}
\label{ex:5}
    Show that it is impossible to write
    \[
        \R = \bigcup_{n=1}^{\infty} F_n,
    \]
    where for each \( n \in \N, F_n \) is a closed set containing no nonempty open intervals.
\end{exercise}

\begin{solution}
    Suppose that \( \{ F_n : n \in \N \} \) is a collection of closed sets, each of which contains no non-empty open intervals. Then for each \( n \in \N, \setcomp{F_n} \) is an open set. Furthermore, we claim that \( \setcomp{F_n} \) is dense. To see this, let \( x < z \) be arbitrary real numbers. By assumption, \( (x, z) \not\subseteq F_n \), so there exists some \( y \in (x, z) \cap \setcomp{F_n} \); the claim follows.

    Thus \( \{ \setcomp{F_n} : n \in \N \} \) is a collection of open, dense sets. Theorem 3.5.2 (\Cref{ex:4}) and De Morgan's Laws now imply that
    \[
        \bigcap_{n=1}^{\infty} \setcomp{F_n} \neq \emptyset \iff \bigcup_{n=1}^{\infty} F_n \neq \R.
    \]
\end{solution}

\begin{exercise}
\label{ex:6}
    Show how the previous exercise implies that the set \( \I \) of irrationals cannot be an \( F_{\sigma} \) set, and \( \Q \) cannot be a \( G_{\delta} \) set.
\end{exercise}

\begin{solution}
    We will argue by contradiction. Suppose that \( \I \) is an \( F_{\sigma} \) set, so that \( \I = \bigcup_{m=1}^{\infty} F_m \), where each \( F_m \) is closed. Note that for any \( m \in \N \), it must be the case that \( F_m \) contains no non-empty open interval; otherwise, \( F_m \) would contain infinitely many rational numbers. Let \( \{ r_1, r_2, \ldots \} \) be an enumeration of \( \Q \), so that \( \Q = \bigcup_{n=1}^{\infty} \{ r_n \} \). For any \( n \in \N \), the singleton \( \{ r_n \} \) is closed and contains no non-empty interval. Observe that
    \[
        \R = \I \cup \Q = \left( \bigcup_{m=1}^{\infty} F_m \right) \cup \left( \bigcup_{n=1}^{\infty} \{ r_n \} \right) = \bigcup_{(m, n) \in \N^2} (F_m \cup \{ r_n \}).
    \]
    For any \( (m, n) \in \N^2 \), the union \( F_m \cup \{ r_n \} \) is closed and contains no non-empty intervals. However, since \( \N^2 \) is countable, this expression for \( \R \) contradicts \Cref{ex:5}. Hence it must be the case that \( \I \) is not an \( F_{\sigma} \) set, which by \Cref{ex:1} implies that \( \Q \) cannot be a \( G_{\delta} \) set.
\end{solution}

\begin{exercise}
\label{ex:7}
    Using \Cref{ex:6} and versions of the statements in \Cref{ex:2}, construct a set that is neither in \( F_{\sigma} \) nor in \( G_{\delta} \).
\end{exercise}

\begin{solution}
    Define \( E := (\I \cap (-\infty, 0]) \cup (\Q \cap [0, \infty)) \); we claim that \( E \) is neither an \( F_{\sigma} \) nor a \( G_{\delta} \) set. Seeking a contradiction, suppose that \( E \) is an \( F_{\sigma} \) set. It is not hard to see that any interval is an \( F_{\sigma} \) set (see \Cref{ex:3}), so by \Cref{ex:2} (b) we have that
    \[
        E \cap (-\infty, 0) = \I \cap (-\infty, 0)
    \]
    is an \( F_{\sigma} \) set, i.e.\ there is a countable collection \( \{ F_m : m \in \N \} \) of closed sets such that
    \[
        \I \cap (-\infty, 0) = \bigcup_{m=1}^{\infty} F_m.  
    \]
    For \( m \in \N \), let \( -F_m = \{ -x : x \in F_m \} \). Then since \( (x_n) \to x \) implies \( (-x_n) \to -x \), each \( -F_m \) is also closed. Furthermore, we have
    \[
        \I \cap (0, \infty) = \bigcup_{m=1}^{\infty} -F_m.
    \]
    It follows that \( \I \cap (0, \infty) \) is an \( F_{\sigma} \) set. However, \Cref{ex:2} (a) now implies that
    \[
        \I = (\I \cap (-\infty, 0)) \cup (\I \cap (0, \infty))
    \]
    is an \( F_{\sigma} \) set, contradicting \Cref{ex:6}. A similar argument with \( \Q \) shows that \( E \) cannot be a \( G_{\delta} \) set either.
\end{solution}

\begin{exercise}
\label{ex:8}
    Show that a set \( E \) is nowhere-dense in \( \R \) if and only if the complement of \( \overline{E} \) is dense in \( \R \).
\end{exercise}

\begin{solution}
    We will show that \( A \subseteq \R \) contains no non-empty open intervals if and only if \( \setcomp{A} \) is dense in \( \R \). By \( A \) containing no non-empty open intervals, we mean that for all \( x, y \in \R \) such that \( x < y \), we have \( (x, y) \not\subseteq A \). This is equivalent to saying that for all \( x, y \in \R \) such that \( x < y \), there exists some \( t \in \R \) such that \( x < t < y \) and \( t \not\in A \). In other words, \( \setcomp{A} \) is dense in \( \R \).
\end{solution}

\begin{exercise}
\label{ex:9}
    Decide whether the following sets are dense in \( \R \), nowhere-dense in \( \R \), or somewhere in between.
    \begin{enumerate}
        \item \( A = \Q \cap [0, 5] \).

        \item \( B = \{ 1/n : n \in \N \} \).

        \item the set of irrationals.

        \item the Cantor set.
    \end{enumerate}
\end{exercise}

\begin{solution}
    \begin{enumerate}
        \item We have \( \overline{A} = [0, 5] \), which is not the entire real line and also contains non-empty open intervals. Thus \( A \) is neither dense nor nowhere-dense.

        \item We have \( \overline{B} = \{ 0 \} \cup B \neq \R \), so that \( B \) is not dense. Note that if \( \overline{B} \) contained a non-empty open interval then \( \overline{B} \) would contain at least one irrational number, but \( \overline{B} \subseteq \Q \). Thus \( \overline{B} \) contains no non-empty open intervals and hence \( B \) is nowhere-dense.

        \item \( \I \) is dense in \( \R \) (see \href{https://lew98.github.io/Mathematics/UA_Section_1_4_Exercises.pdf}{Exercise 1.4.5}) and hence cannot be nowhere-dense (a dense subset \( E \subseteq \R \) certainly cannot be nowhere-dense; \( \overline{E} = \R \) contains every non-empty open interval).

        \item The Cantor set is closed, so \( \overline{C} = C \neq \R \). Thus \( C \) is not dense in \( \R \). \( C \) also does not contain any non-empty open intervals; given any \( x < y \) in \( C \), it is always possible to find some \( t \not\in C \) such that \( x < t < y \) (see \href{https://lew98.github.io/Mathematics/UA_Section_3_4_Exercises.pdf}{Exercise 3.4.8}). Thus \( C \) is nowhere-dense in \( \R \).
    \end{enumerate}
\end{solution}

\begin{exercise}
\label{ex:10}
    Finish the proof by finding a contradiction to the results in this section.
\end{exercise}

\begin{solution}
    Since \( E_n \subseteq \overline{E_n} \) for each \( n \in \N \), we have \( \R = \bigcup_{n=1}^{\infty} \overline{E_n} \). However, each \( \overline{E_n} \) is closed and by assumption contains no non-empty open intervals, so this contradicts \Cref{ex:5}.
\end{solution}

\noindent \hrulefill

\noindent \hypertarget{ua}{\textcolor{blue}{[UA]} Abbott, S. (2015) \textit{Understanding Analysis.} 2\ts{nd} edition.}

\end{document}