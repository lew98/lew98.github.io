\documentclass[12pt]{article}
\usepackage[utf8]{inputenc}
\usepackage[utf8]{inputenc}
\usepackage{amsmath}
\usepackage{amsthm}
\usepackage{geometry}
\usepackage{amsfonts}
\usepackage{mathrsfs}
\usepackage{bm}
\usepackage{hyperref}
\usepackage[dvipsnames]{xcolor}
\usepackage{enumitem}
\usepackage{mathtools}
\usepackage{changepage}
\usepackage{lipsum}
\usepackage{tikz}
\usetikzlibrary{matrix}
\usepackage{tikz-cd}
\usepackage[nameinlink]{cleveref}
\geometry{
headheight=15pt,
left=60pt,
right=60pt
}
\setlength{\emergencystretch}{20pt}
\usepackage{fancyhdr}
\pagestyle{fancy}
\fancyhf{}
\lhead{}
\chead{Section 3.A Exercises}
\rhead{\thepage}
\hypersetup{
    colorlinks=true,
    linkcolor=blue,
    urlcolor=blue
}

\theoremstyle{definition}
\newtheorem*{remark}{Remark}

\newtheoremstyle{exercise}
    {}
    {}
    {}
    {}
    {\bfseries}
    {.}
    { }
    {\thmname{#1}\thmnumber{#2}\thmnote{ (#3)}}
\theoremstyle{exercise}
\newtheorem{exercise}{Exercise 3.A.}

\newtheoremstyle{solution}
    {}
    {}
    {}
    {}
    {\itshape\color{magenta}}
    {.}
    { }
    {\thmname{#1}\thmnote{ #3}}
\theoremstyle{solution}
\newtheorem*{solution}{Solution}

\Crefformat{exercise}{#2Exercise 3.A.#1#3}

\newcommand{\poly}{\mathcal{P}}
\newcommand{\lmap}{\mathcal{L}}
\newcommand{\ts}{\textsuperscript}
\newcommand{\Span}{\text{span}}
\newcommand{\quand}{\quad \text{and} \quad}
\newcommand{\setcomp}[1]{#1^{\mathsf{c}}}
\newcommand{\N}{\mathbf{N}}
\newcommand{\Z}{\mathbf{Z}}
\newcommand{\Q}{\mathbf{Q}}
\newcommand{\R}{\mathbf{R}}
\newcommand{\C}{\mathbf{C}}
\newcommand{\F}{\mathbf{F}}

\DeclarePairedDelimiter\abs{\lvert}{\rvert}
% Swap the definition of \abs* and \norm*, so that \abs
% and \norm resizes the size of the brackets, and the 
% starred version does not.
\makeatletter
\let\oldabs\abs
\def\abs{\@ifstar{\oldabs}{\oldabs*}}
%
\let\oldnorm\norm
\def\norm{\@ifstar{\oldnorm}{\oldnorm*}}
\makeatother

\setlist[enumerate,1]{label={(\alph*)}}

\begin{document}

\section{Section 3.A Exercises}

Exercises with solutions from Section 3.A of \hyperlink{ladr}{[LADR]}.

\begin{exercise}
\label{ex:1}
    Suppose \( b, c \in \R \). Define \( T : \R^3 \to \R^2 \) by
    \[
        T(x, y, z) = (2x - 4y + 3z + b, 6x + cxyz).
    \]
    Show that \( T \) is linear if and only if \( b = c = 0 \).
\end{exercise}

\begin{solution}
    Suppose that \( b = c = 0 \), so that \( T \) is the map
    \[
        T(x, y, z) = (2x - 4y + 3z, 6x).
    \]
    Let \( (x_1, y_1, z_1), (x_2, y_2, z_2) \in \R^3 \) and \( \lambda \in \R \) be given. Then
    \begin{align*}
        T(x_1 + x_2, y_1 + y_2, z_1 + z_2) &= (2(x_1 + x_2) - 4(y_1 + y_2) + 3(z_1 + z_2), 6(x_1 + x_2)) \\
        &= (2 x_1 + 2 x_2 - 4 y_1 - 4 y_2 + 3 z_1 + 3 z_2, 6 x_1 + 6 x_2) \\
        &= (2 x_1 - 4 y_1 + 3 z_1, 6 x_1) + (2 x_2 - 4 y_2 + 3 z_2, 6 x_2) \\
        &= T(x_1, y_1, z_1) + T(x_2, y_2, z_2). \\[5mm]
        T(\lambda x_1, \lambda y_1, \lambda z_1) &= (2 \lambda x_1 - 4 \lambda y_1 + 3 \lambda z_1, 6 \lambda x_1) \\
        &= (\lambda (2 x_1 - 4 y_1 + 3 z_1), \lambda (6 x_1)) \\
        &= \lambda (2 x_1 - 4 y_1 + 3 z_1, 6 x_1) \\
        &= \lambda T(x_1, y_1, z_1).
    \end{align*}
    Thus \( T \) is linear.

    Suppose that \( b \neq 0 \). Then \( T(0, 0, 0) = (b, 0) \neq (0, 0) \), so \( T \) cannot be linear by 3.11. Now suppose that \( c \neq 0 \). Then
    \[
        T(1, 1, 1) = (1 + b, 6 + c) \quand T(2, 2, 2) = (2 + b, 12 + 8c).
    \]
    Since \( 2(6 + c) = 12 + 2c \neq 12 + 8c \) for \( c \neq 0 \), we see that \( 2 T(1, 1, 1) \neq T(2, 2, 2) \). Thus \( T \) is not linear.
\end{solution}

\begin{exercise}
\label{ex:2}
    Suppose \( b, c \in \R \). Define \( T : \poly(\R) \to \R^2 \) by
    \[
        T p = \left( 3 p(4) + 5 p'(6) + b p(1) p(2), \int_{-1}^2 x^3 p(x) dx + c \sin p(0) \right).
    \]
    Show that \( T \) is linear if and only if \( b = c = 0 \).
\end{exercise}

\begin{solution}
    Suppose that \( b = c = 0 \), so that \( T \) is the map
    \[
        T p = \left( 3 p(4) + 5 p'(6), \int_{-1}^2 x^3 p(x) dx \right).
    \]
    Let \( p, q \in \poly(\R) \) and \( \lambda \in \R \) be given. Then
    \begin{align*}
        T(p + q) &= \left( 3 (p + q)(4) + 5 (p + q)'(6), \int_{-1}^2 x^3 (p + q)(x) dx \right) \\
        &= \left( 3 (p(4) + q(4)) + 5 (p'(6) + q'(6)), \int_{-1}^2 x^3 (p(x) + q(x)) dx \right) \\
        &= \left( 3 p(4) + 3 q(4) + 5 p'(6) + 5 q'(6), \int_{-1}^2 x^3 p(x) dx + \int_{-1}^2 x^3 q(x) dx \right) \\
        &= \left( 3 p(4) + 5 p'(6), \int_{-1}^2 x^3 p(x) dx \right) + \left( 3 q(4) + 5 q'(6), \int_{-1}^2 x^3 q(x) dx \right) \\
        &= T p + T q. \\[5mm]
        T(\lambda p) &= \left( 3 (\lambda p)(4) + 5 (\lambda p)'(6), \int_{-1}^2 x^3 (\lambda p)(x) dx \right) \\
        &= \left( 3 (\lambda p(4)) + 5 (\lambda p'(6)), \int_{-1}^2 x^3 (\lambda p(x)) dx \right) \\
        &= \left( \lambda (3 p(4) + 5 p'(6)), \lambda \int_{-1}^2 x^3 p(x) dx \right) \\
        &= \lambda \left( 3 p(4) + 5 p'(6), \int_{-1}^2 x^3 p(x) dx \right) \\
        &= \lambda T p.
    \end{align*}
    Thus \( T \) is linear.

    Now suppose that \( T \) is linear. Observe that
    \begin{align*}
        T(\pi) &= \left( 3 \pi + b \pi^2, \frac{15 \pi}{4} + c \right), \\
        2 T(\pi) &= \left( 6 \pi + 2 b \pi^2, \frac{15 \pi}{2} + 2c \right), \\
        T(2 \pi) &= \left( 6 \pi + 4 b \pi^2, \frac{15 \pi}{2} \right).
    \end{align*}
    Since \( T \) is linear, we must have \( 2 T(\pi) = T(2 \pi) \), i.e.\
    \[
        \left( 6 \pi + 2 b \pi^2, \frac{15 \pi}{2} + 2c \right) = \left( 6 \pi + 4 b \pi^2, \frac{15 \pi}{2} \right) \iff (2 b \pi^2, 2c) = (4 b \pi^2, 0) \iff b = c = 0.
    \]
\end{solution}

\begin{exercise}
\label{ex:3}
    Suppose \( T \in \lmap(\F^n, \F^m) \). Show that there exist scalars \( A_{j,k} \in \F \) for \( j = 1, \ldots, m \) and \( k = 1, \ldots, n \) such that
    \[
        T(x_1, \ldots, x_n) = (A_{1,1} x_1 + \cdots + A_{1,n} x_n, \ldots, A_{m,1} x_1 + \cdots + A_{m,n} x_n)
    \]
    for every \( (x_1, \ldots, x_n) \in \F^n \).

    \noindent [\textit{The exercise above shows that \( T \) has the form promised in the last item of Example 3.4.}]
\end{exercise}

\begin{solution}
    Let \( e_1, \ldots, e_n \) be the standard basis of \( \F^n \) and let \( f_1, \ldots, f_m \) be the standard basis of \( \F^m \). For any \( k \in \{ 1, \ldots, n \} \), we have \( T e_k \in \F^m \). Thus there exist scalars \( A_{1,k}, \ldots, A_{m,k} \) such that
    \[
        T e_k = \sum_{j=1}^m A_{j,k} f_j.
    \]
    Let \( x = (x_1, \ldots, x_n) = \sum_{k=1}^n x_k e_k \) be given. Then by linearity,
    \begin{align*}
        T x &= T \left( \sum_{k=1}^n x_k e_k \right) \\
        &= \sum_{k=1}^n x_k T e_k \\
        &= \sum_{k=1}^n x_k \sum_{j=1}^m A_{j,k} f_j \\
        &= \sum_{j=1}^m \left( \sum_{k=1}^n A_{j,k} x_k \right) f_j \\
        &= \left( \sum_{k=1}^n A_{1,k} x_k, \ldots, \sum_{k=1}^n A_{m,k} x_k \right) \\
        &= (A_{1,1} x_1 + \cdots + A_{1,n} x_n, \ldots, A_{m,1} x_1 + \cdots + A_{m,n} x_n).
    \end{align*}
\end{solution}

\begin{exercise}
\label{ex:4}
    Suppose \( T \in \lmap(V, W) \) and \( v_1, \ldots, v_m \) is a list of vectors in \( V \) such that \( T v_1, \ldots, T v_m \) is a linearly independent list in \( W \). Prove that \( v_1, \ldots, v_m \) is linearly independent.
\end{exercise}

\begin{solution}
    Suppose we have scalars \( a_1, \ldots, a_m \) such that
    \[
        a_1 v_1 + \cdots + a_m v_m = 0.
    \]
    Applying \( T \) to both sides of this equation and using linearity, we obtain
    \[
        T(a_1 v_1 + \cdots + a_m v_m) = T(0) \iff a_1 T v_1 + \cdots + a_m T v_m = 0.
    \]
    Since the list \( T v_1, \ldots, T v_m \) is linearly independent, this implies that \( a_1 = \cdots = a_m = 0 \). Thus the list \( v_1, \ldots, v_m \) is linearly independent.
\end{solution}

\begin{exercise}
\label{ex:5}
    Prove the assertion in 3.7.
\end{exercise}

\begin{solution}
    The assertion is that, if \( V \) and \( W \) are vector spaces, then \( \lmap(V, W) \) is a vector space. First, let us show that \( \lmap(V, W) \) is closed under addition and scalar multiplication, i.e.\ if \( S \) and \( T \) are linear maps from \( V \) to \( W \) and \( \lambda \in \F \) is a scalar, then \( S + T \) and \( \lambda S \) are linear maps from \( V \) to \( W \). Let \( u, v \in V \) and \( \alpha \in \F \) be given. Then
    \begin{multline*}
        (S + T)(u + v) = S(u + v) + T(u + v) = Su + Sv + Tu + Tv \\ = Su + Tu + Sv + Tv = (S + T)(u) + (S + T)(v).
    \end{multline*}
    \[
        (S + T)(\alpha u) = S(\alpha u) + T(\alpha u) = \alpha S u + \alpha T u = \alpha (S u + T u) = \alpha (S + T)(u).
    \]
    Thus \( S + T \in \lmap(V, W) \). Similarly,
    \begin{gather*}
        (\lambda S)(u + v) = \lambda S(u + v) = \lambda (S u + S v) = \lambda S u + \lambda S v = (\lambda S)(u) + (\lambda S)(v). \\[3mm]
        (\lambda S)(\alpha u) = \lambda S(\alpha u) = \lambda (\alpha S u) = \alpha (\lambda S u) = \alpha (\lambda S)(u).
    \end{gather*}
    Thus \( \lambda S \in \lmap(V, W) \). To prove that \( \lmap(V, W) \) is a vector space with these operations, we will verify each of the requirements from 1.19.
    \begin{description}
        \item[Commutativity.] Suppose \( S, T \in \lmap(V, W) \) and \( u \in V \). Then
        \[
            (S + T)(u) = Su + Tu = Tu + Su = (T + S)(u).
        \]
        Thus \( S + T = T + S \).
        
        \item[Associativity.] Suppose \( R, S, T \in \lmap(V, W), a, b \in \F, \) and \( u \in V \). Then
        \begin{multline*}
            ((R + S) + T)(u) = (R + S)(u) + Tu = (Ru + Su) + Tu = Ru + (Su + Tu) \\ = Ru + (S + T)(u) = (R + (S + T))(u).
        \end{multline*}
        \[
            ((ab)R)(u) = (ab) Ru = a (b Ru) = a ((bR)(u)) = (a(bR))(u).
        \]
        Thus \( (R + S) + T = R + (S + T) \) and \( (ab)R = a(bR) \).

        \item[Additive identity.] It is easily verified that the map \( 0 : V \to W \) given by \( v \mapsto 0 \) belongs to \( \lmap(V, W) \). We claim that this map is the additive identity in \( \lmap(V, W) \). Let \( S \in \lmap(V, W) \) and \( u \in V \) be given. Then
        \[
            (S + 0)(u) = Su + 0u = Su + 0 = Su.
        \]
        Thus \( S + 0 = S \).

        \item[Additive inverse.] Suppose that \( S \in \lmap(V, W) \). Define \( T : V \to W \) by \( Tu = -Su \); it is not hard to see that \( T \) is linear. We claim that \( T \) is the additive inverse to \( S \). Indeed, for any \( u \in V \),
        \[
            (S + T)(u) = Su + Tu = Su + (-Su) = 0.
        \]
        Thus \( S + T = 0 \).

        \item[Multiplicative identity.] Let \( S \in \lmap(V, W) \) and \( u \in V \) be given. Then
        \[
            (1S)(u) = 1 Su = Su.
        \]
        Thus \( 1S = S \).

        \item[Distributive properties.] Let \( S, T \in \lmap(V, W), a, b \in \F, \) and \( u \in V \) be given. Then
        \begin{gather*}
            (a(S + T))(u) = a (S + T)(u) = a (Su + Tu) = a Su + a Tu = (aS)(u) + (aT)(u). \\[3mm]
            ((a + b)S)(u) = (a + b) Su = a Su + b Su = (aS)(u) + (bS)(u).
        \end{gather*}
        Thus \( a(S + T) = aS + aT \) and \( (a + b)S = aS + bS \).
    \end{description}
\end{solution}

\begin{exercise}
\label{ex:6}
    Prove the assertions in 3.9.
\end{exercise}

\begin{solution}
    The first assertion is that if the products make sense, then \( (T_1 T_2) T_3 = T_1 (T_2 T_3) \). This is certainly the case, since the composition of functions is associative.

    The second assertion is that if \( T \in \lmap(V, W) \), \( I_V \) is the identity map on \( V \), and \( I_W \) is the identity map on \( W \), then \( T I_V = I_W T = T \). Indeed, let \( v \in V \) be given. Then
    \[
        (T I_V)(v) = T(I_V v) = Tv \quand (I_W T)(v) = I_W(Tv) = Tv.
    \]

    The third assertion is that if \( T, T_1, T_2 \in \lmap(U, V) \) and \( S, S_1, S_2 \in \lmap(V, W) \), then
    \[
        (S_1 + S_2)T = S_1 T + S_2 T \quand S(T_1 + T_2) = S T_1 + S T_2.
    \]
    Let \( u \in U \) be given. Then
    \begin{gather*}
        ((S_1 + S_2)T)(u) = (S_1 + S_2)(Tu) = S_1(Tu) + S_2(Tu) = (S_1 T)(u) + (S_2 T)(u). \\[3mm]
        (S(T_1 + T_2))(u) = S((T_1 + T_2)(u)) = S(T_1 u + T_2 u) = S(T_1 u) + S(T_2 u) = (S T_1)(u) + (S T_2)(u).
    \end{gather*}
\end{solution}

\begin{exercise}
\label{ex:7}
    Show that every linear map from a 1-dimensional vector space to itself is multiplication by some scalar. More precisely, prove that if \( \dim V = 1 \) and \( T \in \lmap(V, V) \), then there exists \( \lambda \in \F \) such that \( Tv = \lambda v \) for all \( v \in V \).
\end{exercise}

\begin{solution}
    Since \( \dim V = 1 \), there exists a basis \( u \) for \( V \). Then since \( T u \in V \), it must be of the form \( \lambda u \) for some \( \lambda \in \F \). Let \( v = \alpha u \in V \) be given. Then
    \[
        Tv = T(\alpha u) = \alpha Tu = \alpha (\lambda u) = \lambda (\alpha u) = \lambda v.
    \]
\end{solution}

\begin{exercise}
\label{ex:8}
    Give an example of a function \( \varphi : \R^2 \to \R \) such that
    \[
        \varphi(av) = a \varphi(v)
    \]
    for all \( a \in \R \) and all \( v \in \R^2 \) but \( \varphi \) is not linear.

    \noindent [\textit{The exercise above and the next exercise show that neither homogeneity nor additivity alone is enough to imply that a function is a linear map.}]
\end{exercise}

\begin{solution}
    Let \( \varphi : \R^2 \to \R \) be given by \( \varphi(x, y) = \left( x^3 + y^3 \right)^{\tfrac{1}{3}} \). Then for any \( a \in \R \) and \( (x, y) \in \R^2 \), we have
    \[
        \varphi(ax, ay) = \left( (ax)^3 + (ay)^3 \right)^{\tfrac{1}{3}} = \left( a^3 \right)^{\tfrac{1}{3}} \left( x^3 + y^3 \right)^{\tfrac{1}{3}} = a \left( x^3 + y^3 \right)^{\tfrac{1}{3}} = a \varphi(x, y).
    \]
    However, observe that
    \[
        \varphi(1, 0) + \varphi(0, 1) = 1 + 1 = 2 \neq 2^{\tfrac{1}{3}} = \varphi(1, 1).
    \]
    Thus \( \varphi \) is not linear.
\end{solution}

\begin{exercise}
\label{ex:9}
    Give an example of a function \( \varphi : \C \to \C \) such that
    \[
        \varphi(w + z) = \varphi(w) + \varphi(z)
    \]
    for all \( w, z \in \C \) but \( \varphi \) is not linear. (Here \( \C \) is thought of as a complex vector space.)

    \noindent [\textit{There also exists a function \( \varphi : \R \to \R \) such that \( \varphi \) satisfies the additivity condition above but \( \varphi \) is not linear. However, showing the existence of such a function involves considerably more advanced tools.}]
\end{exercise}

\begin{solution}
    Let \( \varphi : \C \to \C \) be given by \( \varphi(x + iy) = x \), i.e.\ \( \varphi \) takes a complex number to its real part. Then
    \[
        \varphi((x + iy) + (u + iv)) = \varphi((x + u) + i(y + v)) = x + u = \varphi(x + iy) + \varphi(u + iv).
    \]
    However, \( \varphi(i) = 0 \) and \( \varphi(i^2) = \varphi(-1) = -1 \neq i \varphi(i) \). Thus \( \varphi \) is not linear.
\end{solution}

\begin{exercise}
\label{ex:10}
    Suppose \( U \) is a subspace of \( V \) with \( U \neq V \). Suppose \( S \in \lmap(U, W) \) and \( S \neq 0 \) (which means that \( Su \neq 0 \) for some \( u \in U \)). Define \( T : V \to W \) by
    \[
        Tv = \begin{cases}
            Sv & \text{if } v \in U, \\
            0 & \text{if } v \in V \text{ and } v \not\in U.
        \end{cases}
    \]
    Prove that \( T \) is not a linear map on \( V \).
\end{exercise}

\begin{solution}
    There is some \( u \in U \) such that \( Su \neq 0 \), and since \( U \neq V \) there is some \( v \in V \) such that \( v \not\in U \). This implies that \( u - v \not\in U \), otherwise \( v = -(u - v) + u \in U \). Then we have
    \[
        Tv + T(u - v) = 0 + 0 = 0 \neq Su = Tu = T(v + u - v).
    \]
    Thus \( T \) is not linear.
\end{solution}

\begin{exercise}
\label{ex:11}
    Suppose \( V \) is finite-dimensional. Prove that every linear map on a subspace of \( V \) can be extended to a linear map on \( V \). In other words, show that if \( U \) is a subspace of \( V \) and \( S \in \lmap(U, W) \), then there exists \( T \in \lmap(V, W) \) such that \( Tu = Su \) for all \( u \in U \).
\end{exercise}

\begin{solution}
    Let \( u_1, \ldots, u_m \) be a basis of \( U \), which we extend to a basis \( u_1, \ldots, u_m, v_1, \ldots, v_n \) of \( V \). By 3.5, we may define a (unique) linear map \( T : V \to W \) by specifying
    \[
        Tu_j = Su_j \text{ for } 1 \leq j \leq m \quand Tv_j = 0 \text{ for } 1 \leq j \leq n.
    \]
    If \( u \in U \), then there are scalars \( a_1, \ldots, a_m \) such that \( u = a_1 u_1 + \cdots + a_m u_m \). Observe that
    \begin{multline*}
        Tu = T(a_1 u_1 + \cdots + a_m u_m) = a_1 Tu_1 + \cdots + a_m Tu_m \\ = a_1 Su_1 + \cdots + a_m Su_m = S(a_1 u_1 + \cdots + a_m u_m) = Su.
    \end{multline*}
    Thus \( T \) extends \( S \).
\end{solution}

\begin{exercise}
\label{ex:12}
    Suppose \( V \) is finite-dimensional with \( \dim V > 0 \), and suppose \( W \) is infinite-dimensional. Prove that \( \lmap(V, W) \) is infinite-dimensional.
\end{exercise}

\begin{solution}
    For a finite-dimensional vector space \( V \) with \( \dim V > 0 \), we wish to prove that
    \[
        W \text{ is infinite-dimensional} \implies \lmap(V, W) \text{ is infinite-dimensional.}
    \]
    We will prove the contrapositive statement:
    \[
        \lmap(V, W) \text{ is finite-dimensional} \implies W \text{ is finite-dimensional.}
    \]
    Suppose therefore that \( \lmap(V, W) = \Span(S_1, \ldots, S_m) \) for some (possibly empty) list \( S_1, \ldots, S_m \) in \( \lmap(V, W) \). Since \( V \) is finite-dimensional with \( \dim V > 0 \), there is a non-empty basis \( v_1, \ldots, v_n \) for \( V \). Consider the (possibly empty) list \( w_1, \ldots, w_m \) where \( w_j := S_j v_1 \). We claim that this list spans \( W \). To see this, let \( w \in W \) be given. By 3.5, there is a (unique) linear map \( T : \Span(v_1) \to W \) such that \( T v_1 = w \). Then by \Cref{ex:11}, \( T \) can be extended to a linear map \( S : V \to W \). Since \( \lmap(V, W) = \Span(S_1, \ldots, S_m) \), there are scalars \( a_1, \ldots, a_m \) such that \( S = a_1 S_1 + \cdots + a_m S_m \) (this is to be understood as the ``empty linear combination'' if \( \lmap(V, W) = \{ 0 \} \), so that \( S = 0 \)). Then observe that
    \[
        w = S v_1 = (a_1 S_1 + \cdots + a_m S_m)(v_1) = a_1 S_1 v_1 + \cdots + a_m S_m v_1 = a_1 w_1 + \cdots + a_m w_m.
    \]
    Thus \( W = \Span(w_1, \ldots, w_m) \). This shows that \( W \) is finite-dimensional and moreover that \( \dim W \leq \dim \lmap(V, W) \).
\end{solution}

\begin{exercise}
\label{ex:13}
    Suppose \( v_1, \ldots, v_m \) is a linearly dependent list of vectors in \( V \). Suppose also that \( W \neq \{ 0 \} \). Prove that there exist \( w_1, \ldots, w_m \in W \) such that no \( T \in \lmap(V, W) \) satisfies \( T v_k = w_k \) for each \( k = 1, \ldots, m \).
\end{exercise}

\begin{solution}
    Since \( W \neq \{ 0 \} \), there is a \( w \in W \) such that \( w \neq 0 \), and by the Linear Dependence Lemma, there is a \( j \in \{ 1, \ldots, m \} \) such that \( v_j \in \Span(v_1, \ldots, v_{j-1}) \). If \( j = 1 \) then \( v_1 = 0 \). Consider the list \( w, \ldots, w \) of length \( m \) in \( W \). If \( T \in \lmap(V, W) \), then \( T v_1 = T(0) = 0 \neq w \), and so we have found the desired list.
    
    If \( j > 1 \), then there are scalars \( a_1, \ldots, a_{j-1} \) such that \( v_j = a_1 v_1 + \cdots + a_{j-1} v_{j-1} \).  Consider the list \( w_1, \ldots, w_m \) where \( w_k = w \) if \( k \neq j \) and \( w_j = (a_1 + \cdots + a_{j-1} + 1) w \). Suppose we have some \( T \in \lmap(V, W) \) such that \( T v_k = w_k \) for \( 1 \leq k < j \). Then observe that
    \begin{multline*}
        T v_j = T(a_1 v_1 + \cdots + a_{j-1} v_{j-1}) = a_1 T v_1 + \cdots + a_{j-1} T v_{j-1} \\ = a_1 w + \cdots + a_{j-1} w = (a_1 + \cdots + a_{j-1}) w.
    \end{multline*}
    Since \( w \neq 0 \), we have \( T v_j = (a_1 + \cdots + a_{j-1}) w \neq (a_1 + \cdots + a_{j-1} + 1) w = w_j \). Thus no \( T \in \lmap(V, W) \) can possibly satisfy \( T v_k = w_k \) for each \( k = 1, \ldots, m \).
\end{solution}

\begin{exercise}
\label{ex:14}
    Suppose \( V \) is finite-dimensional with \( \dim V \geq 2 \). Prove that there exist \( S, T \in \lmap(V, V) \) such that \( ST \neq TS \).
\end{exercise}

\begin{solution}
    There is a basis \( v_1, v_2, \ldots, v_n \) for \( V \) with \( n \geq 2 \). By 3.5, to define \( S, T \in \lmap(V, V) \), it is enough to specify where the basis vectors \( v_1, v_2, \ldots, v_n \) get mapped to. So let \( S \) be the linear map defined by \( S v_1 = v_2, S v_2 = v_1, \) and \( S v_j = v_j \) for \( j \geq 2 \), and let \( T \) be the linear map defined by \( T v_1 = 2 v_2, T v_2 = v_1, \) and \( T v_j = v_j \) for \( j \geq 2 \). Then observe that
    \[
        (ST - TS)(v_1) = S(T v_1) - T(S v_1) = S(2 v_2) - T v_2 = 2 v_1 - v_1 = v_1 \neq 0.
    \]
    Thus \( ST \neq TS \).
\end{solution}

\noindent \hrulefill

\noindent \hypertarget{ladr}{\textcolor{blue}{[LADR]} Axler, S. (2015) \textit{Linear Algebra Done Right.} 3\ts{rd} edition.}

\end{document}