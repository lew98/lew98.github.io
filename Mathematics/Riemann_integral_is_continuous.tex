\documentclass[12pt]{article}
\usepackage[utf8]{inputenc}
\usepackage{amsmath}
\usepackage{amsthm}
\usepackage{geometry}
\usepackage{amsfonts}
\usepackage{bm}
\geometry{
headheight=15pt,
left=60pt,
right=60pt
}
\usepackage{fancyhdr}
\pagestyle{fancy}
\fancyhf{}
\lhead{}
\chead{Riemann integral is continuous}
\rhead{}

\pagenumbering{gobble}
\setlength{\parindent}{0pt}

\newcommand{\newp}{\vspace{5mm}}

\theoremstyle{definition}
\newtheorem{theorem}{Theorem}

\newtheorem*{remark}{Remark}

\begin{document}

\section{Riemann integral is continuous}

This small theorem supports the claim that ``integrating functions makes them nicer", since \( f \) is required only to be Riemann-integrable and not necessarily continuous on its domain.

\begin{theorem}

Suppose we have some interval \( I \), some \( t_0 \in I \) and some Riemann-integrable function \( f : I \to \mathbb{R} \). Define \( g : I \to \mathbb{R} \) by
\[
g(x) = \int_{t_0}^x{ f(t) \, \text{d}t }.
\]
Then \( g \) is continuous.

\end{theorem}

\begin{proof}

Since \( f \) is Riemann-integrable, it is bounded by some \( M > 0 \). Let \( y \in I \) be given and fix some \( \varepsilon > 0 \). Then provided we take \( x \in I \) such that \( |x - y| < \frac{\varepsilon}{M} \), we will have
\begin{align*}
|g(x) - g(y)| &= \left| \int_{t_0}^x f(t) \, \text{d}t - \int_{t_0}^y f(t) \, \text{d}t \right| \\
&= \left| \int_y^x f(t) \, \text{d}t \right| \\
&\leq \int_y^x |f(t)| \, \text{d}t \\
&\leq M \, |x - y| \\
&< \varepsilon,
\end{align*}
as desired.
\end{proof}

\end{document}
