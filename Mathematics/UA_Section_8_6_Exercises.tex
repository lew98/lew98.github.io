\documentclass[12pt]{article}
\usepackage[utf8]{inputenc}
\usepackage[utf8]{inputenc}
\usepackage{amsmath}
\usepackage{amsthm}
\usepackage{amssymb}
\usepackage{array}
\usepackage{geometry}
\usepackage{amsfonts}
\usepackage{mathrsfs}
\usepackage{bm}
\usepackage{hyperref}
\usepackage{float}
\usepackage[dvipsnames]{xcolor}
\usepackage[inline]{enumitem}
\usepackage{mathtools}
\usepackage{changepage}
\usepackage{graphicx}
\usepackage{systeme}
\usepackage{caption}
\usepackage{subcaption}
\usepackage[breakable]{tcolorbox}
\usepackage[linguistics]{forest}
\usepackage{tikz}
\usetikzlibrary{matrix, patterns, decorations.pathreplacing, calligraphy}
\usepackage{tikz-cd}
\usepackage[nameinlink]{cleveref}
\geometry{
headheight=15pt,
left=60pt,
right=60pt
}
\setlength{\emergencystretch}{20pt}
\usepackage{fancyhdr}
\pagestyle{fancy}
\fancyhf{}
\lhead{}
\chead{Section 8.6 Exercises}
\rhead{\thepage}
\hypersetup{
    colorlinks=true,
    linkcolor=blue,
    urlcolor=blue
}

\theoremstyle{definition}
\newtheorem*{remark}{Remark}

\newtheoremstyle{exercise}
    {}
    {}
    {}
    {}
    {\bfseries}
    {.}
    { }
    {\thmname{#1}\thmnumber{#2}\thmnote{ (#3)}}
\theoremstyle{exercise}
\newtheorem{exercise}{Exercise 8.6.}

\newtheoremstyle{solution}
    {}
    {}
    {}
    {}
    {\itshape\color{magenta}}
    {.}
    { }
    {\thmname{#1}\thmnote{ #3}}
\theoremstyle{solution}
\newtheorem*{solution}{Solution}

\Crefformat{exercise}{#2Exercise 8.6.#1#3}

\tcbset{colback=blue!5!white, breakable}

\newcommand{\interior}[1]{%
  {\kern0pt#1}^{\mathrm{o}}%
}
\newcommand{\ts}{\textsuperscript}
\newcommand{\setcomp}[1]{#1^{\mathsf{c}}}
\newcommand{\poly}{\mathcal{P}}
\newcommand{\quand}{\quad \text{and} \quad}
\newcommand{\quimplies}{\quad \implies \quad}
\newcommand{\quiff}{\quad \iff \quad}
\newcommand{\N}{\mathbf{N}}
\newcommand{\Z}{\mathbf{Z}}
\newcommand{\Q}{\mathbf{Q}}
\newcommand{\I}{\mathbf{I}}
\newcommand{\R}{\mathbf{R}}
\newcommand{\C}{\mathbf{C}}

\DeclarePairedDelimiter\abs{\lvert}{\rvert}
\makeatletter
\let\oldabs\abs
\def\abs{\@ifstar{\oldabs}{\oldabs*}}

\DeclarePairedDelimiter\norm{\lVert}{\rVert}
\makeatletter
\let\oldnorm\norm
\def\norm{\@ifstar{\oldnorm}{\oldnorm*}}
\makeatother

\DeclarePairedDelimiter\paren{(}{)}
\makeatletter
\let\oldparen\paren
\def\paren{\@ifstar{\oldparen}{\oldparen*}}
\makeatother

\DeclarePairedDelimiter\bkt{[}{]}
\makeatletter
\let\oldbkt\bkt
\def\bkt{\@ifstar{\oldbkt}{\oldbkt*}}
\makeatother

\DeclarePairedDelimiter\set{\{}{\}}
\makeatletter
\let\oldset\set
\def\set{\@ifstar{\oldset}{\oldset*}}
\makeatother

\setlist[enumerate,1]{label={(\alph*)}}

\begin{document}

\section{Section 8.6 Exercises}

Exercises with solutions from Section 8.6 of \hyperlink{ua}{[UA]}.

\begin{exercise}
\label{ex:1}
    \begin{enumerate}
        \item Fix \( r \in \Q \). Show that the set \( C_r = \{ t \in \Q : t < r \} \) is a cut.

        The temptation to think of all cuts as being of this form should be avoided. Which of the following subsets of \( \Q \) are cuts?

        \item \( S = \{ t \in \Q : t \leq 2 \} \)

        \item \( T = \{ t \in \Q : t^2 < 2 \text{ or } t < 0 \} \)

        \item \( U = \{ t \in \Q : t^2 \leq 2 \text{ or } t < 0 \} \)
    \end{enumerate}
\end{exercise}

\begin{solution}
    \begin{enumerate}
        \item It is clear that \( C_r \) satisfies (c1) and (c2). To see that \( C_r \) satisfies (c3), observe that if \( t \in C_r \), then \( t < \tfrac{t + r}{2} \) and \( \tfrac{t + r }{2} \in C_r \).

        \item This is not a cut, since it has 2 as a maximum element.

        \item This is a cut. \( T \) satisfies (c1) since \( 0 \in T \) and \( 2 \not\in T \).
        
        Suppose \( t \in T \) and \( r \) is a rational such that \( r < t \). If \( r < 0 \) then certainly \( r \in T \), so suppose that \( r \geq 0 \), which implies that \( t > 0 \). It follows that \( r^2 < t^2 < 2 \) and so \( r \in T \). Thus \( T \) satisfies (c2).

        Suppose \( t \in T \). If \( t \leq 0 \) then let \( r = 1 \) and if \( t > 0 \) then let \( r = \frac{2t + 2}{t + 2} \). In either case, one can verify that \( t < r \) and \( r \in T \). Thus \( T \) satisfies (c3).

        \item By Theorem 1.1.1 we have \( U = T \) and hence by part (c) \( U \) is a cut.
    \end{enumerate}
\end{solution}

\begin{exercise}
\label{ex:2}
    Let \( A \) be a cut. Show that if \( r \in A \) and \( s \not\in A \), then \( r < s \).
\end{exercise}

\begin{solution}
    Given that \( r \in A \), the implication \( s \not\in A \implies r < s \) is the contrapositive of (c2).
\end{solution}

\begin{exercise}
\label{ex:3}
    Using the usual definitions of addition and multiplication, determine which of these properties are possessed by \( \N, \Z, \) and \( \Q \), respectively.
\end{exercise}

\begin{solution}
    \( \N \) satisfies (f1), (f2), and (f5). It fails (f3) since there is no additive identity and it fails (f4) since no element has an additive inverse and only 1 has a multiplicative inverse (1 is its own inverse).

    \( \Z \) satisfies (f1), (f2), (f3), and (f5). It fails (f4) since, while each element has an additive inverse, only 1 and \( -1 \) have multiplicative inverses (they are their own inverses).

    \( \Q \) satisfies each property (f1) - (f5).
\end{solution}

\begin{exercise}
\label{ex:4}
    Show that this defines an ordering on \( \R \) by verifying properties (o1), (o2), and (o3) from Definition 8.6.5.
\end{exercise}

\begin{solution}
    Properties (o2) and (o3) are clear, so let us verify property (o1). It will suffice to show that if \( B \not\subseteq A \), then \( A \subseteq B \). Since \( B \) is not a subset of \( A \), there exists some \( s \in B \) such that \( s \not\in A \). Let \( r \in A \) be given. By \Cref{ex:2} we must have \( r < s \) and so by (c2) we have \( r \in B \). Thus \( A \subseteq B \).
\end{solution}

\begin{exercise}
\label{ex:5}
    \begin{enumerate}
        \item Show that (c1) and (c3) also hold for \( A + B \). Conclude that \( A + B \) is a cut.

        \item Check that addition in \( \R \) is commutative (f1) and associative (f2).
        
        \item Show that property (o4) holds.

        \item Show that the cut
        \[
            O = \{ p \in \Q : p < 0 \}
        \]
        successfully plays the role of the additive identity (f3). (Showing \( A + O = A \) amounts to proving that these two sets are the same. The standard way to prove such a thing is to show two inclusions: \( A + O \subseteq A \) and \( A \subseteq A + O \).)
    \end{enumerate}
\end{exercise}

\begin{solution}
    \begin{enumerate}
        \item Since \( A \) and \( B \) are non-empty, \( A + B \) must also be non-empty. Since neither \( A \) nor \( B \) contains every rational number, there exist rationals \( r \not\in A \) and \( s \not\in B \). It follows from \Cref{ex:2} that \( a + b < r + s \) for every \( a \in A \) and \( b \in B \), so that \( r + s \not\in A + B \). Thus \( A + B \neq \Q \) and we have now shown that \( A + B \) satisfies (c1).

        Let \( a + b \in A + B \) be given. By (c3), there exist rationals \( r \in A \) and \( s \in B \) such that \( a < r \) and \( b < s \). It follows that \( a + b < r + s \) and \( r + s \in A + B \). Thus \( A + B \) satisfies (c3).

        \item Commutativity and associativity of addition in \( \R \) follow immediately from commutativity and associativity of addition in \( \Q \).

        \item Let \( A, B, \) and \( C \) be cuts such that \( B \subseteq C \). If \( a + b \in A + B \), then \( a + b \in A + C \) also since \( B \subseteq C \). Thus \( A + B \subseteq A + C \).

        \item Let \( a + p \in A + O \) be given. Then \( p < 0 \), so \( a + p < a \) and it follows from (c2) that \( a + p \in A \); thus \( A + O \subseteq A \).

        Now let \( a \in A \) be given. By (c3) there exists some \( b \in A \) such that \( a < b \). Notice that \( a = b + (a - b) \in A + O \), since \( a - b < 0 \). It follows that \( A \subseteq A + O \) and we may conclude that \( A + O = A \).
    \end{enumerate}
\end{solution}

\begin{exercise}
\label{ex:6}
    \begin{enumerate}
        \item Prove that \( -A \) defines a cut.

        \item What goes wrong if we set \( -A = \{ r \in \Q : -r \not\in A \} \)?

        \item If \( a \in A \) and \( r \in -A \), show \( a + r \in O \). This shows \( A + (-A) \subseteq O \). Now, finish the proof of property (f4) for addition in Definition 8.6.4.
    \end{enumerate}
\end{exercise}

\begin{solution}
    \begin{enumerate}
        \item Since \( A \neq \Q \), there exists a \( t \not\in A \). Then \( -t - 1 \in -A \), since \( t < -(-t - 1) = t + 1 \). Thus \( -A \) is non-empty. Since \( A \) is non-empty, there exists some \( r \in A \). Then \( -r \not\in -A \), since if \( t \not\in A \) then \( t > -(-r) = r \) by \Cref{ex:2}. Thus \( -A \neq \Q \) and we see that \( -A \) satisfies (c1).

        Suppose that \( r \in -A \), so that there is some \( t \not\in A \) such that \( t < -r \), and suppose that \( s < r \). Then \( t < -r < -s \), demonstrating that \( s \in -A \) also. Thus \( -A \) satisfies (c2).

        Suppose that \( r \in -A \), so that there is some \( t \not\in A \) such that \( t < -r \). Define \( s = r - \tfrac{r + t}{2} \) and notice that \( r < s \) since \( 0 < -r - t \). Furthermore, \( s \in -A \) since
        \[
            t \not\in A \quand t < \frac{t - r}{2} = -s.
        \]
        Thus \( -A \) satisfies (c3) and we may conclude that \( -A \) is a cut.

        \item This does not necessarily define a cut. For example, let \( C_2 \) be the cut \( \{ r \in \Q : r < 2 \} \). Then using this definition, we find that \( -C_2 = \{ r \in \Q : r \leq -2 \} \), which fails property (c3).

        \item There exists a \( t \not\in A \) such that \( t < -r \). By \Cref{ex:2} it must be the case that \( a < t < -r \) and thus \( a + r < 0 \), i.e.\ \( a + r \in O \). Thus \( A + (-A) \subseteq O \).
        
        For the reverse inclusion, let \( p < 0 \) be a given rational number in \( O \). We claim that there must exist some \( r \in A \) such that \( r - \tfrac{p}{2} \not\in A \), and we will prove this by contradiction. Suppose that \( r - \tfrac{p}{2} \in A \) for all \( r \in A \). Since \( A \) is a cut, there is some \( r_0 \in A \). An induction argument shows that \( r_0 - \tfrac{np}{2} \in A \) for all \( n \in \N \). Let \( q \in \Q \) be given and use the Archimedean property of \( \Q \) to obtain an \( n \in \N \) such that \( r_0 - \tfrac{np}{2} > q \); it follows from (c2) that \( q \in A \). The conclusion is that \( A = \Q \), contradicting (c1).

        Thus there is some \( r \in A \) such that \( r - \tfrac{p}{2} \not\in A \). Since \( r - \tfrac{p}{2} < r - p \), it follows that \( p - r \in -A \). Then \( p = r + p - r \in A + (-A) \), demonstrating that \( O \subseteq A + (-A) \). We may conclude that \( A + (-A) = O \).
    \end{enumerate}
\end{solution}

\begin{exercise}
\label{ex:7}
    \begin{enumerate}
        \item Show that \( AB \) is a cut and that property (o5) holds.

        \item Propose a good candidate for the multiplicative identity (1) on \( \R \) and show that this works for all cuts \( A \geq O \).

        \item Show the distributive property (f5) holds for non-negative cuts.
    \end{enumerate}
\end{exercise}

\begin{solution}
    \begin{enumerate}
        \item It is clear that \( AB \) is non-empty. If either \( A = O \) or \( B = O \), then it is straightforward to verify that \( AB = O \neq \Q \). Suppose that \( A > O \) and \( B > O \). There exist rationals \( r \not\in A \) and \( s \not\in B \); clearly, \( r, s > 0 \). If \( q \in AB \), then either \( q < 0 \) or \( q = ab \) for \( a \in A, b \in B \) and \( a, b \geq 0 \). By \Cref{ex:2} we must have \( a < r \) and \( b < s \), so that \( ab < rs \). In either case, we have \( q < rs \) and thus \( rs \not\in AB \), demonstrating that \( AB \neq \Q \). Thus \( AB \) satisfies (c1).

        Suppose \( r \in AB \) and \( q < r \). If \( q < 0 \) then \( q \in AB \), so suppose that \( q \geq 0 \), which implies that \( r > 0 \). We must then have \( r = ab \) for some \( a \in A, b \in B \) with \( a, b > 0 \). Notice that \( \tfrac{q}{b} < a \); (c2) then implies that \( \tfrac{q}{b} \in A \) and hence \( q = \tfrac{q}{b} \cdot b \in AB \). Thus \( AB \) satisfies (c2).

        If \( A = O \) or \( B = O \) then \( AB = O \), which has no maximum element. Suppose that \( A > O \) and \( B > O \) and let \( r \in AB \) be given. If \( r \leq 0 \) then let \( q \) be any positive rational in \( AB \). If \( r > 0 \), then \( r = ab \) for some \( a \in A, b \in B \) with \( a, b > 0 \). By (c3), there exist rationals \( s \in A, t \in B \) such that \( a < s \) and \( b < t \). Let \( q = st \) and notice that \( q \in AB \) and \( r = ab < st = q \). In either case, there exists a \( q \in AB \) with \( r < q \). Thus \( AB \) satisfies (c3) and we may conclude that \( AB \) is a cut.

        Property (o5) is clear from the definition of \( AB \).

        \item Define \( I = \{ p \in \Q : p < 1 \} \) and let \( A \geq O \) be given. We claim that \( AI = A \). Suppose that \( r \in AI \). If \( r < 0 \), then \( r \in A \), so suppose that \( r \geq 0 \). Thus \( r = ab \) for some \( a \in A \) such that \( a \geq 0 \) and some \( 0 \leq b < 1 \). It follows that \( ab < a \) and so by (c2) we have \( r = ab \in A \). Thus \( AI \subseteq A \).

        Now suppose that \( a \in A \). If \( a \leq 0 \), then (c2) implies that \( 2a \in A \) and thus \( a = (2a) \cdot \tfrac{1}{2} \in AI \). If \( a > 0 \), then (c3) implies there is some \( r \in A \) with \( a < r \). Thus \( \tfrac{a}{r} \in I \) and we see that \( a = r \cdot \tfrac{a}{r} \in AI \). Hence \( A \subseteq AI \) and we may conclude that \( AI = A \).

        \item Let \( A, B, C \geq O \) be cuts. If \( ABC = O \) then the equality \( A(B + C) = AB + AC \) is clear, so suppose that \( A, B, C > O \) and suppose that \( q \in A(B + C) \). If \( q < 0 \) then \( q = \tfrac{q}{2} + \tfrac{q}{2} \in AB + AC \). Suppose that \( q \geq 0 \). Then \( q = a(b + c) = ab + ac \), where \( a \in A, b \in B, c \in C \) and \( a, b + c \geq 0 \). There are three cases: \( b, c \geq 0 \), \( b \geq 0 \) and \( c < 0 \), or \( b < 0 \) and \( c \geq 0 \). In any of these cases, it is straightforward to verify that \( ab + ac \in AB + AC \). Thus \( A(B + C) \subseteq AB + AC \).

        Now suppose that \( p + q \in AB + AC \). If \( p + q < 0 \), then \( p + q \in A(B + C) \), so suppose that \( p + q \geq 0 \). If \( p, q \geq 0 \), then \( p = a_1 b \) and \( q = a_2 c \), for some \( a_1, a_2 \in A, b \in B, \) and \( c \in C \) such that \( a_1, a_2, b, c \geq 0 \). Let \( a = \max \{ a_1, a_2 \} \) and notice that \( a(b + c) \in A(B + C) \). Furthermore, \( p + q = a_1 b + a_2 c \leq a b + a c = a(b + c) \). It follows from (c2) that \( p + q \in A(B + C) \).
        
        Next, suppose that \( p < 0 \) and \( q \geq 0 \), so that \( q = ac \) for some \( a \in A, c \in C \) with \( a, c \geq 0 \). Let \( b \in B \) be such that \( b \geq 0 \); such a \( b \) exists since \( B > O \). Now notice that
        \[
            p + q = p + ac < ac \leq a(b + c) \in A(B + C).
        \]
        It follows from (c2) that \( p + q \in A(B + C) \). The case where \( p \geq 0 \) and \( q < 0 \) is handled similarly. Thus \( AB + AC \subseteq A(B + C) \) and we may conclude that \( A(B + C) = AB + AC \).
    \end{enumerate}
\end{solution}

\begin{exercise}
\label{ex:8}
    Let \( \mathcal{A} \subseteq \R \) be nonempty and bounded above, and let \( S \) be the \textit{union} of all \( A \in \mathcal{A} \).
    \begin{enumerate}
        \item First, prove that \( S \in \R \) by showing that it is a cut.

        \item Now, show that \( S \) is the least upper bound for \( \mathcal{A} \).
    \end{enumerate}
\end{exercise}

\begin{solution}
    \begin{enumerate}
        \item Since \( \mathcal{A} \) is non-empty, it contains some cut \( A \), so that \( A \subseteq S \). It follows that \( S \) is non-empty as \( A \) is non-empty. Since \( \mathcal{A} \) is bounded above, there exists some cut \( B \) such that \( A \subseteq B \) for all \( A \in \mathcal{A} \). It follows that \( S \subseteq B \) and hence that \( S \neq \Q \) since \( B \neq \Q \). Thus \( S \) satisfies (c1).

        Suppose \( r \in S \), so that \( r \in A \) for some \( A \in \mathcal{A} \), and suppose \( q < r \). Since \( A \) is a cut we must have \( q \in A \), which gives \( q \in S \). Thus \( S \) satisfies (c2).

        Suppose \( r \in S \), so that \( r \in A \) for some \( A \in \mathcal{A} \). Since \( A \) is a cut there exists some \( q \in A \) such that \( r < q \); note that \( q \in S \) also. Thus \( S \) satisfies (c3). We may conclude that \( S \) is a cut.

        \item It is clear that \( S \) is an upper bound for \( \mathcal{A} \). If \( B \) is any upper bound for \( \mathcal{A} \), then \( B \) contains every \( A \in \mathcal{A} \) and hence must contain the union of all \( A \in \mathcal{A} \), i.e.\ \( S \subseteq B \). It follows that \( S \) is the least upper bound for \( \mathcal{A} \).
    \end{enumerate}
\end{solution}

\begin{exercise}
\label{ex:9}
    Consider the collection of so-called ``rational'' cuts of the form
    \[
        C_r = \{ t \in \Q : t < r \}
    \]
    where \( r \in \Q \). (See \Cref{ex:1}.)
    \begin{enumerate}
        \item Show that \( C_r + C_s = C_{r+s} \) for all \( r, s \in \Q \). Verify \( C_r C_s = C_{rs} \) for the case when \( r, s \geq 0 \).

        \item Show that \( C_r \leq C_s \) if and only if \( r \leq s \) in \( \Q \).
    \end{enumerate}
\end{exercise}

\begin{solution}
    \begin{enumerate}
        \item Let \( r, s \in \Q \) be given and suppose \( a + b \in C_r + C_s \), i.e.\ \( a < r \) and \( b < s \). It follows that \( a + b < r + s \) and hence that \( a + b \in C_{r+s} \). Thus \( C_r + C_s \subseteq C_{r+s} \). Now suppose that \( t \in C_{r + s} \), so that \( t < r + s \). Choose a positive integer \( n \in \N \) such that \( t + \tfrac{1}{n} < r + s \) and note that:
        \begin{itemize}
            \item \( s - \tfrac{1}{n} < s \), so that \( s - \tfrac{1}{n} \in C_s \);

            \item \( t + \tfrac{1}{n} - s < r \), so that \( t + \tfrac{1}{n} - s \in C_r \);

            \item \( t = \paren{ t + \tfrac{1}{n} - s } + \paren{ s - \tfrac{1}{n} } \in C_r + C_s \).
        \end{itemize}
        Thus \( C_{r+s} \subseteq C_r + C_s \) and we may conclude that \( C_r + C_s = C_{r+s} \).

        It is clear that \( C_r C_s = C_{rs} \) if \( rs = 0 \), so suppose that \( r, s > 0 \) and let \( q \in C_r C_s \) be given. If \( q \leq 0 \) then \( q < rs \), i.e.\ \( q \in C_{rs} \). If \( q > 0 \) then \( q = ab \) for some \( 0 < a < r \) and \( 0 < b < s \). It follows that \( 0 < ab < rs \) and thus \( q = ab \in C_{rs} \). Hence \( C_r C_s \subseteq C_{rs} \).

        Now let \( q \in C_{rs} \) be given. If \( q \leq 0 \) then certainly \( q \in C_r C_s \), so suppose that \( q > 0 \) and define \( p = \tfrac{1}{2} \paren{ \tfrac{q}{s} + r } \). Notice that:
        \begin{itemize}
            \item \( 0 < \tfrac{q}{s} < p < r \), so that \( p \in C_r \);

            \item \( 0 < \tfrac{q}{p} < s \), so that \( \tfrac{q}{p} \in C_s \);

            \item \( q = p \cdot \tfrac{q}{p} \in C_r C_s \).
        \end{itemize}
        Thus \( C_{rs} \subseteq C_r C_s \) and we may conclude that \( C_r C_s = C_{rs} \).

        \item If \( r \leq s \) then it is clear that \( C_r \subseteq C_s \). If \( s < r \), then it is again clear that \( C_s \subseteq C_r \). Furthermore, notice that \( C_s \neq C_r \) since \( \tfrac{s + r}{2} \) belongs to \( C_r \) but not to \( C_s \). Thus \( C_s \subsetneq C_r \).
    \end{enumerate}
\end{solution}

\noindent \hrulefill

\noindent \hypertarget{ua}{\textcolor{blue}{[UA]} Abbott, S. (2015) \textit{Understanding Analysis.} 2\ts{nd} edition.}

\end{document}