\documentclass[12pt]{article}
\usepackage[utf8]{inputenc}
\usepackage[utf8]{inputenc}
\usepackage{amsmath}
\usepackage{amsthm}
\usepackage{geometry}
\usepackage{amsfonts}
\usepackage{mathrsfs}
\usepackage{bm}
\usepackage{hyperref}
\usepackage[dvipsnames]{xcolor}
\usepackage{enumitem}
\usepackage{mathtools}
\usepackage{changepage}
\usepackage{lipsum}
\usepackage{tikz}
\usetikzlibrary{matrix}
\usepackage{tikz-cd}
\usepackage[nameinlink]{cleveref}
\geometry{
headheight=15pt,
left=60pt,
right=60pt
}
\setlength{\emergencystretch}{10pt}
\usepackage{fancyhdr}
\pagestyle{fancy}
\fancyhf{}
\lhead{}
\chead{Section 2.6 Exercises}
\rhead{\thepage}
\hypersetup{
    colorlinks=true,
    linkcolor=blue,
    urlcolor=blue
}

\theoremstyle{definition}
\newtheorem*{remark}{Remark}

\newtheoremstyle{exercise}
    {}
    {}
    {}
    {}
    {\bfseries}
    {.}
    { }
    {\thmname{#1}\thmnumber{#2}\thmnote{ (#3)}}
\theoremstyle{exercise}
\newtheorem{exercise}{Exercise 2.6.}

\newtheoremstyle{solution}
    {}
    {}
    {}
    {}
    {\itshape\color{magenta}}
    {.}
    { }
    {\thmname{#1}\thmnote{ #3}}
\theoremstyle{solution}
\newtheorem*{solution}{Solution}

\Crefformat{exercise}{#2Exercise 2.6.#1#3}

\newcommand{\setcomp}[1]{#1^{\mathsf{c}}}
\newcommand{\N}{\mathbf{N}}
\newcommand{\Z}{\mathbf{Z}}
\newcommand{\Q}{\mathbf{Q}}
\newcommand{\R}{\mathbf{R}}
\newcommand{\C}{\mathbf{C}}

\DeclarePairedDelimiter\abs{\lvert}{\rvert}
% Swap the definition of \abs* and \norm*, so that \abs
% and \norm resizes the size of the brackets, and the 
% starred version does not.
\makeatletter
\let\oldabs\abs
\def\abs{\@ifstar{\oldabs}{\oldabs*}}
%
\let\oldnorm\norm
\def\norm{\@ifstar{\oldnorm}{\oldnorm*}}
\makeatother

\setlist[enumerate,1]{label={(\alph*)}}

\begin{document}

\section{Section 2.6 Exercises}

Exercises with solutions from Section 2.6 of \hyperlink{ua}{[UA]}.

\begin{exercise}
\label{ex:1}
    Supply a proof for Theorem 2.6.2.
\end{exercise}

\begin{solution}
    Let \( (x_n) \) be a convergent sequence; we will show that \( (x_n) \) is Cauchy. Let \( \epsilon > 0 \) be given. There is an \( N \in \N \) such that \( n \geq N \) implies that \( \abs{x_n - x} < \tfrac{\epsilon}{2} \). Then for \( m, n \geq N \) we have
    \[
        \abs{x_n - x_m} \leq \abs{x_n - x} + \abs{x_m - x} < \tfrac{\epsilon}{2} + \tfrac{\epsilon}{2} = \epsilon.
    \]
    It follows that \( (x_n) \) is a Cauchy sequence.
\end{solution}

\begin{exercise}
\label{ex:2}
    Give an example of each of the following, or argue that such a request is impossible.
    \begin{enumerate}
        \item A Cauchy sequence that is not monotone.

        \item A Cauchy sequence with an unbounded subsequence.

        \item A divergent monotone sequence with a Cauchy subsequence.

        \item An unbounded sequence containing a subsequence that is Cauchy.
    \end{enumerate}
\end{exercise}

\begin{solution}
    \begin{enumerate}
        \item Consider the sequence \( (x_n) \) given by \( x_n = \tfrac{(-1)^n}{n} \). The sequence is convergent (\( \lim x_n = 0 \)) and hence Cauchy (\Cref{ex:1}), but is certainly not monotone.

        \item This is impossible. A Cauchy sequence \( (x_n) \) is necessarily convergent (Theorem 2.6.4) and hence all subsequences of \( (x_n) \) must be convergent (Theorem 2.5.2); each subsequence must then be bounded.

        \item First, let us show that if \( (x_n) \) is an unbounded monotone sequence, then all subsequences of \( (x_n) \) are also unbounded and monotone. Suppose \( (x_n) \) is increasing; the case where \( (x_n) \) is decreasing is handled similarly. Let \( (x_{n_k}) \) be a subsequence of \( (x_n) \). Suppose \( k > l \); then \( n_k > n_l \) and so \( x_{n_k} \geq x_{n_l} \) since \( (x_n) \) is increasing; it follows that \( (x_{n_k}) \) is an increasing sequence. Now let \( M > 0 \) be given. Since \( (x_n) \) is unbounded, there is an \( N \in \N \) such that \( x_N > M \), and since \( (x_{n_k}) \) is a subsequence we can find a \( K \in \N \) such that \( n_K > N \). It follows that \( x_{n_K} \geq x_N > M \) since \( (x_n) \) is increasing. We may conclude that \( (x_{n_k}) \) is unbounded.

        We can now show that the given request is impossible. If \( (x_n) \) is a divergent monotone sequence, then by the Monotone Convergence Theorem \( (x_n) \) must be unbounded. Then by the previous paragraph, all subsequences of \( (x_n) \) must be unbounded, hence divergent, and hence not Cauchy.

        \item Consider the unbounded sequence \( (0, 1, 0, 2, 0, 3, \ldots) \); the subsequence \( (0, 0, 0, \ldots) \) is convergent and hence Cauchy.
    \end{enumerate}
\end{solution}

\begin{exercise}
\label{ex:3}
    If \( (x_n) \) and \( (y_n) \) are Cauchy sequences, then one easy way to prove that \( (x_n + y_n) \) is Cauchy is to use the Cauchy Criterion. By Theorem 2.6.4, \( (x_n) \) and \( (y_n) \) must be convergent, and the Algebraic Limit Theorem then implies \( (x_n + y_n) \) is convergent and hence Cauchy.
    \begin{enumerate}
        \item Give a direct argument that \( (x_n + y_n) \) is a Cauchy sequence that does not use the Cauchy Criterion or the Algebraic Limit Theorem.

        \item Do the same for the product \( (x_n y_n) \).
    \end{enumerate}
\end{exercise}

\begin{solution}
    \begin{enumerate}
        \item Let \( \epsilon > 0 \) be given. There are positive integers \( N_1 \) and \( N_2 \) such that
        \[
            m, n \geq N_1 \implies \abs{x_n - x_m} < \tfrac{\epsilon}{2} \quad \text{and} \quad m, n \geq N_2 \implies \abs{y_n - y_m} < \tfrac{\epsilon}{2}.
        \]
        Let \( N = \max \{ N_1, N_2 \} \) and observe that for \( m, n \geq N \) we have
        \[
            \abs{x_n + y_n - x_m - y_m} \leq \abs{x_n - x_m} + \abs{y_n - y_m} < \tfrac{\epsilon}{2} + \tfrac{\epsilon}{2} = \epsilon.
        \]
        It follows that \( (x_n + y_n) \) is a Cauchy sequence.

        \item By Lemma 2.6.3, Cauchy sequences are bounded. So there are positive real numbers \( M_1 \) and \( M_2 \) such that \( \abs{x_n} \leq M_1 \) and \( \abs{y_n} \leq M_2 \) for all \( n \in \N \). Let \( \epsilon > 0 \) be given. There are positive integers \( N_1 \) and \( N_2 \) such that
        \[
            m, n \geq N_1 \implies \abs{x_n - x_m} < \frac{\epsilon}{2 M_2} \quad \text{and} \quad m, n \geq N_2 \implies \abs{y_n - y_m} < \frac{\epsilon}{2 M_1}.
        \]
        Let \( N = \max \{ N_1, N_2 \} \) and observe that for \( m, n \geq N \) we have
        \begin{multline*}
            \abs{x_n y_n - x_m y_m} = \abs{x_n y_n - x_m y_n + x_m y_n - x_m y_m} \leq \abs{y_n} \abs{x_n - x_m} + \abs{x_m} \abs{y_n - y_m} \\
            \leq M_2 \frac{\epsilon}{2 M_2} + M_1 \frac{\epsilon}{2 M_1} = \epsilon.
        \end{multline*}
        It follows that \( (x_n y_n) \) is a Cauchy sequence.
    \end{enumerate}
\end{solution}

\begin{exercise}
\label{ex:4}
    Let \( (a_n) \) and \( (b_n) \) be Cauchy sequences. Decide whether each of the following sequences is a Cauchy sequence, justifying each conclusion.
    \begin{enumerate}
        \item \( c_n = \abs{a_n - b_n} \)

        \item \( c_n = (-1)^n a_n \)

        \item \( c_n = [[a_n]] \), where \( [[x]] \) refers to the greatest integer less than or equal to \( x \).
    \end{enumerate}
\end{exercise}

\begin{solution}
    By the Cauchy Criterion, we have \( \lim a_n = a \) and \( \lim b_n = b \) for some real numbers \( a \) and \( b \). Again by the Cauchy Criterion, it will suffice to consider convergence of the given sequences \( (c_n) \).
    \begin{enumerate}
        \item By \href{https://lew98.github.io/Mathematics/UA_Section_2_3_Exercises.pdf}{Exercise 2.3.10} (b) and the Algebraic Limit Theorem, we have
        \[
            \lim c_n = \lim \abs{a_n - b_n} = \abs{ \lim a_n - \lim b_n} = \abs{a - b}.
        \]
        So \( (c_n) \) is convergent and hence Cauchy.

        \item Suppose that \( a = 0 \). By \href{https://lew98.github.io/Mathematics/UA_Section_2_3_Exercises.pdf}{Exercise 2.3.9} (a) we then have \( \lim c_n = 0 \). It follows that \( (c_n) \) is Cauchy. If \( a \neq 0 \), then observe that
        \[
            \lim c_{2n} = \lim a_{2n} = a \neq -a = \lim (- a_{2n-1}) = \lim c_{2n-1}.
        \]
        So \( (c_n) \) has two subsequences which converge to different limits. It follows that \( (c_n) \) is not convergent (Theorem 2.5.2) and hence not Cauchy.

        \item Suppose that \( a \) is not an integer, so that \( [[a]] < a < [[a]] + 1 \). Let
        \[
            \delta = \min \{ a - [[a]], [[a]] + 1 - a \}.
        \]
        Since \( \lim a_n = a \), there is a positive integer \( N \) such that \( n \geq N \) implies that \( a_n \in (a - \delta, a + \delta) \). Observe that \( [[a]] \leq a - \delta \) and \( a + \delta \leq [[a]] + 1 \). Then for \( n \geq N \) we have \( [[a]] < a_n < [[a]] + 1 \), which gives us \( [[a_n]] = [[a]] \). So the sequence \( [[a_n]] \) is eventually constant with value \( [[a]] \); it follows that \( [[a_n]] \) is convergent with limit \( [[a]] \) and hence Cauchy.

        If \( a \) is an integer, then the sequence \( ([[a_n]]) \) may or may not be convergent (and so may or may not be Cauchy). For example, if \( (a_n) \) is the sequence \( (0, 0, 0, \ldots) \) then clearly \( \lim [[a_n]] = 0 \). However, consider the sequence \( a_n = \tfrac{(-1)^n}{n} \), which also satisfies \( \lim a_n = 0 \). This gives
        \[
            ([[a_n]]) = (-1, 0, -1, 0, -1, 0, \ldots),
        \]
        which is divergent.
    \end{enumerate}
\end{solution}

\begin{exercise}
\label{ex:5}
    Consider the following (invented) definition: A sequence \( (s_n) \) is \textit{pseudo-Cauchy} if, for all \( \epsilon > 0 \), there exists an \( N \) such that if \( n \geq N \), then \( \abs{s_{n+1} - s_n} < \epsilon \).

    Decide which one of the following two propositions is actually true. Supply a proof for the valid statement and a counterexample for the other.
    \begin{enumerate}[label = (\roman*)]
        \item Psuedo-Cauchy sequences are bounded.

        \item If \( (x_n) \) and \( (y_n) \) are pseudo-Cauchy, then \( (x_n + y_n) \) is pseudo-Cauchy as well.
    \end{enumerate}
\end{exercise}

\begin{solution}
    \begin{enumerate}[label = (\roman*)]
        \item This statement is false in general. Consider the sequence \( (s_n) \) given by \( s_n = \sum_{m=1}^n \tfrac{1}{m} \). Then \( s_{n+1} - s_n = \tfrac{1}{n+1} \to 0 \); it follows that \( (s_n) \) is pseudo-Cauchy. However, as shown in Example 2.4.5, \( (s_n) \) is unbounded.

        \item This statement is true. Let \( \epsilon > 0 \) be given. There are positive integers \( N_1 \) and \( N_2 \) such that
        \[
            n \geq N_1 \implies \abs{x_{n+1} - x_n} < \tfrac{\epsilon}{2} \quad \text{and} \quad n \geq N_2 \implies \abs{y_{n+1} - y_n} < \tfrac{\epsilon}{2}.
        \]
        Let \( N = \max \{ N_1, N_2 \} \) and observe that for \( n \geq N \) we have
        \[
            \abs{x_{n+1} + y_{n+1} - x_n - y_n} \leq \abs{x_{n+1} - x_n} + \abs{y_{n+1} - y_n} < \tfrac{\epsilon}{2} + \tfrac{\epsilon}{2} = \epsilon.
        \]
        It follows that \( (x_n + y_n) \) is pseudo-Cauchy.
    \end{enumerate}
\end{solution}

\begin{exercise}
\label{ex:6}
    Let's call a sequence \( (a_n) \) \textit{quasi-increasing} if for all \( \epsilon > 0 \) there exists an \( N \) such that whenever \( n > m \geq N \) it follows that \( a_n > a_m - \epsilon \).
    \begin{enumerate}
        \item Give an example of a sequence that is quasi-increasing but not monotone or eventually monotone.

        \item Give an example of a quasi-increasing sequence that is divergent and not monotone or eventually monotone.

        \item Is there an analogue of the Monotone Convergence Theorem for quasi-increasing sequences? Give an example of a bounded, quasi-increasing sequence that doesn't converge, or prove that no such sequence exists.
    \end{enumerate}
\end{exercise}

\begin{solution}
    \begin{enumerate}
        \item Consider the sequence \( (a_n) \) given by
        \[
            a_n = \begin{cases}
                \tfrac{n+1}{2} & \text{if } n \text{ is odd}, \\
                \tfrac{n}{2} - \tfrac{2}{n} & \text{if } n \text{ is even}.
            \end{cases}
        \]
        Let \( m \) be a positive integer, and suppose that \( m \) is even. We claim that \( a_n > a_m \) for all \( n > m \). If \( n \) is even, then observe that
        \[
            n > m \implies \left( \tfrac{n}{2} > \tfrac{m}{2} \text{  and  } -\tfrac{2}{n} > -\tfrac{2}{m} \right) \implies \tfrac{n}{2} - \tfrac{2}{n} > \tfrac{m}{2} - \tfrac{2}{m},
        \]
        i.e.\ \( a_n > a_m \). If \( n \) is odd, then
        \[
            n > m \implies n + 1 > m \implies \tfrac{n+1}{2} > \tfrac{m}{2} \implies \tfrac{n+1}{2} > \tfrac{m}{2} - \tfrac{2}{m},
        \]
        i.e.\ \( a_n > a_m \). Now suppose that \( m \) is odd. We claim that \( a_n > a_m \) for all \( n > m + 1 \). If \( n \) is even, first observe that we must have \( n > m + 2 \). It follows that
        \[
            n > m + 1 + 1 \implies \tfrac{n}{2} > \tfrac{m+1}{2} + \tfrac{1}{2} \implies \tfrac{n}{2} - \tfrac{1}{2} > \tfrac{m+1}{2}.
        \]
        Since \( n > m + 2 \), it must be the case that \( n \geq 4 \). Hence
        \[
            n \geq 4 \implies -\tfrac{2}{n} \geq -\tfrac{1}{2} \implies \tfrac{n}{2} - \tfrac{2}{n} \geq \tfrac{n}{2} - \tfrac{1}{2} > \tfrac{m+1}{2},
        \]
        i.e.\ \( a_n > a_m \). If \( n \) is odd, then
        \[
            n > m + 1 \implies n + 1 > m + 1 \implies \tfrac{n+1}{2} > \tfrac{m+1}{2}
        \]
        i.e.\ \( a_n > a_m \). Finally, observe that
        \[
            a_m - a_{m+1} = \tfrac{m+1}{2} - \tfrac{m+1}{2} + \tfrac{2}{m+1} = \tfrac{2}{m+1}.
        \]

        To summarise, let \( m \) be a positive integer.
        \begin{itemize}
            \item If \( m \) is even, then \( a_n > a_m \) for all \( n > m \).

            \item If \( m \) is odd, then \( a_n > a_m \) for all \( n > m + 1 \) and \( a_m - a_{m+1} = \tfrac{2}{m+1} > 0 \).
        \end{itemize}
        Then \( (a_n) \) is not eventually monotone, for if \( N \) is a positive integer, choose an odd integer \( m \) such that \( m > N \); then \( a_m > a_{m+1} \) and \( a_m < a_{m+2} \). Furthermore, \( (a_n) \) is quasi-increasing. To see this, let \( \epsilon > 0 \) be given. Choose a positive integer \( N \) such that \( \tfrac{2}{N+1} < \epsilon \) and suppose that \( n > m \geq N \). By the summary above, we have
        \[
            a_m - a_n < 0 < \epsilon \implies a_n > a_m - \epsilon
        \]
        unless \( m \) is odd and \( n = m + 1 \). In that case we have
        \[
            a_m - a_{m+1} = \tfrac{2}{m+1} \leq \tfrac{2}{N+1} < \epsilon \implies a_n > a_m - \epsilon.
        \]

        \item The sequence \( (a_n) \) given in part (a) is also divergent, since it is clearly unbounded.

        \item There is an analogue of the Monotone Convergence Theorem for bounded quasi-increasing sequences. Let \( (a_n) \) be such a sequence; we will show that \( (a_n) \) converges to \( \limsup a_n \).

        Set \( s = \limsup a_n \) and \( y_n = \sup \{ a_l : l \geq n \} \), so that \( \lim y_n = s \). By \href{https://lew98.github.io/Mathematics/UA_Section_2_5_Exercises.pdf}{Exercise 2.5.2} (c), there is a subsequence \( (a_{n_k}) \) converging to \( s \). Let \( \epsilon > 0 \) be given. There is an \( N_1 \in \N \) such that \( n \geq N_1 \) implies that \( \abs{y_n - s} < \epsilon \). Since \( a_n \leq y_n \) for all \( n \in \N \), we have
        \[
            n \geq N_1 \implies a_n < s + \epsilon. \tag{1}
        \]
        Since \( (a_n) \) is quasi-increasing, there is an \( N_2 \in \N \) such that
        \[
            n > m \geq N_2 \implies a_m - \tfrac{\epsilon}{2} < a_n, \tag{2}
        \]
        and since \( (a_{n_k}) \to s \), there is a \( K \in \N \) such that
        \[
            k \geq K \implies \abs{a_{n_k} - s} < \tfrac{\epsilon}{2}. \tag{3}
        \]
        Since \( (a_{n_k}) \) is a subsequence, there must be some \( k' \in \N \) such that both \( k' \geq K \) and \( n_{k'} \geq N_2 \). Then
        \[
            n > n_{k'} \implies a_{n_{k'}} - \tfrac{\epsilon}{2} < a_n \qquad \text{by } (2),
        \]
        and \( s - \epsilon < a_{n_{k'}} - \tfrac{\epsilon}{2} \) by (3). Combining these gives
        \[
            n > n_{k'} \implies s - \epsilon < a_n. \tag{4}
        \]
        Let \( N = \max \{ N_1, n_{k'} \} \). Then by (1) and (4), we have
        \[
            n > N \implies s - \epsilon < a_n < s + \epsilon.
        \]
        It follows that \( \lim a_n = s \).
    \end{enumerate}
\end{solution}

\begin{exercise}
\label{ex:7}
    Exercises 2.4.4 and 2.5.4 establish the equivalence of the Axiom of Completeness and the Monotone Convergence Theorem. They also show that the Nested Interval Property is equivalent to these other two in the presence of the Archimedean Property.
    \begin{enumerate}
        \item Assume the Bolzano-Weierstrass Theorem is true and use it to construct a proof of the Monotone Convergence Theorem without making any appeal to the Archimedean Property. This shows that BW, AoC, and MCT are all equivalent.

        \item Use the Cauchy Criterion to prove the Bolzano-Weierstrass Theorem, and find the point in the argument where the Archimedean Property is implicitly required. This establishes the final link in the equivalence of the five characterizations of completeness discussed at the end of Section 2.6.

        \item How do we know it is impossible to prove the Axiom of Completeness starting from the Archimedean Property?
    \end{enumerate}
\end{exercise}

\begin{solution}
    \begin{enumerate}
        \item Suppose \( (x_n) \) is bounded and increasing (the case where \( (x_n) \) is decreasing is handled similarly). By assumption, there is a convergent subsequence \( (x_{n_k}) \), say \( \lim_{k \to \infty} x_{n_k} = x \). Let \( \epsilon > 0 \) be given. There is a \( K \in \N \) such that
        \[
            k \geq K \implies \abs{x_{n_k} - x} < \epsilon. \tag{1}
        \]
        Suppose \( n \in \N \) is such that \( n \geq n_K \). Since \( (x_n) \) is increasing, we then have \( x - \epsilon < x_{n_K} \leq x_n \). Furthermore, it must be the case that \( x_n < x + \epsilon \). If \( x_n \geq x + \epsilon \), then since \( (x_{n_k}) \) is a subsequence, there is some \( k \in \N \) such that \( n_k \geq n \geq n_K \). This implies that \( k \geq K \) and, since \( (x_n) \) is increasing, that \( x_{n_k} \geq x_n \geq x + \epsilon \); this contradicts (1). So we have shown that
        \[
            n \geq n_K \implies x - \epsilon < x_n < x + \epsilon.
        \]
        It follows that \( \lim x_n = x \).

        \item Let \( (x_n) \) be sequence bounded by some \( M > 0 \). As in the proof of the Bolzano-Weierstrass Theorem (Theorem 2.5.5) given in \hyperlink{ua}{[UA]}, construct a sequence of nested intervals \( (I_k) \) with length \( M(1/2)^{k-1} \) and a subsequence \( (x_{n_k}) \) such that \( x_{n_k} \in I_k \). Let \( \epsilon > 0 \) be given. Assuming that \( (1/2)^k \to 0 \) (this is equivalent to assuming the Archimedean Property), there is a \( K \in \N \) such that \( M(1/2)^{K-1} < \epsilon \). Suppose that \( k > l \geq K \). Then since the intervals are nested, both \( x_{n_k} \) and \( x_{n_l} \) belong to \( I_K \). It follows that \( x_{n_k} \) and \( x_{n_l} \) are no further apart than the width of \( I_K \), i.e.\
        \[
            \abs{x_{n_k} - x_{n_l}} \leq M(1/2)^{K-1} < \epsilon.
        \]
        This demonstrates that \( (x_{n_k}) \) is a Cauchy sequence. By assumption, this is equivalent to \( (x_{n_k}) \) being convergent.

        \item The ordered field \( \Q \) has the Archimedean Property but does not satisfy the Axiom of Completeness.
    \end{enumerate}
\end{solution}

\noindent \hrulefill

\noindent \hypertarget{ua}{\textcolor{blue}{[UA]} Abbott, S. (2015) \textit{Understanding Analysis.} 2nd edn.}

\end{document}