\documentclass[12pt]{article}
\usepackage[utf8]{inputenc}
\usepackage[utf8]{inputenc}
\usepackage{amsmath}
\usepackage{amsthm}
\usepackage{geometry}
\usepackage{amsfonts}
\usepackage{mathrsfs}
\usepackage{bm}
\usepackage{hyperref}
\usepackage[dvipsnames]{xcolor}
\usepackage{enumitem}
\usepackage{mathtools}
\usepackage{changepage}
\usepackage{lipsum}
\usepackage{tikz}
\usetikzlibrary{matrix}
\usepackage{tikz-cd}
\usepackage[nameinlink]{cleveref}
\geometry{
headheight=15pt,
left=60pt,
right=60pt
}
\usepackage{fancyhdr}
\pagestyle{fancy}
\fancyhf{}
\lhead{}
\chead{Section 1.A Exercises}
\rhead{\thepage}
\hypersetup{
    colorlinks=true,
    linkcolor=blue,
    urlcolor=blue
}

\theoremstyle{definition}
\newtheorem*{remark}{Remark}

\newtheoremstyle{exercise}
    {}
    {}
    {}
    {}
    {\bfseries}
    {.}
    { }
    {\thmname{#1}\thmnumber{#2}\thmnote{ (#3)}}
\theoremstyle{exercise}
\newtheorem{exercise}{Exercise 1.A.}

\newtheoremstyle{solution}
    {}
    {}
    {}
    {}
    {\itshape\color{magenta}}
    {.}
    { }
    {\thmname{#1}\thmnote{ #3}}
\theoremstyle{solution}
\newtheorem*{solution}{Solution}

\Crefformat{exercise}{#2Exercise 1.A.#1#3}

\newcommand{\setcomp}[1]{#1^{\mathsf{c}}}
\newcommand{\N}{\mathbf{N}}
\newcommand{\Z}{\mathbf{Z}}
\newcommand{\Q}{\mathbf{Q}}
\newcommand{\R}{\mathbf{R}}
\newcommand{\C}{\mathbf{C}}
\newcommand{\F}{\mathbf{F}}

\DeclarePairedDelimiter\abs{\lvert}{\rvert}
% Swap the definition of \abs* and \norm*, so that \abs
% and \norm resizes the size of the brackets, and the 
% starred version does not.
\makeatletter
\let\oldabs\abs
\def\abs{\@ifstar{\oldabs}{\oldabs*}}
%
\let\oldnorm\norm
\def\norm{\@ifstar{\oldnorm}{\oldnorm*}}
\makeatother

\setlist[enumerate,1]{label={(\alph*)}}

\begin{document}

\section{Section 1.A Exercises}

Exercises with solutions from Section 1.A of \hyperlink{ladr}{[LADR]}.

\begin{exercise}
\label{ex:1}
    Suppose \( a \) and \( b \) are real numbers, not both 0. Find real numbers \( c \) and \( d \) such that
    \[
        1/(a + bi) = c + di.
    \]
\end{exercise}

\begin{solution}
    Observe that
    \[
        \frac{1}{a + bi} = \frac{a - bi}{(a + bi)(a - bi)} = \frac{a - bi}{a^2 + b^2}.
    \]
    So the desired real numbers are \( c = a/(a^2 + b^2) \) and \( d = -b/(a^2 + b^2) \).
\end{solution}

\begin{exercise}
\label{ex:2}
    Show that
    \[
        \frac{-1 + \sqrt{3} i}{2}
    \]
    is a cube root of 1 (meaning that its cube equals 1).
\end{exercise}

\begin{solution}
    Let \( z = \tfrac{-1 + \sqrt{3} i}{2} \), so that \( 2z = -1 + \sqrt{3} i \). Then
    \begin{align*}
        & (2z)^2 = 4z^2 = (-1 + \sqrt{3} i)^2 = 1 - 2 \sqrt{3} i - 3 = -2 - 2 \sqrt{3} i \\
        \implies & (2z)^3 = (4z^2)(2z) = (-2 - 2 \sqrt{3} i)(-1 + \sqrt{3}i) = 2 - 2 \sqrt{3} i + 2 \sqrt{3} i + 6 = 8,
    \end{align*}
    i.e.\ \( 8z^3 = 8 \). It follows that \( z^3 = 1 \).
\end{solution}

\begin{exercise}
\label{ex:3}
    Find two distinct square roots of \( i \).
\end{exercise}

\begin{solution}
    Let \( z_1 = \frac{1 + i}{\sqrt{2}} \) and \( z_2 = -z_1 \) (\( z_1 \) and \( z_2 \) are distinct since \( z_1 \neq 0 \)). Then
    \[
    2 z_1^2 = (1 + i)^2 = 2i \implies z_1^2 = i,
    \]
    and hence \( z_2^2 = (-z_1)^2 = z_1^2 = i \).
\end{solution}

\begin{exercise}
\label{ex:4}
    Show that \( \alpha + \beta = \beta + \alpha \) for all \( \alpha, \beta \in \C \).
\end{exercise}

\begin{solution}
    Suppose \( \alpha = x + yi \) and \( \beta = u + vi \). Then
    \[
        \alpha + \beta = (x + u) + (y + v)i = (u + x) + (v + y)i = \beta + \alpha,
    \]
    where we have used commutativity of addition in \( \R \).
\end{solution}

\begin{exercise}
\label{ex:5}
    Show that \( (\alpha + \beta) + \lambda = \alpha + (\beta + \lambda) \) for all \( \alpha, \beta, \lambda \in \C \).
\end{exercise}

\begin{solution}
    Suppose that \( \alpha = x + yi, \beta = u + vi \), and \( \lambda = s + ti \). Then
    \begin{multline*}
        (\alpha + \beta) + \lambda = ((x + u) + (y + v)i) + \lambda = ((x + u) + s) + ((y + v) + t)i \\ = (x + (u + s)) + (y + (v + t))i = \alpha + ((u + s) + (v + t)i) = \alpha + (\beta + \lambda),
    \end{multline*}
    where we have used associativity of addition in \( \R \).
\end{solution}

\begin{exercise}
\label{ex:6}
    Show that \( (\alpha \beta) \lambda = \alpha (\beta \lambda) \) for all \( \alpha, \beta, \lambda \in \C \).
\end{exercise}

\begin{solution}
    Suppose that \( \alpha = x + yi, \beta = u + vi \), and \( \lambda = s + ti \). Then
    \begin{align*}
        (\alpha \beta) \lambda &= ((xu - yv) + (xv + yu)i) \lambda \\
        &= ((xu - yv)s - (xv + yu)t) + ((xu - yv)t + (xv + yu)s)i \\
        &= ((xu)s - (yv)s - (xv)t - (yu)t) + ((xu)t - (yv)t + (xv)s + (yu)s)i \\
        &= (x(us) - x(vt) - y(ut) - y(vs)) + (x(ut) + x(vs) + y(us) - y(vt))i \\
        &= (x(us - vt) - y(ut + vs)) + (x(ut + vs) + y(us - vt))i \\
        &= \alpha ((us - vt) + (ut + vs)i) \\
        &= \alpha (\beta \lambda),
    \end{align*}
    where we have used several algebraic properties of \( \R \).
\end{solution}

\begin{exercise}
\label{ex:7}
    Show that for every \( \alpha \in \C \), there exists a unique \( \beta \in \C \) such that \( \alpha + \beta = 0 \).
\end{exercise}

\begin{solution}
    Suppose that \( \alpha = x + yi \) and let \( \beta = -x - yi \). Then
    \[
        \alpha + \beta = (x - x) + (y - y)i = 0 + 0i = 0.
    \]
    To see that \( \beta \) is unique, suppose \( \beta' \) also satisfies \( \alpha + \beta' = 0 \). Then
    \[
        \beta = \beta + 0 = \beta + (\alpha + \beta') = (\alpha + \beta) + \beta' = 0 + \beta' = \beta',
    \]
    where we have used associativity and commutativity of addition in \( \C \).
\end{solution}

\begin{exercise}
\label{ex:8}
    Show that for every \( \alpha \in \C \) with \( \alpha \neq 0 \), there exists a unique \( \beta \in \C \) such that \( \alpha \beta = 1 \).
\end{exercise}

\begin{solution}
    Suppose that \( \alpha = x + yi \). Since \( \alpha \neq 0 \), it must be the case that \( x \) and \( y \) are not both zero. It follows that \( x^2 + y^2 \neq 0 \), so let \( \beta = \tfrac{x}{x^2 + y^2} - \tfrac{y}{x^2 + y^2} i \). Then
    \[
        \alpha \beta = (x + yi)\left( \frac{x}{x^2 + y^2} - \frac{y}{x^2 + y^2} i \right) = \frac{x^2 + y^2}{x^2 + y^2} + \frac{xy - xy}{x^2 + y^2} i = 1 + 0i = 1.
    \]
    To see that \( \beta \) is unique, suppose \( \beta' \) also satisfies \( \alpha \beta' = 1 \). Then
    \[
        \beta = \beta 1 = \beta (\alpha \beta') = (\alpha \beta) \beta' = 1 \beta' = \beta',
    \]
    where we have used associativity and commutativity of multiplication in \( \C \).
\end{solution}

\begin{exercise}
\label{ex:9}
    Show that \( \lambda (\alpha + \beta) = \lambda \alpha + \lambda \beta \) for all \( \lambda, \alpha, \beta \in \C \).
\end{exercise}

\begin{solution}
    Suppose that \( \alpha = x + yi, \beta = u + vi \), and \( \lambda = s + ti \). Then
    \begin{align*}
        \lambda (\alpha + \beta) &= (s(x + u) - t(y + v)) + (s(y + v) + t(x + u)) i \\
        &= (sx + su - ty - tv) + (sy + sv + tx + tu) i \\
        &= [(sx - ty) + (sy + tx)i] + [(su - tv) + (sv + tu)i] \\
        &= \lambda \alpha + \lambda \beta,
    \end{align*}
    where we have used distributivity in \( \R \).
\end{solution}

\begin{exercise}
\label{ex:10}
    Find \( x \in \R^4 \) such that
    \[
        (4, -3, 1, 7) + 2x = (5, 9, -6, 8).
    \]
\end{exercise}

\begin{solution}
    Take \( x = (\tfrac{1}{2}, 6, -\tfrac{7}{2}, \tfrac{1}{2}) \).
\end{solution}

\begin{exercise}
\label{ex:11}
    Explain why there does not exist \( \lambda \in \C \) such that
    \[
        \lambda (2 - 3i, 5 + 4i, -6 + 7i) = (12 - 5i, 7 + 22i, -32 - 9i).
    \]
\end{exercise}

\begin{solution}
    Suppose there was such a \( \lambda \). Then
    \[
        \lambda (2 - 3i) = 12 - 5i \implies \lambda = \frac{12 - 5i}{2 - 3i} = 3 + 2i.
    \]
    However,
    \[
        (3 + 2i)(-6 + 7i) = -32 + 9i \neq -32 - 9i.
    \]
\end{solution}

\begin{exercise}
\label{ex:12}
    Show that \( (x + y) + z = x + (y + z) \) for all \( x, y, z \in \F^n \).
\end{exercise}

\begin{solution}
    Suppose \( x = (x_1, \ldots, x_n), y = (y_1, \ldots, y_n) \), and \( z = (z_1, \ldots, z_n) \). Then
    \begin{align*}
        (x + y) + z &= (x_1 + y_1, \ldots, x_n + y_n) + z \\
        &= ((x_1 + y_1) + z_1, \ldots, (x_n + y_n) + z_n) \\
        &= (x_1 + (y_1 + z_1), \ldots, x_n + (y_n + z_n)) \\
        &= x + (y_1 + z_1, \ldots, y_n + z_n) \\
        &= x + (y + z),
    \end{align*}
    where we have used associativity of addition in \( \F \).
\end{solution}

\begin{exercise}
\label{ex:13}
    Show that \( (ab)x = a(bx) \) for all \( x \in \F^n \) and all \( a, b \in \F \).
\end{exercise}

\begin{solution}
    Suppose that \( x = (x_1, \ldots, x_n) \). Then
    \begin{align*}
        (ab)x &= ((ab)x_1, \ldots, (ab)x_n) \\
        &= (a(b x_1), \ldots, a(b x_n)) \\
        &= a(b x_1, \ldots, b x_n) \\
        &= a(bx),
    \end{align*}
    where we have used associativity of multiplication in \( \F \).
\end{solution}

\begin{exercise}
\label{ex:14}
    Show that \( 1x = x \) for all \( x \in \F^n \).
\end{exercise}

\begin{solution}
    Suppose that \( x = (x_1, \ldots, x_n) \). Then
    \[
        1x = (1 x_1, \ldots, 1 x_n) = (x_1, \ldots, x_n) = x,
    \]
    where we have used that \( 1 x_j = x_j \) for any \( x_j \in \F \).
\end{solution}

\begin{exercise}
\label{ex:15}
    Show that \( \lambda (x + y) = \lambda x + \lambda y \) for all \( \lambda \in \F \) and all \( x, y \in \F^n \).
\end{exercise}

\begin{solution}
    Suppose that \( x = (x_1, \ldots, x_n) \) and \( y = (y_1, \ldots, y_n) \). Then
    \begin{align*}
        \lambda (x + y) &= \lambda (x_1 + y_1, \ldots, x_n + y_n) \\
        &= (\lambda (x_1 + y_1), \ldots, \lambda (x_n + y_n)) \\
        &= (\lambda x_1 + \lambda y_1, \ldots, \lambda x_n + \lambda y_n) \\
        &= (\lambda x_1, \ldots, \lambda x_n) + (\lambda y_1, \ldots, \lambda y_n) \\
        &= \lambda (x_1, \ldots, x_n) + \lambda (y_1, \ldots, y_n) \\
        &= \lambda x + \lambda y,
    \end{align*}
    where we have used distributivity in \( \F \).
\end{solution}

\begin{exercise}
\label{ex:16}
    Show that \( (a + b)x = ax + bx \) for all \( a, b \in \F \) and all \( x \in \F^n \).
\end{exercise}

\begin{solution}
    Suppose that \( x = (x_1, \ldots, x_n) \). Then
    \begin{align*}
        (a + b) x &= (a + b) (x_1, \ldots, x_n) \\
        &= ((a + b) x_1, \ldots, (a + b) x_n) \\
        &= (a x_1 + b x_1, \ldots, a x_n + b x_n) \\
        &= (a x_1, \ldots, a x_n) + (b x_1, \ldots, b x_n) \\
        &= a (x_1, \ldots x_n) + b (x_1, \ldots, x_n) \\
        &= ax + bx,
    \end{align*}
    where we have used distributivity in \( \F \).
\end{solution}

\noindent \hrulefill

\noindent \hypertarget{ladr}{\textcolor{blue}{[LADR]} Axler, S. (2015) \textit{Linear Algebra Done Right.} 3rd edn.}

\end{document}