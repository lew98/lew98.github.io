\documentclass[12pt]{article}
\title{Suprema, infima, maxima, and minima}
\author{}
\date{\vspace{-24mm}}
\usepackage[utf8]{inputenc}
\usepackage{amsmath}
\usepackage{amsthm}
\usepackage{geometry}
\usepackage{amsfonts}
\usepackage{bm}
\usepackage{hyperref}
\usepackage{xcolor}
\usepackage{enumitem}
\usepackage{changepage}
\usepackage{tikz-cd}
\usepackage[nameinlink]{cleveref}
\geometry{
headheight=15pt,
left=60pt,
right=60pt
}
\usepackage{fancyhdr}
\pagestyle{fancy}
\fancyhf{}
\lhead{}
\chead{Suprema, infima, maxima, and minima}
\rhead{\thepage}

% \setlength{\parindent}{0pt}
\hypersetup{
    colorlinks=true,
    linkcolor=blue,
    urlcolor=blue
}

\newcommand{\newp}{\vspace{5mm}}

\theoremstyle{definition}
\newtheorem{definition}{Definition}[section]
\newtheorem{proposition}[definition]{Proposition}

\newtheorem*{remark}{Remark}

\newcommand{\setcomp}[1]{#1^{\mathsf{c}}}
\newcommand{\N}{\mathbf{N}}
\newcommand{\Z}{\mathbf{Z}}
\newcommand{\Q}{\mathbf{Q}}
\newcommand{\R}{\mathbf{R}}
\newcommand{\C}{\mathbf{C}}

\begin{document}

\maketitle

\tableofcontents

\newpage

This document contains definitions and various results on suprema, infima, maxima, and minima in partially ordered sets, totally ordered sets, and ordered fields. We will usually only prove results for suprema; the analogous proofs for infima are similar.

\begin{remark}
    \( \N = \{ 1, 2, 3, \ldots \} \).
\end{remark}

\section{Definitions}
\label{sec:definitions}

Let \( A \) be a \href{https://en.wikipedia.org/wiki/Partially_ordered_set}{partially ordered set}.

\begin{definition}
\label{def:sup_inf_max_min}
    Suppose we have a subset \( E \subseteq A \) and some \( y \in A \) which satisfies the following two conditions:
    
    \begin{enumerate}[label = (\roman*)]
        \item \( y \) is an upper bound of \( E \), i.e.\ for all \( x \in E \), \( x \leq y \);
        
        \item \( y \) is the least upper bound of \( E \), i.e.\ if \( z \in A \) is an upper bound of \( E \) then \( y \leq z \), or equivalently, if \( z \in A \) is such that \( y \not\leq z \) (which is to say \( z < y \) or \( z \) is incomparable with \( y \)), then \( z \) is not an upper bound of \( E \).
    \end{enumerate}
    
    \noindent Then \( y \) is known as a \textbf{supremum} or \textbf{least upper bound} of \( E \). If instead \( y \) satisfies condition (i), i.e.\ \( y \) is an upper bound of \( E \), and \( y \in E \), then \( y \) is known as a \textbf{maximum} of \( E \). Similarly, suppose \( y \) satisfies the following two conditions:
    
    \begin{enumerate}[label = (\roman*), start = 3]
        \item \( y \) is a lower bound of \( E \), i.e.\ for all \( x \in E \), \( y \leq x \);
        
        \item \( y \) is the greatest lower bound of \( E \), i.e.\ if \( z \in A \) is a lower bound of \( E \) then \( z \leq y \), or equivalently, if \( z \in A \) is such that \( z \not\leq y \) (which is to say \( y < z \) or \( z \) is incomparable with \( y \)), then \( z \) is not a lower bound of \( E \).
    \end{enumerate}
    
    \noindent Then \( y \) is known as an \textbf{infimum} or \textbf{greatest lower bound} of \( E \). If instead \( y \) satisfies condition (iii), i.e.\ \( y \) is a lower bound of \( E \), and \( y \in E \), then \( y \) is known as a \textbf{minimum} of \( E \).
\end{definition}

In fact, the supremum of a subset \( E \) is unique. If \( y_1 \) and \( y_2 \) both satisfy conditions (i) and (ii) in \Cref{def:sup_inf_max_min}, then by condition (ii) we must have both \( y_1 \leq y_2 \) and \( y_2 \leq y_1 \); it follows that \( y_1 = y_2 \). Similar reasoning shows that the infimum of a subset is also unique. For this reason, we are justified in saying \textit{the} supremum of \( E \) and \textit{the} infimum of \( E \).

It is straightforward to see that if \( E \) has a maximum \( M \), then \( \sup E = M \), and if \( E \) has a minimum \( m \), then \( \inf E = m \). By the previous paragraph, it follows that the maximum and minimum of a set are unique. However, \( E \) may have a supremum without having a maximum, and may have an infimum without having a minimum. For example, if \( A = \R \) and \( E = (0, 1) \), then one can verify that \( \sup E = 1 \) and \( \inf E = 0 \), but \( E \) has no minimum or maximum element.

A subset need not have a supremum or an infimum. For example, if \( A = E = \Q \), then \( E \) does not have a supremum or an infimum; indeed, \( E \) has no upper or lower bounds. Even if \( E \) does have upper and lower bounds, the supremum and infimum of \( E \) need not exist; see \href{https://lew98.github.io/Mathematics/Q_does_not_have_the_least_upper_bound_property.pdf}{here} for a counterexample.

What if we take \( E \) to be the empty set in \Cref{def:sup_inf_max_min}? Note that any \( a \in A \) is an upper bound of the empty set, since the statement
\[
    \forall x \in \emptyset \quad x \leq a
\]
is vacuously true. Hence if \( y \in A \) is to be the supremum of the empty set, then \( y \) must satisfy \( y \leq a \) for all \( a \in A \), i.e.\ \( y \) must be the minimum of \( A \). It follows that the empty set has a supremum if and only if \( A \) has a minimum, and if they both exist then \( \sup \emptyset = \min A \). A similar discussion shows that the empty set has an infimum if and only if \( A \) has a maximum, and if they both exist then \( \inf \emptyset = \max A \).

\begin{definition}
\label{def:lub_glb_properties}
    If any subset \( E \subseteq A \) which is non-empty and bounded above admits a supremum in \( A \), then \( A \) is said to have the \textbf{least-upper-bound property}, or to be \textbf{Dedekind complete}. If any subset \( E \subseteq A \) which is non-empty and bounded below admits an infimum in \( A \), then \( A \) is said to have the \textbf{greatest-lower-bound property}.
\end{definition}

For example, the real numbers have the least-upper-bound property. This can be taken as an axiom (the Axiom of Completeness), or it can be shown as a theorem while constructing the real numbers from the rational numbers; one approach to do this is via \href{https://lew98.github.io/Mathematics/Construction_of_R_from_Q_via_Dedekind_cuts.pdf}{Dedekind cuts}. On the other hand, the \href{https://lew98.github.io/Mathematics/Q_does_not_have_the_least_upper_bound_property.pdf}{rational numbers do not have the least-upper-bound property}.

Regarding these properties, we have the following proposition.

\begin{proposition}
\label{prop:lub_glb_property_equivalence}
    \( A \) has the least-upper-bound property if and only if \( A \) has the greatest-lower-bound property.
\end{proposition}

\begin{proof}
    We shall only show that if \( A \) has the least-upper-bound property, then \( A \) has the greatest-lower-bound property; the converse implication is proved similarly. Suppose therefore that \( A \) has the least-upper-bound property and let \( E \subseteq A \) be non-empty and bounded below. Define
    \[
        B = \{ l \in A : l \text{ is a lower bound of } E \}.
    \]
    Then \( B \) is non-empty since \( E \) is bounded below, and \( B \) is bounded above by any \( x \in E \); there exists at least one such \( x \) since \( E \) is non-empty. Since \( A \) has the least-upper-bound property, it follows that \( \sup B \) exists in \( A \). We claim that \( \inf E = \sup B \). To see this, we need to show two things.
    
    \begin{enumerate}[label = (\roman*)]
        \item \( \sup B \) is a lower bound of \( E \). Seeking a contradiction, suppose this is not the case, i.e.\ suppose there exists \( x \in E \) with \( \sup B \not\leq x \). Then \( x \) cannot be an upper bound of \( B \), so there must exist some \( l \in B \) such that \( l \not\leq x \); this is a contradiction since \( l \) is a lower bound of \( E \).

        \item \( \sup B \) is the greatest lower bound of \( E \). Suppose \( y \in A \) is a lower bound of \( E \). Then \( y \) belongs to \( B \), so \( y \leq \sup B \).
    \end{enumerate}

    We have now shown that any \( E \subseteq A \) which is non-empty and bounded below admits an infimum in \( A \), i.e.\ that \( A \) has the greatest-lower-bound property.
\end{proof}

\section{Partially ordered sets}
\label{sec:partially_ordered_sets}

Let \( A \) be a \href{https://en.wikipedia.org/wiki/Partially_ordered_set}{partially ordered set}.

\begin{proposition}
\label{prop:subset_implies_sup/inf_order}
    Suppose we have subsets \( E \subseteq G \subseteq A \).
    \begin{enumerate}[label = (\roman*)]
        \item If \( \sup E \) and \( \sup G \) both exist, then \( \sup E \leq \sup G \).
        
        \item If \( \inf E \) and \( \inf G \) both exist, then \( \inf G \leq \inf E \).
    \end{enumerate}
\end{proposition}

\begin{proof}
    \begin{enumerate}[label = (\roman*)]
        \item It is straightforward to see that any upper bound of \( G \) is also an upper bound of \( E \). It follows that \( \sup G \) is an upper bound of \( E \) and hence that \( \sup E \leq \sup G \).

        \item The proof is easily adapted from part (i). \qedhere
    \end{enumerate}
\end{proof}

\begin{proposition}
\label{prop:sup_inf_unions_partial_order}
    Suppose we have subsets \( \{ E_i \subseteq A : i \in I \} \) for some indexing set \( I \).
    
    \begin{enumerate}[label = (\roman*)]
        \item If \( \sup E_i \) exists in \( A \) for each \( i \in I \) and \( \sup \{ \sup E_i : i \in I \} \) also exists in \( A \), then
        \[
            \textstyle{\sup \left( \bigcup_{i \in I} E_i \right) = \sup \{ \sup E_i : i \in I \}.}
        \]

        \item If \( \inf E_i \) exists in \( A \) for each \( i \in I \) and \( \inf \{ \inf E_i : i \in I \} \) also exists in \( A \), then
        \[
            \textstyle{\inf \left( \bigcup_{i \in I} E_i \right) = \inf \{ \inf E_i : i \in I \}.}
        \]
    \end{enumerate}
\end{proposition}

\begin{proof}
    \begin{enumerate}[label = (\roman*)]
        \item Let \( s = \sup \{ \sup E_i : i \in I \} \) and \( E' = \bigcup_{i \in I} E_i \). Suppose \( x \in E' \), i.e.\ there exists some \( j \in I \) such that \( x \in E_j \). Then \( x \leq \sup E_j \leq s \); it follows that \( s \) is an upper bound of \( E' \). Suppose that \( y \in A \)  is such that \( s \not\leq y \). Then \( y \) cannot be an upper bound of \( \{ \sup E_i : i \in I \} \), so there must exist some \( j \in I \) such that \( \sup E_j \not\leq y \), which implies that \( y \) cannot be an upper bound of \( E_j \), i.e.\ there exists some \( x \in E_j \) such that \( x \not\leq y \). Since this \( x \) also belongs to \( E' \), it follows that \( y \) cannot be an upper bound of \( E' \). We may conclude that \( \sup E' = s \).

        \item The proof is easily adapted from part (i). \qedhere
    \end{enumerate}
\end{proof}

Even if \( \sup E_i \) exists for each \( i \in I \), it may not be the case that \( \sup \{ \sup E_i : i \in I \} \) exists. For example, if \( A = \R, I = \N, \) and \( E_i = [0, i] \), then \( \sup E_i = i \) but the set \( \{ \sup E_i : i \in I \} = \N \) has no supremum in \( \R \). A similar discussion holds for infima.

\section{Totally ordered sets}
\label{sec:totally_ordered_sets}

Let \( A \) be a \href{https://en.wikipedia.org/wiki/Total_order}{totally ordered set}. Note that since any two elements of a totally ordered set are comparable, we need not consider the possibility that ``\( z \) is incomparable with \( y \)" in parts (ii) and (iv) of \Cref{def:sup_inf_max_min}.

\begin{proposition}
\label{prop:finite_set_has_min_max_total_order}
    Any finite non-empty subset \( E \subseteq A \) has a minimum and maximum element.
\end{proposition}

\begin{proof}
    Suppose \( E \) has \( n \) elements, where \( n \) is a positive integer. We shall prove the proposition by induction on \( n \). For the base case, \( n = 1 \), we have \( E = \{ x \} \) for some \( x \in A \). Then \( x = \min E = \max E \). Suppose that the result is true for some \( n \), i.e.\ any \( n \) element set has a minimum and a maximum, and suppose that \( E \) has \( n + 1 \) elements. Pick any \( y \in E \) and set \( E' = E \setminus \{ y \} \). Then \( E' \) has \( n \) elements and so by assumption \( m = \min E' \) and \( M = \max E' \) both exist. There are now three cases:
    \begin{itemize}
        \item \( y < m \). Then \( y = \min E \) and \( M = \max E \).

        \item \( M < y \). Then \( m = \min E \) and \( y = \max E \).

        \item \( m < y < M \). Then \( m = \min E \) and \( M = \max E \).
    \end{itemize}

    These cases are exclusive and exhaustive since the order on \( A \) is total. This completes the induction step and hence the result is true for any positive integer \( n \).
\end{proof}

If \( E \) is an infinite set, then \( E \) may have a minimum element, a maximum element, both, or neither. For example, consider the set \( \Q' = \Q \cup \{ -\infty, \infty \} \); the rational numbers with the formal symbols \( -\infty \) and \( \infty \) adjoined. We take the usual order on \( \Q \) and declare that \( -\infty \) is strictly less than any other element of \( \Q' \) and that \( \infty \) is strictly greater than any other element of \( \Q' \). One can verify that this defines a total order on \( \Q' \).

\begin{itemize}
    \item Consider \( E = \{ -\infty \} \cup \Q \subseteq \Q' \). Then \( \min E = -\infty \) but \( E \) has no maximum element. However, \( \sup E = \infty \).

    \item Similarly, consider \( E = \Q \cup \{ \infty \} \subseteq \Q' \). Then \( \max E = \infty \) but \( E \) has no minimum element. However, \( \inf E = -\infty \).

    \item \( \Q \subseteq \Q' \) has no minimum or maximum element, but \( \inf \Q = -\infty \) and \( \sup \Q = \infty \).

    \item \( \Q' \) itself has a minimum and a maximum element: \( \min \Q' = -\infty \) and \( \max \Q' = \infty \).
\end{itemize}
 
It is straightforward to combine \Cref{prop:sup_inf_unions_partial_order} and \Cref{prop:finite_set_has_min_max_total_order} to obtain the following proposition.

\begin{proposition}
\label{prop:sup_inf_unions_total_order}
    Suppose we have subsets \( E_1, \ldots, E_n \subseteq A \).
    \begin{enumerate}[label = (\roman*)]
        \item If \( \sup E_1, \ldots, \sup E_n \) all exist in \( A \), then
        \[
            \textstyle{\sup \left( \bigcup_{i=1}^{n} E_i \right) = \max \{ \sup E_1, \ldots, \sup E_n \}}.
        \]

        \item If \( \inf E_1, \ldots, \inf E_n \) all exist in \( A \), then
        \[
            \textstyle{\inf \left( \bigcup_{i=1}^{n} E_i \right) = \min \{ \inf E_1, \ldots, \inf E_n \}}.
        \]
    \end{enumerate}
\end{proposition}

\begin{proposition}
\label{prop:sup/inf_strictly_comparable_implies_bounds}
    Suppose we have subsets \( E, G \subseteq A \).
    \begin{enumerate}[label = (\roman*)]
        \item If \( \sup E \) and \( \sup G \) both exist and \( \sup E < \sup G \), then there exists \( y \in G \) such that \( y \) is an upper bound of \( E \).

        \item If \( \inf E \) and \( \inf G \) both exist and \( \inf G < \inf E \), then there exists \( y \in G \) such that \( y \) is a lower bound of \( E \).
    \end{enumerate}
\end{proposition}

\begin{proof}
    \begin{enumerate}[label = (\roman*)]
        \item \( \sup E \) is not an upper bound of \( G \), so there exists \( y \in G \) such that \( \sup E < y \). Then for any \( x \in E \), we have \( x \leq \sup E < y \), i.e.\ \( y \) is an upper bound of \( E \).

        \item The proof is easily adapted from part (i). \qedhere
    \end{enumerate}
\end{proof}

Note that an analogous statement to part (i) for partially ordered sets is not necessarily true, i.e.\ if \( A \) is a partially ordered set, the statement

\vspace{6px}
\begin{adjustwidth}{16px}{16px}
    ``If \( \sup E \) and \( \sup G \) both exist and \( \sup G \not\leq \sup E \), then there exists \( y \in G \) such that \( y \) is an upper bound of \( E \)."
\end{adjustwidth}
\vspace{6px}

\noindent is false in general. For a counterexample, consider the partial order which has the following \href{https://en.wikipedia.org/wiki/Hasse_diagram}{Hasse diagram}.

\[
\begin{tikzcd}[column sep = tiny]
    & s & & & & t & \\
    a \arrow[ru, no head] & & b \arrow[lu, no head] & & c \arrow[ru, no head] & & d \arrow[lu, no head]
\end{tikzcd}
\]
Let \( E = \{ a, b \} \) and \( G = \{ c, d \} \). Then \( \sup E = s, \sup G = t, t \not\leq s, \) but neither \( c \) nor \( d \) are upper bounds of \( E \).


\section{Ordered fields}
\label{sec:ordered_fields}

Let \( F \) be an \href{https://en.wikipedia.org/wiki/Ordered_field}{ordered field} (for example, \( \Q \) and \( \R \) are ordered fields).

\begin{proposition}[Approximation property for suprema and infima]
\label{prop:approximation_property}
    Suppose that \( u \in F \) is an upper bound and \( l \in F \) is a lower bound of a subset \( E \subseteq F \).

    \begin{enumerate}[label = (\roman*)]
        \item \( u = \sup E \) if and only if for all \( \varepsilon > 0 \) there exists an \( x \in E \) such that \( u - \varepsilon < x \leq u \).

        \item \( l = \inf E \) if and only if for all \( \varepsilon > 0 \) there exists an \( x \in E \) such that \( l \leq x < l + \varepsilon \).
    \end{enumerate}
\end{proposition}

\begin{proof}

    \begin{enumerate}[label = (\roman*)]
        \item We will prove the forward implication by considering the contrapositive statement. Assume therefore that there exists an \( \varepsilon > 0 \) such that for all \( x \in E \), we have \( x \leq u - \varepsilon \). Then \( u - \varepsilon \) is an upper bound of \( E \) and \( u - \varepsilon < u \). It follows that \( u \) is not the least upper bound of \( E \), i.e.\ \( u \neq \sup E \).

        For the reverse implication, suppose that \( z \in F \) is such that \( z < u \). Then by assumption there exists an \( x \in E \) such that \( u - (u - z) = z < x \). It follows that \( z \) cannot be an upper bound of \( E \) and we may conclude that \( u = \sup E \).

        \item The proof is analogous to part (i). \qedhere
    \end{enumerate}

\end{proof}

\begin{proposition}
\label{prop:add_mult_constant}
    For \( E \subseteq F \) and \( c \in F \), define
    \[
        E + c = \{ x + c : x \in E \} \quad \text{and} \quad cE = \{ cx : x \in E \}.
    \]

    \begin{enumerate}[label = (\roman*)]
        \item If \( \sup E \) exists, then \( \sup (E + c) = \sup E + c \).
        
        \item If \( \inf E \) exists, then \( \inf (E + c) = \inf E + c \).
        
        \item If \( \sup E \) exists and \( c \geq 0 \), then \( \sup (cE) = c \cdot \sup E \).
        
        \item If \( \inf E \) exists and \( c \geq 0 \), then \( \inf (cE) = c \cdot \inf E \).
        
        \item If \( \inf E \) exists and \( c < 0 \), then \( \sup (cE) = c \cdot \inf E \).

        \item If \( \sup E \) exists and \( c < 0 \), then \( \inf (cE) = c \cdot \sup E \).
    \end{enumerate}
\end{proposition}

\begin{proof}
    The proofs of each part of this proposition are similar and mostly involve applying the defining properties of an ordered field.

    \begin{enumerate}[label = (\roman*)]
        \item For any \( x \in E \), we have
        \[
            x \leq \sup E \iff x + c \leq \sup E + c,
        \]
        so that \( \sup E + c \) is an upper bound of \( E + c \). Suppose \( y \in F \) is such that \( y < \sup E + c \). Then \( y - c < \sup E \), so \( y - c \) cannot be an upper bound of \( E \), i.e.\ there exists \( x \in E \) such that \( y - c < x \iff y < x + c \). It follows that \( y \) cannot be an upper bound of \( E + c \) and we may conclude that \( \sup (E + c) = \sup E + c \).

        \item The proof is easily adapted from part (i).

        \item If \( c = 0 \) then the result is clear. Suppose therefore that \( c > 0 \). For any \( x \in E \), we have
        \[
            x \leq \sup E \iff cx \leq c \cdot \sup E,
        \]
        so that \( c \cdot \sup E \) is an upper bound of \( cE \). Suppose \( y \in F \) is such that \( y < c \cdot \sup E \). Then \( c^{-1} y < \sup E \), so \( c^{-1} y \) cannot be an upper bound of \( E \), i.e.\ there exists \( x \in E \) such that \( c^{-1} y < x \iff y < cx \). It follows that \( y \) cannot be an upper bound of \( cE \) and we may conclude that \( \sup (cE) = c \cdot \sup E \).

        \item The proof is easily adapted from part (iii).

        \item For any \( x \in E \), we have
        \[
            \inf E \leq x \iff cx \leq c \cdot \inf E,
        \]
        so that \( c \cdot \inf E \) is an upper bound of \( cE \). Suppose \( y \in F \) is such that \( y < c \cdot \inf E \). Then \( \inf E < c^{-1} y \), so \( c^{-1} y \) cannot be a lower bound of \( E \), i.e.\ there exists \( x \in E \) such that \( x < c^{-1} y \iff y < cx \). It follows that \( y \) cannot be an upper bound of \( cE \) and we may conclude that \( \sup (cE) = c \cdot \inf E \).

        \item The proof is easily adapted from part (v). \qedhere
    \end{enumerate}
\end{proof}

Regarding part (v) of this proposition, even if \( \inf E \) exists, it may not be the case that \( \inf (cE) \) exists. For example, if \( F = \R \), \( E = (0, \infty) \), and \( c = -1 \), then \( \inf E = 0 \) but \( cE = (-\infty, 0) \), which does not have an infimum in \( \R \). A similar discussion holds for part (vi) and suprema.

\begin{proposition}
\label{prop:add_sets}
    For \( E, G \subseteq F \), define
    \[
        E + G = \{ x + y : x \in E, y \in G \}. 
    \]
    \begin{enumerate}[label = (\roman*)]
        \item If \( \sup E \) and \( \sup G \) both exist, then \( \sup(E + G) = \sup E + \sup G \).

        \item If \( \inf E \) and \( \inf G \) both exist, then \( \inf(E + G) = \inf E + \inf G \).
    \end{enumerate}
\end{proposition}

\begin{proof}
    \begin{enumerate}[label = (\roman*)]
        \item Let \( s = \sup E \) and \( t = \sup G \). It is straightforward to see that \( s + t \) is an upper bound of \( E + G \). Suppose \( z \in F \) is such that \( z < s + t \). Then \( z - t < s \), so \( z - t \) cannot be an upper bound of \( E \), i.e.\ there exists \( x \in E \) such that \( z - t < x \). This gives \( z - x < t \), so that \( z - x \) cannot be an upper bound of \( G \), i.e.\ there exists \( y \in G \) such that \( z - x < y \), which gives \( z < x + y \). It follows that \( z \) cannot be an upper bound of \( E + G \) and we may conclude that \( \sup(E + G) = s + t = \sup E + \sup G \).

        \item The proof is analogous to part (i). \qedhere
    \end{enumerate}
\end{proof}

\end{document}
