\documentclass[12pt]{article}
\usepackage[utf8]{inputenc}
\usepackage[utf8]{inputenc}
\usepackage{amsmath}
\usepackage{amsthm}
\usepackage{amssymb}
\usepackage{geometry}
\usepackage{amsfonts}
\usepackage{mathrsfs}
\usepackage{bm}
\usepackage{hyperref}
\usepackage[dvipsnames]{xcolor}
\usepackage[inline]{enumitem}
\usepackage{mathtools}
\usepackage{changepage}
\usepackage{graphicx}
\usepackage{caption}
\usepackage{subcaption}
\usepackage{lipsum}
\usepackage{tikz}
\usetikzlibrary{matrix, patterns, decorations.pathreplacing, calligraphy}
\usepackage{tikz-cd}
\usepackage[nameinlink]{cleveref}
\geometry{
headheight=15pt,
left=60pt,
right=60pt
}
\setlength{\emergencystretch}{20pt}
\usepackage{fancyhdr}
\pagestyle{fancy}
\fancyhf{}
\lhead{}
\chead{Section 5.3 Exercises}
\rhead{\thepage}
\hypersetup{
    colorlinks=true,
    linkcolor=blue,
    urlcolor=blue
}

\theoremstyle{definition}
\newtheorem*{remark}{Remark}

\newtheoremstyle{exercise}
    {}
    {}
    {}
    {}
    {\bfseries}
    {.}
    { }
    {\thmname{#1}\thmnumber{#2}\thmnote{ (#3)}}
\theoremstyle{exercise}
\newtheorem{exercise}{Exercise 5.3.}

\newtheoremstyle{solution}
    {}
    {}
    {}
    {}
    {\itshape\color{magenta}}
    {.}
    { }
    {\thmname{#1}\thmnote{ #3}}
\theoremstyle{solution}
\newtheorem*{solution}{Solution}

\Crefformat{exercise}{#2Exercise 5.3.#1#3}

\newcommand{\interior}[1]{%
  {\kern0pt#1}^{\mathrm{o}}%
}
\newcommand{\ts}{\textsuperscript}
\newcommand{\setcomp}[1]{#1^{\mathsf{c}}}
\newcommand{\quand}{\quad \text{and} \quad}
\newcommand{\N}{\mathbf{N}}
\newcommand{\Z}{\mathbf{Z}}
\newcommand{\Q}{\mathbf{Q}}
\newcommand{\I}{\mathbf{I}}
\newcommand{\R}{\mathbf{R}}
\newcommand{\C}{\mathbf{C}}

\DeclarePairedDelimiter\abs{\lvert}{\rvert}
% Swap the definition of \abs* and \norm*, so that \abs
% and \norm resizes the size of the brackets, and the 
% starred version does not.
\makeatletter
\let\oldabs\abs
\def\abs{\@ifstar{\oldabs}{\oldabs*}}
%
\let\oldnorm\norm
\def\norm{\@ifstar{\oldnorm}{\oldnorm*}}
\makeatother

\DeclarePairedDelimiter\paren{(}{)}
\makeatletter
\let\oldparen\paren
\def\paren{\@ifstar{\oldparen}{\oldparen*}}
\makeatother

\DeclarePairedDelimiter\bkt{[}{]}
\makeatletter
\let\oldbkt\bkt
\def\bkt{\@ifstar{\oldbkt}{\oldbkt*}}
\makeatother

\DeclarePairedDelimiter\set{\{}{\}}
\makeatletter
\let\oldset\set
\def\set{\@ifstar{\oldset}{\oldset*}}
\makeatother

\setlist[enumerate,1]{label={(\alph*)}}

\begin{document}

\section{Section 5.3 Exercises}

Exercises with solutions from Section 5.3 of \hyperlink{ua}{[UA]}.

\begin{exercise}
\label{ex:1}
    Recall from \href{https://lew98.github.io/Mathematics/UA_Section_4_4_Exercises.pdf}{Exercise 4.4.9} that a function \( f : A \to \R \) is Lipschitz on \( A \) if there exists an \( M > 0 \) such that
    \[
        \abs{\frac{f(x) - f(y)}{x - y}} \leq M
    \]
    for all \( x \neq y \) in \( A \).
    \begin{enumerate}
        \item Show that if \( f \) is differentiable on a closed interval \( [a, b] \) and if \( f' \) is continuous on \( [a, b] \), then \( f \) is Lipschitz on \( [a, b] \).

        \item Review the definition of a contractive function in \href{https://lew98.github.io/Mathematics/UA_Section_4_3_Exercises.pdf}{Exercise 4.3.11}. If we add the assumption that \( \abs{f'(x)} < 1 \) on \( [a, b] \), does it follow that \( f \) is contractive on this set?
    \end{enumerate}
\end{exercise}

\begin{solution}
    \begin{enumerate}
        \item Note that \( \abs{f'} \) is continuous on \( [a, b] \) since \( f' \) is continuous on \( [a, b] \). The Extreme Value Theorem then implies that \( \abs{f'} \) attains a maximum on \( [a, b] \), say \( M = \abs{f'(t)} \) for some \( t \in [a, b] \). Let \( x < y \) in \( [a, b] \) be given. The Mean Value Theorem on the interval \( [x, y] \) implies that there is a \( c \in (x, y) \) such that
        \[
            \abs{\frac{f(x) - f(y)}{x - y}} = \abs{f'(c)} \leq M.
        \]
        Thus \( f \) is Lipschitz on \( [a, b] \).

        \item If \( \abs{f'(x)} < 1 \) on \( [a, b] \), then the maximum value \( M = \abs{f'(t)} \) from part (a) must satisfy \( M < 1 \) and thus \( f \) is contractive on \( [a, b] \).
    \end{enumerate}
\end{solution}

\begin{exercise}
\label{ex:2}
    Let \( f \) be differentiable on an interval \( A \). If \( f'(x) \neq 0 \) on \( A \), show that \( f \) is one-to-one on \( A \). Provide an example to show that the converse statement need not be true.
\end{exercise}

\begin{solution}
    We will prove the contrapositive statement. Suppose that there exist \( x < y \) in \( A \) such that \( f(x) = f(y) \). Then Rolle's Theorem implies that there exists some \( c \in (x, y) \) such that \( f'(c) = 0 \).

    For a counterexample to the converse statement, consider the one-to-one function \( f : (-1, 1) \to (-1, 1) \) given by \( f(x) = x^3 \), which satisfies \( f'(0) = 0 \).
\end{solution}

\begin{exercise}
\label{ex:3}
    Let \( h \) be a differentiable function defined on the interval \( [0, 3] \), and assume that \( h(0) = 1, h(1) = 2 \), and \( h(3) = 2 \).
    \begin{enumerate}
        \item Argue that there exists a point \( d \in [0, 3] \) where \( h(d) = d \).

        \item Argue that at some point \( c \) we have \( h'(c) = 1/3 \).

        \item Argue that \( h'(x) = 1/4 \) at some point in the domain.
    \end{enumerate}
\end{exercise}

\begin{solution}
    \begin{enumerate}
        \item Define \( f : [0, 3] \to \R \) by \( f(x) = h(x) - x \) and note that \( f \) is continuous since \( h \) is continuous. Furthermore, since \( f(1) = h(1) - 1 = 1 \) and \( f(3) = h(3) - 3 = -1 \), the Intermediate Value Theorem implies that there exists some \( d \in (1, 3) \) such that \( f(d) = 0 \), i.e.\ \( h(d) = d \).

        \item Since \( h \) is differentiable on \( [0, 3] \), the Mean Value Theorem implies that there exists some point \( c \in (0, 3) \) such that
        \[
            h'(c) = \frac{h(3) - h(0)}{3 - 0} = \frac{1}{3}.
        \]

        \item Similarly to part (b), the Mean Value Theorem implies that there exists some point \( b \in (1, 3) \) such that
        \[
            h'(b) = \frac{h(3) - h(1)}{3 - 1} = 0.
        \]
        Combining this with part (b), we see that \( h' \) takes the values \( 0 \) and \( \tfrac{1}{3} \). Since \( 0 < \tfrac{1}{4} < \tfrac{1}{3} \), Darboux's Theorem implies that \( h' \) takes the value \( \tfrac{1}{4} \) at some point in the domain \( [0, 3] \).
    \end{enumerate}
\end{solution}

\begin{exercise}
\label{ex:4}
    Let \( f \) be differentiable on an interval \( A \) containing zero, and assume \( (x_n) \) is a sequence in \( A \) with \( (x_n) \to 0 \) and \( x_n \neq 0 \).
    \begin{enumerate}
        \item If \( f(x_n) = 0 \) for all \( n \in \N \), show \( f(0) = 0 \) and \( f'(0) = 0 \).

        \item Add the assumption that \( f \) is twice-differentiable at zero and show that \( f''(0) = 0 \) as well.
    \end{enumerate}
\end{exercise}

\begin{solution}
    \begin{enumerate}
        \item We have
        \[
            0 = \lim_{n \to \infty} 0 = \lim_{n \to \infty} f(x_n) = f(0)
        \]
        since \( f \) is continuous at zero.

        Note that, for each \( n \in \N \), since \( x_n \neq 0 \), the difference quotient \( \frac{f(x_n) - f(0)}{x_n - 0} = \frac{f(x_n)}{x_n} \) is well-defined and satisfies \( \frac{f(x_n)}{x_n} = 0 \). Since \( f'(0) \) exists, it must then be the case that
        \[
            f'(0) = \lim_{n \to \infty} \frac{f(x_n)}{x_n} = 0.
        \]

        \item We are given that the limit
        \[
            f''(0) = \lim_{x \to 0} \frac{f'(x) - f'(0)}{x - 0} = \lim_{x \to 0} \frac{f'(x)}{x}
        \]
        exists. Since \( f(0) = 0 \), we may apply L'Hospital's Rule (Theorem 5.3.6) to see that
        \[
            f''(0) = \lim_{x \to 0} \frac{2 f(x)}{x^2}.
        \]
        Similarly to part (a), note that for each \( n \in \N \), since \( x_n \neq 0 \), the quotient \( \frac{2 f(x_n)}{x_n^2} \) is well-defined and satisfies \( \frac{2 f(x_n)}{x_n^2} = 0 \). Since \( \lim_{x \to 0} \frac{2 f(x)}{x^2} \) exists, it must be the case that
        \[
            f''(0) = \lim_{n \to \infty} \frac{2 f(x_n)}{x_n^2} = 0.
        \]
    \end{enumerate}
\end{solution}

\begin{exercise}
\label{ex:5}
    \begin{enumerate}
        \item Supply the details for the proof of Cauchy's Generalized Mean Value Theorem (Theorem 5.3.5).

        \item Give a graphical interpretation of the Generalized Mean Value Theorem analogous to the one given for the Mean Value Theorem at the beginning of Section 5.3. (Consider \( f \) and \( g \) as parametric equations for a curve.)
    \end{enumerate}
\end{exercise}

\begin{solution}
    \begin{enumerate}
        \item Define \( h : [a, b] \to \R \) by
        \[
            h(x) = [f(b) - f(a)] g(x) - [g(b) - g(a)] f(x)
        \]
        and note that \( h \) is continuous on \( [a, b] \) and differentiable on \( (a, b) \) since \( f \) and \( g \) are. The Mean Value Theorem implies that there exists some \( c \in (a, b) \) such that
        \[
            h'(c) = \frac{h(b) - h(a)}{b - a},
        \]
        or equivalently
        \begin{align*}
            & [f(b) - f(a)] g'(c) - [g(b) - g(a)] f'(c) \\
            &= \frac{[f(b) - f(a)] g(b) - [g(b) - g(a)] f(b) - [f(b) - f(a)] g(a) + [g(b) - g(a)] f(a)}{b - a} \\
            &= \frac{[f(b) - f(a)][g(b) - g(a)] - [g(b) - g(a)][f(b) - f(a)]}{b - a} \\
            &= 0.
        \end{align*}

        \item If \( f'(c) \neq 0 \) and \( f(b) \neq f(a) \), so that
        \[
            \frac{g'(c)}{f'(c)} = \frac{g(b) - g(a)}{f(b) - f(a)},
        \]
        then the Generalized Mean Value Theorem can be geometrically interpreted as asserting the existence of a tangent line to the graph of the curve \( \gamma : [a, b] \to \R^2 ; \,\, \gamma(t) = (f(t), g(t)) \) at the point \( (f(c), g(c)) \) which is parallel to the line through the points \( (f(a), g(a)) \) and \( (f(b), g(b)) \).
    \end{enumerate}
\end{solution}

\begin{exercise}
\label{ex:6}
    \begin{enumerate}
        \item Let \( g : [0, a] \to \R \) be differentiable, \( g(0) = 0 \), and \( \abs{g'(x)} \leq M \) for all \( x \in [0, a] \). Show \( \abs{g(x)} \leq Mx \) for all \( x \in [0, a] \).

        \item Let \( h : [0, a] \to \R \) be twice differentiable, \( h'(0) = h(0) = 0 \) and \( \abs{h''(x)} \leq M \) for all \( x \in [0, a] \). Show \( \abs{h(x)} \leq Mx^2 / 2 \) for all \( x \in [0, a] \).

        \item Conjecture and prove an analogous result for a function that is differentiable three times on \( [0, a] \).
    \end{enumerate}
\end{exercise}

\begin{solution}
    \begin{enumerate}
        \item Since \( g(0) = 0 \), the inequality \( \abs{g(x)} \leq Mx \) is clear when \( x = 0 \). Suppose \( x \in (0, a] \). By the Mean Value Theorem on the interval \( [0, x] \), there exists some \( c \in (0, x) \) such that
        \[
            \abs{g'(c)} = \abs{\frac{g(x)}{x}} \implies \abs{g(x)} = \abs{g'(c)}x \leq Mx.
        \]

        \item Since \( h(0) = 0 \), the inequality \( \abs{h(x)} \leq Mx^2 / 2 \) is clear when \( x = 0 \). Suppose \( x \in (0, a] \). Using the Generalized Mean Value Theorem on the interval \( [0, x] \) with the functions \( h \) and \( \tfrac{1}{2} x^2 \), we can find some \( c \in (0, x) \) such that
        \[
            \frac{h(x)}{\tfrac{1}{2} x^2} = \frac{h'(c)}{c}. 
        \]
        Now we can use the Mean Value Theorem on the interval \( [0, c] \) with the function \( h' \) to find some \( d \in (0, c) \) such that
        \[
            h''(d) = \frac{h'(c)}{c}.
        \]
        Combining this with the previous equality, we see that
        \[
            h''(d) = \frac{h(x)}{\tfrac{1}{2} x^2} \implies \abs{h(x)} = \tfrac{1}{2} \abs{h''(d)} x^2 \leq \tfrac{1}{2} M x^2.
        \]

        \item Suppose \( f : [0, a] \to \R \) is three times differentiable, \( f''(0) = f'(0) = f(0) = 0 \), and \( f'''(x) \leq M \) for all \( x \in [0, a] \). We claim that \( \abs{f(x)} \leq \tfrac{1}{6} M x^3 \) for all \( x \in [0, a] \). To see this, we proceed as in part (b). Since \( f(0) = 0 \), the inequality \( \abs{f(x)} \leq \tfrac{1}{6} M x^3 \) is clear when \( x = 0 \). Suppose \( x \in (0, a] \). Using the Generalized Mean Value Theorem on the interval \( [0, x] \) with the functions \( f \) and \( \tfrac{1}{6} x^3 \), we can find some \( b \in (0, x) \) such that
        \[
            \frac{f(x)}{\tfrac{1}{6} x^3} = \frac{f'(b)}{\tfrac{1}{2} b^2}.
        \]
        Using the Generalized Mean Value Theorem on the interval \( [0, b] \) with the functions \( f' \) and \( \tfrac{1}{2} x^2 \), we can find some \( c \in (0, b) \) such that
        \[
            \frac{f'(b)}{\tfrac{1}{2} b^2} = \frac{f''(c)}{c}.
        \]
        Now we can use the Mean Value Theorem on the interval \( [0, c] \) with the function \( f'' \) to find some \( d \in (0, c) \) such that
        \[
            f'''(d) = \frac{f''(c)}{c}. 
        \]
        Combining all of these equalities, we see that
        \[
            f'''(d) = \frac{f(x)}{\tfrac{1}{6} x^3} \implies \abs{f(x)} = \tfrac{1}{6} \abs{f'''(d)} x^3 \leq \tfrac{1}{6} M x^3.
        \]
    \end{enumerate}
\end{solution}

\begin{exercise}
\label{ex:7}
    A \textit{fixed point} of a function \( f \) is a value \( x \) where \( f(x) = x \). Show that if \( f \) is differentiable on an interval with \( f'(x) \neq 1 \), then \( f \) can have at most one fixed point.
\end{exercise}

\begin{solution}
    We will prove the contrapositive statement. Suppose that \( x < y \) belong to the domain of \( f \) and are such that \( f(x) = x \) and \( f(y) = y \). By the Mean Value Theorem on the interval \( [x, y] \), there exists some \( c \in (x, y) \) such that
    \[
        f'(c) = \frac{f(y) - f(x)}{x - y} = \frac{y - x}{y - x} = 1.
    \]
\end{solution}

\begin{exercise}
\label{ex:8}
    Assume \( f \) is continuous on an interval containing zero and differentiable for all \( x \neq 0 \). If \( \lim_{x \to 0} f'(x) = L \), show \( f'(0) \) exists and equals \( L \).
\end{exercise}

\begin{solution}
    We would like to see that the limit
    \[
        \lim_{x \to 0} \frac{f(x) - f(0)}{x}
    \]
    exists and equals \( L \). Letting \( I \) denote the interval domain of \( f \), note that the numerator and denominator of this fraction are both continuous on \( I \), differentiable on \( I \setminus \{ 0 \} \), and vanish at zero. That is, we have satisfied the hypotheses of the \( 0 / 0 \) case of L'Hospital's Rule (Theorem 5.3.6) and hence
    \[
        f'(0) = \lim_{x \to 0} \frac{f(x) - f(0)}{x} = \lim_{x \to 0} f'(x) = L.
    \]
\end{solution}

\begin{exercise}
\label{ex:9}
    Assume \( f \) and \( g \) are as described in Theorem 5.3.6, but now add the assumption that \( f \) and \( g \) are differentiable at \( a \), and \( f' \) and \( g' \) are continuous at \( a \) with \( g'(a) \neq 0 \). Find a short proof for the \( 0 / 0 \) case of L'Hospital's Rule under this stronger hypothesis.
\end{exercise}

\begin{solution}
    Note that for all \( x \neq a \), we have
    \[
        \frac{f(x)}{g(x)} = \frac{f(x) - f(a)}{g(x) - g(a)} = \frac{\frac{f(x) - f(a)}{x - a}}{\frac{g(x) - g(a)}{x - a}}.
    \]
    By assumption, the limits
    \[
        f'(a) = \lim_{x \to a} \frac{f(x) - f(a)}{x - a} \quand g'(a) = \lim_{x \to a} \frac{g(x) - g(a)}{x - a}
    \]
    both exist and \( g'(a) \neq 0 \). It follows from Corollary 4.2.4 (iv) that
    \[
        \lim_{x \to a} \frac{f(x)}{g(x)} = \frac{\lim_{x \to a} \frac{f(x) - f(a)}{x - a}}{\lim_{x \to a} \frac{g(x) - g(a)}{x - a}} = \frac{f'(a)}{g'(a)}.
    \]
    Now we can use our assumption that \( f' \) and \( g' \) are continuous at \( a \) with \( g'(a) \neq 0 \) to see that
    \[
        L = \lim_{x \to a} \frac{f'(x)}{g'(x)} = \frac{f'(a)}{g'(a)} = \lim_{x \to a} \frac{f(x)}{g(x)}.
    \]
\end{solution}

\begin{exercise}
\label{ex:10}
    Let \( f(x) = x \sin(1 / x^4) e^{-1 / x^2} \) and \( g(x) = e^{-1 / x^2} \). Using the familiar properties of these functions, compute the limit as \( x \) approaches zero of \( f(x), g(x), f(x) / g(x) \), and \( f'(x) / g'(x) \). Explain why the results are surprising but not in conflict with the content of Theorem 5.3.6.
\end{exercise}

\begin{solution}
    Some algebra reveals that
    \[
        \frac{f(x)}{g(x)} = x \sin \paren{\tfrac{1}{x^4}} \quand \frac{f'(x)}{g'(x)} = \sin \paren{\tfrac{1}{x^4}} \paren{\tfrac{x^3}{2} + x} - \frac{2 \cos \paren{\tfrac{1}{x^4}}}{x}.
    \]
    Given an \( \epsilon > 0 \), we have
    \[
        \abs{x} < \sqrt{\frac{1}{\log \paren{ \tfrac{1}{\epsilon} }}} \implies e^{-\tfrac{1}{x^2}} < \epsilon
    \]
    and thus \( \lim_{x \to 0} g(x) = 0 \). Combining this with various applications of the Squeeze Theorem, we see that
    \[
        \lim_{x \to 0} f(x) = \lim_{x \to 0} g(x) = \lim_{x \to 0} \frac{f(x)}{g(x)} = 0.
    \]
    However, we claim that \( \tfrac{f'(x)}{g'(x)} \) does not converge to zero as \( x \to 0 \). To see this, consider the sequence \( (x_n) \) given by
    \[
        x_n = \frac{1}{\sqrt[4]{2 n \pi}},
    \]
    which satisfies \( \lim_{n \to \infty} x_n = 0 \). Then
    \[
        \frac{f'(x_n)}{g'(x_n)} = - 2 \sqrt[4]{2 n \pi} \to - \infty \text{ as } n \to \infty.
    \]
    This does not conflict with the content of Theorem 5.3.6 since \( f \) and \( g \) are not continuous at zero; they are not even defined at zero.
\end{solution}

\begin{exercise}
\label{ex:11}
    \begin{enumerate}
        \item Use the Generalized Mean Value Theorem to furnish a proof of the \( 0 / 0 \) case of L'Hospital's Rule (Theorem 5.3.6).

        \item If we keep the first part of the hypothesis of Theorem 5.3.6 the same but we assume that
        \[
            \lim_{x \to a} \frac{f'(x)}{g'(x)} = \infty,
        \]
        does it necessarily follow that
        \[
            \lim_{x \to a} \frac{f(x)}{g(x)} = \infty?
        \]
    \end{enumerate}
\end{exercise}

\begin{solution}
    \begin{enumerate}
        \item Let \( \epsilon > 0 \) be given. Since \( \lim_{x \to a} \frac{f'(x)}{g'(x)} = L \), there is a \( \delta > 0 \) such that
        \[
            0 < \abs{x - a} < \delta \implies \abs{\frac{f'(x)}{g'(x)} - L} < \epsilon.
        \]
        Suppose \( x \in (a - \delta, a) \). By the Generalized Mean Value Theorem on the interval \( [x, a] \), there exists some \( c \in (x, a) \) such that
        \[
            \frac{f'(c)}{g'(c)} = \frac{f(a) - f(x)}{g(a) - g(x)} = \frac{f(x)}{g(x)};
        \]
        note we are using that \( g' \) does not vanish on \( (x, a) \). Since \( c \in (a - \delta, a) \), we then have
        \[
            \abs{\frac{f(x)}{g(x)} - L} = \abs{\frac{f'(c)}{g'(c)} - L} < \epsilon.
        \]
        We can similarly handle the case where \( x \in (a, a + \delta) \) by using the Generalized Mean Value Theorem on the interval \( [a, x] \). In any case, we have shown that if \( x \) satisfies \( 0 < \abs{x - a} < \delta \) then
        \[
            \abs{\frac{f(x)}{g(x)} - L} < \epsilon
        \]
        and thus
        \[
            \lim_{x \to a} \frac{f(x)}{g(x)} = L.
        \]

        \item It does necessarily follow; the proof from part (a) needs only slight modifications. Let \( M > 0 \) be given. Since \( \lim_{x \to a} \frac{f'(x)}{g'(x)} = \infty \), there is a \( \delta > 0 \) such that
        \[
            0 < \abs{x - a} < \delta \implies \frac{f'(x)}{g'(x)} \geq M.
        \]
        Suppose \( x \in (a - \delta, a) \). By the Generalized Mean Value Theorem on the interval \( [x, a] \), there exists some \( c \in (x, a) \) such that
        \[
            \frac{f'(c)}{g'(c)} = \frac{f(a) - f(x)}{g(a) - g(x)} = \frac{f(x)}{g(x)};
        \]
        note we are using that \( g' \) does not vanish on \( (x, a) \). Since \( c \in (a - \delta, a) \), we then have
        \[
            \frac{f(x)}{g(x)} = \frac{f'(c)}{g'(c)} \geq M.
        \]
        We can similarly handle the case where \( x \in (a, a + \delta) \) by using the Generalized Mean Value Theorem on the interval \( [a, x] \). In any case, we have shown that if \( x \) satisfies \( 0 < \abs{x - a} < \delta \) then
        \[
            \frac{f(x)}{g(x)} \geq M
        \]
        and thus
        \[
            \lim_{x \to a} \frac{f(x)}{g(x)} = \infty.
        \]
    \end{enumerate}
\end{solution}

\begin{exercise}
\label{ex:12}
    If \( f \) is twice differentiable on an open interval containing \( a \) and \( f'' \) is continuous at \( a \), show
    \[
        \lim_{h \to 0} \frac{f(a + h) - 2 f(a) + f(a - h)}{h^2} = f''(a).
    \]
    (Compare this to \href{https://lew98.github.io/Mathematics/UA_Section_5_2_Exercises.pdf}{Exercise 5.2.6(b)}.)
\end{exercise}

\begin{solution}
    We have by the \( 0 / 0 \) case of L'Hospital's Rule (Theorem 5.3.6) that
    \[
        \lim_{h \to 0} \frac{f(a + h) - 2 f(a) + f(a - h)}{h^2} = \lim_{h \to 0} \frac{f'(a + h) - f'(a - h)}{2h}.
    \]
    Since \( f' \) is differentiable at \( a \), we may apply \href{https://lew98.github.io/Mathematics/UA_Section_5_2_Exercises.pdf}{Exercise 5.2.6 (b)} to see that
    \[
        \lim_{h \to 0} \frac{f'(a + h) - f'(a - h)}{2h} = f''(a).
    \]
\end{solution}

\noindent \hrulefill

\noindent \hypertarget{ua}{\textcolor{blue}{[UA]} Abbott, S. (2015) \textit{Understanding Analysis.} 2\ts{nd} edition.}

\end{document}