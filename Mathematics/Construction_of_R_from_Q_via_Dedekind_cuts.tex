\documentclass[12pt]{article}
\title{Construction of \texorpdfstring{\(\mathbb{R}\)}{} from \texorpdfstring{\(\mathbb{Q}\)}{} via Dedekind cuts}
\author{}
\date{\vspace{-24mm}}
\usepackage[utf8]{inputenc}
\usepackage{amsmath}
\usepackage{amsthm}
\usepackage{geometry}
\usepackage{amsfonts}
\usepackage{bm}
\usepackage{hyperref}
\usepackage{xcolor}
\usepackage{enumitem}
\usepackage[nameinlink]{cleveref}
\geometry{
headheight=15pt,
left=60pt,
right=60pt
}
\usepackage{fancyhdr}
\pagestyle{fancy}
\fancyhf{}
\lhead{}
\chead{Construction of \texorpdfstring{\(\mathbb{R}\)}{} from \texorpdfstring{\(\mathbb{Q}\)}{} via Dedekind cuts}
\rhead{\thepage}

\setlength{\parindent}{0pt}
\hypersetup{
    colorlinks=true,
    linkcolor=blue,
    urlcolor=blue
}

\newcommand{\newp}{\vspace{5mm}}

\theoremstyle{definition}
\newtheorem{theorem}{Theorem}
\newtheorem{lemma}{Lemma}

\newtheorem*{remark}{Remark}

\begin{document}

\maketitle

\tableofcontents

\newpage

The following is mostly paraphrased from the appendix to Chapter 1 of \hyperlink{pma}{[PMA]}, with some details filled in and some changes to notation.

\section{Construction of \texorpdfstring{\(\mathbb{R}\)}{} from \texorpdfstring{\(\mathbb{Q}\)}{} via Dedekind cuts}

Our aim is to prove the following theorem.

\begin{theorem}
\label{thm:real_number_field}
    There exists an ordered field \( \mathbb{R} \) which has the least-upper-bound property. Moreover, \( \mathbb{R} \) contains \( \mathbb{Q} \) as a subfield.
\end{theorem}

We will assume that \( \mathbb{Q} \) has already been constructed from the integers and is known to be an ordered field.

\subsection{Defining \texorpdfstring{\(\mathbb{R}\)}{}}
\label{sec:defining_R}

Let \( A \) be a subset of \( \mathbb{Q} \). Then \( A \) is a \textbf{Dedekind cut} if it satisfies the following four properties:

\begin{enumerate}[label = (\Roman*)]
    \item \( A \) is non-empty;
    
    \item \( A \neq \mathbb{Q} \);
    
    \item if \( p \in A \), \( q \in \mathbb{Q} \), and \( q < p \), then \( q \in A \) (this property is sometimes known as being ``closed downwards'');
    
    \item if \( p \in A \), then \( p < r \) for some \( r \in A \) (in other words, \( A \) has no greatest element).
\end{enumerate}

We define \( \mathbb{R} \) to be the set of all Dedekind cuts; we will refer to elements of \( \mathbb{R} \) as Dedekind cuts or (perhaps prematurely) as real numbers. For any given rational number \( q \), one can verify that the set \( \{ p \in \mathbb{Q} : p < q \} \) is a Dedekind cut, so that our definition is non-empty. (In fact, this collection of Dedekind cuts will be the subfield \( \mathbb{Q} \) inside of \( \mathbb{R} \). We will make this more precise in \Cref{sec:R_contains_Q}.)

\newp

\textbf{Remark.} In general, we will use uppercase letters \( A, B, C, \ldots \) to refer to Dedekind cuts and lowercase letters \( p, q, r, s, \ldots \) to refer to rational numbers.

\subsection{Ordering \texorpdfstring{\(\mathbb{R}\)}{}}
\label{sec:ordering_R}

For \( A, B \in \mathbb{R} \) (that is, for Dedekind cuts \( A \) and \( B \)), we define \( A < B \) to mean that \( A \) is a proper subset of \( B \) and \( A \leq B \) to mean that \( A < B \) or \( A = B \); clearly these two are mutually exclusive (we are essentially using the symbol \( \leq \) in place of \( \subseteq \)). We claim that \( \leq \) is a total order on \( \mathbb{R} \). For all \( A, B, C \in \mathbb{R} \), we must verify the following four properties:

\begin{enumerate}[label = (O\arabic*)]
    \item \( A \leq A \) (\textbf{reflexivity}). This certainly holds.
    
    \item \( A \leq B \) and \( B \leq A \) implies that \( A = B \) (\textbf{antisymmetry}). \( A \) and \( B \) cannot both be proper subsets of each other, so if \( A \leq B \) and \( B \leq A \) then it must be the case that \( A = B \).
    
    \item \( A \leq B \) and \( B \leq C \) implies that \( A \leq C \) (\textbf{transitivity}). The only interesting case is when each inequality is strict; in that case, transitivity holds since a proper subset of a proper subset is a proper subset.
    
    \item \( A \leq B \) or \( B \leq A \) (\textbf{comparibility}, or the \textbf{trichotomy law}). Note that this does not hold for arbitrary sets; we will have to use the properties of Dedekind cuts. It will suffice to show that if \( A \) is not a proper subset of \( B \) and \( A \neq B \), then \( B \) is a proper subset of \( A \). Assuming therefore that \( A \) is not a subset of \( B \), there must exist some \( q \in A \) such that \( q \not\in B \). Let \( p \) be any element of \( B \). We cannot have \( p = q \) since \( q \not\in B \), and \( p > q \) would violate property (III), so we must have \( p < q \). It then follows from property (III) that \( p \in A \). So \( B \) is a subset of \( A \) but by assumption is not equal to it, i.e.\ \( B \) is a proper subset of \( A \).
\end{enumerate}

\subsection{\texorpdfstring{\(\mathbb{R}\)}{} has the least-upper-bound property}
\label{sec:R_has_lub_property}

We will now show that \( \mathbb{R} \) with this total ordering has the least-upper-bound property. Let \( E \) be a non-empty subset of \( \mathbb{R} \) which is bounded above by some \( B \in \mathbb{R} \), i.e.\ for each \( A \in E \), \( A \) is a subset of \( B \). We must show that such an \( E \) always has a supremum in \( \mathbb{R} \). Let \( C \) be the union of all \( A \in E \); our claim is that \( C \) is the supremum of \( E \). To see this, we need to show the following:

\begin{enumerate}[label = (S\arabic*)]
    \item \( C \in \mathbb{R} \), i.e.\ \( C \) is a Dedekind cut. We shall verify properties (I) - (IV).
    
    \begin{enumerate}[label = (\Roman*)]
        \item Since \( E \) is non-empty, there exists some \( A_0 \in E \) which is a Dedekind cut and hence non-empty. Any \( A \in E \) is a subset of \( C \), so \( C \) has a non-empty subset and hence must be non-empty itself.
        
        \item \( C \) must be a subset of \( B \) since each \( A \in E \) is a subset of \( B \). Since \( B \) is a Dedekind cut, we have \( B \neq \mathbb{Q} \); it follows that \( C \neq \mathbb{Q} \).
        
        \item Suppose \( p \in C, q \in \mathbb{Q} \), and \( q < p \). Then there exists some \( A_0 \in E \) such that \( p \in A_0 \). Since \( A_0 \) is a Dedekind cut, property (III) implies that \( q \in A_0 \), from which it follows that \( q \in C \).
        
        \item Suppose \( p \in C \). Then there exists some \( A_0 \in E \) such that \( p \in A_0 \). Since \( A_0 \) is a Dedekind cut, property (IV) implies that there is some \( r \in A_0 \), which must also belong to \( C \), with \( p < r \).
    \end{enumerate}
    
    \item \( C \) is an upper bound of \( E \). This certainly holds, since any \( A \in E \) is a subset of the union of all such \( A \).
    
    \item \( C \) is the least upper bound of \( E \). To see this, let \( D \in \mathbb{R} \) be such that \( D < C \). Then there must exist some \( p \in C \) such that \( p \not\in D \). Since \( p \in C \), there is an \( A_0 \in E \) such that \( p \in A_0 \). Suppose \( q \in D \). Then it cannot be the case that \( q > p \), otherwise property (III) would imply that \( p \in D \), and it cannot be the case that \( q = p \), since \( p \not\in D \). So we must have \( q < p \), which implies by property (III) that \( q \in A_0 \). Hence \( D \) is a subset of \( A_0 \). In fact, \( D < A_0 \) since \( p \) belongs to \( A_0 \) but not to \( D \), so that \( D \) cannot possibly be an upper bound of \( E \). It follows that \( C \) is the least upper bound of \( E \).
\end{enumerate}

\subsection{Addition in \texorpdfstring{\(\mathbb{R}\)}{}}
\label{sec:addition_in_R}

So far, we have an ordered set \( \mathbb{R} \) with the least-upper-bound property. We will now define the field structure of \( \mathbb{R} \), starting with addition. For \( A, B \in \mathbb{R} \), define
\[
    A + B = \{ r + s : r \in A, s \in B \}.
\]
We will show that with this definition of addition, the five field axioms for addition hold for all \( A, B, C \in \mathbb{R} \):

\begin{enumerate}[label = (A\arabic*)]
    \item \( A + B \in \mathbb{R} \) (\textbf{closure}). In other words, we need to show that \( A + B \) is a Dedekind cut. We shall verify properties (I) - (IV).
    
    \begin{enumerate}[label = (\Roman*)]
        \item \( A + B \) is non-empty since \( A \) and \( B \) are non-empty.
        
        \item Neither \( A \) nor \( B \) contains every rational number, so there exist \( p, q \in \mathbb{Q} \) such that \( p \not\in A \) and \( q \not\in B \). Then for any \( r \in A \) and \( s \in B \), property (III) gives us \( r < p \) and \( s < q \); it follows that \( r + s < p + q \), so that \( p + q \) is greater than any element of \( A + B \) and hence cannot belong to it. We conclude that \( A + B \neq \mathbb{Q} \).
        
        \item Suppose \( r + s \in A + B \), \( q \in \mathbb{Q} \), and \( q < r + s \). Then \( q - s < r \) and so property (III) gives us \( q - s \in A \). Hence \( q = (q - s) + s \) belongs to \( A + B \).
        
        \item Suppose \( r + s \in A + B \). Property (IV) implies that there exists \( p \in A \) such that \( r < p \). It follows that \( p + s \in A + B \) and that \( r + s < p + s \).
    \end{enumerate}
    
    \item \( A + B = B + A \) (\textbf{commutativity}). This follows from commutativity of addition in \( \mathbb{Q} \).
    
    \item \( (A + B) + C = A + (B + C) \) (\textbf{associativity}). This follows from associativity of addition in \( \mathbb{Q} \).
    
    \item There exists an element \( 0 \in \mathbb{R} \) such that \( A + 0 = A \) (\textbf{additive identity}). Let \( 0^* \) be the set of all negative rational numbers (we are using the notation \( 0^* \) to avoid confusion with \( 0 \in \mathbb{Q} \)). As noted in \Cref{sec:defining_R}, sets of the form \( \{ p \in \mathbb{Q} : p < q \} \) for a given rational number \( q \) are Dedekind cuts, so we have \( 0^* \in \mathbb{R} \). We claim that \( 0^* \) is the additive identity in \( \mathbb{R} \). For the inclusion \( A + 0^* \subseteq A \), suppose \( r \in A \) and \( s \in 0^* \), i.e.\ \( s \in \mathbb{Q} \) with \( s < 0 \). Then \( r + s < r \) and so property (III) implies that \( r + s \in A \). For the reverse inclusion \( A \subseteq A + 0^* \), suppose \( r \in A \). Property (IV) implies that there exists \( s \in A \) such that \( r - s < 0 \); it follows that \( r = s + (r - s) \) belongs to \( A + 0^* \). Hence \( A \subseteq A + 0^* \) and we conclude that \( A + 0^* = A \).
    
    \item There exists an element \( -A \in \mathbb{R} \) such that \( A + (-A) = 0^* \) (\textbf{additive inverse}). We define our candidate for the additive inverse of \( A \) as
    \[
        -A = \{ p \in \mathbb{Q} : \text{there exists an } r > 0 \text{ such that} -p - r \not\in A \}.
    \]
    First, we will show that \( -A \) belongs to \( \mathbb{R} \) by verifying properties (I) - (IV).

    \begin{enumerate}[label = (\Roman*)]
        \item Since \( A \neq \mathbb{Q} \), there is a rational number \( p \not\in A \). By property (III), we must have \( p + 1 = -(-p - 2) - 1 \not\in A \). Hence \( -p - 2 \in -A \), so that \( -A \) is non-empty.
        
        \item For any \( p \in A \), property (III) implies that \( p - r \in A \) for any \( r > 0 \); this is exactly the statement that \( -p \not\in -A \). It follows that \( -A \neq \mathbb{Q} \) since \( A \) is non-empty.
        
        \item Suppose \( p \in -A \), \( q \in \mathbb{Q} \), and \( q < p \). Then there is an \( r > 0 \) such that \( -p - r \not\in A \), and the inequality \( q < p \) implies that \( -q - r > -p - r \). By property (III) we must have \( -q - r \not\in A \), whence \( q \in -A \).
        
        \item Suppose \( p \in -A \), i.e.\ there is an \( r > 0 \) such that \( -p - r \not\in A \). Then \( p + \frac{r}{2} > p \) also belongs to \( -A \), since \( -p - r = -\left(p + \frac{r}{2}\right) - \frac{r}{2} \not\in A \).
    \end{enumerate}
    
    Next, we will show that \( A + (-A) = 0^* \). For the inclusion \( A + (-A) \subseteq 0^* \), suppose \( r \in A \) and \( s \in -A \), so that there is a \( u > 0 \) such that \( -s - u \not\in A \). Property (III) implies that \( r - u \in A \), and furthermore that \( r - u < -s - u \); it follows that \( r + s < 0 \), i.e.\ \( r + s \in 0^* \). For the reverse inclusion \( 0^* \subseteq A + (-A) \), suppose that \( r \in 0^* \), i.e.\ \( r \) is a negative rational number. We claim that there must exist some \( p \in A \) such that \( p - \frac{r}{2} \not\in A \). To see this, suppose by way of contradiction that \( p - \frac{r}{2} \in A \) for all \( p \in A \). An induction argument then gives \( p - \frac{nr}{2} \in A \) for all \( p \in A \) and all positive integers \( n \). Let \( q \in \mathbb{Q} \) be given. Since \( -\frac{r}{2} > 0 \), we may invoke the Archimedean property of \( \mathbb{Q} \) to obtain a positive integer \( N \) such that \( p -\frac{Nr}{2} > q \); but since \( p -\frac{Nr}{2} \in A \), property (III) gives \( q \in A \). Since \( q \) was arbitrary, the conclusion is that \( A = \mathbb{Q} \), which is a contradiction. Hence there must exist a \( p \in A \) such that \( p - \frac{r}{2} \not\in A \). This implies that \( r - p \in -A \), since \( -(r - p) - (-\frac{r}{2}) = p - \frac{r}{2} \not\in A \), and it follows that \( r = p + (r - p) \) belongs to \( A + (-A) \). We conclude that \( 0^* \subseteq A + (-A) \) and hence that \( A + (-A) = 0^* \).
\end{enumerate}

Now that we have shown that addition in \( \mathbb{R} \) satisfies the field axioms for addition, we can present the following theorem, given without proof (see, for example, Proposition 1.14 of \hyperlink{pma}{[PMA]}). It contains four statements which are true in any set with a definition of addition which satisfies the field axioms for addition, although we state them in particular for \( \mathbb{R} \).

\begin{theorem}
\label{thm:addition_axioms}
    For all \( A, B, C \in \mathbb{R} \), the following statements hold.
    \begin{enumerate}[label = (\alph*)]
        \item If \( A + B = A + C \) then \( B = C \).

        \item If \( A + B = A \) then \( B = 0^* \).

        \item If \( A + B = 0^* \) then \( B = -A \).

        \item \( -(-A) = A \).
    \end{enumerate}
\end{theorem}

Part (a) of Theorem 2 allows us to prove the following statement, which is the first requirement for \( \mathbb{R} \) to be an \textbf{ordered field}:

\begin{enumerate}[label = (OF\arabic*), left = 6px]
    \item For all \( A, B, C \in \mathbb{R} \), \( B < C \implies A + B < A + C \).
\end{enumerate}

Indeed, for any \( r \in A \) and \( s \in B \) we also have \( s \in C \), so that \( r + s \in A + C \). Hence \( A + B \) is a subset of \( A + C \), and \( A + B \neq A + C\) follows from the contrapositive of part (a) of Theorem 2.

\subsection{Multiplication in \texorpdfstring{\(\mathbb{R}\)}{}}

To complete the field structure of \( \mathbb{R} \), we need to define multiplication of real numbers; this is somewhat more involved than addition. Let \( \mathbb{R}_+ = \{ A \in \mathbb{R} : 0^* < A \} \), i.e.\ the set of those Dedekind cuts which contain the negative rational numbers as a strict subset. We will first define multiplication of elements in \( \mathbb{R}_+ \), show that this definition satisfies the five field axioms for multiplication (with a slight change to the statement on multiplicative inverses; we need not consider \( 0^* \) since it does not belong to \( \mathbb{R}_+ \)), and then extend our definition to all of \( \mathbb{R} \). For \( A, B \in \mathbb{R}_+ \), define
\[
    AB = \{ p \in \mathbb{Q} : p \leq rs \text{ for some choice of } r \in A, s \in B, r > 0, s > 0 \}.
\]
\begin{enumerate}[label = (M\arabic*)]
    \item \( AB \in \mathbb{R}_+ \) (\textbf{closure}). First, let us show that \( AB \) is a Dedekind cut by verifying properties (I) - (IV).
    
    \begin{enumerate}[label = (\Roman*)]
        \item Note that \( A \) must contain some non-negative rational number \( r \) since \( 0^* \) is a strict subset of \( A \), and furthermore we may assume that \( r \) is positive by invoking property (IV) if necessary. So we can always find positive rationals \( r \in A, s \in B \), and there are certainly rational numbers less than \( rs \); it follows that \( AB \) is non-empty.
        
        \item Since \( A \) and \( B \) are not equal to \( \mathbb{Q} \), there exist rationals \( u \not\in A \) and \( v \not\in B \). For any choice of positive rationals \( r \in A \) and \( s \in B \), property (III) implies that \( r < u \) and \( s < v \), from which we obtain \( rs < uv \). It follows that \( uv \not\in AB \) since
        \[
            (AB)^{\mathsf{C}} = \{ p \in \mathbb{Q} : p > rs \text{ for all choices of } r \in A, s \in B, r > 0, s > 0 \}.
        \]
        Hence \( AB \neq \mathbb{Q} \).
        
        \item Suppose \( p \in AB, q \in \mathbb{Q}, \) and \( q < p \); it immediately follows that \( q \in AB \).
        
        \item Suppose \( p \in AB \), so that \( p \leq rs \) for some choice of \( r \in A, s \in B, r > 0, s > 0 \). Property (IV) implies that there is a \( q \in A \) with \( 0 < r < q \), which gives \( rs < qs \). Then \( p < qs \) and \( qs \in AB \).
    \end{enumerate}
    
    Now let us show that \( 0^* < AB \). As noted above, we can always find positive rationals \( r \in A, s \in B \), and it is certainly the case that all non-positive rational numbers \( p \) satisfy \( p \leq rs \); it follows that \( 0^* \) is a strict subset of \( AB \). This also proves that \( \mathbb{R} \) satisfies the second and last requirement to be an ordered field:
    
    \begin{enumerate}[label = (OF\arabic*), left = 6px, start = 2]
        \item For all \( A, B \in \mathbb{R} \), \( A > 0^* \) and \( B > 0 ^* \implies AB > 0^* \).
    \end{enumerate}
    
    (We showed that \( \mathbb{R} \) satisfies the first requirement to be an ordered field at the end of \Cref{sec:addition_in_R}; once we have finished defining multiplication on \( \mathbb{R} \) and verifying the field axioms for multiplication and distributivity, it will follow that \( \mathbb{R} \) is an ordered field.)
    
    \item \( AB = BA \) (\textbf{commutativity}). This follows from commutativity of multiplication in \( \mathbb{Q} \); one has \( p \leq rs \iff p \leq sr \).
    
    \item \( (AB)C = A(BC) \) (\textbf{associativity}). Suppose \( p \in (AB)C \), i.e.\ \( p \leq rs \) for some choice of \( r \in AB, s \in C, r > 0, s > 0 \). Then \( r \leq uv \) for some choice of \( u \in A, v \in B, u > 0, v > 0 \), so that \( p \leq uvs \). But note that \( vs \in BC \), so that \( p \in A(BC) \). The reverse inclusion is similar. (We have of course used that multiplication in \( \mathbb{Q} \) is associative.)
    
    \item There exists an element \( 1 \in \mathbb{R}_+ \) such that \( 1A = A \) (\textbf{multiplicative identity}). Let \( 1^* \) be the set of all rational numbers less than 1; it is clear that \( 1^* \in \mathbb{R}_+ \) (again, we are using the notation \( 1^* \) to avoid confusion with \( 1 \in \mathbb{Q} \)). We claim that \( 1^* \) is the multiplicative identity in \( \mathbb{R}_+ \). For the inclusion \( 1^* A \subseteq A \), suppose that \( p \in 1^* A \), so that \( p \leq rs \) for some \( 0 < r < 1 \) and \( s \in A \) with \( s > 0 \). It follows that \( p < s \), and property (III) then implies that \( p \in A \). For the reverse inclusion \( A \subseteq 1^* A \), suppose \( p \in A \). Combining property (IV) with the fact that \( A > 0^* \), we see that there is an \( s \in A \) such that \( s > 0 \) and \( s > p \). It follows that \( 0 < 1 - \frac{p}{s} \). Using the Archimedean property of \( \mathbb{Q} \), let \( N \) be a positive integer such that \( \frac{1}{N+1} \leq 1 - \frac{p}{s} \). After some algebra, we obtain \( p \leq \frac{N}{N+1} s \). Since \( 0 < \frac{N}{N+1} < 1 \), it follows that \( p \in 1^* A \).
    
    \item There exists an element \( A^{-1} \in \mathbb{R}_+ \) such that \( AA^{-1} = 1^* \) (\textbf{multiplicative inverse}). Let
    \[
        A^{-1} = \{ p \in \mathbb{Q} : p \leq 0 \text{ or there exists } r > 0 \text{ such that } \tfrac{1}{p} - r \not\in A \}.
    \]
    We claim that \( A^{-1} \) is the multiplicative inverse to \( A \). First, we will show that \( A^{-1} \) is a Dedekind cut by verifying properties (I) - (IV).
    
    \begin{enumerate}[label = (\Roman*)]
        \item \( A^{-1} \) contains all non-positive rational numbers, so certainly it is non-empty.

        \item We have
        \[
            \left(A^{-1}\right)^{\mathsf{C}} = \{ p \in \mathbb{Q} : p > 0 \text{ and } \tfrac{1}{p} - r \in A \text{ for all } r > 0 \}.
        \]
        Since \( A \in \mathbb{R}_+ \), there exists \( p \in A \) with \( p > 0 \). It follows from property (III) that \( p - r \in A \) for all \( r > 0 \), and \( \tfrac{1}{p} > 0 \), so \( \tfrac{1}{p} \not\in A^{-1} \). Hence \( A^{-1} \neq \mathbb{Q} \).

        \item Suppose \( p \in A^{-1}, q \in \mathbb{Q} \), and \( q < p \). If \( q \leq 0 \) then \( q \in A^{-1} \), so suppose \( q > 0 \). Then \( p > 0 \), so there must be some \( r > 0 \) such that \( \tfrac{1}{p} - r \not\in A \). Observe that
        \[
            0 < q < p \iff 0 < \tfrac{1}{p} - r < \tfrac{1}{q} - r.
        \]
        It follows from property (III) that \( \tfrac{1}{q} - r \not\in A \), so that \( q \in A^{-1} \).

        \item First, note that there exists \( u \not\in A \) with \( u > 0 \) (this is true of any Dedekind cut; if this were not the case, then the Dedekind cut would be the entire rational line). It follows that \( \tfrac{1}{2u} \in A^{-1} \), since \( 2u - u = u \not\in A \). So \( A^{-1} \) always contains positive rational numbers. Now suppose that \( p \in A^{-1} \). If \( p \leq 0 \), then by the above we can always find a positive \( q \in A^{-1} \) with \( p < q \). Suppose therefore that \( p > 0 \), so that there exists an \( r > 0 \) such that \( \tfrac{1}{p} - r \not\in A \). Let \( q = \tfrac{1}{p} - \tfrac{r}{2} \). Since \( A \in \mathbb{R}_+ \), it must be the case that \( \tfrac{1}{p} - r > 0 \). Observe that
        \[
            0 < \tfrac{1}{p} - r < q < \tfrac{1}{p} \implies 0 < p < \tfrac{1}{q}.
        \]
        It follows that \( \tfrac{1}{q} \in A^{-1} \), since \( q - \tfrac{r}{2} = \tfrac{1}{p} - r \not\in A \), and \( p < \tfrac{1}{q} \).
    \end{enumerate}
    Since \( A^{-1} \) contains all non-positive rationals, we have \( A^{-1} > 0^* \). So we have shown that \( A^{-1} \in \mathbb{R}_+ \); now we need to show that \( AA^{-1} = 1^* \). For the inclusion \( AA^{-1} \subseteq 1^* \), suppose \( p \in AA^{-1} \), i.e.\ \( p \leq rs \) for some choice of \( r \in A, s \in A^{-1}, r > 0, s > 0 \). Then there exists some \( u > 0 \) such that \( \tfrac{1}{s} - u \not\in A \). Property (III) implies that \( \tfrac{1}{s} \not\in A \), and furthermore that \( r < \tfrac{1}{s} \). It follows that \( p \leq rs < 1 \), so that \( p \in 1^* \). For the reverse inclusion \( 1^* \subseteq AA^{-1} \), suppose \( p \in 1^* \). If \( p \leq 0 \), then any choice of positive \( r \in A \) and \( s \in A^{-1} \) will do (as noted before, \( A \) and \( A^{-1} \) always contain positive rational numbers). Suppose therefore that \( 0 < p < 1 \). By the Archimedean property of \( \mathbb{Q} \), there exists a positive integer \( n \) such that
    \[
        p < 1 - \tfrac{1}{m+1} = \tfrac{m}{m+1} \tag{\(*\)}
    \]
    for all integers \( m \geq n \). Let \( r \) be any positive rational number in \( A \), and let \( q = \tfrac{r}{2n} \), so that \( 0 < q < \tfrac{r}{n} \); property (III) implies that both \( q \) and \( nq \in A \). Now we claim the following:
    \[
        \text{there exists a positive integer } m \text{ such that } mq \in A \text{ and } (m+1)q \not\in A.
    \]
    To see this, suppose by way of contradiction that the negation of this statement holds:
    \[
        \text{for all positive integers } m, \text{either }  mq \not\in A \text{ or } (m+1)q \in A.
    \]
    Since \( q \in A \), it follows from the negated statement that \( 2q \in A \). Proceeding by induction, we obtain \( mq \in A \) for all positive integers \( m \). Now let \( u \in \mathbb{Q} \) be given. By the Archimedean property of \( \mathbb{Q} \), there is a positive integer \( M \) such that \( Mq > u \). Property (III) then implies that \( u \in A \). We conclude that \( A = \mathbb{Q} \), which contradicts property (II) of Dedekind cuts. Hence there must be some positive integer \( m \) such that \( mq \in A \) and \( (m+1)q \not\in A \). Since \( nq \in A \), property (III) gives us \( nq < (m+1)q \). It follows that \( n \leq m \), so that inequality \( (*) \) holds for this \( m \). Then observe that
    \[
        0 < p < \tfrac{m}{m+1} \implies 0 < \tfrac{p}{mq} < \tfrac{1}{(m+1)q} \implies 0 < (m+1)q < \tfrac{mq}{p} \implies 0 < \tfrac{mq}{p} - (m+1)q.
    \]
    Hence \( \tfrac{p}{mq} \in A^{-1} \), since \( (m+1)q = \tfrac{mq}{p} - (\tfrac{mq}{p} - (m+1)q) \not\in A \). It follows that \( p \in AA^{-1} \), since \( p = mq \cdot \tfrac{p}{mq} \). We conclude that \( 1^* \subseteq AA^{-1} \) and hence that \( AA^{-1} = 1^* \).    
\end{enumerate}

We will now show that multiplication distributes over addition in \( \mathbb{R}_+ \), i.e.\ for all \( A, B, C \in \mathbb{R}_+ \), we have \( A(B + C) = AB + AC \). For the inclusion \( A(B + C) \subseteq AB + AC \), let \( p \in A(B + C) \) be given, so that \( p \leq rs \) for some choice of \( r \in A, s \in B + C, r > 0, s > 0 \). Then \( s \) is of the form \( u + v \) for some \( u \in B \) and \( v \in C \). Since \( B, C > 0^* \), there exist positive rational numbers \( u' \in B \) and \( v' \in C \) such that \( u \leq u' \) and \( v \leq v' \). We then have
\[
    p \leq rs = r(u + v) = ru + rv \leq ru' + rv'.
\]
The sum \( ru' + rv' \) belongs to \( AB + AC \), which we have shown is a Dedekind cut in (A1) and (M1). It follows from property (III) that \( p \in AB + AC \). For the reverse inclusion \( AB + AC \subseteq A(B + C) \), suppose \( p + q \in AB + AC \), i.e.\
\begin{gather*}
    p \leq r_1 s_1 \text { for some } r_1 \in A, s_1 \in B, r_1 > 0, s_1 > 0, \\
    q \leq r_2 s_2 \text { for some } r_2 \in A, s_2 \in C, r_2 > 0, s_2 > 0.
\end{gather*}
Let \( r = \max\{ r_1, r_2 \} \). Then \( r \in A, r > 0 \), and \( p + q \leq r_1 s_1 + r_2 s_2 \leq r(s_1 + s_2) \). Since \( s_1 + s_2 \in B + C \) and \( s_1 + s_2 > 0 \), it follows that \( p + q \in A(B + C) \). We conclude that \( AB + AC \subseteq A(B + C) \) and hence that \( A(B + C) = AB + AC \).

\newp

We are now in a position to define multiplication on all of \( \mathbb{R} \). For \( A, B \in \mathbb{R} \), set \( A 0^* = 0^* A = 0^* \), and
\[
    AB = \begin{cases}
        (-A)(-B) & \text{if } A < 0^*, B < 0^*, \\
        -[(-A)B] & \text{if } A < 0^*, B > 0^*, \\
        -[A(-B)] & \text{if } A > 0^*, B < 0^*.
    \end{cases}
\]
At the end of \Cref{sec:addition_in_R}, we showed that (OF1) holds for elements of \( \mathbb{R} \). A consequence of this is that \( A > 0^* \implies -A < 0^* \) (add \( -A \) to both sides of \( A > 0^* \)); hence the products on the right-hand side of our extended definition of multiplication are happening in \( \mathbb{R}_+ \). Showing that the field axioms for multiplication hold in \( \mathbb{R} \) with this extended definition of multiplication mostly amounts to casework. For all \( A, B, C \in \mathbb{R} \):
\begin{enumerate}[label = (M\arabic*)]
    \item \( AB \in \mathbb{R} \) (\textbf{closure}). If either of \( A \) and \( B \) are \( 0^* \), then \( AB = 0^* \in \mathbb{R} \). Otherwise, we consider the following cases.
    
    \begin{itemize}
        \item \( A > 0^*, B > 0^* \). We have already shown that a product of positive real numbers is a (positive) real number.
        
        \item \( A < 0^*, B < 0^* \). Then \( AB = (-A)(-B) \), which is again a product of positive real numbers.
        
        \item \( A < 0^*, B > 0^* \). Then \( AB = -[(-A)B] \), which is the additive inverse of a product of positive real numbers and hence is a real number itself.
        
        \item \( A > 0^*, B < 0^* \). Then \( AB = -[A(-B)] \), which is the additive inverse of a product of positive real numbers and hence is a real number itself.
    \end{itemize}

    \item \( AB = BA \) (\textbf{commutativity}). This follows from commutativity of products in \( \mathbb{R}_+ \).

    \item \( A(BC) = (AB)C \) (\textbf{associativity}). This follows from associativity of products in \( \mathbb{R}_+ \).

    \item There exists an element \( 1 \in \mathbb{R} \) such that \( 1A = A \) (\textbf{multiplicative identity}). Of course, we claim that \( 1^* \) is the multiplicative identity for all of \( \mathbb{R} \).

    \begin{itemize}
        \item \( A > 0^* \). We have already shown that \( 1^* A = A \) for positive \( A \).

        \item \( A = 0^* \). Then \( 1^* 0^* = 0^* \).

        \item \( A < 0^* \). Then \( 1^* A = -[1^* (-A)] = -[(-A)] = A \), where we have used part (d) of \Cref{thm:addition_axioms} (it is clear from the definitions of \( 0^* \) and \( 1^* \) that \( 1^* > 0^* \)).
    \end{itemize}

    \item If \( A \neq 0^* \), then there exists an element \( A^{-1} \in \mathbb{R} \) such that \( A A^{-1} = 1^* \) (\textbf{multiplicative inverse}).

    \begin{itemize}
        \item \( A > 0^* \). We have already shown that \( A^{-1} \) exists in \( \mathbb{R} \).

        \item \( A < 0^* \). Then \( -A > 0^* \), so \( (-A)^{-1} \) exists in \( \mathbb{R}_+ \). We claim that \( A^{-1} = -(-A)^{-1} \) (note that this is negative). Indeed,
        \[
            A A^{-1} = (-A)(-\left(A^{-1}\right)) = (-A)[-[-(-A)^{-1}]] = (-A)(-A)^{-1} = 1^*,
        \]
        where we have used part (d) of \Cref{thm:addition_axioms}.
    \end{itemize}
\end{enumerate}

Finally, we need to show that multiplication distributes over addition in \( \mathbb{R} \), i.e.\ for all \( A, B, C \in \mathbb{R} \), \( A(B + C) = AB + AC \); we already showed that this holds in \( \mathbb{R}_+ \). There are a number of cases to check, a couple of which are shown below. The remaining cases are handled similarly.

\begin{itemize}
    \item \( A > 0^*, B < 0^*, B + C > 0^* \). Then \( C = (B + C) + (-B) > 0^* \), so
    \[
        AC = A[(B + C) + (-B)] = A(B + C) + A(-B),
    \]
    since distributivity holds in \( \mathbb{R}_+ \). Now observe that
    \[
        AB + A(-B) = -[A(-B)] + A(-B) = 0^*.
    \]
    It follows from part (c) of \Cref{thm:addition_axioms} that \( A(-B) = -(AB) \). Hence we see that
    \[
        AC = A(B + C) + A(-B) \iff AC = A(B + C) + [-(AB)] \iff A(B + C) = AB + AC.
    \]

    \item \( A > 0^*, B < 0^*, C < 0^* \). Note that by commutativity and associativity of addition, we have
    \[
        (-B) + (-C) + (B + C) = (B + (-B)) + (C + (-C)) = 0^* + 0^* = 0^*.
    \]
    Part (c) of \Cref{thm:addition_axioms} then implies that \( -(B + C) = (-B) + (-C) \). Since \( B \) and \( C \) are both negative, \( B + C \) is also negative. Then
    \[
        A(B + C) = -[A(-(B + C))] = -[A((-B) + (-C))] = -[A(-B) + A(-C)],
    \]
    since distributivity holds in \( \mathbb{R}_+ \). Similarly to the previous case, it can be verified that \( A(-B) = -(AB) \) and \( A(-C) = -(AC) \). Hence
    \begin{align*}
        A(B + C) &= -[A(-B) + A(-C)] \\
        &= -[-(AB) + (-(AC))] \\
        &= -[-(AB)] + (-[-(AC)]) \\
        &= AB + AC,
    \end{align*}
    where we have used part (d) of \Cref{thm:addition_axioms}.
\end{itemize}

We have now shown that \( \mathbb{R} \) is an ordered field with the least-upper-bound property. This allows us to present another theorem, given without proof (see, for example, Proposition 1.16 of \hyperlink{pma}{[PMA]}). It contains two statements which are true in any field, although we state them in particular for \( \mathbb{R} \).

\begin{theorem}
\label{thm:multiplication_axioms}
    For all \( A, B, C \in \mathbb{R} \), the following statements hold.
    \begin{enumerate}[label = (\alph*)]
        \item \( (-A)B = -(AB) = A(-B) \).

        \item \( (-A)(-B) = AB \).
    \end{enumerate}
\end{theorem}

\subsection{\texorpdfstring{\(\mathbb{R}\)}{} contains \texorpdfstring{\(\mathbb{Q}\)}{} as a subfield}
\label{sec:R_contains_Q}

Finally, we will prove the last part of \Cref{thm:real_number_field}, which says that \( \mathbb{R} \) contains \( \mathbb{Q} \) as a subfield. More precisely, we will demonstrate the existence of a function \( \psi : \mathbb{Q} \to \mathbb{R} \) such that the following statements hold for all \( p, q \in \mathbb{Q} \):
\begin{enumerate}[label = (H\arabic*)]
    \item \( \psi(0) = 0^* \) and \( \psi(1) = 1^* \);
    \item \( \psi(p) < \psi(q) \) if and only if \( p < q \);
    \item \( \psi(p + q) = \psi(p) + \psi(q) \);
    \item \( \psi(pq) = \psi(p) \psi(q) \).
\end{enumerate}
Such a function is said to be an \textbf{ordered field homomorphism}; it preserves both the order and field structure of \( \mathbb{Q} \). It can be shown that field homomorphisms are necessarily injective, so \( \psi \) will in fact be an ordered field isomorphism onto its image \( \psi(\mathbb{Q}) \). This permits us to make an identification of \( \mathbb{Q} \) with \( \psi(\mathbb{Q}) \subseteq \mathbb{R} \), which is what we mean by `\( \mathbb{R} \) contains \( \mathbb{Q} \) as a subfield'. We will show at the end of this section that \( \psi \) cannot be surjective, so that \( \mathbb{Q} \) is strictly contained inside of \( \mathbb{R} \).

\newp

To define \( \psi \), let \( \psi(p) = \{ u \in \mathbb{Q} : u < p \} \) for \( p \in \mathbb{Q} \). One can verify that \( \psi(p) \) is a Dedekind cut and that (H1) holds. We will now show that the statements (H2) - (H4) hold.

\begin{enumerate}[label = (H\arabic*), start = 2]
    \item Suppose that \( p < q \); it is then clear that \( \psi(p) \) is a subset of \( \psi(q) \). This containment is strict since \( \tfrac{p + q}{2} \) belongs to \( \psi(q) \) but not to \( \psi(p) \). Hence \( \psi(p) < \psi(q) \). Conversely, suppose \( \psi(p) \) is a strict subset of \( \psi(q) \). Then there must exist some \( u \in \mathbb{Q} \) such that \( u \in \psi(q) \) and \( u \not\in \psi(p) \), i.e.\ \( u < q \) and \( p \leq u \). It follows that \( p < q \).

    \item We have
    \begin{align*}
        \psi(p + q) &= \{ u \in \mathbb{Q} : u < p + q \}, \\
        \psi(p) + \psi(q) &= \{ r + s : r \in \psi(p), s \in \psi(q) \} \\
        &= \{ r + s : r < p, s < q \}.
    \end{align*}
    The inclusion \( \psi(p) + \psi(q) \subseteq \psi(p + q) \) is clear. For the reverse inclusion, suppose \( u \in \psi(p + q) \), i.e.\ \( u < p + q \). Using the Archimedean property of \( \mathbb{Q} \), choose a positive integer \( N \) such that \( \tfrac{1}{N} < p + q - u \). Then \( p - \tfrac{1}{N} < p \), \( u - p + \tfrac{1}{N} < q \), and
    \[
        u = \left(p - \tfrac{1}{N}\right) + \left(u - p + \tfrac{1}{N}\right) \in \psi(p) + \psi(q).
    \]

    A useful consequence of (H1) and (H2) is the following. For any \( p \in \mathbb{Q} \), we have
    \[
        0^* = \psi(0) = \psi(p + (-p)) = \psi(p) + \psi(-p).
    \]
    It follows from \Cref{thm:addition_axioms} (c) that \( \psi(-p) = -\psi(p) \).

    \item First, suppose \( p \) and \( q \) are both positive. It then follows from (H1) and (H2) that both of \( \psi(p) \) and \( \psi(q) \) are also positive. Hence we have
    \begin{align*}
        \psi(pq) &= \{ u \in \mathbb{Q} : u < pq \}, \\
        \psi(p) \psi(q) &= \{ u \in \mathbb{Q} : u \leq rs \text{ for some choice of } r \in \psi(p), s \in \psi(q), r > 0, s > 0 \} \\
        &= \{ u \in \mathbb{Q} : u \leq rs \text{ for some choice of } 0 < r < p, 0 < s < q \}.
    \end{align*}
    The inclusion \( \psi(p) \psi(q) \subseteq \psi(pq) \) is clear. For the reverse inclusion, suppose \( u \in \psi(pq) \), i.e.\ \( u < pq \). If \( u \leq 0 \), then any choice of \( r \) and \( s \) with \( 0 < r < p \) and \( 0 < s < q \) will do, say \( r = \tfrac{p}{2} \) and \( s = \tfrac{q}{2} \). If \( 0 < u < pq \), then set \( r = \tfrac{1}{2}\left(\tfrac{u}{q} + p\right) \). It follows that
    \[
        0 < \tfrac{u}{q} < r < p \implies 0 < \tfrac{u}{r} < q.
    \]
    Then since \( u = r \cdot \tfrac{u}{r} \), we have \( u \in \psi(p) \psi(q) \), so that \( \psi(pq) \subseteq \psi(p) \psi(q) \). Hence we have \( \psi(pq) = \psi(p) \psi(q) \) in the special case when both of \( p \) and \( q \) are positive.

    \newp

    If either of \( p \) or \( q \) are 0, then the equality \( \psi(pq) = \psi(p) \psi(q) \) is clear since \( \psi(0) = 0^* \). Suppose that \( p \) and \( q \)  have opposite signs, say \( p > 0 \) and \( q < 0 \). Then we have
    \begin{align*}
        \psi(pq) &= \psi(-[p(-q)]) \\
        &= -\psi(p(-q)) \\
        &= -[\psi(p) \psi(-q)] \\
        &= -[\psi(p) (-\psi(q))] \\
        &= -(-[\psi(p) \psi(q)]) \\
        &= \psi(p) \psi(q),
    \end{align*}
    where we have used \Cref{thm:multiplication_axioms} (a) and \Cref{thm:addition_axioms} (d). Now suppose that \( p < 0 \) and \( q < 0 \). Then
    \[
        \psi(pq) = \psi((-p)(-q)) = \psi(-p) \psi(-q) = [-\psi(p)] [-\psi(q)] = \psi(p) \psi(q),
    \]
    where we have used \Cref{thm:multiplication_axioms} (b). We have now shown that (H4) holds in all cases.
\end{enumerate}

Now we will show that \( \psi \) cannot be surjective. One approach is to use the following lemma.

\begin{lemma}
    Suppose \( A \) and \( B \) are totally ordered sets and \( f : A \to B \) is a bijection with the following property: for all \( a \in A \) and \( b \in B \),
    \[
        a < b \iff f(a) < f(b).
    \]
    Then if \( A \) has the least-upper-bound property, so does \( B \).
\end{lemma}

\begin{proof}
    Let \( E \subseteq B \) be non-empty and bounded above by some \( b \in B \). Then since \( f \) is a bijection, \( f^{-1}(E) \subseteq A \) is also non-empty. Furthermore, it is bounded above by \( f^{-1}(b) \in A \):
    \[
        a \in f^{-1}(E) \iff f(a) \in E \implies f(a) < b \iff a < f^{-1}(b).
    \]
    Hence \( s = \sup f^{-1}(E) \) exists in \( A \). We claim that \( \sup E = f(s) \). To prove this, we need to show two things.
    \begin{itemize}
        \item \( f(s) \) is an upper bound for \( E \). This follows since
        \[
            y \in E \iff f^{-1}(y) \in f^{-1}(E) \implies f^{-1}(y) < s \iff y < f(s).
        \]

        \item If \( y \in B \) is such that \( y < f(s) \), then \( y \) is not an upper bound of \( E \). For such a \( y \), we have \( f^{-1}(y) < s \). Hence \( f^{-1}(y) \) is not an upper bound of \( f^{-1}(E) \), i.e.\ there must exist some \( x \in f^{-1}(E) \) such that \( f^{-1}(y) < x \). It follows that \( y < f(x) \), with \( f(x) \in E \), so that \( y \) cannot be an upper bound of \( E \).
    \end{itemize}
    We conclude that \( \sup E = f(s) \) and hence that \( B \) has the least-upper-bound property.
\end{proof}

This lemma rules out the possibility of \( \psi \) being surjective, since this would imply the existence of \( \psi^{-1} : \mathbb{R} \to \mathbb{Q} \) satisfying the hypotheses of Lemma 1, which would in turn imply that \( \mathbb{Q} \) has the least-upper-bound property; but \( \mathbb{Q} \) does not have the least-upper-bound property (see Chapter 1 of \hyperlink{pma}{[PMA]} or \href{https://lew98.github.io/Mathematics/Q_does_not_have_the_least_upper_bound_property.pdf}{here}).

\newp

Another approach would be to consider square roots of 2. It can be shown that there is a real number whose square is 2 (see Chapter 1 of \hyperlink{pma}{[PMA]} or \href{https://lew98.github.io/Mathematics/Consequences_of_the_least_upper_bound_property_of_R.pdf}{here}). Combining such a real number with the existence of \( \psi^{-1} : \mathbb{R} \to \mathbb{Q} \) would imply that there was a rational number whose square is 2; but it is well-known that there is no such rational number.

\hrulefill

\hypertarget{pma}{\textcolor{blue}{[PMA]} Rudin, W. (1976) \textit{Principles of Mathematical Analysis.} 3rd edn.}

\end{document}
