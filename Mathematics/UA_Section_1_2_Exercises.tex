\documentclass[12pt]{article}
\usepackage[utf8]{inputenc}
\usepackage[utf8]{inputenc}
\usepackage{amsmath}
\usepackage{amsthm}
\usepackage{geometry}
\usepackage{amsfonts}
\usepackage{mathrsfs}
\usepackage{bm}
\usepackage{hyperref}
\usepackage[dvipsnames]{xcolor}
\usepackage{enumitem}
\usepackage{changepage}
\usepackage{lipsum}
\usepackage{tikz}
\usetikzlibrary{matrix}
\usepackage{tikz-cd}
\usepackage[nameinlink]{cleveref}
\geometry{
headheight=15pt,
left=60pt,
right=60pt
}
\usepackage{fancyhdr}
\pagestyle{fancy}
\fancyhf{}
\lhead{}
\chead{Section 1.2 Exercises}
\rhead{\thepage}
\hypersetup{
    colorlinks=true,
    linkcolor=blue,
    urlcolor=blue
}

\theoremstyle{definition}

\newtheorem*{remark}{Remark}

\newtheoremstyle{exercise}
    {}
    {}
    {}
    {}
    {\bfseries}
    {.}
    { }
    {\thmname{#1}\thmnumber{#2}\thmnote{ (#3)}}
\theoremstyle{exercise}
\newtheorem{exercise}{Exercise 1.2.}

\newtheoremstyle{solution}
    {}
    {}
    {}
    {}
    {\itshape\color{magenta}}
    {.}
    { }
    {\thmname{#1}\thmnote{ #3}}
\theoremstyle{solution}
\newtheorem*{solution}{Solution}

\Crefformat{exercise}{#2Exercise 1.2.#1#3}

\newcommand{\setcomp}[1]{#1^{\mathsf{c}}}

\begin{document}

\section{Section 1.2 Exercises}

Exercises with solutions from Section 1.2 of \hyperlink{ua}{[UA]}.

\begin{exercise}
\label{ex:1}
    \begin{enumerate}[label = (\alph*)]
        \item Prove that \( \sqrt{3} \) is irrational. Does the same argument work to show that \( \sqrt{6} \) is irrational?

        \item Where does the proof of Theorem 1.1.1 break down if we try to use it to prove \( \sqrt{4} \) is irrational?
    \end{enumerate}
\end{exercise}

\begin{solution}
    \begin{enumerate}[label = (\alph*), listparindent = \parindent, parsep = 0em]
        \item Suppose there was a rational number \( p = \tfrac{m}{n} \), which we may assume is in lowest terms, such that \( p^2 = 3 \). Then \( m^2 = 3 n^2 \), so that \( m^2 \) is divisible by 3. Observe that for any \( k \in \mathbf{N} \) we have
        \[
            (3k + 1)^2 = 3(3k^2 + 2k) + 1 \quad \text{and} \quad (3k + 2)^2 = 3(3k^2 + 4k + 1) + 1.
        \]
        It follows that
        \[
            m \text{ is not divisible by 3} \implies m^2 \text{ is not divisible by 3},
        \]
        the contrapositive of which is
        \[
            m^2 \text{ is divisible by 3} \implies m \text{ is divisible by 3}.
        \]
        Hence we may write \( m = 3k \) for some \( k \in \mathbf{N} \) and substitute this into the equation \( m^2 = 3 n^2 \) to obtain the equation \( n^2 = 3 k^2 \), which similarly implies that \( n \) is divisble by 3. This is a contradiction since we assumed that \( m \) and \( n \) had no common factors. We may conclude that there is no rational number whose square is 3.

        The same argument works to show that there is no rational number whose square is 6; the crux of this argument is that
        \[
            m^2 \text{ is divisible by 6} \implies m \text{ is divisible by 6}.
        \]

        \item The argument breaks down when we try to assert that
        \[
            m^2 \text{ is divisible by 4} \implies m \text{ is divisible by 4}.
        \]
        This is false; for example, \( 2^2 \) = 4 is divisible by 4 but 2 is not divisible by 4.
    \end{enumerate}
\end{solution}

\begin{exercise}
\label{ex:2}
    Show that there is no rational number \( r \) satisfying \( 2^r = 3 \).
\end{exercise}

\begin{solution}
    Suppose there was a rational number \( r = \tfrac{m}{n} \), which we may assume is in lowest terms with \( n > 0 \), such that \( 2^r = 3 \). This implies that \( 2^m = 3^n \). Since \( n > 0 \implies 3^n > 3 \) and \( 2^m < 2 \) for \( m \leq 0 \), we see that \( m > 0 \). Then the equation \( 2^m = 3^n \) is absurd, since the left-hand side is a positive even integer whereas the right-hand side is positive odd integer.
\end{solution}

\begin{exercise}
\label{ex:3}
    Decide which of the following represent true statements about the nature of sets. For any that are false, provide a specific example where the statement in question does not hold.
    \begin{enumerate}[label = (\alph*)]
        \item If \( A_1 \supseteq A_2 \supseteq A_3 \supseteq A_4 \cdots \) are all sets containing an infinite number of elements, then the intersection \( \bigcap_{n=1}^{\infty} A_n \) is infinite as well.

        \item If \( A_1 \supseteq A_2 \supseteq A_3 \supseteq A_4 \cdots \) are all finite, nonempty sets of real numbers, then the intersection \( \bigcap_{n=1}^{\infty} A_n \) is finite and nonempty.

        \item \( A \cap (B \cup C) = (A \cap B) \cup C \).

        \item \( A \cap (B \cap C) = (A \cap B) \cap C \).

        \item \( A \cap (B \cup C) = (A \cap B) \cup (A \cap C) \).
    \end{enumerate}
\end{exercise}

\begin{solution}
    \begin{enumerate}[label = (\alph*)]
        \item This is false in general. Consider \( A_n = [0, 1/n] \) for \( n \in \mathbf{N} \). Then \( A_1 \supseteq A_2 \supseteq A_3 \supseteq A_4 \cdots \), each \( A_n \) contains infinitely many elements, but the intersection \( \bigcap_{n=1}^{\infty} A_n = \{ 0 \} \) is finite.

        \item This is true. The sequence \( A_1, A_2, A_3, A_4, \ldots \) must be eventually constant, i.e.\ there exists an \( N \in \mathbf{N} \) such that \( A_n = A_N \) for all \( n \geq N \). To see this, assume the negation holds; that is, for each \( N \in \mathbf{N} \), there exists an \( n > N \) such that \( A_n \neq A_N \). Then there is an \( n_1 > 1 \) such that \( A_{n_1} \neq A_1 \). Since \( A_{n_1} \subseteq A_1 \), this implies that \( |A_{n_1}| < |A_1| \iff |A_{n_1}| \leq |A_1| - 1 \). There is then an \( n_2 > n_1 \) such that \( |A_{n_2}| < |A_{n_1}| \leq |A_1| - 1 \iff |A_{n_2}| \leq |A_1| - 2 \). Continuing in this fashion, we obtain a positive integer \( m = n_{|A_1|} \) such that \( |A_m| \leq |A_1| - |A_1| = 0 \), which implies that \( A_m = \emptyset \). This is a contradiction since we assumed that the sets \( A_1, A_2, A_3, A_4, \ldots \) were nonempty.

        So we can be sure that there exists an \( N \in \mathbf{N} \) such that \( A_n = A_N \) for all \( n \geq N \). Given this, \( \bigcap_{n=1}^{\infty} A_n = A_N \), which by assumption is finite and nonempty.

        \item This is false in general. Consider \( A = B = \emptyset \) and \( C = \{ 0 \} \). Then \( A \cap (B \cup C) = \emptyset \) but \( (A \cap B) \cup C = \{ 0 \} \).

        \item This is true, since
        \[
            x \in A \cap (B \cap C) \iff x \in A \text{ and } x \in (B \cap C) \iff x \in A \text{ and } (x \in B \text{ and } x \in C),
        \]
        \[
            x \in (A \cap B) \cap C \iff x \in (A \cap B) \text{ and } x \in C \iff x \in (A \text{ and } x \in B) \text{ and } x \in C.
        \]
        It follows that \( x \in A \cap (B \cap C) \iff x \in (A \cap B) \cap C \) since logical conjunction (``and") is associative.

        \item This is true, since
        \[
            x \in A \cap (B \cup C) \iff x \in A \text{ and } x \in (B \cup C) \iff x \in A \text{ and } (x \in B \text{ or } x \in C)
        \]
        \[
            \iff (x \in A \text{ and } x \in B) \text{ or } (x \in A \text{ and } x \in C) \iff x \in (A \cap B) \text{ or } x \in (A \cap C)
        \]
        \[
            \iff x \in (A \cap B) \cup (A \cap C),
        \]
        where we have used that logical conjunction distributes over logical disjunction (``or").
    \end{enumerate}
\end{solution}

\begin{exercise}
\label{ex:4}
    Produce an infinite collection of sets \( A_1, A_2, A_3, \ldots \) with the property that every \( A_i \) has an infinite number of elements, \( A_i \cap A_j = \emptyset \) for all \( i \neq j \), and \( \bigcup_{i=1}^{\infty} A_i = \mathbf{N} \).
\end{exercise}

\begin{solution}[1]
    Arrange \( \mathbf{N} \) in a grid like so:
    \[
        \begin{matrix}
            1 & 3 & 6 & 10 & \cdots \\
            2 & 5 & 9 & 14 & \cdots \\
            4 & 8 & 13 & 19 & \cdots \\
            7 & 12 & 18 & 25 & \cdots \\
            \vdots & \vdots & \vdots & \vdots & \ddots
        \end{matrix}
    \]
    Now take \( A_i \) to be the set of numbers appearing in the \( i \)th column.
\end{solution}

\begin{solution}[2]
    Assuming some basic knowledge about prime numbers, we can produce another solution as follows. Let \( p_i \) be the \(i\)th prime number (\( p_1 = 2, p_2 = 3, p_3 = 5 \), and so on), and define
    \begin{align*}
        A_1 &= \{ n \in \mathbf{N} : n \text{ is divisible by 2} \} \cup \{ 1 \} \\
        A_2 &= \{ n \in \mathbf{N} : n \text{ is divisible by 3 but not by 2} \} \\
        A_3 &= \{ n \in \mathbf{N} : n \text{ is divisible by 5 but not by 3 or 2} \} \\
        \vdots & \\
        A_i &= \{ n \in \mathbf{N} : n \text{ is divisible by } p_i \text { but not by } p_{i-1}, \ldots, \text{3, or 2} \} \\
        \vdots &
    \end{align*}
    \begin{itemize}
        \item Each \( A_i \) is infinite since \( p_i^k \in A_i \) for each \( k \in \mathbf{N} \).
        \item For \( i < j \), one has \( n \in A_i \) only if \( p_i \) divides \( n \); but in order to have \( n \in A_j \) it is necessary that \( p_i \) does \textit{not} divide \( n \). It follows that \( n \not\in A_j \), so that \( A_i \cap A_j = \emptyset \).
        \item It is clear that \( \bigcup_{i=1}^{\infty} A_i \subseteq \mathbf{N} \). For the reverse inclusion, suppose \( n \in \mathbf{N} \). If \( n = 1 \), then \( n \in A_1 \). If \( n > 1 \), then let \( j \) be the index of the smallest prime appearing in the unique prime factorization of \( n \). It follows that \( n \in A_j \), so that \( n \in \bigcup_{i=1}^{\infty} A_i \).
    \end{itemize}
\end{solution}

\begin{exercise}[De Morgan's Laws]
\label{ex:5}
    Let \( A \) and \( B \) be subsets of \( \mathbf{R} \).
    \begin{enumerate}[label = (\alph*)]
        \item If \( x \in \setcomp{(A \cap B)} \), explain why \( x \in \setcomp{A} \cup \setcomp{B} \). This shows that \( \setcomp{(A \cap B)} \subseteq \setcomp{A} \cup \setcomp{B} \).

        \item Prove the reverse inclusion \( \setcomp{(A \cap B)} \supseteq \setcomp{A} \cup \setcomp{B} \), and conclude that \( \setcomp{(A \cap B)} = \setcomp{A} \cup \setcomp{B} \).

        \item Show \( \setcomp{(A \cup B)} = \setcomp{A} \cap \setcomp{B} \) by demonstrating inclusion both ways.
    \end{enumerate}
\end{exercise}

\begin{solution}
    \begin{enumerate}[label = (\alph*)]
        \item and (b). The negation of (\( x \in A \) and \( x \in B \)) is (\( x \not\in A \) or \( x \not\in B \)).

        \item The negation of (\( x \in A \) or \( x \in B \)) is (\( x \not\in A \) and \( x \not\in B \)).
    \end{enumerate}
\end{solution}

\begin{exercise}
\label{ex:6}
    \begin{enumerate}[label = (\alph*)]
        \item Verify the triangle inequality in the special case where \( a \) and \( b \) have the same sign.

        \item Find an efficient proof for all the cases at once by first demonstrating \( (a + b)^2 \leq (|a| + |b|)^2 \).

        \item Prove \( |a - b| \leq |a - c| + |c - d| + |d - b| \) for all \( a, b, c, \) and \( d \).

        \item Prove \( ||a| - |b|| \leq |a - b| \). (The unremarkable identity \( a = a - b + b \) may be useful.)
    \end{enumerate}
\end{exercise}

\begin{solution}
    \begin{enumerate}[label = (\alph*)]
        \item Suppose that \( a, b \geq 0 \). Then \( a + b \geq 0 \), so
        \[
            |a + b| \leq |a| + |b| \iff a + b \leq a + b,
        \]
        which is certainly true. Now suppose that \( a, b < 0 \). Then \( a + b < 0 \), so
        \[
            |a + b| \leq |a| + |b| \iff -a - b \leq -a - b,
        \]
        which is certainly true.

        \item Observe that
        \[
            2ab \leq 2|ab| \iff a^2 + 2ab + b^2 \leq |a|^2 + 2|a||b| + |b|^2 \iff (a + b)^2 \leq (|a| + |b|)^2.
        \]
        Since \( (a + b)^2 = |a + b|^2 \), we have shown that \( |a + b|^2 \leq (|a| + |b|)^2 \). This implies that \( |a + b| \leq |a| + |b| \), because both \( |a + b| \) and \( |a| + |b| \) are non-negative.

        \item Apply the triangle inequality twice:
        \[
            |a - b| \leq |a - c| + |c - b| \leq |a - c| + |c - d| + |d - b|.
        \]

        \item We have
        \[
            |a| = |a - b + b| \leq |a - b| + |b| \iff |a| - |b| \leq |a - b|,
        \]
        \[
            |b| = |b - a + a| \leq |a - b| + |a| \iff |b| - |a| \leq |a - b|.
        \]
        Since \( ||a| - |b|| \) equals either \( |a| - |b| \) or \( |b| - |a| \), it follows that \( ||a| - |b|| \leq |a - b| \).
    \end{enumerate}
\end{solution}

\begin{exercise}
\label{ex:7}
    Given a function \( f \) and a subset \( A \) of its domain, let \( f(A) \) represent the range of \( f \) over the set \( A \); that is, \( f(A) = \{ f(x) : x \in A \} \).
    \begin{enumerate}[label = (\alph*)]
        \item Let \( f(x) = x^2 \). If \( A = [0, 2] \) (the closed interval \( \{ x \in \mathbf{R} : 0 \leq x \leq 2 \} \)) and \( B = [1, 4] \), find \( f(A) \) and \( f(B) \). Does \( f(A \cap B) = f(A) \cap f(B) \) in this case? Does \( f(A \cup B) = f(A) \cup f(B) \)?

        \item Find two sets \( A \) and \( B \) for which \( f(A \cap B) \neq f(A) \cap f(B) \).

        \item Show that, for an arbitrary function \( g : \mathbf{R} \to \mathbf{R} \), it is always true that \( g(A \cap B) \subseteq g(A) \cap g(B) \) for all sets \( A, B \subseteq \mathbf{R} \).

        \item Form and prove a conjecture about the relationship between \( g(A \cup B) \) and \( g(A) \cup g(B) \) for an arbitrary function \( g \).
    \end{enumerate}
\end{exercise}

\begin{solution}
    \begin{enumerate}[label = (\alph*)]
        \item \( f(A) = [0, 4] \) and \( f(B) = [1, 16] \) since
        \[
            0 \leq x \leq 2 \iff 0 \leq x^2 \leq 4 \quad \text{and} \quad 1 \leq x \leq 4 \iff 1 \leq x^2 \leq 16.
        \]
        Similarly, we have \( f(A \cap B) = f([1, 2]) = [1, 4] \) and \( f(A) \cap f(B) = [0, 4] \cap [1, 16] = [1, 4] \). So in this case we do have \( f(A \cap B) = f(A) \cap f(B) \).
    
        \( f(A \cup B) = f([0, 4]) = [0, 16] \) and \( f(A) \cup f(B) = [0, 4] \cup [1, 16] = [0, 16] \), so we also have \( f(A \cup B) = f(A) \cup f(B) \).
    
        \item Let \( A = \{ -1 \} \) and \( B = \{ 1 \} \). Then \( f(A \cap B) = f(\emptyset) = \emptyset \) but \( f(A) \cap f(B) = \{ 1 \} \cap \{ 1 \} = \{ 1 \} \neq \emptyset \).

        \item We have
        \begin{gather*}
            y \in g(A \cap B) \iff y = f(x) \text{ for some } x \in A \cap B \\
            \implies (y = f(x_1) \text{ for some } x_1 \in A) \text{ and } (y = f(x_2) \text{ for some } x_2 \in B) \\
            \iff y \in g(A) \text{ and } y \in g(B) \iff y \in g(A) \cap g(B).
        \end{gather*}
        Hence \( y \in g(A \cap B) \implies y \in g(A) \cap g(B) \), i.e.\ \( g(A \cap B) \subseteq g(A) \cap g(B) \).

        \item We always have \( g(A \cup B) = g(A) \cup g(B) \); indeed,
        \begin{gather*}
            y \in g(A \cup B) \iff y = f(x) \text{ for some } x \in A \cup B \\
            \iff y = f(x) \text{ for some } x \text{ such that } (x \in A \text{ or } x \in B) \\
            \iff (y = f(x_1) \text{ for some } x_1 \in A) \text{ or } (y = f(x_2) \text{ for some } x_2 \in B) \\
            \iff y \in g(A) \text{ or } y \in g(B) \iff y \in g(A) \cup g(B).
        \end{gather*}
        Hence \( y \in g(A \cup B) \iff y \in g(A) \cup g(B) \), i.e.\ \( g(A \cup B) = g(A) \cup g(B) \).
    \end{enumerate}
\end{solution}

\begin{exercise}
\label{ex:8}
    Here are two important definitions related to a function \( f : A \to B \). The function \( f \) is \textit{one-to-one} (1-1) if \( a_1 \neq a_2 \) in \( A \) implies that \( f(a_1) \neq f(a_2) \) in \( B \). The function \( f \) is \textit{onto} if, given any \( b \in B \), it is possible to find an element \( a \in A \) for which \( f(a) = b \).

    Give an example of each or state that the request is impossible:
    \begin{enumerate}[label = (\alph*)]
        \item \( f : \mathbf{N} \to \mathbf{N} \) that is 1-1 but not onto.

        \item \( f : \mathbf{N} \to \mathbf{N} \) that is onto but not 1-1.

        \item \( f : \mathbf{N} \to \mathbf{Z} \) that is 1-1 and onto.
    \end{enumerate}
\end{exercise}

\begin{solution}
    \begin{enumerate}[label = (\alph*)]
        \item Let \( f(n) = 2n \). Then \( f \) is 1-1 since \( n = m \iff 2n = 2m \), but not onto since the range of \( f \) contains only even numbers.

        \item Let \( f(1) = 1 \) and \( f(n) = n - 1 \) for \( n \geq 2 \). Then \( f(n + 1) = n \) for \( n \in \mathbf{N} \), so \( f \) is onto, but \( f(1) = f(2) = 1 \), so \( f \) is not 1-1.

        \item Let \( f \) be given by
        \[
            f(n) = \begin{cases}
                \frac{n}{2} & \text{if } n \text{ is even}, \\
                -\frac{n-1}{2} & \text{if } n \text{ is odd}.
            \end{cases}
        \]
        \[
            \begin{tikzcd}[column sep = tiny, row sep = small]
                \mathbf{N} : & 1 \arrow[d, leftrightarrow] & 2 \arrow[d, leftrightarrow] & 3 \arrow[d, leftrightarrow] & 4 \arrow[d, leftrightarrow] & 5 \arrow[d, leftrightarrow] & \cdots \\
                \mathbf{Z} : & 0 & 1 & -1 & 2 & -2 & \cdots \\
            \end{tikzcd}
        \]
        To see that \( f \) is 1-1, let \( n \neq m \) be given. If \( n \) and \( m \) are both even, then \( f(n) \neq f(m) \) since \( n \neq m \iff \tfrac{n}{2} \neq \tfrac{m}{2} \); if \( n \) and \( m \) are both odd, then \( f(n) \neq f(m) \) since \( n \neq m \iff -\tfrac{n-1}{2} \neq -\tfrac{m-1}{2} \); and if \( n \) and \( m \) have opposite signs, say \( n \) is even and \( m \) is odd, then \( f(n) \neq f(m) \) since \( f(n) > 0 \) and \( f(m) \leq 0 \). To see that \( f \) is onto, let \( k \in \mathbf{Z} \) be given. If \( k > 0 \), then \( f(2k) = k \), and if \( k \leq 0 \) then \( f(-2k + 1) = k \).
    \end{enumerate}
\end{solution}

\begin{exercise}
\label{ex:9}
    Given a function \( f : D \to \mathbf{R} \) and a subset \( B \subseteq \mathbf{R} \), let \( f^{-1}(B) \) be the set of all points from the domain \( D \) that get mapped into \( B \); that is, \( f^{-1}(B) = \{ x \in D : f(x) \in B \} \). This set is called the \textit{preimage} of \( B \).
    \begin{enumerate}[label = (\alph*)]
        \item Let \( f(x) = x^2 \). If \( A \) is the closed interval \( [0, 4] \) and \( B \) is the closed interval \( [-1, 1] \), find \( f^{-1}(A) \) and \( f^{-1}(B) \). Does \( f^{-1}(A \cap B) = f^{-1}(A) \cap f^{-1}(B) \) in this case? Does \( f^{-1}(A \cup B) = f^{-1}(A) \cup f^{-1}(B) \)?

        \item The good behavior of preimages demonstrated in (a) is completely general. Show that for an arbitrary function \( g : \mathbf{R} \to \mathbf{R} \), it is always true that \( g^{-1}(A \cap B) = g^{-1}(A) \cap g^{-1}(B) \) and \( g^{-1}(A \cup B) = g^{-1}(A) \cup g^{-1}(B) \) for all sets \( A, B \subseteq \mathbf{R} \).
    \end{enumerate}
\end{exercise}

\begin{solution}
    \begin{enumerate}[label = (\alph*)]
        \item We have \( f^{-1}(A) = [-2, 2], f^{-1}(B) = [-1, 1], f^{-1}(A \cap B) = f^{-1}([0, 1]) = [-1, 1], \) and \( f^{-1}(A) \cap f^{-1}(B) = [-1, 1] \). So in this case we do have \( f^{-1}(A \cap B) = f^{-1}(A) \cap f^{-1}(B) \). We also have \( f^{-1}(A \cup B) = f^{-1}([-1, 4]) = [-2, 2] \) and \( f^{-1}(A) \cup f^{-1}(B) = [-2, 2] \), so we also have \( f^{-1}(A \cup B) = f^{-1}(A) \cup f^{-1}(B) \).

        \item Observe that
        \begin{gather*}
            x \in g^{-1}(A \cap B) \iff g(x) \in A \cap B \iff (g(x) \in A) \text{ and } (g(x) \in B) \\
            \iff (x \in g^{-1}(A)) \text{ and } (x \in g^{-1}(B)) \iff x \in g^{-1}(A) \cap g^{-1}(B).
        \end{gather*}
        Similarly,
        \begin{gather*}
            x \in g^{-1}(A \cup B) \iff g(x) \in A \cup B \iff (g(x) \in A) \text{ or } (g(x) \in B) \\
            \iff (x \in g^{-1}(A)) \text{ or } (x \in g^{-1}(B)) \iff x \in g^{-1}(A) \cup g^{-1}(B).
        \end{gather*}
    \end{enumerate}
\end{solution}

\begin{exercise}
\label{ex:10}
    Decide which of the following are true statements. Provide a short justification for those that are valid and a counterexample for those that are not:
    \begin{enumerate}[label = (\alph*)]
        \item Two real numbers satisfy \( a < b \) if and only if \( a < b + \epsilon \) for every \( \epsilon > 0 \).

        \item Two real numbers satisfy \( a < b \) if \( a < b + \epsilon \) for every \( \epsilon > 0 \).

        \item Two real numbers satisfy \( a \leq b \) if and only if \( a < b + \epsilon \) for every \( \epsilon > 0 \).
    \end{enumerate}
\end{exercise}

\begin{solution}
    \begin{enumerate}[label = (\alph*)]
        \item This is false in general; the implication
        \[
            a < b + \epsilon \text{ for every } \epsilon > 0 \implies a < b
        \]
        does not hold. The problem occurs when we consider the case \( a = b \). In that case, for every \( \epsilon > 0 \) one has \( a + \epsilon > a \) but of course \( a < a \) is absurd.

        \item See part (a).

        \item This is true. First, let us prove the implication
        \[
            a \leq b \implies a < b + \epsilon \text{ for every } \epsilon > 0.
        \]
        Let \( \epsilon > 0 \) be given. Then \( a \leq b < b + \epsilon \). Now let us prove the reverse implication by considering the contrapositive statement:
        \[
            a > b \implies a \geq b + \epsilon \text{ for some } \epsilon > 0.
        \]
        Take \( \epsilon = a - b > 0 \). Then \( b + \epsilon = a \leq a \).
    \end{enumerate}
\end{solution}

\begin{exercise}
\label{ex:11}
    Form the logical negation of each claim. One trivial way to do this is to simply add ``It is not the case that..." in front of each assertion. To make this interesting, fashion the negation into a positive statement that avoids using the word ``not" altogether. In each case, make an intuitive guess as to whether the claim or its negation is the true statement.
    \begin{enumerate}[label = (\alph*)]
        \item For all real numbers satisfying \( a < b \), there exists an \( n \in \mathbf{N} \) such that \( a + 1/n < b \).

        \item There exists a real number \( x > 0 \) such that \( x < 1/n \) for all \( n \in \mathbf{N} \).

        \item Between every two distinct real numbers there is a rational number.
    \end{enumerate}
\end{exercise}

\begin{solution}
    \begin{enumerate}[label = (\alph*)]
        \item The negated statement is:
        \begin{center}
            there exist real numbers satisfying \( a < b \) such that \( a + 1/n \geq b \) for all \( n \in \mathbf{N} \).
        \end{center}
        The original statement is true and follows from the \href{https://en.wikipedia.org/wiki/Archimedean_property}{Archimedean property} of \( \mathbf{R} \) (Theorem 1.4.2 of \hyperlink{ua}{[UA]}).

        \item The negated statement is:
        \begin{center}
            for all \( x > 0 \), there exists an \( n \in \mathbf{N} \) such that \( x \geq 1/n \).
        \end{center}
        The negated statement is true and again follows from the Archimedean property of \( \mathbf{R} \).

        \item The negated statement is:
        \begin{center}
            there are two distinct real numbers with no rational number between them.
        \end{center}
        The original statement is true; this is the density of \( \mathbf{Q} \) in \( \mathbf{R} \) (Theorem 1.4.3 of \hyperlink{ua}{[UA]}).
    \end{enumerate}
\end{solution}

\begin{exercise}
\label{ex:12}
    Let \( y_1 = 6 \), and for each \( n \in \mathbf{N} \) define \( y_{n+1} = (2y_n - 6)/3 \).
    \begin{enumerate}[label = (\alph*)]
        \item Use induction to prove that the sequence satisfies \( y_n > -6 \) for all \( n \in \mathbf{N} \).

        \item Use another induction argument to show the sequence \( (y_1, y_2, y_3, \ldots) \) is decreasing.
    \end{enumerate}
\end{exercise}

\begin{solution}
    \begin{enumerate}[label = (\alph*)]
        \item For the base case \( n = 1 \), we have \( y_1 = 6 > -6 \). Suppose that for some \( n \in \mathbf{N} \) we have \( y_n > -6 \). Then observe that
        \[
            y_{n+1} = \tfrac{2}{3} y_n - 2 > \tfrac{2}{3} (-6) - 2 = -6.
        \]
        Hence by induction we have \( y_n > -6 \) for all \( n \in \mathbf{N} \).

        \item We want to show that for all \( n \in \mathbf{N} \) we have \( y_{n+1} \leq y_n \). Since \( y_1 = 6 \) and \( y_2 = 2 \), the statement is true for \( n = 1 \). Suppose that \( y_{n+1} \leq y_n \) for some \( n \in \mathbf{N} \). Then observe that
        \[
            y_{n+2} = \tfrac{2}{3} y_{n+1} - 2 \leq \tfrac{2}{3} y_n - 2 = y_{n+1}.
        \]
        Hence by induction we have \( y_{n+1} \leq y_n \) for all \( n \in \mathbf{N} \).
    \end{enumerate}
\end{solution}

\begin{exercise}
\label{ex:13}
    For this exercise, assume \Cref{ex:5} has been successfully completed.
    \begin{enumerate}[label = (\alph*)]
        \item Show how induction can be used to conclude that
        \[
            \setcomp{(A_1 \cup A_2 \cup \cdots \cup A_n)} = \setcomp{A_1} \cap \setcomp{A_2} \cap \cdots \cap \setcomp{A_n}
        \]
        for any finite \( n \in \mathbf{N} \).

        \item It is tempting to appeal to induction to conclude that
        \[
            \setcomp{\left( \bigcup_{i=1}^{\infty} A_i \right)} = \bigcap_{i=1}^{\infty} \setcomp{A_i},
        \]
        but induction does not apply here. Induction is used to prove that a particular statement holds for every value of \( n \in \mathbf{N} \), but this does not imply the validity of the infinite case. To illustrate this point, find an example of a collection of sets \( B_1, B_2, B_3, \ldots \) where \( \bigcap_{i=1}^n B_i \neq \emptyset \) is true for every \( n \in \mathbf{N} \), but \( \bigcap_{i=1}^{\infty} B_i \neq \emptyset \) fails.

        \item Nevertheless, the infinite version of De Morgan's Law stated in (b) is a valid statement. Provide a proof that does not use induction.
    \end{enumerate}
\end{exercise}

\begin{solution}
    \begin{enumerate}[label = (\alph*)]
        \item We want to prove the statement
        \[
            P(n) : \quad \setcomp{(A_1 \cup A_2 \cup \cdots \cup A_n)} = \setcomp{A_1} \cap \setcomp{A_2} \cap \cdots \cap \setcomp{A_n}
        \]
        for all \( n \in \mathbf{N} \). The truth of \( P(1) \) is clear. Suppose that \( P(n) \) holds for some \( n \in \mathbf{N} \). Then observe that
        \begin{align*}
            \setcomp{(A_1 \cup A_2 \cup \cdots \cup A_n \cup A_{n+1})} &= \setcomp{((A_1 \cup A_2 \cup \cdots \cup A_n) \cup (A_{n+1}))} \tag{union associativity} \\
            &= \setcomp{(A_1 \cup A_2 \cup \cdots \cup A_n)} \cap \setcomp{A_{n+1}} \tag{\Cref{ex:5}} \\
            &= \setcomp{A_1} \cap \setcomp{A_2} \cap \cdots \cap \setcomp{A_n} \cap \setcomp{A_{n+1}}, \tag{induction hypothesis}
        \end{align*}
        which is the statement \( P(n+1) \). Hence by induction \( P(n) \) holds for all \( n \in \mathbf{N} \).

        \item Let \( B_i = (0, 1/i) \). Then \( \bigcap_{i=1}^{n} B_i = (0, 1/n) \), but \( \bigcap_{i=1}^{\infty} B_i = \emptyset \).

        \item We have
        \begin{gather*}
            x \in \setcomp{\left( \bigcup_{i=1}^{\infty} A_i \right)} \iff x \not\in \bigcup_{i=1}^{\infty} A_i \iff x \not\in A_i \text{ for every } i \in \mathbf{N} \iff x \in \bigcap_{i=1}^{\infty} \setcomp{A_i}.
        \end{gather*}
        Hence the infinite version of De Morgan's Law holds.
    \end{enumerate}
\end{solution}

\noindent \hrulefill

\noindent \hypertarget{ua}{\textcolor{blue}{[UA]} Abbott, S. (2015) \textit{Understanding Analysis.} 2nd edn.}

\end{document}
