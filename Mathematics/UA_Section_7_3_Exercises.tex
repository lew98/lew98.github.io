\documentclass[12pt]{article}
\usepackage[utf8]{inputenc}
\usepackage[utf8]{inputenc}
\usepackage{amsmath}
\usepackage{amsthm}
\usepackage{amssymb}
\usepackage{array}
\usepackage{geometry}
\usepackage{amsfonts}
\usepackage{mathrsfs}
\usepackage{bm}
\usepackage{hyperref}
\usepackage{float}
\usepackage[dvipsnames]{xcolor}
\usepackage[inline]{enumitem}
\usepackage{mathtools}
\usepackage{changepage}
\usepackage{graphicx}
\usepackage{systeme}
\usepackage{caption}
\usepackage{subcaption}
\usepackage{lipsum}
\usepackage{tikz}
\usetikzlibrary{matrix, patterns, decorations.pathreplacing, calligraphy}
\usepackage{tikz-cd}
\usepackage[nameinlink]{cleveref}
\geometry{
headheight=15pt,
left=60pt,
right=60pt
}
\setlength{\emergencystretch}{20pt}
\usepackage{fancyhdr}
\pagestyle{fancy}
\fancyhf{}
\lhead{}
\chead{Section 7.3 Exercises}
\rhead{\thepage}
\hypersetup{
    colorlinks=true,
    linkcolor=blue,
    urlcolor=blue
}

\theoremstyle{definition}
\newtheorem*{remark}{Remark}

\newtheoremstyle{exercise}
    {}
    {}
    {}
    {}
    {\bfseries}
    {.}
    { }
    {\thmname{#1}\thmnumber{#2}\thmnote{ (#3)}}
\theoremstyle{exercise}
\newtheorem{exercise}{Exercise 7.3.}

\newtheoremstyle{solution}
    {}
    {}
    {}
    {}
    {\itshape\color{magenta}}
    {.}
    { }
    {\thmname{#1}\thmnote{ #3}}
\theoremstyle{solution}
\newtheorem*{solution}{Solution}

\Crefformat{exercise}{#2Exercise 7.3.#1#3}

\newcommand{\interior}[1]{%
  {\kern0pt#1}^{\mathrm{o}}%
}
\newcommand{\ts}{\textsuperscript}
\newcommand{\setcomp}[1]{#1^{\mathsf{c}}}
\newcommand{\poly}{\mathcal{P}}
\newcommand{\quand}{\quad \text{and} \quad}
\newcommand{\quimplies}{\quad \implies \quad}
\newcommand{\quiff}{\quad \iff \quad}
\newcommand{\upd}{\,\text{d}}
\newcommand{\N}{\mathbf{N}}
\newcommand{\Z}{\mathbf{Z}}
\newcommand{\Q}{\mathbf{Q}}
\newcommand{\I}{\mathbf{I}}
\newcommand{\R}{\mathbf{R}}
\newcommand{\C}{\mathbf{C}}

\DeclarePairedDelimiter\abs{\lvert}{\rvert}
% Swap the definition of \abs* and \norm*, so that \abs
% and \norm resizes the size of the brackets, and the 
% starred version does not.
\makeatletter
\let\oldabs\abs
\def\abs{\@ifstar{\oldabs}{\oldabs*}}
%
\let\oldnorm\norm
\def\norm{\@ifstar{\oldnorm}{\oldnorm*}}
\makeatother

\DeclarePairedDelimiter\paren{(}{)}
\makeatletter
\let\oldparen\paren
\def\paren{\@ifstar{\oldparen}{\oldparen*}}
\makeatother

\DeclarePairedDelimiter\bkt{[}{]}
\makeatletter
\let\oldbkt\bkt
\def\bkt{\@ifstar{\oldbkt}{\oldbkt*}}
\makeatother

\DeclarePairedDelimiter\set{\{}{\}}
\makeatletter
\let\oldset\set
\def\set{\@ifstar{\oldset}{\oldset*}}
\makeatother

\setlist[enumerate,1]{label={(\alph*)}}

\begin{document}

\section{Section 7.3 Exercises}

Exercises with solutions from Section 7.3 of \hyperlink{ua}{[UA]}.

\begin{exercise}
\label{ex:1}
    Consider the function
    \[
        h(x) = \begin{cases}
            1 & \text{for } 0 \leq x < 1 \\
            2 & \text{for } x = 1
        \end{cases}
    \]
    over the interval \( [0, 1] \).
    \begin{enumerate}
        \item Show that \( L(f, P) = 1 \) for every partition \( P \) of \( [0, 1] \).

        \item Construct a partition \( P \) for which \( U(f, P) < 1 + 1/10 \).

        \item Given \( \epsilon > 0 \), construct a partition \( P_{\epsilon} \) for which \( U(f, P_{\epsilon}) < 1 + \epsilon \).
    \end{enumerate}
\end{exercise}

\begin{solution}
    \begin{enumerate}
        \item Let \( P = \{ x_0, x_1, \ldots, x_n \} \) be a partition of \( [0, 1] \). For any \( 1 \leq k \leq n \),
        \[
            m_k = \inf \{ f(x) : x \in [x_{k-1}, x_k] \} = 1
        \]
        and thus
        \[
            L(f, P) = \sum_{k=1}^n m_k \Delta x_k = \sum_{k=1}^n \Delta x_k = 1 - 0 = 1.
        \]

        \item Set \( x_0 = 0, x_1 = \tfrac{19}{20}, x_2 = 1 \), and let \( P \) be the partition \( \{ x_0, x_1, x_2 \} \) of \( [0, 1] \). Since
        \[
            M_1 = \sup \{ f(x) : x \in [x_0, x_1] \} = 1 \quand M_2 = \sup \{ f(x) : x \in [x_1, x_2] \} = 2,
        \]
        we have
        \[
            U(f, P) = M_1 (x_1 - x_0) + M_2 (x_2 - x_1) = 2 - x_1 = 2 - \frac{19}{20} = \frac{21}{20} = 1 + \frac{1}{20} < 1 + \frac{1}{10}.
        \]

        \item Set \( x_0 = 0, x_1 = \max \set{ \tfrac{1}{2}, 1 - \tfrac{\epsilon}{2} }, x_2 = 1 \), and let \( P \) be the partition \( \{ x_0, x_1, x_2 \} \) of \( [0, 1] \). Since
        \[
            M_1 = \sup \{ f(x) : x \in [x_0, x_1] \} = 1 \quand M_2 = \sup \{ f(x) : x \in [x_1, x_2] \} = 2,
        \]
        we have
        \[
            U(f, P) = M_1 (x_1 - x_0) + M_2 (x_2 - x_1) = 2 - x_1 \leq 1 + \frac{\epsilon}{2} < 1 + \epsilon.
        \]
    \end{enumerate}
\end{solution}

\begin{exercise}
\label{ex:2}
    Recall that Thomae's function
    \[
        t(x) = \begin{cases}
            1 & \text{if } x = 0 \\
            1/n & \text{if } x = m/n \in \Q \setminus \{ 0 \} \text{ is in lowest terms with } n > 0 \\
            0 & \text{if } x \not\in \Q
        \end{cases}
    \]
    has a countable set of discontinuities occurring at precisely every rational number. Follow these steps to prove \( t(x) \) is integrable on \( [0, 1] \) with \( \int_0^1 t = 0 \).
    \begin{enumerate}
        \item First argue that \( L(t, P) = 0 \) for any partition \( P \) of \( [0, 1] \).
        
        \item Let \( \epsilon > 0 \), and consider the set of points \( D_{\epsilon/2} = \{ x \in [0, 1] : t(x) \geq \epsilon/2 \} \). How big is \( D_{\epsilon/2} \)?

        \item To complete the argument, explain how to construct a partition \( P_{\epsilon} \) of \( [0, 1] \) so that \( U(t, P_{\epsilon}) < \epsilon \).
    \end{enumerate}
\end{exercise}

\begin{solution}
    \begin{enumerate}
        \item Let \( P = \{ x_0, x_1, \ldots x_n \} \) be an arbitrary partition of \( [0, 1] \). The irrationals are dense in \( \R \), so any subinterval \( [x_{k-1}, x_k] \) contains an irrational number \( y \). Since \( t(y) = 0 \) and \( t(x) \geq 0 \) for all \( x \in [0, 1] \), it follows that \( m_k = 0 \), from which we see that \( L(t, P) = 0 \).

        \item Since \( 0 \leq t(x) \leq 1 \) for all \( x \in [0, 1] \), if \( \tfrac{\epsilon}{2} > 1 \) then \( D_{\epsilon/2} \) is empty. Suppose therefore that \( 0 < \tfrac{\epsilon}{2} \leq 1 \) and let \( N \) be the smallest positive integer such that \( \tfrac{1}{N} < \tfrac{\epsilon}{2} \). It follows that \( D_{\epsilon/2} \) consists precisely of those rational numbers \( \tfrac{m}{n} \in [0, 1] \) (in lowest terms with \( n > 0 \)) with \( 1 \leq n \leq N \), of which there are only finitely many. Thus \( D_{\epsilon/2} \) is finite for any \( \epsilon > 0 \).

        \item Let \( \epsilon > 0 \) be given. If \( D_{\epsilon/2} \) is empty, i.e.\ if \( 0 \leq t(x) < \tfrac{\epsilon}{2} \) for all \( x \in [0, 1] \), then let \( P_{\epsilon} \) be the partition \( \{ 0, 1 \} \) of \( [0, 1] \). For this partition we have
        \[
            U(t, P_{\epsilon}) = \sup \{ t(x) : x \in [0, 1] \} \leq \frac{\epsilon}{2} < \epsilon.
        \]

        Now suppose that \( D_{\epsilon/2} \) is not empty; by part (b) it must be the case that \( D_{\epsilon/2} = \{ y_1, \ldots, y_m \} \) for some \( m \in \N \) and some \( y_1, \ldots, y_m \in [0, 1] \). Let \( P_{\epsilon} = \{ x_0, \ldots, x_n \} \) be the evenly spaced partition of \( [0, 1] \) such that \( \Delta x_k < \tfrac{\epsilon}{2(m+1)} \) for each \( k \in \{ 1, \ldots, n \} \). Decompose the set \( \{ 1, \ldots, n \} \) into the disjoint union \( A \cup \setcomp{A} \), where
        \[
            A = \{ k \in \{ 1, \ldots, n \} : \text{there exists } j \in \{ 1, \ldots, m \} \text{ such that } y_j \in [x_{k-1}, x_k] \},
        \]
        so that
        \[
            U(t, P_{\epsilon}) = \sum_{k=1}^n M_k \Delta x_k = \sum_{k \in A} M_k \Delta x_k + \sum_{k \not\in A} M_k \Delta x_k. \tag{1}
        \]
        Note that \( A \) can contain at most \( m + 1 \) elements and also that \( M_k \leq 1 \) for any \( k \in \{ 1, \ldots, n \} \). It follows that
        \[
            \sum_{k \in A} M_k \Delta x_k < \sum_{k \in A} \frac{\epsilon}{2(m + 1)} \leq (m + 1) \frac{\epsilon}{2(m + 1)} = \frac{\epsilon}{2}. \tag{2}
        \]
        Now suppose that \( k \in \{ 1, \ldots, n \} \) is such that \( k \not\in A \), so that \( f(x) < \tfrac{\epsilon}{2} \) for all \( x \in [x_{k-1}, x_k] \). Then \( M_k \leq \tfrac{\epsilon}{2} \) and it follows that
        \[
            \sum_{k \not\in A} M_k \Delta x_k \leq \frac{\epsilon}{2} \sum_{k \not\in A} \Delta x_k \leq \frac{\epsilon}{2} \sum_{k=1}^n \Delta x_k = \frac{\epsilon}{2}. \tag{3}
        \]
        Combining (1), (2), and (3), we see that \( U(t, P_{\epsilon}) < \tfrac{\epsilon}{2} + \tfrac{\epsilon}{2} = \epsilon \). 

        We have now shown that for any \( \epsilon > 0 \) there exists a partition \( P_{\epsilon} \) of \( [0, 1] \) such that \( U(t, P_{\epsilon}) < \epsilon \). From part (a) we have \( L(t, P_{\epsilon}) = 0 \) and hence \( U(t, P_{\epsilon}) - L(t, P_{\epsilon}) < \epsilon \); it follows that \( t \) is integrable on \( [0, 1] \). Part (a) also shows that
        \[
            \int_0^1 t = L(t) = 0.
        \]
    \end{enumerate}
\end{solution}

\begin{exercise}
\label{ex:3}
    Let
    \[
        f(x) = \begin{cases}
            1 & \text{if } x = 1/n \text{ for some } n \in \N \\
            0 & \text{otherwise}. 
        \end{cases}
    \]
    Show that \( f \) is integrable on \( [0, 1] \) and compute \( \int_0^1 f \).
\end{exercise}

\begin{solution}
    Let \( P = \{ x_0, \ldots, x_n \} \) be an arbitrary partition of \( [0, 1] \). The irrationals are dense in \( \R \), so any subinterval \( [x_{k-1}, x_k] \) contains an irrational number \( y \). Since \( f(y) = 0 \) and \( f(x) \geq 0 \) for all \( x \in [0, 1] \), it follows that \( m_k = 0 \), from which we see that \( L(f, P) = 0 \). Because \( P \) was an arbitrary partition of \( [0, 1] \), we have also shown that \( L(f) = 0 \); once we show that \( f \) is integrable on \( [0, 1] \) it will follow that \( \int_0^1 f = 0 \).

    Let \( \epsilon > 0 \) be given. If \( \tfrac{\epsilon}{2} > 1 \), then \( f(x) \leq \tfrac{\epsilon}{2} \) for all \( x \in [0, 1] \). Take the partition \( P_{\epsilon} = \{ 0, 1 \} \) of \( [0, 1] \), so that
    \[
        U(f, P_{\epsilon}) = \sup \{ f(x) : x \in [0, 1] \} \leq \frac{\epsilon}{2} < \epsilon.
    \]
    As noted above, we have \( L(f, P_{\epsilon}) = 0 \) and thus \( U(f, P_{\epsilon}) - L(f, P_{\epsilon}) < \epsilon \).
    
    Now suppose that \( \tfrac{\epsilon}{2} \leq 1 \). Our argument here is similar to the one we gave in \Cref{ex:2} (c). Choose \( N \in \N \) such that \( \tfrac{1}{N} < \tfrac{\epsilon}{2} \); note that \( N \geq 2 \). Let \( P_{\epsilon} = \{ x_0, x_1, \ldots, x_n \} \) be the partition of \( [0, 1] \) where \( x_0 = 0, x_1 = \tfrac{1}{N}, x_n = 1 \), and \( x_2, \ldots, x_{n-1} \) are chosen to be evenly spaced between \( \tfrac{1}{N} \) and 1, such that \( \Delta x_k < \tfrac{\epsilon}{2N} \) for \( k \geq 2 \). Then
    \[
        U(f, P_{\epsilon}) = \sum_{k=1}^n M_k \Delta x_k = M_1 \Delta x_1 + \sum_{k=2}^n M_k \Delta x_k = \frac{1}{N} + \sum_{k=2}^n M_k \Delta x_k < \frac{\epsilon}{2} + \sum_{k=2}^n M_k \Delta x_k. \tag{1}
    \]
    Decompose the set \( \{ 2, \ldots, n \} \) into the disjoint union \( A \cup \setcomp{A} \), where
    \[
        A = \set{ k \in \{ 2, \ldots, n \} : \text{there exists } j \in \{ 1, \ldots, N - 1 \} \text{ such that } \tfrac{1}{j} \in [x_{k-1}, x_k] },
    \]
    so that
    \[
        \sum_{k=2}^n M_k \Delta x_k = \sum_{k \in A} M_k \Delta x_k + \sum_{k \not\in A} M_k \Delta x_k. \tag{2}
    \]
    Note that \( A \) can contain at most \( N \) elements and also that \( M_k \leq 1 \) for any \( k \in \{ 2, \ldots, n \} \). It follows that
    \[
        \sum_{k \in A} M_k \Delta x_k < \sum_{k \in A} \frac{\epsilon}{2N} \leq N \frac{\epsilon}{2N} = \frac{\epsilon}{2}. \tag{3}
    \]
    Now suppose that \( k \in \{ 2, \ldots, n \} \) is such that \( k \not\in A \), so that \( f(x) = 0 \) for all \( x \in [x_{k-1}, x_k] \). Thus \( M_k = 0 \) and it follows that
    \[
        \sum_{k \not\in A} M_k \Delta x_k = 0. \tag{4}
    \]
    Combining (1), (2), (3), and (4), we see that \( U(f, P_{\epsilon}) < \epsilon \). As noted above, we have \( L(f, P_{\epsilon}) = 0 \) and thus \( U(f, P_{\epsilon}) - L(f, P_{\epsilon}) < \epsilon \).

    We have now shown that for any \( \epsilon > 0 \) there exists a partition \( P_{\epsilon} \) of \( [0, 1] \) such that \( U(f, P_{\epsilon}) - L(f, P_{\epsilon}) < \epsilon \). We may conclude that \( f \) is integrable on \( [0, 1] \).
\end{solution}

\begin{exercise}
\label{ex:4}
    Let \( f \) and \( g \) be functions defined on (possibly different) closed intervals, and assume the range of \( f \) is contained in the domain of \( g \) so that the composition \( g \circ f \) is properly defined.
    \begin{enumerate}
        \item Show, by example, that it is not the case that if \( f \) and \( g \) are integrable, then \( g \circ f \) is integrable.

        Now decide on the validity of each of the following conjectures, supplying a proof or counterexample as appropriate.

        \item If \( f \) is increasing and \( g \) is integrable, then \( g \circ f \) is integrable.

        \item If \( f \) is integrable and \( g \) is increasing, then \( g \circ f \) is integrable.
    \end{enumerate}
\end{exercise}

\begin{solution}
    \begin{enumerate}
        \item Let \( f : [0, 1] \to \R \) be Thomae's function as defined in \Cref{ex:2}; as we showed there, \( f \) is integrable. Let \( g : [0, 1] \to \R \) be given by
        \[
            g(x) = \begin{cases}
                0 & \text{if } x = 0, \\
                1 & \text{if } 0 < x \leq 1.
            \end{cases}
        \]
        Theorem 7.3.2 shows that \( g \) is also integrable. However, note that since \( f(x) = 0 \) for irrational \( x \) and \( 0 < f(x) \leq 1 \) for rational \( x \), the composition \( g \circ f : [0, 1] \to \R \) is nothing but Dirichlet's function, which was shown to be non-integrable in Example 7.3.3.

        \item This is actually false, however the only \href{https://math.stackexchange.com/questions/1833028/if-g-is-riemann-integrable-in-a-closed-interval-and-f-is-a-increasing-functi/1834357#1834357}{counterexample} I know of is quite involved and uses material from Section 7.6.

        \item See part (a) for a counterexample.
    \end{enumerate}
\end{solution}

\begin{exercise}
\label{ex:5}
    Provide an example or give a reason why the request is impossible.
    \begin{enumerate}
        \item A sequence \( (f_n) \to f \) pointwise, where each \( f_n \) has at most a finite number of discontinuities but \( f \) is not integrable.

        \item A sequence \( (g_n) \to g \) uniformly where each \( g_n \) has at most a finite number of discontinuities and \( g \) is not integrable.

        \item A sequence \( (h_n) \to h \) uniformly where each \( h_n \) is not integrable but \( h \) is integrable.
    \end{enumerate}
\end{exercise}

\begin{solution}
    \begin{enumerate}
        \item For each \( n \in \N \) define \( f_n : [0, 1] \to \R \) by
        \[
            f_n(x) = \begin{cases}
                \tfrac{1}{x} & \text{if } x \in \bkt{ \tfrac{1}{n}, 1 }, \\
                0 & \text{if } x \in \left[ 0, \tfrac{1}{n} \right),
            \end{cases}
        \]
        and define \( f : [0, 1] \to \R \) by
        \[
            f(x) = \begin{cases}
                \tfrac{1}{x} & \text{if } x \in \left( 0, 1 \right], \\
                0 & \text{if } x = 0.
            \end{cases}
        \]
        Then \( (f_n) \to f \) pointwise, each \( f_n \) has exactly one discontinuity at \( x = \tfrac{1}{n} \), but \( f \) is not bounded and hence is not integrable.

        \item This is impossible. As discussed after Theorem 7.3.2, each \( g_n \) must be integrable. \href{https://lew98.github.io/Mathematics/UA_Section_7_2_Exercises.pdf}{Exercise 7.2.5} then implies that \( g \) is integrable.

        \item For each \( n \in \N \) define \( h_n : [0, 1] \to \R \) by
        \[
            h_n(x) = \begin{cases}
                \tfrac{1}{n} & \text{if } x \in \Q, \\
                0 & \text{if } x \not\in \Q,
            \end{cases}
        \]
        and let \( h : [0, 1] \to \R \) be identically zero. Then \( h \) is certainly integrable and a small modification of the argument given in Example 7.3.3 shows that each \( h_n \) is not integrable. Furthermore, since
        \[
            \sup \{ \abs{h_n(x) - h(x)} : x \in [0, 1] \} = \frac{1}{n} \to 0,
        \]
        we have uniform convergence \( (h_n) \to h \).
    \end{enumerate}
\end{solution}

\begin{exercise}
\label{ex:6}
    Let \( \{ r_1, r_2, r_3, \ldots \} \) be an enumeration of all the rationals in \( [0, 1] \), and define
    \[
        g_n(x) = \begin{cases}
            1 & \text{if } x = r_n \\
            0 & \text{otherwise}.
        \end{cases}
    \]
    \begin{enumerate}
        \item Is \( G(x) = \sum_{n=1}^{\infty} g_n(x) \) integrable on \( [0, 1] \)?

        \item Is \( F(x) = \sum_{n=1}^{\infty} g_n(x) / n \) integrable on \( [0, 1] \)?
    \end{enumerate}
\end{exercise}

\begin{solution}
    \begin{enumerate}
        \item For irrational \( x \in [0, 1] \), we have \( g_n(x) = 0 \) for all \( n \in \N \) and thus \( G(x) = 0 \). If \( x \in [0, 1] \) is rational, then \( x = r_N \) for some \( N \in \N \). Since \( g_N(r_N) = 1 \) and \( g_n(r_N) = 0 \) for \( n \neq N \), we then have \( G(r_N) = 1 \). Hence \( G \) is in fact Dirichlet's function, which is not integrable (Example 7.3.3).

        \item We claim that \( F \) is integrable on \( [0, 1] \); notice that
        \[
            F(x) = \begin{cases}
                \tfrac{1}{n} & \text{if } x = r_n \in \Q, \\
                0 & \text{if } x \not\in \Q.
            \end{cases}
        \]
        The density of the irrationals in \( \R \) implies that \( L(F, P) = 0 \) for any partition \( P \) of \( [0, 1] \). Let \( \epsilon > 0 \) be given and set
        \[
            D_{\epsilon/2} = \set{ x \in [0, 1] : F(x) \geq \tfrac{\epsilon}{2} }.
        \]
        If \( \tfrac{\epsilon}{2} > 1 \) then \( D_{\epsilon/2} \) is empty, since \( 0 \leq F(x) \leq 1 \) for all \( x \in [0, 1] \). If \( \tfrac{\epsilon}{2} \leq 1 \) then let \( N \) be the smallest positive integer such that \( \tfrac{1}{N} < \tfrac{\epsilon}{2} \); note that \( N \geq 2 \). It follows that
        \[
            D_{\epsilon/2} = \{ r_1, \ldots, r_{N-1} \},
        \]
        so that \( D_{\epsilon/2} \) is finite. We may now argue as in \Cref{ex:2} (c) to obtain a partition \( P_{\epsilon} \) of \( [0, 1] \) such that \( U(F, P_{\epsilon}) < \epsilon \). Since \( L(F, P_{\epsilon}) = 0 \) we then have
        \[
            U(F, P_{\epsilon}) - L(F, P_{\epsilon}) < \epsilon
        \]
        and thus \( F \) is integrable on \( [0, 1] \). Furthermore, \( \int_0^1 F = L(F) = 0 \).
    \end{enumerate}
\end{solution}

\begin{exercise}
\label{ex:7}
    Assume \( f : [a, b] \to \R \) is integrable.
    \begin{enumerate}
        \item Show that if \( g \) satisfies \( g(x) = f(x) \) for all but a finite number of points in \( [a, b] \), then \( g \) is integrable as well.

        \item Find an example to show that \( g \) may fail to be integrable if it differs from \( f \) at a countable number of points.
    \end{enumerate}
\end{exercise}

\begin{solution}
    \begin{enumerate}
        \item Let \( D = \{ x \in [a, b] : f(x) \neq g(x) \} \). If \( D \) is empty then it is clear that \( g \) is integrable, so suppose that \( D = \{ c_1, \ldots, c_d \} \) for some \( d \in \N \) and \( c_1, \ldots, c_d \in [a, b] \). Let \( \epsilon > 0 \) be given. Because \( f \) is integrable, there exists a partition \( Q_{\epsilon} \) of \( [a, b] \) such that \( U(f, Q_{\epsilon}) - L(f, Q_{\epsilon}) < \tfrac{\epsilon}{2} \). The integrability of \( f \) also implies that \( f \) is bounded; since \( g \) differs from \( f \) at only finitely many points, \( g \) must also be bounded, say by \( R > 0 \). Let \( Q'_{\epsilon} = \{ y_0, \ldots, y_l \} \) be the evenly spaced partition of \( [a, b] \) such that
        \[
            \Delta y_k < \frac{\epsilon}{4 R (d + 1)}
        \]
        for each \( k \in \{ 1, \ldots, l \} \), and let \( P_{\epsilon} = Q_{\epsilon} \cup Q'_{\epsilon} = \{ x_0, \ldots, x_n \} \) be the common refinement of \( Q_{\epsilon} \) and \( Q'_{\epsilon} \), so that
        \[
            \Delta x_k < \frac{\epsilon}{4 R (d + 1)}
        \]
        for each \( k \in \{ 1, \ldots, n \} \). Let
        \[
            M^g_k = \sup \{ g(x) : x \in [x_{k-1}, x_k] \} \quand m^g_k = \inf \{ g(x) : x \in [x_{k-1}, x_k] \}
        \]
        for each \( k \in \{ 1, \ldots, n \} \), and define \( M^f_k \) and \( m^f_k \) similarly. Decompose the set \( \{ 1, \ldots, n \} \) into the disjoint union \( A \cup \setcomp{A} \), where
        \[
            A = \{ k \in \{ 1, \ldots, n \} : \text{there exists } j \in \{ 1, \ldots, d \} \text{ such that } c_j \in [x_{k-1}, x_k] \},
        \]
        so that
        \[
            U(g, P_{\epsilon}) - L(g, P_{\epsilon}) = \sum_{k=1}^n (M^g_k - m^g_k) \Delta x_k = \sum_{k \in A} (M^g_k - m^g_k) \Delta x_k + \sum_{k \not\in A} (M^g_k - m^g_k) \Delta x_k. \tag{1}
        \]
        Note that \( A \) can contain at most \( d + 1 \) elements and also that \( M^g_k - m^g_k \leq 2R \) for any \( k \in \{ 1, \ldots, n \} \). It follows that
        \[
            \sum_{k \in A} (M^g_k - m^g_k) \Delta x_k < \sum_{k \in A} 2R \frac{\epsilon}{4 R (d + 1)} \leq (d + 1) \frac{\epsilon}{2(d + 1)} = \frac{\epsilon}{2}. \tag{2}
        \]
        Now suppose that \( k \in \{ 1, \ldots, n \} \) is such that \( k \not\in A \), so that \( f(x) = g(x) \) for all \( x \in [x_{k-1}, x_k] \). It follows that \( M^g_k - m^g_k = M^f_k - m^f_k \) and thus
        \begin{multline*}
            \sum_{k \not\in A} (M^g_k - m^g_k) \Delta x_k = \sum_{k \not\in A} (M^f_k - m^f_k) \Delta x_k \leq \sum_{k=1}^n (M^f_k - m^f_k) \Delta x_k \\
            = U(f, P_{\epsilon}) - L(f, P_{\epsilon}) \leq U(f, Q_{\epsilon}) - L(f, Q_{\epsilon}) < \frac{\epsilon}{2}. \tag{3}
        \end{multline*}
        Combining (1), (2), and (3), we see that \( U(g, P_{\epsilon}) - L(g, P_{\epsilon}) < \epsilon \). Because \( \epsilon > 0 \) was arbitrary, it follows that \( g \) is integrable on \( [a, b] \).

        \item Let \( f : [0, 1] \to \R \) be identically zero, so that \( f \) is certainly integrable, and let \( g : [0, 1] \to \R \) be Dirichlet's function. Then \( g \) differs from \( f \) precisely on the countable set \( \Q \cap [0, 1] \) and yet \( g \) is not integrable.
    \end{enumerate}
\end{solution}

\begin{exercise}
\label{ex:8}
    As in \Cref{ex:6}, let \( \{ r_1, r_2, r_3, \ldots \} \) be an enumeration of the rationals in \( [0, 1] \), but this time define
    \[
        h_n(x) = \begin{cases}
            1 & \text{if } r_n < x \leq 1 \\
            0 & \text{if } 0 \leq x \leq r_n.
        \end{cases}
    \]
    Show \( H(x) = \sum_{n=1}^{\infty} h_n(x) / 2^n \) is integrable on \( [0, 1] \) even though it has discontinuities at every rational point.
\end{exercise}

\begin{solution}
    For a given \( N \in \N \) let \( H_N(x) = \sum_{n=1}^N h_n(x) / 2^n \) and order the rationals \( \{ r_1, \ldots, r_N \} \) as \( 0 \leq r_{i_1} < \cdots < r_{i_N} \leq 1 \). Then
    \[
        H_N(x) = \begin{cases}
            0 & \text{if } x \in [0, r_{i_1}], \\
            \tfrac{1}{2} & \text{if } x \in (r_{i_1}, r_{i_2}], \\
            \tfrac{3}{4} & \text{if } x \in (r_{i_2}, r_{i_3}], \\
            \, \vdots & \qquad \vdots \\
            1 - \tfrac{1}{2^N} & \text{if } x \in (r_{i_N}, 1].
        \end{cases}
    \]
    Thus \( H_N \) is piecewise-constant on \( [0, 1] \). It is straightforward to argue that such functions are integrable (this is implied by Theorem 7.4.1). Now observe that
    \[
        \abs{ \frac{h_n(x)}{2^n} } \leq \frac{1}{2^n}
    \]
    for each \( n \in \N \). Since the series \( \sum_{n=1}^{\infty} 2^{-n} \) is a convergent geometric series, the Weierstrass M-Test (Corollary 6.4.5) implies that \( H_N \) converges uniformly to \( H \); it follows from \href{https://lew98.github.io/Mathematics/UA_Section_7_2_Exercises.pdf}{Exercise 7.2.5} that \( H \) is integrable on \( [0, 1] \).
\end{solution}

\begin{exercise}[Content Zero]
\label{ex:9}
    A set \( A \subseteq [a, b] \) has \textit{content zero} if for every \( \epsilon > 0 \) there exists a finite collection of open intervals \( \{ O_1, O_2, \ldots, O_N \} \) that contain \( A \) in their union and whose lengths sum to \( \epsilon \) or less. Using \( \abs{O_n} \) to refer to the length of each interval, we have
    \[
        A \subseteq \bigcup_{n=1}^N O_n \quand \sum_{n=1}^N \abs{O_n} \leq \epsilon.
    \]
    \begin{enumerate}
        \item Let \( f \) be bounded on \( [a, b] \). Show that if the set of discontinuous points of \( f \) has content zero, then \( f \) is integrable.
        
        \item Show that any finite set has content zero.

        \item Content zero sets do not have to be finite. They do not have to be countable. Show that the Cantor set \( C \) defined in Section 3.1 has content zero.

        \item Prove that
        \[
            h(x) = \begin{cases}
                1 & \text{if } x \in C \\
                0 & \text{if } x \not\in C,
            \end{cases}
        \]
        is integrable, and find the value of the integral.
    \end{enumerate}
\end{exercise}

\begin{solution}
    \begin{enumerate}
        \item Suppose \( f \) is bounded by \( R > 0 \) on \( [a, b] \) and let \( \epsilon > 0 \) be given. Because the set of discontinuous points of \( f \) has content zero, we can choose a partition \( Q \) of \( [a, b] \) such that the discontinuities of \( f \) are contained in the interiors of subintervals whose total length is strictly less than \( \tfrac{\epsilon}{4R} \). Letting \( K \) be the union of the remaining subintervals, we see that \( f \) is continuous on \( K \) and also that \( K \) is compact, being a finite union of closed and bounded intervals. Thus \( f \) is uniformly continuous on \( K \) and, as in the proof of Theorem 7.2.9, we may refine the partition \( Q \), subdividing \( K \) as necessary, to obtain a partition \( P = \{ x_0, \ldots, x_n \} \) of \( [a, b] \) such that the indices \( \{ 1, \ldots, n \} \) can be expressed as the disjoint union \( A \cup B \), where:
        \begin{enumerate}[label=(\roman*)]
            \item \( f \) is continuous on \( \bigcup_{k \in A} [x_{k-1}, x_k] \) and \( M_k - m_k < \tfrac{\epsilon}{2 (b - a)} \) for \( k \in A \);

            \item the discontinuities of \( f \) are contained inside \( \bigcup_{k \in B} (x_{k-1}, x_k) \) and \( \sum_{k \in B} \Delta x_k < \tfrac{\epsilon}{4R} \).
        \end{enumerate}
        It follows that
        \begin{align*}
            U(f, P) - L(f, P) &= \sum_{k=1}^n (M_k - m_k) \Delta x_k \\
            &= \sum_{k \in A} (M_k - m_k) \Delta x_k + \sum_{k \in B} (M_k - m_k) \Delta x_k \\
            &< \frac{\epsilon}{2 (b - a)} \sum_{k \in A} \Delta x_k + 2R \sum_{k \in B} \Delta x_k \\
            &< \frac{\epsilon}{2} + \frac{\epsilon}{2} \\
            &= \epsilon.
        \end{align*}
        Thus \( f \) is integrable on \( [a, b] \).

        \item Let \( A \subseteq \R \) be finite and let \( \epsilon > 0 \) be given. If \( A \) is empty then the open interval \( \paren{ -\tfrac{\epsilon}{2}, \tfrac{\epsilon}{2} } \) suffices to show that \( A \) has content zero. Suppose therefore that \( A \) is not empty, say \( A = \{ x_1, \ldots, x_N \} \). For each \( 1 \leq n \leq N \), let
        \[
            O_n = \paren{ x_n - \frac{\epsilon}{2N}, x_n + \frac{\epsilon}{2N} }.
        \]
        Then \( A \subseteq \bigcup_{n=1}^N O_n \) and
        \[
            \sum_{n=1}^N \abs{O_n} = \sum_{n=1}^N \frac{\epsilon}{N} = \epsilon.
        \]
        Thus \( A \) has content zero.

        \item Recall from Section 3.1 that the Cantor set \( C \) is defined as the intersection \( C = \bigcap_{n=0}^{\infty} C_n \), where each \( C_n \) consists of \( 2^n \) closed intervals each of length \( 3^{-n} \) and such that
        \[
            \cdots \subseteq C_2 \subseteq C_1 \subseteq C_0 = [0, 1].
        \]
        Let \( \epsilon > 0 \) be given and choose \( N \in \N \) such that
        \[
            \paren{ \frac{2}{3} }^N + \paren{ \frac{1}{10} }^N < \epsilon.
        \]
        The set \( C_N \) consists of \( 2^N \) closed intervals each of length \( 3^{-N} \); suppose these intervals are \( [x_k, y_k] \) for \( 1 \leq k \leq 2^N \), so that \( y_k - x_k = 3^{-N} \). For each \( 1 \leq k \leq 2^N \), let
        \[
            O_k = \paren{ x_k - \frac{1}{2^{N+1} 10^N}, y_k + \frac{1}{2^{N+1} 10^N} },
        \]
        so that \( [x_k, y_k] \subseteq O_k \) and
        \[
            \abs{O_k} = \frac{1}{3^N} + \frac{1}{2^N 10^N}.
        \]
        Then
        \begin{multline*}
            C = \bigcap_{n=0}^{\infty} C_n \subseteq C_N = \bigcup_{k=1}^{2^N} [x_k, y_k] \subseteq \bigcup_{k=1}^{2^N} O_k \\
            \text{and} \quad \sum_{k=1}^{2^N} \abs{O_k} = \sum_{k=1}^{2^N} \paren{ \frac{1}{3^N} + \frac{1}{2^N 10^N} } = \paren{ \frac{2}{3} }^N + \paren{ \frac{1}{10} }^N < \epsilon.
        \end{multline*}
        Thus \( C \) has content zero.

        \item Let
        \[
            D_h = \{ x \in \R : h \text{ is not continuous at } x \}.
        \]
        We claim that \( D_h = C \). First, suppose that \( x \not\in C \). Since \( C \) is closed, the complement of \( C \) is open and so there exists some \( \delta > 0 \) such that \( (x - \delta, x + \delta) \subseteq \setcomp{C} \). Thus \( h \) is constant on the proper interval \( (x - \delta, x + \delta) \); it follows that \( h \) is continuous at \( x \). Now suppose that \( x \in C \). To show that \( h \) is not continuous at \( x \), it will suffice to show that for any \( \delta > 0 \) there exists some \( y \in (x - \delta, x + \delta) \) such that \( y \not\in C \). The existence of some \( \delta \) such that this does not hold implies that \( C \) contains a proper interval. However, \( C \) cannot contain any proper intervals since it is totally disconnected (\href{https://lew98.github.io/Mathematics/UA_Section_3_4_Exercises.pdf}{Exercise 3.4.8}). Thus \( h \) is not continuous at \( x \) and our claim follows.

        Abbott does not specify an interval to integrate \( h \) over, but in fact \( h \) is integrable over any interval \( [a, b] \) for \( a < b \). Let \( g : [a, b] \to \R \) be the restriction of \( h \) to \( [a, b] \). Then
        \[
            D_g = \{ x \in [a, b] : g \text{ is not continuous at } x \} = D_h \cap [a, b] = C \cap [a, b].
        \]
        It is straightforward to verify that if a set has content zero, then the intersection of that set with any other set also has content zero. Thus, by part (c), \( D_g \) has content zero and it follows from part (a) that \( g \) is integrable. To calculate the integral of \( g \), let \( P \) be any partition of \( [a, b] \). As we noted before, \( C \) does not contain any proper intervals. It follows that any subinterval \( [x_{k-1}, x_k] \) of the partition \( P \) contains some \( x \not\in C \), whence \( g(x) = 0 \). Thus \( L(g, P) = 0 \) and, because \( P \) was an arbitrary partition of \( [a, b] \), it follows that
        \[
            \int_a^b g = L(g) = 0.
        \]
    \end{enumerate}
\end{solution}

\noindent \hrulefill

\noindent \hypertarget{ua}{\textcolor{blue}{[UA]} Abbott, S. (2015) \textit{Understanding Analysis.} 2\ts{nd} edition.}

\end{document}