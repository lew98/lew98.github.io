\documentclass[12pt]{article}
\usepackage[utf8]{inputenc}
\usepackage[utf8]{inputenc}
\usepackage{amsmath}
\usepackage{amsthm}
\usepackage{amssymb}
\usepackage{geometry}
\usepackage{amsfonts}
\usepackage{mathrsfs}
\usepackage{bm}
\usepackage{hyperref}
\usepackage{float}
\usepackage[dvipsnames]{xcolor}
\usepackage[inline]{enumitem}
\usepackage{mathtools}
\usepackage{changepage}
\usepackage{graphicx}
\usepackage{caption}
\usepackage{subcaption}
\usepackage{lipsum}
\usepackage{tikz}
\usetikzlibrary{matrix, patterns, decorations.pathreplacing, calligraphy}
\usepackage{tikz-cd}
\usepackage[nameinlink]{cleveref}
\geometry{
headheight=15pt,
left=60pt,
right=60pt
}
\setlength{\emergencystretch}{20pt}
\usepackage{fancyhdr}
\pagestyle{fancy}
\fancyhf{}
\lhead{}
\chead{Section 6.3 Exercises}
\rhead{\thepage}
\hypersetup{
    colorlinks=true,
    linkcolor=blue,
    urlcolor=blue
}

\theoremstyle{definition}
\newtheorem*{remark}{Remark}

\newtheoremstyle{exercise}
    {}
    {}
    {}
    {}
    {\bfseries}
    {.}
    { }
    {\thmname{#1}\thmnumber{#2}\thmnote{ (#3)}}
\theoremstyle{exercise}
\newtheorem{exercise}{Exercise 6.3.}

\newtheoremstyle{solution}
    {}
    {}
    {}
    {}
    {\itshape\color{magenta}}
    {.}
    { }
    {\thmname{#1}\thmnote{ #3}}
\theoremstyle{solution}
\newtheorem*{solution}{Solution}

\Crefformat{exercise}{#2Exercise 6.3.#1#3}

\newcommand{\interior}[1]{%
  {\kern0pt#1}^{\mathrm{o}}%
}
\newcommand{\ts}{\textsuperscript}
\newcommand{\setcomp}[1]{#1^{\mathsf{c}}}
\newcommand{\quand}{\quad \text{and} \quad}
\newcommand{\N}{\mathbf{N}}
\newcommand{\Z}{\mathbf{Z}}
\newcommand{\Q}{\mathbf{Q}}
\newcommand{\I}{\mathbf{I}}
\newcommand{\R}{\mathbf{R}}
\newcommand{\C}{\mathbf{C}}

\DeclarePairedDelimiter\abs{\lvert}{\rvert}
% Swap the definition of \abs* and \norm*, so that \abs
% and \norm resizes the size of the brackets, and the 
% starred version does not.
\makeatletter
\let\oldabs\abs
\def\abs{\@ifstar{\oldabs}{\oldabs*}}
%
\let\oldnorm\norm
\def\norm{\@ifstar{\oldnorm}{\oldnorm*}}
\makeatother

\DeclarePairedDelimiter\paren{(}{)}
\makeatletter
\let\oldparen\paren
\def\paren{\@ifstar{\oldparen}{\oldparen*}}
\makeatother

\DeclarePairedDelimiter\bkt{[}{]}
\makeatletter
\let\oldbkt\bkt
\def\bkt{\@ifstar{\oldbkt}{\oldbkt*}}
\makeatother

\DeclarePairedDelimiter\set{\{}{\}}
\makeatletter
\let\oldset\set
\def\set{\@ifstar{\oldset}{\oldset*}}
\makeatother

\setlist[enumerate,1]{label={(\alph*)}}

\begin{document}

\section{Section 6.3 Exercises}

Exercises with solutions from Section 6.3 of \hyperlink{ua}{[UA]}.

\begin{exercise}
\label{ex:1}
    Consider the sequence of functions defined by
    \[
        g_n(x) = \frac{x^n}{n}.  
    \]
    \begin{enumerate}
        \item Show \( (g_n) \) converges uniformly on \( [0, 1] \) and find \( g = \lim g_n \). Show that \( g \) is differentiable and compute \( g'(x) \) for all \( x \in [0, 1] \).

        \item Now, show that \( (g_n') \) converges on \( [0, 1] \). Is the convergence uniform? Set \( h = \lim g_n' \) and compare \( h \) and \( g' \). Are they the same?
    \end{enumerate}
\end{exercise}

\begin{solution}
    \begin{enumerate}
        \item The limit function \( \lim g_n = g : [0, 1] \to \R \) is given by \( g(x) = 0 \). Note that for any \( x \in [0, 1] \) we have
        \[
            \abs{g_n(x) - g(x)} = \frac{x^n}{n} \leq \frac{1}{n}.
        \]
        Since this bound converges to zero and does not depend on \( x \), the convergence \( g_n \to g \) is uniform on \( [0, 1] \). Evidently \( g \) is differentiable on \( [0, 1] \) and satisfies \( g'(x) = 0 \) for all \( x \in [0, 1] \).

        \item The sequence \( (g_n') \) is given by \( g_n'(x) = x^{n-1} \) for \( x \in [0, 1] \). This sequence converges pointwise to the function \( h : [0, 1] \to \R \) given by
        \[
            h(x) = \begin{cases}
                0 & \text{if } 0 \leq x < 1, \\
                1 & \text{if } x = 1.
            \end{cases}
        \]
        The convergence cannot be uniform since each \( g_n' \) is continuous at 1 but \( h \) is not. Note that \( h \neq g' \); this gives an alternative proof for showing that the convergence \( g_n' \to h \) is not uniform, as uniform convergence \( g_n' \to h \) would imply that \( g' = h \) by Theorem 6.3.1/6.3.3.
    \end{enumerate}
\end{solution}

\begin{exercise}
\label{ex:2}
    Consider the sequence of functions
    \[
        h_n(x) = \sqrt{x^2 + \frac{1}{n}}.
    \]
    \begin{enumerate}
        \item Compute the pointwise limit of \( (h_n) \) and then prove that the convergence is uniform on \( \R \).

        \item Note that each \( h_n \) is differentiable. Show \( g(x) = \lim h_n'(x) \) exists for all \( x \), and explain how we can be certain that the convergence is \textit{not} uniform on any neighborhood of zero.
    \end{enumerate}
\end{exercise}

\begin{solution}
    \begin{enumerate}
        \item The pointwise limit is the function \( h : \R \to \R \) given by \( h(x) = \sqrt{x^2} = \abs{x} \). Note that for any \( x \in \R \) we have
        \[
            \abs{h_n(x) - h(x)} = \sqrt{x^2 + n^{-1}} - \sqrt{x^2} = \frac{n^{-1}}{\sqrt{x^2 + n^{-1}} + \sqrt{x^2}} \leq \frac{n^{-1}}{n^{-1/2}} = \frac{1}{\sqrt{n}}.
        \]
        Since this bound converges to zero and does not depend on \( x \), we see that the convergence \( h_n \to h \) is uniform on \( \R \).

        \item Note that \( h_n' : \R \to \R \) is given by
        \[
            h_n'(x) = \frac{x}{\sqrt{x^2 + n^{-1}}}.
        \]
        This sequence converges pointwise to the function \( g : \R \to \R \) given by
        \[
            g(x) = \begin{cases}
                -1 & \text{if } x < 0, \\
                0 & \text{if } x = 0, \\
                1 & \text{if } x > 0.
            \end{cases}
        \]
        The convergence \( h_n' \to g \) cannot be uniform on any neighbourhood of zero since each \( h_n' \) is continuous at zero but \( g \) is not. Alternatively, if the convergence \( h_n' \to g \) was uniform, then Theorem 6.3.1/6.3.3 would imply that \( h \) was differentiable at zero; but \( h \) fails to be differentiable precisely at zero.
    \end{enumerate}
\end{solution}

\begin{exercise}
\label{ex:3}
    Consider the sequence of functions
    \[
        f_n(x) = \frac{x}{1 + nx^2}.
    \]
    \begin{enumerate}
        \item Find the points on \( \R \) where each \( f_n(x) \) attains its maximum and minimum value. Use this to prove \( (f_n) \) converges uniformly on \( \R \). What is the limit function?

        \item Let \( f = \lim f_n \). Compute \( f_n'(x) \) and find all the values of \( x \) for which \( f'(x) = \lim f_n'(x) \).
    \end{enumerate}
\end{exercise}

\begin{solution}
    \begin{enumerate}
        \item From the observation
        \[
            \frac{1}{2 \sqrt{n}} - \frac{x}{1 + nx^2} = \frac{nx^2 - 2 \sqrt{n} x + 1}{2 \sqrt{n} (1 + nx^2)} = \frac{(\sqrt{n} x - 1)^2}{2 \sqrt{n} (1 + nx^2)} \geq 0
        \]
        we can see that \( 0 \leq f_n(x) \leq \tfrac{1}{2 \sqrt{n}} \) for all \( x \geq 0 \) and also that \( f_n(x) = \tfrac{1}{2 \sqrt{n}} \) precisely when \( x = \tfrac{1}{\sqrt{n}} \). Combining this with the fact that each \( f_n \) is an odd function, we see that
        \[
            -\frac{1}{2 \sqrt{n}} \leq f_n(x) \leq \frac{1}{2 \sqrt{n}}
        \]
        for all \( x \in \R \) and furthermore that
        \[
            f_n(x) = -\tfrac{1}{2 \sqrt{n}} \iff x = -\tfrac{1}{\sqrt{n}} \quand f_n(x) = \tfrac{1}{2 \sqrt{n}} \iff x = \tfrac{1}{\sqrt{n}}.
        \]
        The bound \( \abs{f_n(x)} \leq \tfrac{1}{2 \sqrt{n}} \) converges to zero and does not depend on \( x \), demonstrating that \( f_n \) converges uniformly to the zero function.

        \item The quotient rule gives us
        \[
            f_n'(x) = \frac{1 - nx^2}{(1 + nx^2)^2}.
        \]
        For \( x \neq 0 \), we have
        \[
            f_n'(x) = \frac{\frac{1}{n^2 x^4} - \frac{1}{n x^2}}{\paren{ \frac{1}{nx^2} + 1 }^2} \to 0 \text{ as } n \to \infty,
        \]
        and for \( x = 0 \) we have \( f_n'(0) = 1 \). In part (a) we showed that \( \lim f_n = f : \R \to \R \) was given by \( f(x) = 0 \). Thus \( f'(x) = \lim f_n'(x) = 0 \) for all \( x \neq 0 \), and \( f'(0) = 0 \neq 1 = \lim f_n'(0) \).
    \end{enumerate}
\end{solution}

\begin{exercise}
\label{ex:4}
    Let
    \[
        h_n(x) = \frac{\sin(nx)}{\sqrt{n}}.
    \]
    Show that \( h_n \to 0 \) uniformly on \( \R \) but that the sequence of derivatives \( (h_n') \) diverges for every \( x \in \R \).
\end{exercise}

\begin{solution}
    Observe that
    \[
        \abs{h_n(x)} \leq \frac{1}{\sqrt{n}}
    \]
    for any \( x \in \R \). Since this bound converges to zero and does not depend on \( x \), we see that \( h_n \to 0 \) uniformly on \( \R \). The sequence of derivatives \( (h_n') \) is given by
    \[
        a_n := h_n'(x) = \sqrt{n} \cos(nx).
    \]
    We claim that \( (a_n) \) does not converge for any \( x \in \R \); to see this, we will consider three cases.
    \begin{description}
        \item[Case 1.] Suppose \( x = k \pi \), where \( k \) is an even integer. In this case, we have \( a_n = \sqrt{n} \), which clearly diverges.

        \item[Case 2.] Suppose \( x = k \pi \), where \( k \) is an odd integer.
        In this case, we have \( a_n = (-1)^n \sqrt{n} \), which clearly diverges.

        \item[Case 3.] Suppose \( x \) is not of the form \( k \pi \) for any integer \( k \) and suppose by way of contradiction that \( a_n \to L \) for some \( L \in \R \). It follows that
        \[
            \frac{a_n}{\sqrt{n}} = \cos(nx) \to 0,
        \]
        which also implies that \( \cos((n+1)x) \to 0 \). Consider the trigonometric identity
        \[
            \sin(nx) = \frac{\cos(nx) \cos(x) - \cos((n+1)x)}{\sin(x)};
        \]
        the fact that \( x \neq k \pi \) for any integer \( k \) means we are not dividing by zero here. Since both \( \cos(nx) \to 0 \) and \( \cos((n+1)x) \to 0 \), we see that \( \sin(nx) \to 0 \), which in turn implies that
        \[
            \sin^2(nx) + \cos^2(nx) \to 0.
        \]
        This is a contradiction since \( \sin^2(nx) + \cos^2(nx) = 1 \) for all \( n \in \N \).
    \end{description}
\end{solution}

\begin{exercise}
\label{ex:5}
    Let
    \[
        g_n(x) = \frac{nx + x^2}{2n},
    \]
    and set \( g(x) = \lim g_n(x) \). Show that \( g \) is differentiable in two ways:
    \begin{enumerate}
        \item Compute \( g(x) \) by algebraically taking the limit as \( n \to \infty \) and then find \( g'(x) \).

        \item Compute \( g_n'(x) \) for each \( n \in \N \) and show that the sequence of derivatives \( (g_n') \) converges uniformly on every interval \( [-M, M] \). Use Theorem 6.3.3 to conclude \( g'(x) = \lim g_n'(x) \).

        \item Repeat parts (a) and (b) for the sequence \( f_n(x) = (nx^2 + 1)/(2n + x) \).
    \end{enumerate}
\end{exercise}

\begin{solution}
    \begin{enumerate}
        \item For a fixed \( x \in \R \) we have
        \[
            g_n(x) = \frac{x}{2} + \frac{x^2}{2n} \to \frac{x}{2} \text{ as } n \to \infty.  
        \]
        It follows that \( g'(x) = \tfrac{1}{2} \) for any \( x \in \R \).

        \item The sequence of derivatives \( (g_n') \) is given by
        \[
            g_n'(x) = \frac{1}{2} + \frac{x}{n}.
        \]
        For \( x \in [-M, M] \) we have
        \[
            \abs{g_n'(x) - \frac{1}{2}} = \frac{\abs{x}}{n} \leq \frac{M}{n}.
        \]
        Since this bound converges to zero as \( n \to \infty \) and does not depend on \( x \), we see that \( g_n' \to \tfrac{1}{2} \) uniformly on any interval of the form \( [-M, M] \). Observe that \( 0 \in [-M, M] \) and \( g_n(0) = 0 \) is convergent. Theorem 6.3.3 implies that \( g_n \to g \) uniformly on \( [-M, M] \) and furthermore that \( g'(x) = \lim g_n'(x) = \tfrac{1}{2} \) for any \( x \in [-M, M] \). By taking \( M \) sufficiently large, this shows that \( g'(x) = \tfrac{1}{2} \) for all \( x \in \R \).

        \item The sequence \( (f_n) \) is given by
        \[
            f_n(x) = \frac{nx^2 + 1}{2n + x}.
        \]
        (Strictly speaking this is only defined on \( \R \setminus \{ -2n \} \), but since we are only interested in the limit as \( n \to \infty \), this isn't a problem; eventually the sequence is defined on any interval of the form \( [-M, M] \).)
        
        Note that
        \[
            f_n(x) = \frac{x^2 + \tfrac{1}{n}}{2 + \tfrac{x}{n}} \to \frac{x^2}{2} \text{ as } n \to \infty,
        \]
        so that the pointwise limit function is \( f(x) = \tfrac{x^2}{2} \), which satisfies \( f'(x) = x \).

        The sequence of derivatives \( (f_n') \) is given by
        \[
            f_n'(x) = \frac{nx^2 + 4n^2x - 1}{x^2 + 4nx + 4n^2} = \frac{\tfrac{x^2}{n} + 4x - \tfrac{1}{n^2}}{\tfrac{x^2}{n^2} + \tfrac{4x}{n} + 4} \to x \text{ as } n \to \infty.
        \]
        For any \( x \in [-M, M] \), observe that
        \[
            \abs{f_n'(x) - x} = \abs{\frac{x^3 + 3nx^2 + 1}{4n^2 + 4nx + x^2}} \leq \frac{M^3 + 3M^2 + 1}{\abs{x + n}} \leq \frac{M^3 + 3M^2 + 1}{n - M}
        \]
        provided \( n > M \). Since this bound converges to zero as \( n \to \infty \) and does not depend on \( x \), we see that \( f_n' \to x \) uniformly on \( [-M, M] \). Observe that \( 0 \in [-M, M] \) and \( f_n(0) = \tfrac{1}{2n} \to 0 \) as \( n \to \infty \). Theorem 6.3.3 implies that \( f_n \to f \) uniformly on \( [-M, M] \) and furthermore that \( f'(x) = \lim f_n'(x) = x \) for any \( x \in [-M, M] \). By taking \( M \) sufficiently large, this shows that \( f'(x) = x \) for all \( x \in \R \).
    \end{enumerate}
\end{solution}

\begin{exercise}
\label{ex:6}
    Provide an example or explain why the request is impossible. Let's take the domain of the functions to be all of \( \R \).
    \begin{enumerate}
        \item A sequence \( (f_n) \) of nowhere differentiable functions with \( f_n \to f \) uniformly and \( f \) everywhere differentiable.

        \item A sequence \( (f_n) \) of differentiable functions such that \( (f_n') \) converges uniformly but the original sequence \( (f_n) \) does not converge for any \( x \in \R \).

        \item A sequence \( (f_n) \) of differentiable functions such that both \( (f_n) \) and \( (f_n') \) converge uniformly but \( f = \lim f_n \) is not differentiable at some point.
    \end{enumerate}
\end{exercise}

\begin{solution}
    \begin{enumerate}
        \item Define a sequence \( (f_n : \R \to \R) \) by
        \[
            f_n(x) = \begin{cases}
                \tfrac{1}{n} & \text{if } x \in \Q, \\
                0 & \text{if } x \not\in \Q.
            \end{cases}
        \]
        Then \( \abs{f_n(x)} \leq \tfrac{1}{n} \) for any \( x \in \R \), demonstrating that \( f_n \to 0 \) uniformly on \( \R \). Clearly the zero function is differentiable everywhere, but each \( f_n \) is nowhere continuous and hence nowhere differentiable.

        \item Define a sequence \( (f_n : \R \to \R) \) by
        \[
            f_n(x) = n
        \]
        for all \( x \in \R \). Then each \( f_n \) is differentiable and the sequence \( (f_n') \) is given by \( f_n'(x) = 0 \), which converges uniformly to the zero function. However, \( (f_n(x)) \) is divergent for every \( x \in \R \).

        \item This is impossible. Any point \( x \in \R \) is contained in some interval of the form \( [-M, M] \); applying Theorem 6.3.3 to this interval shows that \( f \) is differentiable at \( x \).
    \end{enumerate}
\end{solution}

\begin{exercise}
\label{ex:7}
    Use the Mean Value Theorem to supply a proof for Theorem 6.3.2. To get started, observe that the triangle inequality implies that, for any \( x \in [a, b] \) and \( m, n \in \N \),
    \[
        \abs{f_n(x) - f_m(x)} \leq \abs{(f_n(x) - f_m(x)) - (f_n(x_0) - f_m(x_0))} + \abs{f_n(x_0) - f_m(x_0)}.
    \]
\end{exercise}

\begin{solution}
    Let \( \epsilon > 0 \) be given. Since the sequence \( (f_n(x_0)) \) is convergent, there exists an \( N_1 \in \N \) such that
    \[
        n, m \geq N_1 \implies \abs{f_n(x_0) - f_m(x_0)} < \frac{\epsilon}{2},
    \]
    and since the sequence \( (f_n') \) converges uniformly on \( [a, b] \), there exists an \( N_2 \in \N \) such that
    \[
        x \in [a, b] \text{ and } n, m \geq N_2 \implies \abs{f_n'(x) - f_m'(x)} < \frac{\epsilon}{2(b-a)}.
    \]
    Set \( N = \max \{ N_1, N_2 \} \) and suppose that \( n, m \geq N \) and \( x \in (x_0, b] \) (the argument is easily modified if \( x \in [a, x_0) \)). Note that \( f_n - f_m \) is differentiable on the interval \( [x_0, x] \); the Mean Value Theorem then implies that there is some \( c \in (x_0, x) \) such that
    \[
        \abs{x - x_0} \abs{f_n'(c) - f_m'(c)} = \abs{(f_n(x) - f_m(x)) - (f_n(x_0) - f_m(x_0))}.
    \]
    It follows that
    \begin{align*}
        \abs{f_n(x) - f_m(x)} &\leq \abs{(f_n(x) - f_m(x)) - (f_n(x_0) - f_m(x_0))} + \abs{f_n(x_0) - f_m(x_0)} \\[2mm]
        &= \abs{x - x_0} \abs{f_n'(c) - f_m'(c)} + \abs{f_n(x_0) - f_m(x_0)} \\[2mm]
        &\leq (b - a) \abs{f_n'(c) - f_m'(c)} + \abs{f_n(x_0) - f_m(x_0)} \\[2mm]
        &< \frac{\epsilon}{2} + \frac{\epsilon}{2} \\[2mm]
        &= \epsilon.
    \end{align*}
    We have now shown that for any \( n, m \geq N \) and \( x \in [a, b] \) it holds that
    \[
        \abs{f_n(x) - f_m(x)} < \epsilon;
    \]
    it follows from Theorem 6.2.5 that the sequence \( (f_n) \) is uniformly convergent on \( [a, b] \).
\end{solution}

\noindent \hrulefill

\noindent \hypertarget{ua}{\textcolor{blue}{[UA]} Abbott, S. (2015) \textit{Understanding Analysis.} 2\ts{nd} edition.}

\end{document}