\documentclass[12pt]{article}
\usepackage[utf8]{inputenc}
\usepackage[utf8]{inputenc}
\usepackage{amsmath}
\usepackage{amsthm}
\usepackage{tabularray}
\usepackage{geometry}
\usepackage{amsfonts}
\usepackage{mathrsfs}
\usepackage{bm}
\usepackage{hyperref}
\usepackage[dvipsnames]{xcolor}
\usepackage{enumitem}
\usepackage{mathtools}
\usepackage{changepage}
\usepackage{lipsum}
\usepackage{float}
\usepackage{tikz}
\usetikzlibrary{matrix}
\usepackage{tikz-cd}
\usepackage[nameinlink]{cleveref}
\geometry{
headheight=15pt,
left=60pt,
right=60pt
}
\setlength{\emergencystretch}{20pt}
\usepackage{fancyhdr}
\pagestyle{fancy}
\fancyhf{}
\lhead{}
\chead{Section 7.A Exercises}
\rhead{\thepage}
\hypersetup{
    colorlinks=true,
    linkcolor=blue,
    urlcolor=blue
}

\theoremstyle{definition}
\newtheorem*{remark}{Remark}

\newtheoremstyle{exercise}
    {}
    {}
    {}
    {}
    {\bfseries}
    {.}
    { }
    {\thmname{#1}\thmnumber{#2}\thmnote{ (#3)}}
\theoremstyle{exercise}
\newtheorem{exercise}{Exercise 7.A.}

\newtheoremstyle{solution}
    {}
    {}
    {}
    {}
    {\itshape\color{magenta}}
    {.}
    { }
    {\thmname{#1}\thmnote{ #3}}
\theoremstyle{solution}
\newtheorem*{solution}{Solution}

\Crefformat{exercise}{#2Exercise 7.A.#1#3}

\newcommand{\upd}{\,\text{d}}
\newcommand{\re}{\text{Re}\,}
\newcommand{\im}{\text{Im}\,}
\newcommand{\poly}{\mathcal{P}}
\newcommand{\lmap}{\mathcal{L}}
\newcommand{\mat}{\mathcal{M}}
\newcommand{\ts}{\textsuperscript}
\newcommand{\Span}{\text{span}}
\newcommand{\Null}{\text{null\,}}
\newcommand{\Range}{\text{range\,}}
\newcommand{\Rank}{\text{rank\,}}
\newcommand{\quand}{\quad \text{and} \quad}
\newcommand{\quimplies}{\quad \implies \quad}
\newcommand{\quiff}{\quad \iff \quad}
\newcommand{\ipanon}{\langle \cdot, \cdot \rangle}
\newcommand{\normanon}{\lVert \, \cdot \, \rVert}
\newcommand{\setcomp}[1]{#1^{\mathsf{c}}}
\newcommand{\tpose}[1]{#1^{\text{t}}}
\newcommand{\ocomp}[1]{#1^{\perp}}
\newcommand{\N}{\mathbf{N}}
\newcommand{\Z}{\mathbf{Z}}
\newcommand{\Q}{\mathbf{Q}}
\newcommand{\R}{\mathbf{R}}
\newcommand{\C}{\mathbf{C}}
\newcommand{\F}{\mathbf{F}}

\DeclarePairedDelimiter\abs{\lvert}{\rvert}
% Swap the definition of \abs* and \norm*, so that \abs
% and \norm resizes the size of the brackets, and the 
% starred version does not.
\makeatletter
\let\oldabs\abs
\def\abs{\@ifstar{\oldabs}{\oldabs*}}

\DeclarePairedDelimiter\norm{\lVert}{\rVert}
\makeatletter
\let\oldnorm\norm
\def\norm{\@ifstar{\oldnorm}{\oldnorm*}}
\makeatother

\DeclarePairedDelimiter\paren{(}{)}
\makeatletter
\let\oldparen\paren
\def\paren{\@ifstar{\oldparen}{\oldparen*}}
\makeatother

\DeclarePairedDelimiter\bkt{[}{]}
\makeatletter
\let\oldbkt\bkt
\def\bkt{\@ifstar{\oldbkt}{\oldbkt*}}
\makeatother

\DeclarePairedDelimiter\Set{\{}{\}}
\makeatletter
\let\oldSet\Set
\def\Set{\@ifstar{\oldSet}{\oldSet*}}
\makeatother

\DeclarePairedDelimiter\ip{\langle}{\rangle}
\makeatletter
\let\oldip\ip
\def\set{\@ifstar{\oldip}{\oldip*}}
\makeatother

\setlist[enumerate,1]{label={(\alph*)}}

\begin{document}

\section{Section 7.A Exercises}

Exercises with solutions from Section 7.A of \hyperlink{ladr}{[LADR]}.

\begin{exercise}
\label{ex:1}
    Suppose \( n \) is a positive integer. Define \( T \in \lmap(\F^n) \) by
    \[
        T(z_1, \ldots, z_n) = (0, z_1, \ldots, z_{n-1}).
    \]
    Find a formula for \( T^*(z_1, \ldots, z_n) \).
\end{exercise}

\begin{solution}
    Observe that
    \begin{align*}
       \ip{(w_1, \ldots, w_n), T^*(z_1, \ldots, z_n)} &= \ip{T(w_1, \ldots, w_n), (z_1, \ldots, z_n)} \\[2mm]
       &= \ip{(0, w_1, \ldots, w_{n-1}), (z_1, \ldots, z_n)} \\[2mm]
       &= w_1 z_2 + \cdots + w_{n-1} z_n \\[2mm]
       &= \ip{(w_1, \ldots, w_n), (z_2, \ldots, z_n, 0)}.
    \end{align*}
    Thus \( T^*(z_1, \ldots, z_n) = (z_2, \ldots, z_n, 0) \).
\end{solution}

\begin{exercise}
\label{ex:2}
    Suppose \( T \in \lmap(V) \) and \( \lambda \in \F \). Prove that \( \lambda \) is an eigenvalue of \( T \) if and only if \( \overline{\lambda} \) is an eigenvalue of \( T^* \).
\end{exercise}

\begin{solution}
    Observe that
    \begin{align*}
        \lambda \text{ is an eigenvalue of } T &\iff T - \lambda I \text{ is not surjective} \tag{5.6 (c)} \\[2mm]
        &\iff \Range(T - \lambda I) \neq V \\[2mm]
        &\iff \ocomp{(\Range(T - \lambda I))} \neq \{ 0 \} \tag{\href{https://lew98.github.io/Mathematics/LADR_Section_6_C_Exercises.pdf}{Exercise 6.C.2}} \\[2mm]
        &\iff \Null (T - \lambda I)^* \neq \{ 0 \} \tag{7.7 (a)} \\[2mm]
        &\iff \Null \paren{ T^* - \overline{\lambda} I } \neq \{ 0 \} \tag{7.6 (a), (b), (d)} \\[2mm]
        &\iff T^* - \overline{\lambda} I \text{ is not injective} \\[2mm]
        &\iff \overline{\lambda} \text{ is an eigenvalue of } T^*. \tag{5.6 (b)}
    \end{align*}
\end{solution}

\begin{exercise}
\label{ex:3}
    Suppose \( T \in \lmap(V) \) and \( U \) is a subspace of \( V \). Prove that \( U \) is invariant under \( T \) if and only if \( \ocomp{U} \) is invariant under \( T^* \).
\end{exercise}

\begin{solution}
    Suppose that \( U \) is invariant under \( T \) and let \( v \in \ocomp{U} \) be given. Observe that
    \[
        \ip{u, T^* v} = \ip{Tu, v} = 0
    \]
    for any \( u \in U \), where the last equality follows since \( Tu \in U \) and \( v \in \ocomp{U} \). Thus \( T^* v \in \ocomp{U} \) and we see that \( \ocomp{U} \) is invariant under \( T^* \).

    Now suppose that \( \ocomp{U} \) is invariant under \( T^* \). The previous paragraph shows that \( \ocomp{(\ocomp{U})} \) is invariant under \( (T^*)^* \), which by 6.51 and 7.6 (c) is exactly the statement that \( U \) is invariant under \( T \).
\end{solution}

\begin{exercise}
\label{ex:4}
    Suppose \( T \in \lmap(V, W) \). Prove that
    \begin{enumerate}
        \item \( T \) is injective if and only if \( T^* \) is surjective;

        \item \( T \) is surjective if and only if \( T^* \) is injective.
    \end{enumerate}
\end{exercise}

\begin{solution}
    \begin{enumerate}
        \item Observe that
        \begin{align*}
            T \text{ is injective} &\iff \Null T = \{ 0 \} \\[2mm]
            &\iff \ocomp{(\Range T^*)} = \{ 0 \} \tag{7.7 (c)} \\[2mm]
            &\iff \Range T^* = V \tag{\href{https://lew98.github.io/Mathematics/LADR_Section_6_C_Exercises.pdf}{Exercise 6.C.2}} \\[2mm]
            &\iff T^* \text{ is surjective}.
        \end{align*}

        \item Part (a) shows that \( T^* \) is injective if and only if \( (T^*)^* \) is surjective, which by 7.6 (c) is equivalent to \( T^* \) being injective if and only if \( T \) is surjective.
    \end{enumerate}
\end{solution}

\begin{exercise}
\label{ex:5}
    Prove that
    \[
        \dim \Null T^* = \dim \Null T + \dim W - \dim V
    \]
    and
    \[
        \dim \Range T^* = \dim \Range T
    \]
    for every \( T \in \lmap(V, W) \).
\end{exercise}

\begin{solution}
    We have
    \begin{align*}
        \dim \Null T^* &= \dim \ocomp{(\Range T)} \tag{7.7 (a)} \\[2mm]
        &= \dim W - \dim \Range T \tag{6.50} \\[2mm]
        &= \dim \Null T + \dim W - \dim V \tag{3.22}.
    \end{align*}
    Similarly,
    \begin{align*}
        \dim \Range T^* &= \dim \ocomp{(\Null T)} \tag{7.7 (b)} \\[2mm]
        &= \dim V - \dim \Null T \tag{6.50} \\[2mm]
        &= \dim \Range T \tag{3.22}.
    \end{align*}
\end{solution}

\begin{exercise}
\label{ex:6}
    Make \( \poly_2(\R) \) into an inner product space by defining
    \[
        \ip{p, q} = \int_0^1 p(x) q(x) \upd x.
    \]
    Define \( T \in \lmap(\poly_2(\R)) \) by \( T(a_0 + a_1 x + a_2 x^2) = a_1 x \).
    \begin{enumerate}
        \item Show that \( T \) is not self-adjoint.

        \item The matrix of \( T \) with respect to the basis \( (1, x, x^2) \) is
        \[
            \begin{pmatrix}
                0 & 0 & 0 \\
                0 & 1 & 0 \\
                0 & 0 & 0
            \end{pmatrix}.
        \]
        This matrix equals its conjugate transpose, even though \( T \) is not self-adjoint. Explain why this is not a contradiction.
    \end{enumerate}
\end{exercise}

\begin{solution}
    \begin{enumerate}
        \item Let \( p, q \in \poly_2(\R) \) be given by \( p(x) = 2x \) and \( q(x) = 1 \), so that \( Tp = p \) and \( Tq = 0 \). Then
        \[
            \ip{Tp, q} = \int_0^1 2x \upd x = 1 \neq 0 = \ip{p, Tq}.
        \]
        Thus \( T \) is not self-adjoint.

        \item The result in 7.10 requires that the basis of \( \poly_2(\R) \) is orthonormal, but \( (1, x, x^2) \) is not an orthonormal basis:
        \[
            \ip{1, x} = \int_0^1 x \upd x = \tfrac{1}{2} \neq 0.
        \]
    \end{enumerate}
\end{solution}

\begin{exercise}
\label{ex:7}
    Suppose \( S, T \in \lmap(V) \) are self-adjoint. Prove that \( ST \) is self-adjoint if and only if \( ST = TS \).
\end{exercise}

\begin{solution}
    Observe that
    \[
        (ST)^* = T^* S^* = TS,
    \]
    where the first equality is 7.6 (e). It follows that \( (ST)^* = ST \) if and only if \( TS = ST \).
\end{solution}

\begin{exercise}
\label{ex:8}
    Suppose \( V \) is a real inner product space. Show that the set of self-adjoint operators on \( V \) is a subspace of \( \lmap(V) \).
\end{exercise}

\begin{solution}
    Clearly the zero operator is self-adjoint. Closure under additivity and scalar multiplication follows from 7.6 (a) and (b).
\end{solution}

\begin{exercise}
\label{ex:9}
    Suppose \( V \) is a complex inner product space with \( V \neq \{ 0 \} \). Show that the set of self-adjoint operators on \( V \) is not a subspace of \( \lmap(V) \).
\end{exercise}

\begin{solution}
    7.6 (d) shows that the identity operator \( I \) is self-adjoint. Since \( V \neq \{ 0 \} \), there is some non-zero \( v \in V \). Observe that
    \[
        (iI)(v) = iv \neq -iv = \paren{\overline{i} I}(v) = (iI)^*(v),
    \]
    where we have used 7.6 (b) and (d). It follows that \( iI \) is not self-adjoint and hence that the set of self-adjoint operators on \( V \) is not closed under scalar multiplication.
\end{solution}

\begin{exercise}
\label{ex:10}
    Suppose \( \dim V \geq 2 \). Show that the set of normal operators on \( V \) is not a subspace of \( \lmap(V) \).
\end{exercise}

\begin{solution}
    Let \( e_1, e_2, \ldots, e_n \) be an orthonormal basis of \( V \) and let \( S, T \in \lmap(V) \) be the operators whose matrices with respect to this basis are
    \[
        A = \begin{pmatrix}
            1 & 0 & 0 & \cdots & 0 \\
            0 & 0 & 0 & \cdots & 0 \\
            0 & 0 & 0 & \cdots & 0 \\
            \vdots & \vdots & \vdots & \ddots & \vdots \\
            0 & 0 & 0 & \cdots & 0
        \end{pmatrix}
        \quand
        B = \begin{pmatrix}
            0 & 1 & 0 & \cdots & 0 \\
            -1 & 0 & 0 & \cdots & 0 \\
            0 & 0 & 0 & \cdots & 0 \\
            \vdots & \vdots & \vdots & \ddots & \vdots \\
            0 & 0 & 0 & \cdots & 0
        \end{pmatrix},
    \]
    respectively. Note that \( S \) is self-adjoint and hence normal. Note further that \( T \) satisfies \( T^* = -T \), so that \( TT^* = T^*T = -T^2 \); it follows that \( T \) is also normal. However, some calculations reveal that
    \[
        (A + B)(A + B)^* = \begin{pmatrix}
            2 & -1 & 0 & \cdots & 0 \\
            -1 & 1 & 0 & \cdots & 0 \\
            0 & 0 & 0 & \cdots & 0 \\
            \vdots & \vdots & \vdots & \ddots & \vdots \\
            0 & 0 & 0 & \cdots & 0
        \end{pmatrix}
        \quand
        (A + B)^*(A + B) = \begin{pmatrix}
            2 & 1 & 0 & \cdots & 0 \\
            1 & 1 & 0 & \cdots & 0 \\
            0 & 0 & 0 & \cdots & 0 \\
            \vdots & \vdots & \vdots & \ddots & \vdots \\
            0 & 0 & 0 & \cdots & 0
        \end{pmatrix}.
    \]
    Thus \( S + T \) is not normal.
\end{solution}

\begin{exercise}
\label{ex:11}
    Suppose \( P \in \lmap(V) \) is such that \( P^2 = P \). Prove that there is a subspace \( U \) of \( V \) such that \( P = P_U \) if and only if \( P \) is self-adjoint.
\end{exercise}

\begin{solution}
    Suppose that \( P = P_U \) for some subspace \( U \) of \( V \). Let \( v = u + x \) and \( w = u' + x' \) be given, where \( u, u' \in U \) and \( x, x' \in \ocomp{U} \). Then
    \[
        \ip{P_U v, w} = \ip{u, u' + x'} = \ip{u, u'} \quand \ip{v, P_U w} = \ip{u + x, u'} = \ip{u, u'}.
    \]
    It follows that \( P_U \) is self-adjoint.

    Now suppose that \( P \) is self-adjoint and let \( U = \Range P \). We claim that \( P = P_U \). Let \( v = Px + w \) be given, where \( Px \in \Range P \) and \( w \in \ocomp{(\Range P)} \). Note that
    \[
        \ocomp{(\Range P)} = \Null P^* = \Null P,
    \]
    where the first equality follows from 7.7 (a) and the second equality follows since \( P \) is self-adjoint. Hence \( w \in \Null P \) and we see that
    \[
        Pv = P^2 x = Px = P_U v.
    \]
\end{solution}

\begin{exercise}
\label{ex:12}
    Suppose that \( T \) is a normal operator on \( V \) and that 3 and 4 are eigenvalues of \( T \). Prove that there exists a vector \( v \in V \) such that \( \norm{v} = \sqrt{2} \) and \( \norm{Tv} = 5 \).
\end{exercise}

\begin{solution}
    There are eigenvectors \( u, w \in V \) satisfying
    \[
        Tu = 3u, \quad Tw = 4w, \quand \ip{u, w} = 0;
    \]
    this last equality follows from 7.22. Define
    \[
        v := \frac{u}{\norm{u}} + \frac{w}{\norm{w}}
    \]
    and note that
    \[
        \norm{v}^2 = \norm{\frac{u}{\norm{u}}}^2 + \norm{\frac{w}{\norm{w}}}^2 = 2 \quimplies \norm{v} = \sqrt{2},
    \]
    where we have used the Pythagorean theorem (6.13). Furthermore,
    \[
        \norm{Tv}^2 = \norm{T \frac{u}{\norm{u}} + T \frac{w}{\norm{w}}}^2 = \norm{\frac{3u}{\norm{u}} + \frac{4w}{\norm{w}}}^2 = \norm{\frac{3u}{\norm{u}}} + \norm{\frac{4w}{\norm{w}}}^2 = 3^2 + 4^2,
    \]
    where we have again used the Pythagorean theorem (6.13). It follows that \( \norm{Tv} = 5 \).
\end{solution}

\begin{exercise}
\label{ex:13}
    Give an example of an operator \( T \in \lmap(\C^4) \) such that \( T \) is normal but not self-adjoint.
\end{exercise}

\begin{solution}
    Let \( T \in \lmap(\C^4) \) be the operator whose matrix with respect to the standard orthonormal basis is
    \[
        \begin{pmatrix}
            0 & 1 & 0 & 0 \\
            -1 & 0 & 0 & 0 \\
            0 & 0 & 0 & 0 \\
            0 & 0 & 0 & 0
        \end{pmatrix}.
    \]
    As we showed in \Cref{ex:10}, this operator satisfies \( T^* = -T \), which implies that \( TT^* = T^*T = -T^2 \). It follows that \( T \) is normal but not self-adjoint.
\end{solution}

\begin{exercise}
\label{ex:14}
    Suppose \( T \) is a normal operator on \( V \). Suppose also that \( v, w \in V \) satisfy the equations
    \[
        \norm{v} = \norm{w} = 2, \quad Tv = 3v, \quad Tw = 4w.
    \]
    Show that \( \norm{T(v + w)} = 10 \).
\end{exercise}

\begin{solution}
    Since \( v \) and \( w \) are eigenvectors of \( T \) corresponding to distinct eigenvalues, they must be orthogonal (7.22). The Pythagorean theorem (6.13) then implies that
    \[
        \norm{T(v + w)}^2 = \norm{3v + 4w}^2 = \norm{3v}^2 + \norm{4w}^2 = 9 \norm{v}^2 + 16 \norm{w}^2 = 36 + 64 = 100;
    \]  
    it follows that \( \norm{T(v + w)} = 10 \).
\end{solution}

\begin{exercise}
\label{ex:15}
    Fix \( u, x \in V \). Define \( T \in \lmap(V) \) by
    \[
        Tv = \ip{v, u} x
    \]
    for every \( v \in V \).
    \begin{enumerate}
        \item Suppose \( \F = \R \). Prove that \( T \) is self-adjoint if and only if \( u, x \) is linearly dependent.

        \item Prove that \( T \) is normal if and only if \( u, x \) is linearly dependent.
    \end{enumerate}
\end{exercise}

\begin{solution}
    Note that Example 7.4 gives us the formula
    \[
        T^*v = \ip{v, x} u.
    \]
    \begin{enumerate}
        \item Suppose that \( u, x \) is linearly dependent, say \( x = \lambda u \) for some \( \lambda \in \R \). Then for any \( v \in V \)
        \[
            Tv = \ip{v, u} x = \lambda \ip{v, u} u = \ip{v, \lambda u} u = \ip{v, x} u = T^* v.
        \]
        Now suppose that \( T \) is self-adjoint. If \( u = 0 \) we are done, so suppose that \( u \neq 0 \). Since \( T \) is self-adjoint, we must have
        \[
            Tv = \ip{v, u} x = \ip{v, x} u = T^*v
        \]
        for every \( v \in V \). In particular,
        \[
            \ip{u, u} x = \ip{u, x} u \quimplies x = \frac{\ip{u, x}}{\ip{u, u}} u,
        \]
        demonstrating that \( u, x \) is linearly dependent.

        \item Note that
        \[
            (TT^* - T^*T)(v) = \ip{v, x} \ip{u, u} x - \ip{v, u} \ip{x, x} u
        \]
        for any \( v \in V \). Suppose that \( u, x \) is linearly dependent, say \( x = \lambda u \) for some \( \lambda \in \F \). Then
        \begin{multline*}
            (TT^* - T^*T)(v) = \ip{v, \lambda u} \ip{u, u} \lambda u - \ip{v, u} \ip{\lambda u, \lambda u} u \\[2mm]
            = \abs{\lambda}^2 \ip{v, u} \ip{u, u} u - \abs{\lambda}^2 \ip{v, u} \ip{u, u} u = 0.
        \end{multline*}
        Thus \( T \) is normal. Conversely, suppose that \( T \) is normal. If \( u = 0 \) we are done, so suppose that \( u \neq 0 \). Then
        \[
            (TT^* - T^*T)(x) = \ip{x, x} \ip{u, u} x - \ip{x, u} \ip{x, x} u = 0 \quimplies x = \frac{\ip{x, u}}{\ip{u, u}} u,
        \]
        demonstrating that \( u, x \) is linearly dependent.
    \end{enumerate}
\end{solution}

\begin{exercise}
\label{ex:16}
    Suppose \( T \in \lmap(V) \) is normal. Prove that
    \[
        \Range T = \Range T^*.
    \]
\end{exercise}

\begin{solution}
    Observe that 7.20 implies \( \Null T = \Null T^* \). It then follows from 7.7 that
    \[
        \Range T^* = \ocomp{(\Null T)} = \ocomp{(\Null T^*)} = \Range T.
    \]
\end{solution}

\begin{exercise}
\label{ex:17}
    Suppose \( T \in \lmap(V) \) is normal. Prove that
    \[
        \Null T^k = \Null T \quand \Range T^k = \Range T
    \]
    for every positive integer \( k \).
\end{exercise}

\begin{solution}
    Let us prove by induction that \( \Null T^k = \Null T \) for every positive integer \( k \). The base case \( k = 1 \) is clear, so suppose that the result is true for some positive integer \( k \). The containment \( \Null T^k \subseteq \Null T^{k+1} \) is evident. Suppose that \( v \in \Null T^{k+1} \). By 7.20 we then have \( T^*T^k v = 0 \) and it follows that
    \[
        \ip{T^*T^k v, T^{k-1}v} = 0 \quiff \ip{T^k v, T^k v} = 0 \quiff T^k v = 0.
    \]
    Thus \( \Null T^{k+1} = \Null T^k = \Null T \); the last equality is our induction hypothesis. This completes the induction step and the proof.

    The second half of the exercise is a quick corollary of the first half. Evidently, \( T \) normal implies \( T^* \) normal. Furthermore, 7.6 (e) implies that \( (T^k)^* = (T^*)^k \) for all positive integers \( k \). It follows that
    \begin{align*}
        \Range T^k &= \ocomp{\paren{\Null \paren{T^k}^*}} \tag{7.7 (d)} \\[2mm]
        &= \ocomp{\paren{\Null (T^*)^k}} \\[2mm]
        &= \ocomp{(\Null T^*)} \tag{\( T^* \) is normal} \\[2mm]
        &= \Range T. \tag{7.7 (d)}
    \end{align*}
\end{solution}

\begin{exercise}
\label{ex:18}
    Prove or give a counterexample: If \( T \in \lmap(V) \) and there exists an orthonormal basis \( e_1, \ldots, e_n \) of \( V \) such that \( \norm{T e_j} = \norm{T^*e_j} \) for each \( j \), then \( T \) is normal.
\end{exercise}

\begin{solution}
    This is false. Let \( T \) be the operator on \( \F^2 \) whose matrix with respect to the standard orthonormal basis \( e_1, e_2 \) is
    \[
        \begin{pmatrix}
            1 & 1 \\
            -1 & 0 
        \end{pmatrix};
    \]
    as we showed in \Cref{ex:10}, \( T \) is not normal. However,
    \[
        \norm{T e_1} = \norm{T^* e_1} = \sqrt{2} \quand \norm{T e_2} = \norm{T^* e_2} = 1.
    \]
\end{solution}

\begin{exercise}
\label{ex:19}
    Suppose \( T \in \lmap(\C^3) \) is normal and \( T(1, 1, 1) = (2, 2, 2) \). Suppose \( (z_1, z_2, z_3) \in \Null T \). Prove that \( z_1 + z_2 + z_3 = 0 \).
\end{exercise}

\begin{solution}
    If \( u := (z_1, z_2, z_3) = (0, 0, 0) \) then we are done, so suppose that \( u \neq 0 \). It follows that \( u \) is an eigenvector of \( T \) corresponding to the eigenvalue 0. Note that \( v := (1, 1, 1) \) is an eigenvector of \( T \) corresponding to the eigenvalue 2. Since these are eigenvectors of a normal operator corresponding to distinct eigenvalues, they must be orthogonal (7.22). That is,
    \[
        \ip{u, v} = z_1 + z_2 + z_3 = 0.
    \]
\end{solution}

\begin{exercise}
\label{ex:20}
    Suppose \( T \in \lmap(V, W) \) and \( \F = \R \). Let \( \Phi_V \) be the isomorphism from \( V \) onto the dual space \( V' \) given by \href{https://lew98.github.io/Mathematics/LADR_Section_6_B_Exercises.pdf}{Exercise 17} in Section 6.B, and let \( \Phi_W \) be the corresponding isomorphism from \( W \) onto \( W' \). Show that if \( \Phi_V \) and \( \Phi_W \) are used to identify \( V \) and \( W \) with \( V' \) and \( W' \), then \( T^* \) is identified with the dual map \( T' \). More precisely, show that \( \Phi_V \circ T^* = T' \circ \Phi_W \).
\end{exercise}

\begin{solution}
    Let \( w \in W \) be given. Then
    \[
        (\Phi_V \circ T^*)(w) = \Phi_V T^* w
    \]
    is the map \( V \to \R \) given by
    \[
        (\Phi_V T^* w)(v) = \ip{v, T^* w}.
    \]
    On the other hand,
    \[
        (T' \circ \Phi_W)(w) = T' \Phi_W w = \Phi_W w \circ T
    \]
    is the map \( V \to \R \) given by
    \[
        (\Phi_W w \circ T)(v) = (\Phi_W w)(Tv) = \ip{Tv, w}.
    \]
    Since \( \ip{v, T^* w} = \ip{Tv, w} \) for all \( v \in V \), we see that \( (\Phi_V \circ T^*)(w) = (T' \circ \Phi_W)(w) \) for all \( w \in W \).
\end{solution}

\begin{exercise}
\label{ex:21}
    Fix a positive integer \( n \). In the inner product space of continuous real-valued functions on \( [-\pi, \pi] \) with inner product
    \[
        \ip{f, g} = \int_{-\pi}^{\pi} f(x) g(x) \upd x,
    \]
    let
    \[
        V = \Span(1, \cos x, \cos 2x, \ldots, \cos nx, \sin x, \sin 2x, \ldots, \sin nx).
    \]
    \begin{enumerate}
        \item Define \( D \in \lmap(V) \) by \( Df = f' \). Show that \( D^* = -D \). Conclude that \( D \) is normal but not self-adjoint.

        \item Define \( T \in \lmap(V) \) by \( Tf = f'' \). Show that \( T \) is self-adjoint.
    \end{enumerate}
\end{exercise}

\begin{solution}
    \begin{enumerate}
        \item Let
        \[
            v = \frac{1}{\sqrt{2 \pi}}, \quad e_j = \frac{\cos jx}{\sqrt{\pi}}, \quand f_j = \frac{\sin jx}{\sqrt{\pi}}
        \]
        for each \( 1 \leq j \leq n \), and let \( B := v, e_1, \ldots, e_n, f_1, \ldots, f_n \). Then \( B \) is an orthonormal basis of \( V \) (\href{https://lew98.github.io/Mathematics/LADR_Section_6_B_Exercises.pdf}{Exercise 6.B.4}). Observe that \( Dv = 0, De_j = -j f_j, \) and \( Df_j = j e_j \) for each \( 1 \leq j \leq n \). It follows that the matrix of \( D \) with respect to \( B \) is
        \[
            \begin{pmatrix}
                0 & 0 & 0 & \cdots & 0 & 0 & 0 & \cdots & 0 \\
                0 & 0 & 0 & \cdots & 0 & 1 & 0 & \cdots & 0 \\
                0 & 0 & 0 & \cdots & 0 & 0 & 2 & \cdots & 0 \\
                \vdots & \vdots & \vdots & \ddots & \vdots & \vdots & \vdots & \ddots & \vdots \\
                0 & 0 & 0 & \cdots & 0 & 0 & 0 & \cdots & n \\
                0 & -1 & 0 & \cdots & 0 & 0 & 0 & \cdots & 0 \\
                0 & 0 & -2 & \cdots & 0 & 0 & 0 & \cdots & 0 \\
                \vdots & \vdots & \vdots & \ddots & \vdots & \vdots & \vdots & \ddots & \vdots \\
                0 & 0 & 0 & \cdots & -n & 0 & 0 & \cdots & 0 \\
            \end{pmatrix},
        \]
        from which we see that \( D^* = -D \). Such operators are normal but not self-adjoint, as we showed in \Cref{ex:10}.

        \item Observe that \( Tv = 0, Te_j = -j^2 e_j, \) and \( Tf_j = -j^2 f_j \) for each \( 1 \leq j \leq n \). It follows that the matrix of \( T \) with respect to \( B \) is
        \[
            \begin{pmatrix}
                0 & 0 & 0 & \cdots & 0 & 0 & 0 & \cdots & 0 \\
                0 & -1 & 0 & \cdots & 0 & 0 & 0 & \cdots & 0 \\
                0 & 0 & -4 & \cdots & 0 & 0 & 0 & \cdots & 0 \\
                \vdots & \vdots & \vdots & \ddots & \vdots & \vdots & \vdots & \ddots & \vdots \\
                0 & 0 & 0 & \cdots & -n^2 & 0 & 0 & \cdots & 0 \\
                0 & 0 & 0 & \cdots & 0 & -1 & 0 & \cdots & 0 \\
                0 & 0 & 0 & \cdots & 0 & 0 & -4 & \cdots & 0 \\
                \vdots & \vdots & \vdots & \ddots & \vdots & \vdots & \vdots & \ddots & \vdots \\
                0 & 0 & 0 & \cdots & 0 & 0 & 0 & \cdots & -n^2 \\
            \end{pmatrix},
        \]
        from which we see that \( T \) is self-adjoint (indeed, diagonal).
    \end{enumerate}
\end{solution}

\noindent \hrulefill

\noindent \hypertarget{ladr}{\textcolor{blue}{[LADR]} Axler, S. (2015) \textit{Linear Algebra Done Right.} 3\ts{rd} edition.}

\end{document}