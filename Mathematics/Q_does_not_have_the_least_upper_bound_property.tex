\documentclass[12pt]{article}
\usepackage[utf8]{inputenc}
\usepackage{amsmath}
\usepackage{amsthm}
\usepackage{geometry}
\usepackage{amsfonts}
\usepackage{mathrsfs}
\usepackage{bm}
\usepackage{hyperref}
\usepackage{xcolor}
\usepackage{enumitem}
\usepackage{changepage}
\usepackage{tikz}
\usetikzlibrary{matrix}
\usepackage{tikz-cd}
\usepackage[nameinlink]{cleveref}
\geometry{
headheight=15pt,
left=60pt,
right=60pt
}
\usepackage{fancyhdr}
\pagestyle{fancy}
\fancyhf{}
\lhead{}
\chead{\texorpdfstring{\( \mathbb{Q} \)}{} does not have the least-upper-bound property}
\rhead{\thepage}

\hypersetup{
    colorlinks=true,
    linkcolor=blue
}

\theoremstyle{definition}
\newtheorem{theorem}{Theorem}
\newtheorem{lemma}[theorem]{Lemma}
\newtheorem{corollary}[theorem]{Corollary}

\newtheorem*{remark}{Remark}

\begin{document}

The following is paraphrased from Chapter 1 of \hyperlink{pma}{[PMA]}.

\section{\texorpdfstring{\( \mathbb{Q} \)}{} does not have the least-upper-bound property}

Let \( A \) be the set of rational numbers which are either negative or have square less than 2 and let \( B \) be the set of positive rational numbers with square greater than 2, i.e.
\[
    A = \{ p \in \mathbb{Q} : p < 0 \text{ or } p^2 < 2 \}, \quad B = \{ p \in \mathbb{Q} : p > 0 \text{ and } p^2 > 2 \}.
\]
Note that \( A \) and \( B \) partition \( \mathbb{Q} \) since there is no rational number whose square is 2.

\begin{lemma}
\label{lem:no_greatest/least_element}
    \( A \) contains no greatest element and \( B \) contains no least element. That is, for any \( p \in A \) there exists a \( q \in A \) with \( q > p \) and for any \( p \in B \) there exists a \( q \in B \) with \( q < p \).
\end{lemma}

\begin{proof}
    If \( p \in A \) and \( p \leq 0 \), then take \( q = 1 \). Otherwise, for a positive rational number \( p \), define
    \[ q = p + \frac{2 - p^2}{p + 2} = \frac{2p + 2}{p + 2}. \]
    Then
    \[ 2 - q^2 = 2 - \frac{(2p + 2)^2}{(p+2)^2} = \frac{2(2 - p^2)}{(p + 2)^2}. \]
    For \( p \in A \) with \( p > 0 \), we have \( 2 - p^2 > 0 \), so that \( q > p \) and \( q \in A \); for \( p \in B \) we have \( 2 - p^2 < 0 \), so that \( q < p \) and \( q \in B \).
\end{proof}

\begin{lemma}
\label{lem:set_of_upper_bounds_of_A_is_B}
    The upper bounds of \( A \) are exactly the elements of \( B \).
\end{lemma}

\begin{proof}
    Suppose \( r \in \mathbb{Q} \) is an upper bound for \( A \). Then certainly \( r \) is positive, since \( 1 \in A \). Furthermore, exactly one of the following is true: \( r^2 < 2 \), \( r^2 = 2 \), or \( r^2 > 2 \). If \( r^2 < 2 \), then \( r \in A \); but this implies that \( r \) is the greatest element of \( A \), contradicting Lemma 1. So \( r^2 \geq 2 \) and since \( r^2 = 2 \) is impossible for rational \( r \), we must have \( r^2 > 2 \), i.e. \( r \in B \).
    
    Now suppose \( r \in B \) and let \( p \) be any element of \( A \). If \( r < p \), then \( r^2 < p^2 < 2 \) since \( r \) is positive. This contradicts \( r \in B \), so in fact we must have \( r \geq p \), so that \( r \) is an upper bound for \( A \).
\end{proof}

Since \( A \) and \( B \) partition \( \mathbb{Q} \), \Cref{lem:no_greatest/least_element} and \Cref{lem:set_of_upper_bounds_of_A_is_B} give us the following corollary.

\begin{corollary}
    The lower bounds of \( B \) are exactly the elements of \( A \).
\end{corollary}

\begin{theorem}
    The set of rational numbers \( \mathbb{Q} \) does not have the least-upper-bound property.
\end{theorem}

\begin{proof}
    \( A \) is non-empty and bounded above; the upper bounds of \( A \) are exactly the elements of \( B \). Since \( B \) has no least element, it follows that \( A \) has no least upper bound.
\end{proof}

\noindent \hrulefill

\noindent \hypertarget{pma}{\textcolor{blue}{[PMA]} Rudin, W. (1976) \textit{Principles of Mathematical Analysis.} 3rd edn.}

\end{document}
