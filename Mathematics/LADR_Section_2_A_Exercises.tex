\documentclass[12pt]{article}
\usepackage[utf8]{inputenc}
\usepackage[utf8]{inputenc}
\usepackage{amsmath}
\usepackage{amsthm}
\usepackage{geometry}
\usepackage{amsfonts}
\usepackage{mathrsfs}
\usepackage{bm}
\usepackage{hyperref}
\usepackage[dvipsnames]{xcolor}
\usepackage{enumitem}
\usepackage{mathtools}
\usepackage{changepage}
\usepackage{lipsum}
\usepackage{tikz}
\usetikzlibrary{matrix}
\usepackage{tikz-cd}
\usepackage[nameinlink]{cleveref}
\geometry{
headheight=15pt,
left=60pt,
right=60pt
}
\usepackage{fancyhdr}
\pagestyle{fancy}
\fancyhf{}
\lhead{}
\chead{Section 2.A Exercises}
\rhead{\thepage}
\hypersetup{
    colorlinks=true,
    linkcolor=blue,
    urlcolor=blue
}

\theoremstyle{definition}
\newtheorem*{remark}{Remark}

\newtheoremstyle{exercise}
    {}
    {}
    {}
    {}
    {\bfseries}
    {.}
    { }
    {\thmname{#1}\thmnumber{#2}\thmnote{ (#3)}}
\theoremstyle{exercise}
\newtheorem{exercise}{Exercise 2.A.}

\newtheoremstyle{solution}
    {}
    {}
    {}
    {}
    {\itshape\color{magenta}}
    {.}
    { }
    {\thmname{#1}\thmnote{ #3}}
\theoremstyle{solution}
\newtheorem*{solution}{Solution}

\Crefformat{exercise}{#2Exercise 2.A.#1#3}

\newcommand{\ts}{\textsuperscript}
\newcommand{\Span}{\text{span}}
\newcommand{\setcomp}[1]{#1^{\mathsf{c}}}
\newcommand{\N}{\mathbf{N}}
\newcommand{\Z}{\mathbf{Z}}
\newcommand{\Q}{\mathbf{Q}}
\newcommand{\R}{\mathbf{R}}
\newcommand{\C}{\mathbf{C}}
\newcommand{\F}{\mathbf{F}}

\DeclarePairedDelimiter\abs{\lvert}{\rvert}
% Swap the definition of \abs* and \norm*, so that \abs
% and \norm resizes the size of the brackets, and the 
% starred version does not.
\makeatletter
\let\oldabs\abs
\def\abs{\@ifstar{\oldabs}{\oldabs*}}
%
\let\oldnorm\norm
\def\norm{\@ifstar{\oldnorm}{\oldnorm*}}
\makeatother

\setlist[enumerate,1]{label={(\alph*)}}

\begin{document}

\section{Section 2.A Exercises}

Exercises with solutions from Section 2.A of \hyperlink{ladr}{[LADR]}.

\begin{exercise}
\label{ex:1}
    Suppose \( v_1, v_2, v_3, v_4 \) spans \( V \). Prove that the list
    \[
        v_1 - v_2, v_2 - v_3, v_3 - v_4, v_4
    \]
    also spans \( V \).
\end{exercise}

\begin{solution}
    Let \( v \in V \) be given. Since \( V = \Span (v_1, v_2, v_3, v_4) \), there are scalars \( a_1, a_2, a_3, a_4 \) such that \( v = a_1 v_1 + a_2 v_2 + a_3 v_3 + a_4 v_4 \). Observe that
    \begin{align*}
        & a_1 (v_1 - v_2) + (a_1 + a_2) (v_2 - v_3) + (a_1 + a_2 + a_3) (v_3 - v_4) + (a_1 + a_2 + a_3 + a_4) v_4 \\
        = \,\, & a_1 v_1 + (a_1 + a_2 - a_1) v_2 + (a_1 + a_2 + a_3 - a_1 - a_2) v_3 + (a_1 + a_2 + a_3 + a_4 - a_1 - a_2 - a_3) v_4 \\
        = \,\, & a_1 v_1 + a_2 v_2 + a_3 v_3 + a_4 v_4 \\
        = \,\, & v.
    \end{align*}
    Hence \( v \in \Span (v_1 - v_2, v_2 - v_3, v_3 - v_4, v_4) \). It follows that \( V = \Span (v_1 - v_2, v_2 - v_3, v_3 - v_4, v_4) \).
\end{solution}

\begin{exercise}
\label{ex:2}
    Verify the assertions in Example 2.18.
\end{exercise}

\begin{solution}
    \begin{enumerate}
        \item The assertion is that a list \( v \) of one vector \( v \in V \) is linearly independent if and only if \( v \neq 0 \). To see this, first suppose that \( v \neq 0 \) and \( a \in \F \) is such that \( av = 0 \). Then by \href{https://lew98.github.io/Mathematics/LADR_Section_1_B_Exercises.pdf}{Exercise 1.B.2}, it must be the case that \( a = 0 \), demonstrating that the list \( v \) is linearly independent. Now suppose that \( v = 0 \), so that any choice of non-zero \( a \in \F \) will satisfy \( av = 0 \); \( 1v = 0 \) for example. It follows that the list \( v \) is linearly dependent.

        \item The assertion is that a list of two vectors in \( V \) is linearly independent if and only if neither vector is a scalar multiple of the other. Suppose that \( u, v \) is the list of two vectors in \( V \), and suppose that one is a scalar multiple of the other, say \( v = \lambda u \) for some scalar \( \lambda \). Then \( v - \lambda u = 0 \); the coefficient of \( v \) in this linear combination is non-zero, so the list \( u, v \) is linearly dependent. Now suppose that the list \( u, v \) is linearly dependent, so that \( \mu v + \lambda u = 0 \) where at least one of \( \mu \) and \( \lambda \) is non-zero, say \( \mu \neq 0 \): then \( v = -\tfrac{\lambda}{\mu} u \).

        \item The assertion is that the list \( (1, 0, 0, 0), (0, 1, 0, 0), (0, 0, 1, 0) \) is linearly independent in \( \F^4 \). This is easily seen, since
        \[
            a(1, 0, 0, 0) + b(0, 1, 0, 0) + c(0, 0, 1, 0) = (a, b, c, 0) = (0, 0, 0, 0)
        \]
        forces \( a = b = c = 0 \).

        \item The assertion is that the list \( 1, z, \ldots, z^m \) is linearly independent in \( \mathcal{P}(\F) \) for each nonnegative integer \( m \). Suppose that we have scalars \( a_0, a_1, \ldots, a_m \) such that
        \[
            a_0 + a_1 z + \cdots + a_m z^m = 0
        \]
        for all \( z \in \F \). Since the degree of the right-hand side is \( -\infty \) and any non-zero coefficient on the left-hand side would result in a polynomial with degree zero or greater, it must be the case that \( a_0 = a_1 = \cdots = a_m = 0 \).
    \end{enumerate}
\end{solution}

\begin{exercise}
\label{ex:3}
    Find a number \( t \) such that
    \[
        (3, 1, 4), (2, -3, 5), (5, 9, t)
    \]
    is not linearly independent in \( \R^3 \).
\end{exercise}

\begin{solution}
    Let \( t = 2 \). Then observe that
    \[
        3 (3, 1, 4) - 2(2, -3, 5) - (5, 9, 2) = (0, 0, 0).
    \]
\end{solution}

\begin{exercise}
\label{ex:4}
    Verify the assertion in the second bullet point in Example 2.20.
\end{exercise}

\begin{solution}
    The assertion is that the list \( (2, 3, 1), (1, -1, 2), (7, 3, c) \) is linearly dependent in \( \F^3 \) if and only if \( c = 8 \). If \( c = 8 \) then, as shown in the first bullet point in Example 2.20, we have
    \[
        2 (2, 3, 1) + 3 (1, -1, 2) + (-1) (7, 3, 8) = (0, 0, 0).
    \]
    Now suppose that the list is linearly dependent. Since \( (1, -1, 2) \) is clearly not a scalar multiple of \( (2, 3, 1) \), the Linear Dependence Lemma implies that \( (7, 3, c) \) lies in the span of \( (2, 3, 1) \) and \( (1, -1, 2) \), i.e.\ there are scalars \( x \) and \( y \) such that
    \[
        x (2, 3, 1) + y (1, -1, 2) = (7, 3, c).
    \]
    Solving the equations \( 2x + y = 7 \) and \( 3x - y = 3 \) gives \( x = 2 \) and \( y = 3 \), whence \( c = x + 2y = 8 \).
\end{solution}

\begin{exercise}
\label{ex:5}
    \begin{enumerate}
        \item Show that if we think of \( \C \) as a vector space over \( \R \), then the list \( (1 + i), (1 - i) \) is linearly independent.

        \item Show that if we think of \( \C \) as a vector space over \( \C \), then the list \( (1 + i), (1 - i) \) is linearly dependent.
    \end{enumerate}
\end{exercise}

\begin{solution}
    \begin{enumerate}
        \item Suppose that \( x \) and \( y \) are real numbers such that
        \[
            x(1 + i) + y(1 - i) = (x + y) + (x - y)i = 0.
        \]
        The real part of the left-hand side is \( x + y \) and the imaginary part is \( x - y \). A complex number is zero if and only if both its real and imaginary parts are zero, so we must have
        \[
            x + y = 0 \text{ and } x - y = 0 \iff x = y = 0.
        \]
        Hence the list \( (1 + i), (1 - i) \) is linearly independent.

        \item Observe that \( i(1 - i) = 1 + i \), so that \( 1 + i \) is a scalar multiple of \( 1 - i \). Then by \Cref{ex:2} (b), the list \( (1 + i), (1 - i) \) is linearly dependent.
    \end{enumerate}
\end{solution}

\begin{exercise}
\label{ex:6}
    Suppose \( v_1, v_2, v_3, v_4 \) is linearly independent in \( V \). Prove that the list
    \[
        v_1 - v_2, v_2 - v_3, v_3 - v_4, v_4
    \]
    is also linearly independent.
\end{exercise}

\begin{solution}
    Suppose that \( a_1, a_2, a_3, a_4 \) are scalars such that
    \[
        a_1 (v_1 - v_2) + a_2 (v_2 - v_3) + a_3 (v_3 - v_4) + a_4 v_4 = 0.
    \]
    This is equivalent to
    \[
        a_1 v_1 + (a_2 - a_1) v_2 + (a_3 - a_2) v_3 + (a_4 - a_3) v_4 = 0.
    \]
    Since the list \( v_1, v_2, v_3, v_4 \) is linearly independent, we must have
    \[
        a_1 = a_2 - a_1 = a_3 - a_2 = a_4 - a_3 = 0,
    \]
    which implies that \( a_1 = a_2 = a_3 = a_4 = 0 \). It follows that the list \( v_1 - v_2, v_2 - v_3, v_3 - v_4, v_4 \) is linearly independent.
\end{solution}

\begin{exercise}
\label{ex:7}
    Prove or give a counterexample: If \( v_1, v_2, \ldots, v_m \) is a linearly independent list of vectors in \( V \), then
    \[
        5 v_1 - 4 v_2, v_2, v_3, \ldots, v_m
    \]
    is linearly independent.
\end{exercise}

\begin{solution}
    (Assuming \( m \geq 2 \)). Suppose that \( a_1, a_2, \ldots, a_m \) are scalars such that
    \[
        a_1 (5 v_1 - 4 v_2) + a_2 v_2 + a_3 v_3 + \cdots + a_m v_m = 0,
    \]
    which is equivalent to
    \[
        5 a_1 v_1 + (a_2 - 4 a_1) v_2 + a_3 v_3 + \cdots + a_m v_m = 0.
    \]
    Since the list \( v_1, v_2, \ldots, v_m \) is linearly independent, this implies that
    \[
        5 a_1 = a_2 - 4 a_1 = a_3 = \cdots = a_m = 0,
    \]
    which implies that \( a_1 = a_2 = a_3 = \cdots = a_m = 0 \). It follows that the list \( 5 v_1 - 4 v_2, v_2, v_3, \ldots, v_m \) is linearly independent.
\end{solution}

\begin{exercise}
\label{ex:8}
    Prove or give a counterexample: If \( v_1, v_2, \ldots, v_m \) is a linearly independent list of vectors in \( V \) and \( \lambda \in \F \) with \( \lambda \neq 0 \), then \( \lambda v_1, \lambda v_2, \ldots, \lambda v_m \) is linearly independent.
\end{exercise}

\begin{solution}
    Suppose that \( a_1, a_2, \ldots, a_m \) are scalars such that
    \[
        a_1 \lambda v_1 + a_2 \lambda v_2 + \cdots + a_m \lambda v_m = 0.
    \]
    Since \( \lambda \neq 0 \), we may multiply both sides of this equation by \( \lambda^{-1} \) to obtain
    \[
        a_1 v_1 + a_2 v_2 + \cdots + a_m v_m = 0.
    \]
    Since the list \( v_1, v_2, \ldots, v_m \) is linearly independent, this implies that \( a_1 = a_2 = \cdots = a_m = 0 \). It follows that the list \( \lambda v_1, \lambda v_2, \ldots, \lambda v_m \) is linearly independent.
\end{solution}

\begin{exercise}
\label{ex:9}
    Prove or give a counterexample: If \( v_1, \ldots, v_m \) and \( w_1, \ldots, w_m \) are linearly independent lists of vectors in \( V \), then \( v_1 + w_1, \ldots, v_m + w_m \) is linearly independent.
\end{exercise}

\begin{solution}
    This is false. Consider \( \R \) as a vector space over itself and the two linearly independent lists \( 1 \) and \( -1 \). Then the list \( 1 + (-1) = 0 \) is linearly dependent.
\end{solution}

\begin{exercise}
\label{ex:10}
    Suppose \( v_1, \ldots, v_m \) is linearly independent in \( V \) and \( w \in V \). Prove that if \( v_1 + w, \ldots, v_m + w \) is linearly dependent, then \( w \in \Span (v_1, \ldots, v_m) \).
\end{exercise}

\begin{solution}
    By the Linear Dependence Lemma, there is a \( j \in \{ 1, 2, \ldots, m \} \) such that \( v_j + w \in \Span (v_1 + w, \ldots, v_{j-1} + w) \). If \( j = 1 \), then \( v_1 + w \in \Span () = \{ 0 \} \), so that \( w = -v_1 \) and hence \( w \in \Span (v_1, \ldots, v_m) \). If \( j \geq 2 \), then there are scalars \( a_1, \ldots, a_{j-1} \) such that
    \[
        v_j + w = a_1 (v_1 + w) + \cdots + a_{j-1} (v_{j-1} + w),
    \]
    which is equivalent to
    \[
        v_j + \lambda w = a_1 v_1 + \cdots + a_{j-1} v_{j-1},
    \]
    where \( \lambda = 1 - (a_1 + \cdots + a_{j-1}) \). Note that \( \lambda \) must be non-zero; if this were not the case, then \( v_j \) would lie in the span of \( v_1, \ldots, v_{j-1} \), which cannot happen since the list \( v_1, \ldots, v_j \) is linearly independent. It follows that
    \[
        w = \lambda^{-1} (a_1 v_1 + \cdots + a_{j-1} v_{j-1} - v_j),
    \]
    whence \( w \in \Span (v_1, \ldots, v_m) \).
\end{solution}

\begin{exercise}
\label{ex:11}
    Suppose \( v_1, \ldots, v_m \) is linearly independent in \( V \) and \( w \in V \). Show that \( v_1, \ldots, v_m, w \) is linearly independent if and only if
    \[
        w \not\in \Span (v_1, \ldots, v_m).
    \]
\end{exercise}

\begin{solution}
    If \( w \in \Span (v_1, \ldots, v_m) \), then the list \( v_1, \ldots, v_m, w \) is linearly dependent (third bullet point of Example 2.20). Conversely, suppose that the list \( v_1, \ldots, v_m, w \) is linearly dependent. By the Linear Dependence Lemma, one of the vectors in the list must be in the span of the previous ones. It cannot be the case that some \( v_j \) belongs to \( \Span (v_1, \ldots, v_{j-1}) \) since that would contradict the linear independence of the list \( v_1, \ldots, v_m \), so it must be the case that \( w \in \Span (v_1, \ldots, v_m) \).
\end{solution}

\begin{exercise}
\label{ex:12}
    Explain why there does not exist a list of six polynomials that is linearly independent in \( \mathcal{P}_4 (\F) \).
\end{exercise}

\begin{solution}
    It is easily verified that \( \mathcal{P}_4 (\F) \) is spanned by the list \( 1, z, z^2, z^3, z^4 \), which has length 5. Then by (2.23), any linearly independent list in \( \mathcal{P}_4 (\F) \) can have length at most 5.
\end{solution}

\begin{exercise}
\label{ex:13}
    Explain why no list of four polynomials spans \( \mathcal{P}_4 (\F) \).
\end{exercise}

\begin{solution}
    It is easily verified that the list \( 1, z, z^2, z^3, z^4 \) is linearly independent in \( \mathcal{P}_4 (\F) \). Then by (2.23), any list of vectors which spans \( \mathcal{P}_4 (\F) \) must have length at least 5.
\end{solution}

\begin{exercise}
\label{ex:14}
    Prove that \( V \) is infinite-dimensional if and only if there is a sequence \( v_1, v_2, \ldots \) of vectors in \( V \) such that \( v_1, \ldots, v_m \) is linearly independent for every positive integer \( m \).
\end{exercise}

\begin{solution}
    Suppose that \( V \) is finite-dimensional, so that it is spanned by some list \( w_1, \ldots, w_m \) of vectors, and let \( v_1, v_2, \ldots \) be any sequence of vectors in \( V \). Then by (2.23), the list \( v_1, v_2, \ldots, v_{m+1} \) cannot be linearly independent.
    
    Now suppose that \( V \) is infinite-dimensional, so that no list of vectors in \( V \) spans \( V \). Certainly \( V \neq \{ 0 \} \), so pick any \( v_1 \neq 0 \) in \( V \); the list \( v_1 \) is linearly independent. Suppose that after \( m \) steps, we have chosen vectors \( v_1, \ldots, v_m \) such that the list \( v_1, \ldots, v_m \) is linearly independent. Since \( V \neq \Span (v_1, \ldots, v_m) \), pick any \( v_{m+1} \not\in \Span (v_1, \ldots, v_m) \). By \Cref{ex:11}, the list \( v_1, \ldots, v_m, v_{m+1} \) is linearly independent. In this way, we inductively obtain a sequence \( v_1, v_2, \ldots \) such that \( v_1, \ldots, v_m \) is linearly independent for each positive integer \( m \).
\end{solution}

\begin{exercise}
\label{ex:15}
    Prove that \( \F^{\infty} \) is infinite-dimensional.
\end{exercise}

\begin{solution}
    Consider the sequence of vectors \( v_1, v_2, \ldots \), where \( v_i \) is the vector with a \( 1 \) in the \( i \)\ts{th} position and \( 0 \)'s elsewhere. For each positive integer \( m \), it is not hard to see that the list \( v_1, \ldots, v_m \) is linearly independent. It follows from \Cref{ex:14} that \( \F^{\infty} \) is infinite-dimensional.
\end{solution}

\begin{exercise}
\label{ex:16}
    Prove that the real vector space of all continuous real-valued functions on the interval \( [0, 1] \) is infinite-dimensional.
\end{exercise}

\begin{solution}
    Consider the sequence of continuous functions \( f_1, f_2, \ldots \), where \( f_1(x) = 1 \) for all \( x \in [0, 1] \) and for \( i \geq 2 \), \( f_i : [0, 1] \to \R \) is given by \( f_i(x) = 1 \) for \( x \in [0, 1/i] \), \( f_i(x) = 0 \) for \( x \in [1/(i-1), 1] \). On the interval \( (1/i, 1/(i-1)) \), \( f_i \) is a straight line segment joining the points \( (1/i, 1) \) and \( (1/(i-1), 0) \) to give a continuous function (see \Cref{fig:1}).

    \begin{figure}[h]
        \centering
        \begin{tikzpicture}
            \draw[->] (0,0) -- (6,0);
            \draw[->] (0,0) -- (0,6);
            \draw[very thick] (0,5) -- (2,5) -- (3,0) -- (5,0);
            \draw[dashed] (2,5) -- (2,0);
            \filldraw (0,5) circle [radius=0.05] node[anchor=east] {$1$} (2,5) circle [radius=0.05] (2,0) circle [radius=0.05] node[anchor=north] {$\frac{1}{i}$} (3,0) circle [radius=0.05] node[anchor=north] {$\frac{1}{i-1}$} (5,0) circle [radius=0.05] node[anchor=north] {$1$};
        \end{tikzpicture}
        \caption{$f_i$} \label{fig:1}
    \end{figure}
    
    \noindent Let \( m \) be a positive integer and suppose we have real numbers \( a_1, \ldots, a_m \) such that
    \[
        a_1 f_1(x) + a_2 f_2(x) + \cdots + a_m f_m(x) = 0
    \]
    for all \( x \in [0, 1] \). In particular, taking \( x = 1 \) gives
    \[
        a_1 f_1(1) + a_2 f_2(1) + \cdots + a_m f_m(1) = a_1 = 0.
    \]
    Then taking \( x = 1/2 \) gives
    \[
        a_2 f_2(1/2) + a_3 f_3(1/2) + \cdots + a_m f_m(1/2) = a_2 = 0.
    \]
    Continuing in this way, taking \( x = 1/i \) for each \( i \in \{ 1, 2, \ldots, m \} \), we see that \( a_1 = a_2 = \cdots = a_m = 0 \). It follows that \( f_1, f_2, \ldots, f_m \) is a linearly independent list for each positive integer \( m \) and hence by \Cref{ex:14} the vector space in question is infinite-dimensional.
\end{solution}

\begin{exercise}
\label{ex:17}
    Suppose \( p_0, p_1, \ldots, p_m \) are polynomials in \( \mathcal{P}_m (\F) \) such that \( p_j(2) = 0 \) for each \( j \). Prove that \( p_0, p_1, \ldots, p_m \) is not linearly independent in \( \mathcal{P}_m (\F) \).
\end{exercise}

\begin{solution}
    Consider the list \( p_0, p_1, \ldots, p_m, f \) of length \( m + 2 \), where \( f(x) = x \). Since \( \mathcal{P}_m (\F) \) is spanned by a list of length \( m + 1 \) (Example 2.14), (2.23) implies that the list \( p_0, p_1, \ldots, p_m, f \) is linearly dependent. Suppose that \( f \) belongs to \( \Span (p_0, p_1, \ldots, p_m) \), i.e.\ there are scalars \( a_0, a_1, \ldots, a_m \) such that
    \[
        x = a_0 p_0(x) + a_1 p_1(x) + \cdots + a_m p_m(x)
    \]
    for all \( x \in \F \). In particular,
    \[
        2 = a_0 p_0(2) + a_1 p_1(2) + \cdots + a_m p_m(2) = 0,
    \]
    which is a contradiction; it follows that \( f \not\in \Span (p_0, p_1, \ldots, p_m) \). Hence by the Linear Dependence Lemma, there exists some \( j \in \{ 0, 1, \ldots, m \} \) such that \( p_j \in \Span (p_0, p_1, \ldots, p_{j-1}) \). The third bullet point of Example 2.20 now implies that the list \( p_0, p_1, \ldots, p_m \) is linearly dependent.
\end{solution}

\noindent \hrulefill

\noindent \hypertarget{ladr}{\textcolor{blue}{[LADR]} Axler, S. (2015) \textit{Linear Algebra Done Right.} 3rd edn.}

\end{document}