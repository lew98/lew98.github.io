\documentclass[12pt]{article}
\usepackage[utf8]{inputenc}
\usepackage[utf8]{inputenc}
\usepackage{amsmath}
\usepackage{amsthm}
\usepackage{geometry}
\usepackage{amsfonts}
\usepackage{mathrsfs}
\usepackage{bm}
\usepackage{hyperref}
\usepackage[dvipsnames]{xcolor}
\usepackage{enumitem}
\usepackage{changepage}
\usepackage{lipsum}
\usepackage{tikz}
\usetikzlibrary{matrix}
\usepackage{tikz-cd}
\usepackage[nameinlink]{cleveref}
\geometry{
headheight=15pt,
left=60pt,
right=60pt
}
\usepackage{fancyhdr}
\pagestyle{fancy}
\fancyhf{}
\lhead{}
\chead{Section 1.6 Exercises}
\rhead{\thepage}
\hypersetup{
    colorlinks=true,
    linkcolor=blue,
    urlcolor=blue
}

\theoremstyle{definition}
\newtheorem*{remark}{Remark}

\newtheoremstyle{exercise}
    {}
    {}
    {}
    {}
    {\bfseries}
    {.}
    { }
    {\thmname{#1}\thmnumber{#2}\thmnote{ (#3)}}
\theoremstyle{exercise}
\newtheorem{exercise}{Exercise 1.6.}

\newtheoremstyle{solution}
    {}
    {}
    {}
    {}
    {\itshape\color{magenta}}
    {.}
    { }
    {\thmname{#1}\thmnote{ #3}}
\theoremstyle{solution}
\newtheorem*{solution}{Solution}

\Crefformat{exercise}{#2Exercise 1.6.#1#3}

\newcommand{\setcomp}[1]{#1^{\mathsf{c}}}
\newcommand{\N}{\mathbf{N}}
\newcommand{\Z}{\mathbf{Z}}
\newcommand{\Q}{\mathbf{Q}}
\newcommand{\R}{\mathbf{R}}
\newcommand{\C}{\mathbf{C}}

\setlist[enumerate,1]{label={(\alph*)}}

\begin{document}

\section{Section 1.6 Exercises}

Exercises with solutions from Section 1.6 of \hyperlink{ua}{[UA]}.

\begin{exercise}
\label{ex:1}
    Show that \( (0, 1) \) is uncountable if and only if \( \R \) is uncountable. This shows that Theorem 1.6.1 is equivalent to Theorem 1.5.6.
\end{exercise}

\begin{solution}
    We have \( (0, 1) \sim \R \) by \href{https://lew98.github.io/Mathematics/UA_Section_1_5_Exercises.pdf}{Exercise 1.5.4} (a).
\end{solution}

\begin{exercise}
\label{ex:2}
    \begin{enumerate}
        \item Explain why the real number \( x = .b_1 b_2 b_3 b_4 \ldots \) cannot be \( f(1) \).

        \item Now, explain why \( x \neq f(2) \), and in general why \( x \neq f(n) \) for any \( n \in \N \).

        \item Point out the contradiction that arises from these observations and conclude that \( (0, 1) \) is uncountable.
    \end{enumerate}
\end{exercise}

\begin{solution}
    \begin{enumerate}
        \item We have decimal expansions
        \[
            f(1) = .a_{11} a_{12} a_{13} a_{14} \ldots \quad \text{and} \quad x = .b_1 b_2 b_3 b_4 \ldots.
        \]
        By construction, \( b_1 \neq a_{11} \). This implies that \( f(1) \neq x \), provided these decimal expansions are not two different representations of the same real number (for example, .3 and .2999 \ldots). However, since the only way this can occur is when one decimal expansion terminates in repeating 0's and the other terminates in repeating 9's, and the digits \( b_n \) are always either 2 or 3, we see that the \( .b_1 b_2 b_3 b_4 \ldots \) must be the unique decimal representation of a real number.

        \item Since \( .b_1 b_2 b_3 b_4 \ldots \) is the unique decimal expansion of a real number (see part (a)) and \( b_n \neq a_{nn} \), we have \( x \neq f(n) \) for every \( n \in \N \).

        \item The real number \( x \) belongs to \( (0, 1) \) but not to the image of \( f \). This contradicts our assumption that \( f \) was onto. It follows that there cannot exist a 1-1 and onto function between \( \N \) and \( (0, 1) \). Since \( (0, 1) \) is clearly infinite, we may conclude that \( (0,1 ) \) is uncountable.
    \end{enumerate}
\end{solution}

\begin{exercise}
\label{ex:3}
    Supply rebuttals to the following complaints about the proof of Theorem 1.6.1.
    \begin{enumerate}
        \item Every rational number has a decimal expansion, so we could apply this same argument to show that the set of rational numbers between 0 and 1 is uncountable. However, because we know that any subset of \( \Q \) must be countable, the proof of Theorem 1.6.1 must be flawed.

        \item Some numbers have \textit{two} different decimal representations. Specifically, any decimal expansion that terminates can also be written with repeating 9's. For instance, \( 1/2 \) can also be written as \( .5 \) or as \( .4999 \ldots \). Doesn't this cause some problems?
    \end{enumerate}
\end{exercise}

\begin{solution}
    \begin{enumerate}
        \item The problem with this reasoning is that the real number
        \[
            x = .b_1 b_2 b_3 b_4 \ldots
        \]
        that we construct may not be rational. For example, consider the function \( f : \N \to (0, 1) \cap \Q \) given by
        \[
        \begin{aligned}
            f(1) &= .3, \\
            f(2) &= .02, \\
            f(3) &= .003, \\
            f(4) &= .0003, \\
            f(5) &= .00002, \\
        \end{aligned}
        \qquad
        \begin{aligned}
            f(6) &= .000003, \\
            f(7) &= .0000003, \\
            f(8) &= .00000003, \\
            f(9) &= .000000002, \\
            f(10) &= .0000000003, \\
        \end{aligned}
        \qquad \cdots
        \]
        This results in \( x = .2322322232 \ldots \), which is not rational since its decimal expansion does not repeat. So while \( x \) does not belong to the image of \( f \), this is not a problem because \( x \) does not belong to \( (0, 1) \cap \Q \) either.

        \item We have addressed half of this issue in \Cref{ex:2} part (a). The only other place this could cause problems is when we represent \( f(m) \) with the decimal expansion
        \[
            f(m) = . a_{m1} a_{m2} a_{m3} a_{m4} \ldots.
        \]
        We can avoid any issues by simply choosing the expansion which terminates in 0's if \( f(m) \) is a real number with two decimal expansions.
    \end{enumerate}
\end{solution}

\begin{exercise}
\label{ex:4}
    Let \( S \) be the set consisting of all sequences of 0's and 1's. Observe that \( S \) is not a particular sequence, but rather a large set whose elements are sequences; namely
    \[
        S = \{ (a_1, a_2, a_3, \ldots) : a_n = 0 \text{ or } 1 \}.
    \]
    As an example, the sequence \( (1, 0, 1, 0, 1, 0, 1, 0, \ldots) \) is an element of \( S \), as is the sequence \( (1, 1, 1, 1, 1, 1, \ldots) \).

    Give a rigorous argument showing that \( S \) is uncountable.
\end{exercise}

\begin{solution}
    Suppose that \( f : \N \to S \) is 1-1 and onto. For each \( m \in \N \), let \( a_{mn} \) be the element in the \( n \)th position of \( f(m) \), so that
    \[
        f(m) = (a_{m1}, a_{m2}, a_{m3}, a_{m4}, \ldots) \in S.
    \]
    Let \( b = (b_1, b_2, b_3, b_4, \ldots) \) be the sequence given by
    \[
        b_n = \begin{cases}
            0 & \text{if } a_{nn} = 1, \\
            1 & \text{if } a_{nn} = 0.
        \end{cases}
    \]
    Then \( b \in S \) but \( b \neq f(n) \) for any \( n \in \N \), since \( b \) differs from \( f(n) \) in the \( n \)th position. This is a contradiction since we assumed that \( f \) was onto. Hence there can be no 1-1 and onto function between \( \N \) and \( S \). It is clear that \( S \) is infinite, so we may conclude that \( S \) is uncountable.
\end{solution}

\begin{exercise}
\label{ex:5}
    \begin{enumerate}
        \item Let \( A = \{ a, b, c \} \). List the eight elements of \( P(A) \). (Do not forget that \( \emptyset \) is considered to be a subset of every set.)

        \item If \( A \) is finite with \( n \) elements, show that \( P(A) \) has \( 2^n \) elements.
    \end{enumerate}
\end{exercise}

\begin{solution}
    \begin{enumerate}
        \item We have
        \[
            P(A) = \{ \emptyset, \{ a \}, \{ b \}, \{ c \}, \{ a, b \}, \{ a, c \}, \{ b, c \}, \{ a, b, c \} \}.
        \]

        \item To form a subset \( B \) of \( A \), for each element \( a \in A \), we must decide whether to include \( a \) in \( B \) or not. This is a binary choice to be made for each of the \( n \) elements of \( A \); it follows that there are \( 2^n \) subsets of \( A \).
    \end{enumerate}
\end{solution}

\begin{exercise}
\label{ex:6}
    \begin{enumerate}
        \item Using the particular set \( A = \{ a, b, c \} \), exhibit two different 1-1 mappings from \( A \) into \( P(A) \).

        \item Letting \( C = \{ 1, 2, 3, 4 \} \), produce an example of a 1-1 map \( g : C \to P(C) \).

        \item Explain why, in parts (a) and (b), it is impossible to construct mappings that are \textit{onto}.
    \end{enumerate}
\end{exercise}

\begin{solution}
    \begin{enumerate}
        \item Here are two 1-1 functions \( f : A \to P(A) \) and \( g : A \to P(A) \).
        \[
            \begin{aligned}
                f(a) &= \{ a \}, \\
                f(b) &= \{ b \}, \\
                f(c) &= \{ c \},
            \end{aligned}
            \qquad
            \begin{aligned}
                g(a) &= \{ a, b \}, \\
                g(b) &= \{ b, c \}, \\
                g(c) &= \{ a, c \}.
            \end{aligned}
        \]

        \item Let \( g \) be given by
        \[
            \begin{aligned}
                g(1) &= \{ 1 \}, \\
                g(2) &= \{ 2 \},
            \end{aligned}
            \qquad
            \begin{aligned}
                g(3) &= \{ 3 \}, \\
                g(4) &= \{ 4 \}.
            \end{aligned}
        \]

        \item If \( A \) has \( n \) elements, then by \Cref{ex:5} (b), \( P(A) \) has \( 2^n \) elements. So the power set of a finite set always contains strictly more elements than that finite set. For finite sets, it is impossible to construct an onto function from a set \( A \) to a set \( B \) if \( B \) contains strictly more elements than \( A \).
    \end{enumerate}
\end{solution}

\begin{exercise}
\label{ex:7}
    Return to the particular functions constructed in \Cref{ex:6} and construct the subset \( B \) that results using the preceding rule. In each case, note that \( B \) is not in the range of the function used.
\end{exercise}

\begin{solution}
    For all three functions from \Cref{ex:6}, we have \( B =  \emptyset \), which does not belong to the range of any of the functions.
\end{solution}

\begin{exercise}
\label{ex:8}
    \begin{enumerate}
        \item First, show that the case \( a' \in B \) leads to a contradiction

        \item Now, finish the argument by showing that the case \( a' \not\in B \) is equally unacceptable.
    \end{enumerate}
\end{exercise}

\begin{solution}
    \begin{enumerate}
        \item and (b). We have \( a' \in B \) if and only if \( a' \not\in f(a') = B \), which is clearly a contradiction since \( a' \) either does or does not belong to \( B \).
    \end{enumerate}
\end{solution}

\begin{exercise}
\label{ex:9}
    Using the various tools and techniques developed in the last two sections (including the exercises from Section 1.5), give a compelling argument showing that \( P(\N) \sim \R \).
\end{exercise}

\begin{solution}
    First, let us show that \( P(\N) \sim S \), where \( S \) is the set of all binary sequences defined in \Cref{ex:4}. Consider the function \( f : P(\N) \to S \) given by \( f(E) = (a_1, a_2, a_3, \ldots) \) where
    \[
        a_n = \begin{cases}
            1 & \text{if } n \in E, \\
            0 & \text{if } n \not\in E.
        \end{cases}
    \]
    This function is 1-1 and onto since it has an inverse \( f^{-1} : S \to P(\N) \) given by \( f^{-1}(a_1, a_2, a_3, \ldots) = \{ n \in \N : a_n = 1 \} \).

    Now let us show that \( S \sim (0, 1) \). Consider the function \( g : S \to (0, 1) \) given by
    \[
        g(a_1, a_2, a_3, \ldots) = 0.5 a_1 a_2 a_3 \ldots,
    \]
    where \( 0.5 a_1 a_2 a_3 \ldots \) is a decimal expansion (for example, \( g(1, 0, 1, 0, 0, 0, \ldots) = 0.5101 \)). This function is 1-1 since if \( a = (a_1, a_2, a_3, \ldots) \neq b = (b_1, b_2, b_3, \ldots) \), there must exist some \( n \in \N \) such that \( a_n \neq b_n \). It follows that \( g(a) \neq g(b) \), provided \( g(a) = 0.5 a_1 a_2 a_3 \ldots  \) and \( g(b) = 0.5 b_1 b_2 b_3 \ldots \) are not two different decimal expansions of the same real number. This cannot be the case since each \( a_i \) and \( b_i \) is either 0 or 1, and never 9.

    Now consider the function \( h : (0, 1) \to S \) given by
    \[
        h(x) = h(0.a_1 a_2 a_3 \ldots) = (a_1, a_2, a_3, \ldots),
    \]
    where \( 0.a_1 a_2 a_3 \ldots \) is the \textbf{binary} expansion of \( x \in (0, 1) \), choosing that expansion which terminates in 0's if \( x \) has two different binary expansions. This function is 1-1 since if \( x = 0.a_1 a_2 a_3 \ldots \neq y = 0.b_1 b_2 b_3 \ldots \), then there must be some \( n \in \N \) such that \( a_n \neq b_n \). It follows that \( h(x) \neq h(y) \).

    The Schröder-Bernstein theorem (see \href{https://lew98.github.io/Mathematics/UA_Section_1_5_Exercises.pdf}{Exercise 1.5.11}) now implies that \( S \sim (0, 1) \). We showed in \href{https://lew98.github.io/Mathematics/UA_Section_1_5_Exercises.pdf}{Exercise 1.5.4} that \( (0, 1) \sim \R \) and in \href{https://lew98.github.io/Mathematics/UA_Section_1_5_Exercises.pdf}{Exercise 1.5.5} that \( \sim \) is an equivalence relation, so we may conclude that \( P(\N) \sim \R \).
\end{solution}

\begin{exercise}
\label{ex:10}
    As a final exercise, answer each of the following by establishing a 1-1 correspondence with a set of known cardinality.
    \begin{enumerate}
        \item Is the set of all functions from \( \{ 0, 1 \} \) to \( \N \) countable or uncountable?

        \item Is the set of all functions from \( \N \) to \( \{ 0, 1 \} \) countable of uncountable?

        \item Given a set \( B \), a subset \( \mathcal{A} \) of \( P(B) \) is called an \textit{antichain} if no element of \( \mathcal{A} \) is a subset of any other element of \( \mathcal{A} \). Does \( P(\N) \) contain an uncountable antichain?
    \end{enumerate}
\end{exercise}

\begin{solution}
    \begin{enumerate}
        \item Let \( \N^{\{0,1\}} \) be the set of all functions from \( \{ 0, 1 \} \) to \( \N \). Consider the function \( F : \N^{\{0,1\}} \to \N \times \N \) given by \( F(f) = (f(0), f(1)) \). This function is 1-1 and onto since it has an inverse \( F^{-1} : \N \times \N \to \N^{\{0,1\}} \) given by \( F^{-1}(a, b) = f \), where \( f : \{ 0, 1 \} \to \N \) is the function satisfying \( f(0) = a, f(1) = b \). The product of two countable sets is again countable (see \href{https://lew98.github.io/Mathematics/Cardinality.pdf}{here}, for example), so it follows that \( \N^{\{0,1\}} \) is countable.

        \item The set of all functions from \( \N \) to \( \{ 0, 1 \} \) is nothing but the set of all binary sequences \( S \) defined in \Cref{ex:4}, since a function \( f : \N \to \{ 0, 1 \} \) can be identified with the sequence \( (f(0), f(1), f(2), \ldots) \). So the set of all functions from \( \N \) to \( \{ 0, 1 \} \) is uncountable, since we showed that \( S \) is uncountable in \Cref{ex:4}.

        \item To answer this, let us first state and prove the following lemma.

        \noindent \textbf{Lemma 1.} Suppose \( A \) and \( B \) are sets and \( f : A \to B \) is 1-1. Then if \( \mathcal{A} \subseteq P(A) \) is an antichain, so is \( \mathcal{A}' := \{ f(X) : X \in \mathcal{A} \} \subseteq P(B) \).

        \noindent \textit{Proof.} Suppose we have two elements \( f(X) \) and \( f(Y) \) in \( \mathcal{A}' \), where \( X \) and \( Y \) belong to \( \mathcal{A} \). Since \( \mathcal{A} \) is an antichain, we have \( X \not\subseteq Y \), which can be the case if and only if there is some \( x \in X \) such that \( x \not\in Y \). Then since \( f \) is 1-1, we have \( f(x) \in f(X) \) but \( f(x) \not\in f(Y) \). It follows that \( f(X) \) is not a subset of \( f(Y) \) and we may conclude that \( \mathcal{A}' \) is an antichain. \qed

        Now let us return to the exercise. Consider the following collection of subsets of \( P(\Q) \):
        \[
            \mathcal{A} := \{ (a, a + 1) \cap \Q : a \in \R \}.
        \]
        By the density of \( \Q \) in \( \R \), for real numbers \( a \neq b \) we have \( ((a, a + 1) \cap \Q) \not\subseteq ((b, b + 1) \cap \Q) \), i.e.\ \( \mathcal{A} \) is an antichain. This also implies that each \( a \in \R \) gives rise to a distinct element of \( \mathcal{A} \). Since \( \R \) is uncountable, it follows that \( \mathcal{A} \) is also uncountable (indeed, \( \R \sim \mathcal{A} \)). Then since \( \Q \sim \N \), we may invoke Lemma 1 above to obtain an uncountable antichain \( \mathcal{A}' \subseteq P(\N) \).
    \end{enumerate}
\end{solution}

\noindent \hrulefill

\noindent \hypertarget{ua}{\textcolor{blue}{[UA]} Abbott, S. (2015) \textit{Understanding Analysis.} 2nd edn.}

\end{document}
