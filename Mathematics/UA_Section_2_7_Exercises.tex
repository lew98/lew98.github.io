\documentclass[12pt]{article}
\usepackage[utf8]{inputenc}
\usepackage[utf8]{inputenc}
\usepackage{amsmath}
\usepackage{amsthm}
\usepackage{geometry}
\usepackage{amsfonts}
\usepackage{mathrsfs}
\usepackage{bm}
\usepackage{hyperref}
\usepackage[dvipsnames]{xcolor}
\usepackage[inline]{enumitem}
\usepackage{mathtools}
\usepackage{changepage}
\usepackage{lipsum}
\usepackage{tikz}
\usetikzlibrary{matrix}
\usepackage{tikz-cd}
\usepackage[nameinlink]{cleveref}
\geometry{
headheight=15pt,
left=60pt,
right=60pt
}
\setlength{\emergencystretch}{10pt}
\usepackage{fancyhdr}
\pagestyle{fancy}
\fancyhf{}
\lhead{}
\chead{Section 2.7 Exercises}
\rhead{\thepage}
\hypersetup{
    colorlinks=true,
    linkcolor=blue,
    urlcolor=blue
}

\theoremstyle{definition}
\newtheorem*{remark}{Remark}

\newtheoremstyle{exercise}
    {}
    {}
    {}
    {}
    {\bfseries}
    {.}
    { }
    {\thmname{#1}\thmnumber{#2}\thmnote{ (#3)}}
\theoremstyle{exercise}
\newtheorem{exercise}{Exercise 2.7.}

\newtheoremstyle{solution}
    {}
    {}
    {}
    {}
    {\itshape\color{magenta}}
    {.}
    { }
    {\thmname{#1}\thmnote{ #3}}
\theoremstyle{solution}
\newtheorem*{solution}{Solution}

\Crefformat{exercise}{#2Exercise 2.7.#1#3}

\newcommand{\setcomp}[1]{#1^{\mathsf{c}}}
\newcommand{\N}{\mathbf{N}}
\newcommand{\Z}{\mathbf{Z}}
\newcommand{\Q}{\mathbf{Q}}
\newcommand{\R}{\mathbf{R}}
\newcommand{\C}{\mathbf{C}}

\DeclarePairedDelimiter\abs{\lvert}{\rvert}
% Swap the definition of \abs* and \norm*, so that \abs
% and \norm resizes the size of the brackets, and the 
% starred version does not.
\makeatletter
\let\oldabs\abs
\def\abs{\@ifstar{\oldabs}{\oldabs*}}
%
\let\oldnorm\norm
\def\norm{\@ifstar{\oldnorm}{\oldnorm*}}
\makeatother

\setlist[enumerate,1]{label={(\alph*)}}

\begin{document}

\section{Section 2.7 Exercises}

Exercises with solutions from Section 2.7 of \hyperlink{ua}{[UA]}.

\begin{exercise}
\label{ex:1}
    Proving the Alternating Series Test (Theorem 2.7.7) amounts to showing that the sequence of partial sums
    \[
        s_n = a_1 - a_2 + a_3 - \cdots \pm a_n
    \]
    converges. (The opening example in Section 2.1 includes a typical illustration of \( (s_n) \).) Different characterizations of completeness lead to different proofs.
    \begin{enumerate}
        \item Prove the Alternating Series Test by showing that \( (s_n) \) is a Cauchy sequence.

        \item Supply another proof for this result using the Nested Interval Property (Theorem 1.4.1).

        \item Consider the subsequences \( (s_{2n}) \) and \( (s_{2n+1}) \), and show how the Monotone Convergence Theorem leads to a third proof for the Alternating Series Test.
    \end{enumerate}
\end{exercise}

\begin{solution}
    First note that since \( (a_n) \) is decreasing and converges to zero, \( a_n \geq 0 \) and \( a_n - a_{n+1} \geq 0 \) for any \( n \in \N \).
    \begin{enumerate}
        \item Suppose \( n > m \) are positive integers. If \( n - m \) is even, then
        \[
            \underbrace{a_{m+1} - a_{m+2}}_{\geq 0} + \underbrace{a_{m+3} - a_{m+4}}_{\geq 0} + \cdots + \underbrace{a_{n-1} - a_n}_{\geq 0} \geq 0,
        \]
        and if \( n - m \) is odd, then
        \[
            \underbrace{a_{m+1} - a_{m+2}}_{\geq 0} + \underbrace{a_{m+3} - a_{m+4}}_{\geq 0} + \cdots + \underbrace{a_{n-2} - a_{n-1}}_{\geq 0} + \underbrace{a_n}_{\geq 0} \geq 0.
        \]
        It follows that \( \abs{s_n - s_m} = a_{m+1} - a_{m+2} + \cdots \pm a_n \). If \( n - m \) is even, then
        \[
            a_{m+1} + \underbrace{(-a_{m+2} + a_{m+3})}_{\leq 0} + \cdots + \underbrace{(-a_{n-2} + a_{n-1})}_{\leq 0} + \underbrace{(-a_n)}_{\leq 0} \leq a_{m+1},
        \]
        and if \( n - m \) is odd, then
        \[
            a_{m+1} + \underbrace{(-a_{m+2} + a_{m+3})}_{\leq 0} + \cdots + \underbrace{(-a_{n-1} + a_n)}_{\leq 0} \leq a_{m+1}.
        \]
        It follows that \( \abs{s_n - s_m} \leq a_{m+1} \). Let \( \epsilon > 0 \) be given. Since \( \lim_n a_n = 0 \), there is an \( N \in \N \) such that \( n \geq N \) implies that \( \abs{a_n} = a_n < \epsilon \). Then if we take \( n > m \geq N \) we will have
        \[
            \abs{s_n - s_m} \leq a_{m+1} < \epsilon.
        \]
        Hence \( (s_n) \) is a Cauchy sequence.

        \item Let \( n \) be a positive integer. Observe that
        \begin{align*}
            s_{2n-1} - s_{2n} = a_{2n} \geq 0 &\implies s_{2n} \leq s_{2n-1}, \\
            s_{2n-1} - s_{2n-3} = a_{2n-1} - a_{2n-2} \leq 0 &\implies s_{2n-1} \leq s_{2n-3}, \\
            s_{2n} - s_{2n-2} = a_{2n-1} - a_{2n} \geq 0 &\implies s_{2n-2} \leq s_{2n}.
        \end{align*}
        It follows that if we let \( I_n = [s_{2n}, s_{2n-1}] \), then \( (I_n) \) is a sequence of nested intervals. Hence by the Nested Interval Property, there exists some \( x \in \bigcap_{n=1}^{\infty} I_n \); we claim that \( \lim s_n = x \). To see this, suppose that \( n \in \N \). If \( n \) is even, then \( s_n \in I_{n/2} = [s_n, s_{n-1}] \) and so
        \[
            \abs{s_n - x} \leq \abs{I_{n/2}} = s_{n-1} - s_n = a_n.
        \]
        If \( n \) is odd, then \( s_n \in I_{(n+1)/2} = [s_{n+1}, s_n] \) and so
        \[
            \abs{s_n - x} \leq \abs{I_{(n+1)/2}} = s_n - s_{n+1} = a_{n+1} \leq a_n.
        \]
        It follows that for all \( n \in \N \) we have \( \abs{s_n - x} \leq a_n \). An application of the Squeeze Theorem yields \( \lim s_n = x \).

        \item As shown in (b), the sequence \( (s_{2n}) \) is increasing and bounded above by \( s_1 \), and the sequence \( (s_{2n+1}) \) is decreasing and bounded below by \( s_2 \). The Monotone Convergence Theorem then implies that \( \lim s_{2n} \) and \( \lim s_{2n+1} \) both exist. The relationship \( s_{2n+1} - s_{2n} = a_{2n+1} \) gives
        \[
            \lim s_{2n+1} - \lim s_{2n} = \lim a_{2n+1} = 0,
        \]
        so that \( (s_{2n}) \) and \( (s_{2n+1}) \) both converge to the same limit \( x \in \R \). We claim that \( \lim s_n = x \). To see this, let \( \epsilon > 0 \) be given. Then there are positive integers \( N_1 \) and \( N_2 \) such that
        \begin{equation}
            n \geq N_1 \implies \abs{s_{2n} - x} < \epsilon,
        \end{equation}
        \begin{equation}
            n \geq N_2 \implies \abs{s_{2n+1} - x} < \epsilon.
        \end{equation}
        Let \( N = \max \{ N_1, N_2 \} \) and suppose that \( n \in \N \) is such that \( n \geq 2N + 1 \). If \( n \) is even, then \( n/2 > N \geq N_1 \) and so \( \abs{s_n - x} < \epsilon \) by (1). If \( n \) is odd, then \( (n-1)/2 \geq N \geq N_2 \) and so \( \abs{s_n - x} < \epsilon \). Hence we have
        \[
            n \geq 2N + 1 \implies \abs{s_n - x} < \epsilon.
        \]
        It follows that \( \lim s_n = x \).
    \end{enumerate}
\end{solution}

\begin{exercise}
\label{ex:2}
    Decide whether each of the following series converges or diverges:

    \vspace{3mm}
    \hspace{-4.5mm}
    \( \text{(a)} \, \sum_{n=1}^{\infty} \tfrac{1}{2^n + n} \qquad \text{(b)} \, \sum_{n=1}^{\infty} \tfrac{\sin(n)}{n^2} \)
    \begin{enumerate}[start = 3]
        \item \( 1 - \tfrac{3}{4} + \tfrac{4}{6} - \tfrac{5}{8} + \tfrac{6}{10} - \tfrac{7}{12} + \cdots \)

        \item \( 1 + \tfrac{1}{2} - \tfrac{1}{3} + \tfrac{1}{4} + \tfrac{1}{5} - \tfrac{1}{6} + \tfrac{1}{7} + \tfrac{1}{8} - \tfrac{1}{9} + \cdots \)

        \item \( 1 - \tfrac{1}{2^2} + \tfrac{1}{3} - \tfrac{1}{4^2} + \tfrac{1}{5} - \tfrac{1}{6^2} + \tfrac{1}{7} - \tfrac{1}{8^2} + \cdots \)
    \end{enumerate}
\end{exercise}

\begin{solution}
    \begin{enumerate}
        \item Observe that for each \( n \in \N \) we have
        \[
            0 < \frac{1}{2^n + n} < \frac{1}{2^n}.
        \]
        Then since \( \sum_{n=1}^{\infty} \tfrac{1}{2^n} = 1 \), the Comparison Test implies that \( \sum_{n=1}^{\infty} \tfrac{1}{2^n + n} \) is convergent.

        \item Observe that for each \( n \in \N \) we have
        \[
            0 < \frac{\abs{\sin(n)}}{n^2} < \frac{1}{n^2}.
        \]
        Then since \( \sum_{n=1}^{\infty} \tfrac{1}{n^2} \) is convergent (Example 2.4.4), the Comparison Test implies that \( \sum_{n=1}^{\infty} \tfrac{\sin(n)}{n^2} \) is absolutely convergent and hence convergent.

        \item This is the series \( \sum_{n=1}^{\infty} a_n \), where
        \[
            a_n = (-1)^{n+1} \frac{n + 1}{2n} = (-1)^{n+1} \left( \frac{1}{2} + \frac{1}{2n} \right).
        \]
        The sequence \( (a_n) \) is divergent:
        \[
            \lim a_{2n} = -\tfrac{1}{2} \neq \tfrac{1}{2} = \lim a_{2n+1}.
        \]
        Theorem 2.7.3 then implies that \( \sum_{n=1}^{\infty} a_n \) is divergent.

        \item For the series \( 1 + \tfrac{1}{2} - \tfrac{1}{3} + \tfrac{1}{4} + \tfrac{1}{5} - \tfrac{1}{6} + \tfrac{1}{7} + \tfrac{1}{8} - \tfrac{1}{9} + \cdots \), let \( (s_n) \) be the sequence of partial sums and consider the subsequence \( (s_{3n}) \). Observe that
        \begin{align*}
            s_{3n} = \,\, & \left( 1 + \frac{1}{2} - \frac{1}{3} \right) + \left( \frac{1}{4} + \frac{1}{5} - \frac{1}{6} \right) + \cdots + \left( \frac{1}{3n - 2} + \frac{1}{3n - 1} - \frac{1}{3n} \right) \\[2mm]
            \geq \,\, & \left( 1 + \frac{1}{2} - \frac{1}{2} \right) + \left( \frac{1}{4} + \frac{1}{5} - \frac{1}{5} \right) + \cdots + \left( \frac{1}{3n - 2} + \frac{1}{3n - 1} - \frac{1}{3n - 1} \right) \\[2mm]
            = \,\, & 1 + \frac{1}{4} + \cdots + \frac{1}{3n - 2} \\
            = \,\, & \frac{1}{3} \sum_{k=1}^n \frac{1}{k - \tfrac{2}{3}} \\
            \geq \,\, & \frac{1}{3} \sum_{k=1}^n \frac{1}{k}.
        \end{align*}
        So we have shown that \( s_{3n} \geq \tfrac{1}{3} \sum_{k=1}^n \tfrac{1}{k} \) for all \( n \in \N \). Since \( \sum_{k=1}^n \tfrac{1}{k} \) is unbounded in \( n \) (Example 2.4.5), it follows that \( (s_{3n}) \) is unbounded and hence that \( (s_n) \) is divergent.

        \item For the series \( 1 - \tfrac{1}{2^2} + \tfrac{1}{3} - \tfrac{1}{4^2} + \tfrac{1}{5} - \tfrac{1}{6^2} + \tfrac{1}{7} - \tfrac{1}{8^2} + \cdots \), let \( (s_n) \) be the sequence of partial sums and consider the subsequence \( (s_{2n}) \). For any \( m \geq 2 \), we have
        \[
            \frac{1}{m^2} \leq \frac{1}{m(m-1)} = \frac{1}{m-1} - \frac{1}{m} \implies -\frac{1}{m^2} \geq - \frac{1}{m-1} + \frac{1}{m}.
        \]
        It follows that
        \begin{align*}
            s_{2n} = \,\, & \left( 1 - \frac{1}{2^2} \right) + \left( \frac{1}{3} - \frac{1}{4^2} \right) + \cdots + \left( \frac{1}{2n - 1} - \frac{1}{(2n)^2} \right) \\[2mm]
            \geq \,\, & \left( 1 - 1 + \frac{1}{2} \right) + \left( \frac{1}{3} - \frac{1}{3} + \frac{1}{4} \right) + \cdots + \left( \frac{1}{2n-1} - \frac{1}{2n-1} + \frac{1}{2n} \right) \\[2mm]
            = \,\, & \frac{1}{2} + \frac{1}{4} + \cdots + \frac{1}{2n} \\[2mm]
            = \,\, & \frac{1}{2} \sum_{k=1}^n \frac{1}{k}.
        \end{align*}
        So we have shown that \( s_{2n} \geq \tfrac{1}{2} \sum_{k=1}^n \tfrac{1}{k} \) for all \( n \in \N \). Since \( \sum_{k=1}^n \tfrac{1}{k} \) is unbounded in \( n \) (Example 2.4.5), it follows that \( (s_{2n}) \) is unbounded and hence that \( (s_n) \) is divergent.
    \end{enumerate}
\end{solution}

\begin{exercise}
\label{ex:3}
    \begin{enumerate}
        \item Provide the details for the proof of the Comparison Test (Theorem 2.7.4) using the Cauchy Criterion for Series.

        \item Give another proof for the Comparison Test, this time using the Monotone Convergence Theorem.
    \end{enumerate}
\end{exercise}

\begin{solution}
    \begin{enumerate}
        \item Since \( 0 \leq a_k \leq b_k \) for all \( k \in \N \), for any \( n > m \) we have
        \begin{equation}
            \abs{a_{m+1} + \cdots + a_n} = a_{m+1} + \cdots + a_n \leq b_{m+1} + \cdots + b_n = \abs{b_{m+1} + \cdots + b_n}.
        \end{equation}
        Suppose that \( \sum_{k=1}^{\infty} b_k \) is convergent and let \( \epsilon > 0 \) be given. By the Cauchy Criterion for Series, there exists an \( N \in \N \) such that
        \[
            n > m \geq N \implies \abs{b_{m+1} + \cdots + b_n} < \epsilon.
        \]
        By inequality (3), we then have \( \abs{a_{m+1} + \cdots + a_n} < \epsilon \) for all \( n > m \geq N \). The Cauchy Criterion for Series then implies that \( \sum_{k=1}^{\infty} a_k \) is convergent.

        Suppose that \( \sum_{k=1}^{\infty} a_k \) is divergent. By the Cauchy Criterion for Series, there is an \( \epsilon > 0 \) such that for all \( N \in \N \) there exist positive integers \( n \) and \( m \) such that
        \[
            n > m \geq N \quad \text{and} \quad \abs{a_{m+1} + \cdots + a_n} \geq \epsilon.
        \]
        Let \( N \in \N \) be given and let \( n \) and \( m \) be the positive integers so obtained. Inequality (3) then gives us \( \abs{b_{m+1} + \cdots + b_n} \geq \epsilon \); it follows from the Cauchy Criterion for Series that \( \sum_{k=1}^{\infty} b_k \) is divergent.

        \item Define the sequences of partial sums
        \[
            s_n = a_1 + \cdots + a_n \quad \text{and} \quad t_n = b_1 + \cdots + b_n.
        \]
        Since \( 0 \leq a_k \leq b_k \) for all \( k \in \N \), both sequences of partial sums are increasing and satisfy \( 0 \leq s_n \leq t_n \) for all \( n \in \N \). Then by the Monotone Convergence Theorem, the convergence of each sequence is equivalent to the boundedness of that sequence. From the inequality \( 0 \leq s_n \leq t_n \), it is clear that \( (s_n) \) is bounded if \( (t_n) \) is bounded and that \( (t_n) \) is unbounded if \( (s_n) \) is unbounded.
    \end{enumerate}
\end{solution}

\begin{exercise}
\label{ex:4}
    Give an example of each or explain why the request is impossible referencing the proper theorem(s).
    \begin{enumerate}
        \item Two series \( \sum x_n \) and \( \sum y_n \) that both diverge but where \( \sum x_n y_n \) converges.

        \item A convergent series \( \sum x_n \) and a bounded sequence \( (y_n) \) such that \( \sum x_n y_n \) diverges.

        \item Two sequences \( (x_n) \) and \( (y_n) \) where \( \sum x_n \) and \( \sum (x_n + y_n) \) both converge but \( \sum y_n \) diverges.

        \item A sequence \( (x_n) \) satisfying \( 0 \leq x_n \leq 1/n \) where \( \sum (-1)^n x_n \) diverges.
    \end{enumerate}
\end{exercise}

\begin{solution}
    \begin{enumerate}
        \item Take \( (x_n) \) and \( (y_n) \) to be the sequences given by \( x_n = y_n = 1/n \). Then \( \sum_{n=1}^{\infty} x_n = \sum_{n=1}^{\infty} y_n = \sum_{n=1}^{\infty} 1/n \) is the divergent harmonic series (Example 2.4.5), but \( \sum_{n=1}^{\infty} x_n y_n = \sum_{n=1}^{\infty} 1/n^2 \) is convergent (Example 2.4.4).

        \item Let \( (x_n) \) be the sequence given by \( x_n = (-1)^{n+1} / n \) and \( (y_n) \) be the bounded sequence given by \( y_n = (-1)^{n+1} \). Then by the Alternating Series Test \( \sum_{n=1}^{\infty} x_n \) is convergent, but \( \sum_{n=1}^{\infty} x_n y_n = \sum_{n=1}^{\infty} 1/n \) is divergent.

        \item This is impossible; by Theorem 2.7.1 we must have
        \[
            \sum_{n=1}^{\infty} y_n = \sum_{n=1}^{\infty} (x_n + y_n) - \sum_{n=1}^{\infty} x_n.
        \]

        \item Let \( (x_n) \) be the sequence given by
        \[
            x_n = \begin{cases}
                \frac{1}{2(n+1)} & \text{if } n \text{ is odd}, \\
                \frac{1}{n} & \text{if } n \text{ is even},
            \end{cases}
            \quad \text{i.e. } (x_n) = \left( \tfrac{1}{4}, \tfrac{1}{2}, \tfrac{1}{8}, \tfrac{1}{4}, \tfrac{1}{12}, \tfrac{1}{6}, \ldots \right).
        \]
        Then \( 0 \leq x_n \leq 1/n \) for all \( n \in \N \). Let \( (s_n) \) be the sequence of partial sums for the series \( \sum_{n=1}^{\infty} (-1)^n x_n \). Observe that
        \begin{align*}
            s_{2n} = \,\, & \left( -\frac{1}{4} + \frac{1}{2} \right) + \left( -\frac{1}{8} + \frac{1}{4} \right) + \cdots + \left( -\frac{1}{4n} + \frac{1}{2n} \right) \\[2mm]
            = \,\, & \frac{1}{4} + \frac{1}{8} + \cdots + \frac{1}{4n} \\[2mm]
            = \,\, & \frac{1}{4} \sum_{k=1}^n \frac{1}{k}.
        \end{align*}
        It follows that \( (s_{2n}) \) is unbounded and hence that \( \sum_{n=1}^{\infty} (-1)^n x_n \) is divergent.
    \end{enumerate}
\end{solution}

\begin{exercise}
\label{ex:5}
    Now that we have proved the basic facts about geometric series, supply a proof for Corollary 2.4.7.
\end{exercise}

\begin{solution}
    We want to show that the series \( \sum_{n=1}^{\infty} 1/n^p \) converges if and only if \( p > 1 \). If \( p \leq 0 \), then \( (1/n^p) \) does not converge to zero; it follows that \( \sum_{n=1}^{\infty} 1/n^p \) diverges (Theorem 2.7.3). Suppose that \( p > 0 \). Then the sequence \( (1/n^p) \) is positive and decreasing, so the Cauchy Condensation Test implies that \( \sum_{n=1}^{\infty} 1/n^p \) is convergent if and only if the series
    \[
        \sum_{n=0}^{\infty} \frac{2^n}{(2^n)^p} = \sum_{n=0}^{\infty} \left( 2^{1-p} \right)^n
    \]
    is convergent. This is a geometric series with common ratio \( 2^{1-p} \), so by Example 2.7.5 this series is convergent if and only if
    \[
        \abs{2^{1-p}} < 1 \iff 1 - p < 0 \iff p > 1.
    \]
\end{solution}

\begin{exercise}
\label{ex:6}
    Let's say that a series subverges if the sequence of partial sums contains a subsequence that converges. Consider this (invented) definition for a moment, and then decide which of the following statements are valid propositions about subvergent series:
    \begin{enumerate}
        \item If \( (a_n) \) is bounded, then \( \sum a_n \) subverges.

        \item All convergent series are subvergent.

        \item If \( \sum \abs{a_n} \) subverges, then \( \sum a_n \) subverges as well.

        \item If \( \sum a_n \) subverges, then \( (a_n) \) has a convergent subsequence.
    \end{enumerate}
\end{exercise}

\begin{solution}
    \begin{enumerate}
        \item This is false in general. Consider the bounded sequence \( (a_n) = (1, 1, 1, \ldots) \). Then the sequence of partial sums for \( \sum_{n=1}^{\infty} a_n \) is \( (n) \), which has no convergent subsequence (see part (c)).

        \item This is true. If the sequence of partial sums \( (s_n) \) is convergent then any subsequence of \( (s_n) \) is convergent; \( (s_n) \) itself, for example.

        \item This is true; we will prove the contrapositive statement. Define the sequences of partial sums
        \[
            s_n = \abs{a_1} + \cdots + \abs{a_n} \quad \text{and} \quad t_n = a_1 + \cdots + a_n.
        \]
        We want to show that if \( (t_n) \) has no convergent subsequence, then neither does \( (s_n) \). By the Bolzano-Weierstrass Theorem, it must be the case that \( (t_n) \) is unbounded. Since \( t_n \leq s_n \) for all \( n \in \N \), it follows that \( (s_n) \) is unbounded. Then \( (s_n) \) is an increasing unbounded sequence; such sequences do not have convergent subsequences. To see this, suppose \( (s_{n_k}) \) is a subsequence and \( x \) is a real number. Since \( (s_n) \) is unbounded and increasing, there is an \( M \in \N \) such that \( s_n \geq x + 1 \) for all \( n \geq M \). Let \( N \in \N \) be given. Since \( (s_{n_k}) \) is a subsequence, there exists \( K \in \N \) such that \( n_K \geq \max \{ M, N \} \). Then
        \[
            s_{n_K} \geq x + 1 \implies \abs{s_{n_K} - x} \geq 1.
        \]
        It follows that \( (s_{n_k}) \) does not converge to \( x \).

        \item This is false in general. Consider the sequence \( (a_n) = (1, -1, 2, -2, 3, -3, \ldots) \). The sequence of partial sums is \( (s_n) = (1, 0, 2, 0, 3, 0, \ldots) \), which has the convergent subsequence \( (0, 0, 0, \ldots) \); it follows that \( \sum_{n=1}^{\infty} a_n \) subverges. However, \( (a_n) \) has no convergent subsequence. To see this, observe that for any sequence \( (x_n) \) we have
        \[
            (x_n) \text{ has a convergent subsequence} \implies (\abs{x_n}) \text{ has a convergent subsequence},
        \]
        since if \( \lim_k x_{n_k} = x \) then \( \lim_k \abs{x_{n_k}} = \abs{x} \). Then since \( (\abs{a_n}) = (1, 1, 2, 2, 3, 3, \ldots) \) has no convergent subsequence (see part (c)), it follows that \( (a_n) \) has no convergent subsequence.
    \end{enumerate}
\end{solution}

\begin{exercise}
\label{ex:7}
    \begin{enumerate}
        \item Show that if \( a_n > 0 \) and \( \lim (n a_n) = l \) with \( l \neq 0 \), then the series \( \sum a_n \) diverges.

        \item Assume \( a_n > 0 \) and \( \lim (n^2 a_n) \) exists. Show that \( \sum a_n \) converges.
    \end{enumerate}
\end{exercise}

\begin{solution}
    The condition that \( a_n > 0 \) can be relaxed to \( a_n \geq 0 \) for both parts of this exercise.
    \begin{enumerate}
        \item Since \( n a_n \geq 0 \) for all \( n \in \N \), the Order Limit Theorem and the assumption \( l \neq 0 \) implies that \( l > 0 \). Then there is an \( N \in \N \) such that
        \[
            n \geq N \implies 0 < \frac{l}{2} < n a_n \implies 0 < \frac{l}{2 n} < a_n.
        \]
        Since the series \( \sum_{n=1}^{\infty} \tfrac{l}{2n} \) is divergent, the Comparison Test implies that \( \sum_{n=1}^{\infty} a_n \) is also divergent.

        \item Suppose that \( \lim (n^2 a_n) = l \); the Order Limit Theorem implies that \( l \geq 0 \). Then there is an \( N \in \N \) such that
        \[
            n \geq N \implies 0 \leq n^2 a_n < l + 1 \implies 0 \leq a_n < \frac{l + 1}{n^2}.
        \]
        Since the series \( \sum_{n=1}^{\infty} \tfrac{l + 1}{n^2} \) is convergent, the Comparison Test implies that \( \sum_{n=1}^{\infty} a_n \) is also convergent.
    \end{enumerate}
\end{solution}

\begin{exercise}
\label{ex:8}
    Consider each of the following propositions. Provide short proofs for those that are true and counterexamples for any that are not.
    \begin{enumerate}
        \item If \( \sum a_n \) converges absolutely, then \( \sum a_n^2 \) also converges absolutely.

        \item If \( \sum a_n \) converges and \( (b_n) \) converges, then \( \sum a_n b_n \) converges.

        \item If \( \sum a_n \) converges conditionally, then \( \sum n^2 a_n \) diverges.
    \end{enumerate}
\end{exercise}

\begin{solution}
    \begin{enumerate}
        \item This is true. Since the series \( \sum_{n=1}^{\infty} \abs{a_n} \) converges, we must have \( \lim \abs{a_n} = 0 \). There is then an \( N \in \N \) such that \( 0 \leq \abs{a_n} \leq 1 \) for \( n \geq N \); it follows that \( 0 \leq \abs{a_n}^2 \leq \abs{a_n} \) for \( n \geq N \). We may now apply the Comparison Test to conclude that \( \sum_{n=1}^{\infty} a_n^2 \) converges absolutely.

        \item This is false in general. Let \( (a_n) = (b_n) = ((-1)^{n+1} / \sqrt{n}) \). Then \( \lim b_n = 0 \) and \( \sum_{n=1}^{\infty} a_n \) converges by the Alternating Series Test, but \( \sum_{n=1}^{\infty} a_n b_n = \sum_{n=1}^{\infty} 1/n \), which is divergent.

        \item This is true; we will prove that
        \[
            \sum_{n=1}^{\infty} \abs{a_n} \text{ diverges} \implies \sum_{n=1}^{\infty} n^2 a_n \text{ diverges},
        \]
        by proving the contrapositive statement
        \[
            \sum_{n=1}^{\infty} n^2 a_n \text{ converges} \implies \sum_{n=1}^{\infty} \abs{a_n} \text{ converges}.
        \]
        By Theorem 2.7.3 we have \( \lim(n^2 a_n) = 0 \), which implies that \( \lim(n^2 \abs{a_n}) = 0 \). We may now apply \Cref{ex:7} (b) to conclude that \( \sum_{n=1}^{\infty} \abs{a_n} \) is convergent.
    \end{enumerate}
\end{solution}

\begin{exercise}[Ratio Test]
\label{ex:9}
    Given a series \( \sum_{n=1}^{\infty} a_n \) with \( a_n \neq 0 \), the Ratio Test states that if \( (a_n) \) satisfies
    \[
        \lim \abs{\frac{a_{n+1}}{a_n}} = r < 1,
    \]
    then the series converges absolutely.
    \begin{enumerate}
        \item Let \( r' \) satisfy \( r < r' < 1 \). Explain why there exists an \( N \) such that \( n \geq N \) implies \( \abs{a_{n+1}} \leq \abs{a_n} r' \).

        \item Why does \( \abs{a_N} \sum (r')^n \) converge?

        \item Now, show that \( \sum \abs{a_n} \) converges, and conclude that \( \sum a_n \) converges.
    \end{enumerate}
\end{exercise}

\begin{solution}
    \begin{enumerate}
        \item Since \( \lim \abs{\frac{a_{n+1}}{a_n}} = r \) and \( r' - r > 0 \), there is an \( N \in \N \) such that
        \[
            n \geq N \implies \abs{\abs{\frac{a_{n+1}}{a_n}} - r} < r' - r \implies \frac{\abs{a_{n+1}}}{\abs{a_n}} < r' \implies \abs{a_{n+1}} < \abs{a_n} r'.
        \]

        \item \( \sum_{n=0}^{\infty} (r')^n \) is a geometric series; since \( 0 < r' < 1 \), it converges (Example 2.7.5).

        \item By part (a), we have \( \abs{a_{N+1}} < \abs{a_N} r' \), and so \( \abs{a_{N+2}} < \abs{a_{N+1}} r' < \abs{a_N} (r')^2 \); an induction argument shows that \( \abs{a_{N+n}} < \abs{a_N} (r')^n \) for all \( n \in \N \). Then by part (b) and the Comparison Test, the series
        \[
            \sum_{n=0}^{\infty} \abs{a_{N+n}} = \sum_{n=N}^{\infty} \abs{a_n}
        \]
        is convergent. A finite number of terms do not affect convergence, so it follows that the series \( \sum_{n=1}^{\infty} \abs{a_n} \) converges. The convergence of \( \sum_{n=1}^{\infty} a_n \) is then given by Theorem 2.7.6.
    \end{enumerate}
\end{solution}

\begin{exercise}[Infinite Products]
\label{ex:10}
    Review \href{https://lew98.github.io/Mathematics/UA_Section_2_4_Exercises.pdf}{Exercise 2.4.10} about infinite products and then answer the following questions:
    \begin{enumerate}
        \item Does \( \tfrac{2}{1} \cdot \tfrac{3}{2} \cdot \tfrac{5}{4} \cdot \tfrac{9}{8} \cdot \tfrac{17}{16} \cdots \) converge?

        \item The infinite product \( \tfrac{1}{2} \cdot \tfrac{3}{4} \cdot \tfrac{5}{6} \cdot \tfrac{7}{8} \cdot \tfrac{9}{10} \cdots \) certainly converges. (Why?) Does it converge to zero?

        \item In 1655, John Wallis famously derived the formula
        \[
            \left( \frac{2 \cdot 2}{1 \cdot 3} \right) \left( \frac{4 \cdot 4}{3 \cdot 5} \right) \left( \frac{6 \cdot 6}{5 \cdot 7} \right) \left( \frac{8 \cdot 8}{7 \cdot 9} \right) \cdots = \frac{\pi}{2}.
        \]
        Show that the left side of this identity at least converges to something. (A complete proof of this result is taken up in Section 8.3.)
    \end{enumerate}
\end{exercise}

\begin{solution}
    \begin{enumerate}
        \item This is the infinite product
        \[
            \prod_{n=0}^{\infty} \frac{2^n + 1}{2^n} = \prod_{n=0}^{\infty} \left( 1 + \frac{1}{2^n} \right).
        \]
        By \href{https://lew98.github.io/Mathematics/UA_Section_2_4_Exercises.pdf}{Exercise 2.4.10}, this infinite product converges if and only if the series \( \sum_{n=0}^{\infty} \tfrac{1}{2^n} \) converges. This series is geometric with common ratio \( r = \tfrac{1}{2} \) and hence convergent by Example 2.7.5; it follows that the infinite product converges.

        \item This is the infinite product
        \[
            \prod_{n=1}^{\infty} \frac{2n - 1}{2n} = \prod_{n=1}^{\infty} \left( 1 - \frac{1}{2n} \right).
        \]
        The sequence of partial products is positive and decreasing, since each term in the partial product satisfies \(0 < 1 - \tfrac{1}{2n} < 1 \); the Monotone Convergence Theorem then implies that the infinite product converges.

        The infinite product does converge to zero. To see this, let \( (p_m) \) be the sequence of partial products:
        \[
            p_m = \frac{1}{2} \cdot \frac{3}{4} \cdots \frac{2m - 1}{2m}.
        \]
        As stated above, \( (p_m) \) is decreasing and satisfies \( 0 < p_m < 1 \) for all \( m \in \N \), so we can look at the sequence of reciprocals \( (1/p_m) \):
        \begin{align*}
            \frac{1}{p_m} &= \frac{2}{1} \cdot \frac{4}{3} \cdots \frac{2m}{2m - 1} \\
            &= \left( 1 + \frac{1}{1} \right) \left( 1 + \frac{1}{3} \right) \cdots \left( 1 + \frac{1}{2m - 1} \right) \\
            &\geq \sum_{n=1}^m \frac{1}{2n - 1} \\
            &\geq \frac{1}{2} \sum_{n=1}^m \frac{1}{n}.
        \end{align*}
        It follows that \( (1/p_m) \) is unbounded above; if we let \( \epsilon > 0 \) be arbitrary, there is an \( M \in \N \) such that \( 1/p_M > 1/\epsilon \implies p_M < \epsilon \). Since \( (p_m) \) is decreasing, we then have
        \[
            m \geq M \implies \abs{p_m} = p_m \leq p_M < \epsilon.
        \]
        Hence \( \lim p_m = 0 \).

        \item This is the infinite product
        \[
            \prod_{n=1}^{\infty} \frac{(2n)^2}{(2n - 1)(2n + 1)} = \prod_{n=1}^{\infty} \left( 1 + \frac{1}{(2n - 1)(2n + 1)} \right) = \prod_{n=1}^{\infty} \left( 1 + \frac{1}{4n^2 - 1} \right).
        \]
        By \href{https://lew98.github.io/Mathematics/UA_Section_2_4_Exercises.pdf}{Exercise 2.4.10}, this infinite product converges if and only if the series \( \sum_{n=0}^{\infty} \tfrac{1}{4n^2 - 1} \) converges. Observe that for all \( n \in \N \) we have
        \[
            n^2 - 1 \ \geq 0 \implies 4 n^2 - 1 \geq 3 n^2 \implies \frac{1}{4 n^2 - 1} \leq \frac{1}{3 n^2}.
        \]
        Then since the series \( \sum_{n=1}^{\infty} \tfrac{1}{3 n^2} \) is convergent, the Comparison Test implies that the series \( \sum_{n=0}^{\infty} \tfrac{1}{4n^2 - 1} \) is also convergent; it follows that the infinite product \( \left( \tfrac{2 \cdot 2}{1 \cdot 3} \right) \left( \tfrac{4 \cdot 4}{3 \cdot 5} \right) \left( \tfrac{6 \cdot 6}{5 \cdot 7} \right) \left( \tfrac{8 \cdot 8}{7 \cdot 9} \right) \cdots \) converges.
    \end{enumerate}
\end{solution}

\begin{exercise}
\label{ex:11}
    Find examples of two series \( \sum a_n \) and \( \sum b_n \) both of which diverge but for which \( \sum \min \{ a_n, b_n \} \) converges. To make it more challenging, produce examples where \( (a_n) \) and \( (b_n) \) are strictly positive and decreasing.
\end{exercise}

\begin{solution}
    Consider the series
    \begin{gather*}
        \sum_{n=1}^{\infty} a_n = \underbrace{\frac{1}{1}^2}_{\substack{1 \text{ term} \\ \text{sum = } 1}} + \frac{1}{2^2} + \cdots + \frac{1}{5^2} +  \underbrace{\frac{1}{6^2} + \cdots + \frac{1}{6^2}}_{\substack{6^2 \text{ terms} \\ \text{sum = } 1}} + \frac{1}{42^2} + \cdots + \frac{1}{1805^2} + \cdots \\[4mm]
        \sum_{n=1}^{\infty} b_n = \frac{1}{1^2} + \underbrace{\frac{1}{2^2} + \cdots + \frac{1}{2^2}}_{\substack{2^2 \text{ terms} \\ \text{sum = } 1}} + \frac{1}{6^2} + \cdots + \frac{1}{41^2} + \underbrace{\frac{1}{42^2} + \cdots + \frac{1}{42^2}}_{\substack{42^2 \text{ terms} \\ \text{sum = } 1}} + \cdots
    \end{gather*}
    Then both \( (a_n) \) and \( (b_n) \) are strictly positive and decreasing and
    \[
        \sum_{n=1}^{\infty} \min \{ a_n, b_n \} = \sum_{n=1}^{\infty} \frac{1}{n^2},
    \]
    which is a convergent series. Both \( \sum a_n \) and \( \sum b_n \) diverge since their respective sequences of partial sums are unbounded; we can find arbitrarily many groupings of terms which sum to 1 as shown above.
\end{solution}

\begin{exercise}[Summation by parts]
\label{ex:12}
    Let \( (x_n) \) and \( (y_n) \) be sequences, let \( s_n = x_1 + x_2 + \cdots + x_n \) and set \( s_0 = 0 \). Use the observation that \( x_j = s_j - s_{j-1} \) to verify the formula
    \[
        \sum_{j=m}^n x_j y_j = s_n y_{n+1} - s_{m-1} y_m + \sum_{j=m}^n s_j (y_j - y_{j+1}).
    \]
\end{exercise}

\begin{solution}
    For positive integers \( n > m \),
    \begin{align*}
        \sum_{j=m}^n x_j y_j &= \sum_{j=m}^n (s_j - s_{j-1}) y_j \\
        &= \sum_{j=m}^n s_j y_j - \sum_{j=m}^n s_{j-1} y_j \\
        &= \sum_{j=m}^n s_j y_j - \sum_{j=m-1}^{n-1} s_j y_{j+1} \\
        &= \sum_{j=m}^n s_j y_j - \sum_{j=m}^n s_j y_{j+1} + s_n y_{n+1} - s_{m-1} y_m \\
        &= s_n y_{n+1} - s_{m-1} y_m + \sum_{j=m}^n s_j (y_j - y_{j+1}).
    \end{align*}
\end{solution}

\begin{exercise}[Abel's Test]
\label{ex:13}
    Abel's Test for convergence states that if the series \( \sum_{k=1}^{\infty} x_k \) converges, and if \( (y_k) \) is a sequence satisfying
    \[
        y_1 \geq y_2 \geq y_3 \geq \cdots \geq 0,
    \]
    then the series \( \sum_{k=1}^{\infty} x_k y_k \) converges.
    \begin{enumerate}
        \item Use \Cref{ex:12} to show that
        \[
            \sum_{k=1}^n x_k y_k = s_n y_{n+1} + \sum_{k=1}^n s_k (y_k - y_{k+1}),
        \]
        where \( s_n = x_1 + x_2 + \cdots + x_n \).

        \item Use the Comparison Test to argue that \( \sum_{k=1}^{\infty} s_k (y_k - y_{k+1}) \) converges absolutely, and show how this leads directly to a proof of Abel's Test.
    \end{enumerate}
\end{exercise}

\begin{solution}
    \begin{enumerate}
        \item This follows immediately from \Cref{ex:12}, taking \( m = 1 \) and remembering that \( s_0 := 0 \).

        \item First, note that since \( (y_k) \) is decreasing and bounded below, the limit \( y := \lim_k y_k \) exists by the Monotone Convergence Theorem. Then observe that the series \( \sum_{k=1}^{\infty} (y_k - y_{k+1}) \) is absolutely convergent (each term is positive since \( (y_k) \) is decreasing); if we let \( t_m \) be the \( m \)th partial sum, then
        \[
            t_m = (y_1 - y_2) + (y_2 - y_3) + \cdots + (y_m - y_{m+1}) = y_1 - y_{m+1} \to y_1 - y \text{ as } m \to \infty.
        \]
        By assumption, the sequence \( (s_k) \) is convergent and hence bounded by some \( M > 0 \). For each \( k \in \N \) we then have
        \[
            0 \leq \abs{s_k (y_k - y_{k+1})} = \abs{s_k} (y_k - y_{k+1}) \leq M (y_k - y_{k+1}).
        \]
        The Comparison Test then implies that \( \sum_{k=1}^{\infty} s_k(y_k - y_{k+1}) \) is absolutely convergent and hence convergent. From part (a) we have \( \sum_{k=1}^n x_k y_k = s_n y_{n+1} + \sum_{k=1}^n s_k (y_k - y_{k+1}) \); it follows that
        \[
            \sum_{k=1}^{\infty} x_k y_k = \lim_n \left( s_n y_{n+1} + \sum_{k=1}^n s_k (y_k - y_{k+1}) \right) = y \sum_{k=1}^{\infty} x_k + \sum_{k=1}^{\infty} s_k (y_k - y_{k+1}).
        \]
    \end{enumerate}
\end{solution}

\begin{exercise}[Dirichlet's Test]
\label{ex:14}
    Dirichlet's Test for convergence states that if the partial sums of \( \sum_{k=1}^{\infty} x_k \) are bounded (but not necessarily convergent), and if \( (y_k) \) is a sequence satisfying \( y_1 \geq y_2 \geq y_3 \geq \cdots \geq 0 \) with \( \lim y_k = 0 \), then the series \( \sum_{k=1}^{\infty} x_k y_k \) converges.
    \begin{enumerate}
        \item Point out how the hypothesis of Dirichlet's Test differs from that of Abel's Test in \Cref{ex:13}, but show that essentially the same strategy can be used to provide a proof.

        \item Show how the Alternating Series Test (Theorem 2.7.7) can be derived as a special case of Dirichlet's Test.
    \end{enumerate}
\end{exercise}

\begin{solution}
    \begin{enumerate}
        \item Abel's Test has the stronger hypothesis that the sequence of partial sums of \( \sum_{k=1}^{\infty} x_k \) is convergent (and hence bounded), but the weaker hypothesis that \( (y_k) \) only satisfies \( y_1 \geq y_2 \geq y_3 \geq \cdots \geq 0 \) without necessarily converging to zero; by the Monotone Convergence Theorem and the Order Limit Theorem, we have \( \lim y_k \geq 0 \).

        The proof of Dirichlet's Test is almost identical to the proof of Abel's Test given in \Cref{ex:13} (b). The series \( \sum_{k=1}^{\infty} (y_k - y_{k+1}) \) is absolutely convergent since it has \( m \)th partial sum
        \[
            (y_1 - y_2) + (y_2 - y_3) + \cdots + (y_m - y_{m+1}) = y_1 - y_m \to y_1 \text{ as } m \to \infty.
        \]
        Letting \( (s_k) \) be the \( k \)th partial sum of \( \sum_{k=1}^{\infty} x_k \), we are given that \( (s_k) \) is bounded by some \( M > 0 \). Then since
        \[
            0 \leq \abs{s_k (y_k - y_{k+1})} = \abs{s_k} (y_k - y_{k+1}) \leq M (y_k - y_{k+1})
        \]
        for each \( k \in \N \), the Comparison Test implies that \( \sum_{k=1}^{\infty} s_k(y_k - y_{k+1}) \) is absolutely convergent and hence convergent. Since \( (s_k) \) is bounded and \( \lim y_k = 0 \), we have \( \lim (s_k y_{k+1}) = 0 \) also. Hence
        \[
            \sum_{k=1}^{\infty} x_k y_k = \lim_n \left( s_n y_{n+1} + \sum_{k=1}^n s_k (y_k - y_{k+1}) \right) = \sum_{k=1}^{\infty} s_k (y_k - y_{k+1}).
        \]

        \item The Alternating Series Test can be recovered from Dirichlet's Test by taking \( (x_k) = ((-1)^{k+1}) \); the sequence of partial sums of \( \sum_{k=1}^{\infty} x_k \) is then \( (1, 0, 1, 0, \ldots) \), which is certainly bounded.
    \end{enumerate}
\end{solution}

\noindent \hrulefill

\noindent \hypertarget{ua}{\textcolor{blue}{[UA]} Abbott, S. (2015) \textit{Understanding Analysis.} 2nd edn.}

\end{document}