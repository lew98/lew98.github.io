\documentclass[12pt]{article}
\usepackage[utf8]{inputenc}
\usepackage[utf8]{inputenc}
\usepackage{amsmath}
\usepackage{amsthm}
\usepackage{geometry}
\usepackage{amsfonts}
\usepackage{mathrsfs}
\usepackage{bm}
\usepackage{hyperref}
\usepackage[dvipsnames]{xcolor}
\usepackage[inline]{enumitem}
\usepackage{mathtools}
\usepackage{changepage}
\usepackage{lipsum}
\usepackage{tikz}
\usetikzlibrary{matrix, patterns, decorations.pathreplacing, calligraphy}
\usepackage{tikz-cd}
\usepackage[nameinlink]{cleveref}
\geometry{
headheight=15pt,
left=60pt,
right=60pt
}
\setlength{\emergencystretch}{20pt}
\usepackage{fancyhdr}
\pagestyle{fancy}
\fancyhf{}
\lhead{}
\chead{Section 4.2 Exercises}
\rhead{\thepage}
\hypersetup{
    colorlinks=true,
    linkcolor=blue,
    urlcolor=blue
}

\theoremstyle{definition}
\newtheorem*{remark}{Remark}

\newtheoremstyle{exercise}
    {}
    {}
    {}
    {}
    {\bfseries}
    {.}
    { }
    {\thmname{#1}\thmnumber{#2}\thmnote{ (#3)}}
\theoremstyle{exercise}
\newtheorem{exercise}{Exercise 4.2.}

\newtheoremstyle{solution}
    {}
    {}
    {}
    {}
    {\itshape\color{magenta}}
    {.}
    { }
    {\thmname{#1}\thmnote{ #3}}
\theoremstyle{solution}
\newtheorem*{solution}{Solution}

\Crefformat{exercise}{#2Exercise 4.2.#1#3}

\newcommand{\interior}[1]{%
  {\kern0pt#1}^{\mathrm{o}}%
}
\newcommand{\ts}{\textsuperscript}
\newcommand{\setcomp}[1]{#1^{\mathsf{c}}}
\newcommand{\quand}{\quad \text{and} \quad}
\newcommand{\N}{\mathbf{N}}
\newcommand{\Z}{\mathbf{Z}}
\newcommand{\Q}{\mathbf{Q}}
\newcommand{\I}{\mathbf{I}}
\newcommand{\R}{\mathbf{R}}
\newcommand{\C}{\mathbf{C}}

\DeclarePairedDelimiter\abs{\lvert}{\rvert}
% Swap the definition of \abs* and \norm*, so that \abs
% and \norm resizes the size of the brackets, and the 
% starred version does not.
\makeatletter
\let\oldabs\abs
\def\abs{\@ifstar{\oldabs}{\oldabs*}}
%
\let\oldnorm\norm
\def\norm{\@ifstar{\oldnorm}{\oldnorm*}}
\makeatother

\setlist[enumerate,1]{label={(\alph*)}}

\begin{document}

\section{Section 4.2 Exercises}

Exercises with solutions from Section 4.2 of \hyperlink{ua}{[UA]}.

\begin{exercise}
\label{ex:1}
    \begin{enumerate}
        \item Supply the details for how Corollary 4.2.4 part (ii) follows from the Sequential Criterion for Functional Limits in Theorem 4.2.3 and the Algebraic Limit Theorem for sequences proved in Chapter 2.

        \item Now, write another proof of Corollary 4.2.4 part (ii) directly from Definition 4.2.1 without using the sequential criterion in Theorem 4.2.3.

        \item Repeat (a) and (b) for Corollary 4.2.4 part (iii).
    \end{enumerate}
\end{exercise}

\begin{solution}
    \begin{enumerate}
        \item Suppose \( (x_n) \) is a sequence contained in \( A \), satisfying \( x_n \neq c \) and \( \lim_{n \to \infty} x_n = c \). The sequential criterion implies that
        \[
            \lim_{n \to \infty} f(x_n) = L \quand \lim_{n \to \infty} g(x_n) = M,
        \]
        and thus the Algebraic Limit Theorem (for sequences) gives
        \[
            \lim_{n \to \infty} [f(x_n) + g(x_n)] = L + M.
        \]
        The sequential criterion allows us to conclude that \( \lim_{x \to c} [f(x) + g(x)] = L + M \).

        \item Let \( \epsilon > 0 \) be given. Since \( \lim_{x \to c} f(x) = L \) and \( \lim_{x \to c} g(x) = M \), there exist positive real numbers \( \delta_1 \) and \( \delta_2 \) such that
        \begin{gather*}
            0 < \abs{x - c} < \delta_1 \text{ and } x \in A \implies \abs{f(x) - L} < \tfrac{\epsilon}{2}, \\[2mm]
            0 < \abs{x - c} < \delta_2 \text{ and } x \in A \implies \abs{g(x) - M} < \tfrac{\epsilon}{2}.
        \end{gather*}
        Set \( \delta := \min \{ \delta_1, \delta_2 \} \) and suppose that \( x \in A \) is such that \( 0 < \abs{x - c} < \delta \). Then
        \[
            \abs{f(x) + g(x) - (L + M)} \leq \abs{f(x) - L} + \abs{g(x) - M} < \tfrac{\epsilon}{2} + \tfrac{\epsilon}{2} = \epsilon.
        \]
        Thus \( \lim_{x \to c} [f(x) + g(x)] = L + M \).

        \item Suppose \( (x_n) \) is a sequence contained in \( A \), satisfying \( x_n \neq c \) and \( \lim_{n \to \infty} x_n = c \). The sequential criterion implies that
        \[
            \lim_{n \to \infty} f(x_n) = L \quand \lim_{n \to \infty} g(x_n) = M,
        \]
        and thus the Algebraic Limit Theorem gives
        \[
            \lim_{n \to \infty} [f(x_n) g(x_n)] = LM.
        \]
        The sequential criterion allows us to conclude that \( \lim_{x \to c} [f(x) g(x)] = LM \).

        Let \( \epsilon > 0 \) be given. Since \( \lim_{x \to c} f(x) = L \) and \( \lim_{x \to c} g(x) = M \), there exist positive real numbers \( \delta_1, \delta_2, \) and \( \delta_3 \) such that
        \begin{gather*}
            0 < \abs{x - c} < \delta_1 \text{ and } x \in A \implies \abs{f(x) - L} < \frac{\epsilon}{2(\abs{M} + 1)}, \\[2mm]
            0 < \abs{x - c} < \delta_2 \text{ and } x \in A \implies \abs{g(x) - M} < \frac{\epsilon}{2(\abs{L} + 1)}, \\[2mm]
            0 < \abs{x - c} < \delta_3 \text{ and } x \in A \implies \abs{g(x) - M} < 1 \implies \abs{g(x)} < \abs{M} + 1.
        \end{gather*}
        Set \( \delta := \min \{ \delta_1, \delta_2, \delta_3 \} \) and suppose that \( x \in A \) is such that \( 0 < \abs{x - c} < \delta \). Then
        \begin{align*}
            \abs{f(x)g(x) - LM} &= \abs{f(x)g(x) - Lg(x) + Lg(x) - LM} \\[1mm]
            &= \abs{g(x)[f(x) - L] + L[g(x) - M]} \\[1mm]
            &\leq \abs{g(x)} \abs{f(x) - L} + \abs{L} \abs{g(x) - M} \\[1mm]
            &< (\abs{M} + 1) \frac{\epsilon}{2(\abs{M} + 1)} + \abs{L} \frac{\epsilon}{2(\abs{L} + 1)} \\
            &< \tfrac{\epsilon}{2} + \tfrac{\epsilon}{2} \\
            &= \epsilon.
        \end{align*}
        Thus \( \lim_{x \to c} [f(x)g(x)] = LM \).
    \end{enumerate}
\end{solution}

\begin{exercise}
\label{ex:2}
    For each stated limit, find the largest possible \( \delta \)-neighborhood that is a proper response to the given \( \epsilon \) challenge.
    \begin{enumerate}
        \item \( \lim_{x \to 3} (5x - 6) = 9 \), where \( \epsilon = 1 \).

        \item \( \lim_{x \to 4} \sqrt{x} = 2 \), where \( \epsilon = 1 \).

        \item \( \lim_{x \to \pi} [[x]] = 3 \), where \( \epsilon = 1 \). (The function \( [[x]] \) returns the greatest integer less than or equal to \( x \).)

        \item \( \lim_{x \to \pi} [[x]] = 3 \), where \( \epsilon = .01 \).
    \end{enumerate}
\end{exercise}

\begin{solution}
    \begin{enumerate}
        \item Observe that
        \[
            \abs{5x - 6 - 9} = 5 \abs{x - 3} < 1 \iff \abs{x - 3} < \tfrac{1}{5}.
        \]
        Thus \( \delta = \tfrac{1}{5} \) is the largest possible value we can take.

        \item It is easily verified that \( x \in (1, 7) = (4 - 3, 4 + 3) \) gives us \( \sqrt{x} \in (1, 3) = (2 - 1, 2 + 1) \), so that \( \delta = 3 \) is a valid response. No larger value of \( \delta \) will work, since this would give us an \( x \in [0, 1) \), which implies that \( \sqrt{x} \in [0, 1) \not\subseteq (1, 3) \).

        \item Since \( [[x]] \) is always an integer, we have \( \abs{[[x]] - 3} < 1 \) if and only if \( [[x]] = 3 \), which is the case if and only if \( 3 \leq x < 4 \). So we should choose the largest possible \( \delta \) such that \( V_{\delta}(\pi) \subseteq [3, 4) \), which is
        \[
            \delta = \min \{ \pi - 3, 4 - \pi \} = \pi - 3.
        \]

        \item As in part (c), we have \( \abs{[[x]] - 3} < 0.01 \) if and only if \( [[x]] = 3 \), so the largest possible choice is \( \delta = \pi - 3 \).
    \end{enumerate}
\end{solution}

\begin{exercise}
\label{ex:3}
    Review the definition of Thomae's function \( t(x) \) from Section 4.1.
    \begin{enumerate}
        \item Construct three different sequences \( (x_n), (y_n), \) and \( (z_n) \), each of which converges to 1 without using the number 1 as a term in the sequence.

        \item Now, compute \( \lim t(x_n), \lim t(y_n), \) and \( \lim t(z_n) \).

        \item Make an educated conjecture for \( \lim_{x \to 1} t(x) \), and use Definition 4.2.1B to verify the claim. (Given \( \epsilon > 0 \), consider the set of points \( \{ x \in \R : t(x) \geq \epsilon \} \). Argue that all the points in this set are isolated.)
    \end{enumerate}
\end{exercise}

\begin{solution}
    \begin{enumerate}
        \item Take
        \[
            x_n = 1 + \frac{1}{n}, \quad y_n = 1 - \frac{1}{n}, \quand z_n = 1 + \frac{\sqrt{2}}{n}.
        \]

        \item Since \( x_n = \tfrac{n+1}{n} \), we have \( t(x_n) = \tfrac{1}{n} \) and thus \( \lim t(x_n) = 0 \). Similarly, \( y_n = \tfrac{n-1}{n} \), so \( t(y_1) = t(0) = 1 \) and \( t(y_n) = \tfrac{1}{n} \) for \( n \geq 2 \). Thus \( \lim t(y_n) = 0 \) also. Finally, since \( z_n \in \I \) for all \( n \in \N \), we have \( \lim t(z_n) = 0 \).

        \item We conjecture that \( \lim_{x \to 1} t(x) = 0 \). To see this, first let us prove the following lemma.

        \noindent \textbf{Lemma 1}. Suppose \( x \in \R \) and \( n \in \N \). There exists a \( \delta > 0 \) such that if \( \tfrac{a}{b} \neq x \) is a rational number contained in \( V_{\delta}(x) \), then \( b > n \).

        \noindent \textit{Proof}. Suppose \( b \in \N \) is such that \( 1 \leq b \leq n \). Since \( I := \left[ x - 1, x + 1 \right] \) is an interval of length 2, there are either \( 2b \) or \( 2b + 1 \) rationals of the form \( \tfrac{a}{b} \) contained in \( I \). (To fit the most rationals inside \( I \), we should place the first such rational \( \tfrac{a}{b} \) on the left endpoint \( x - 1 \); then \( \tfrac{a + 2b}{b} = \tfrac{a}{b} + 2 = x + 1, \) the right endpoint. Thus we have the \( 2b + 1 \) rational numbers \( \tfrac{a}{b}, \tfrac{a + 1}{b}, \ldots, \tfrac{a + 2b}{b} \) contained in \( I \). In the general case, the left endpoint will not be of the form \( \tfrac{a}{b} \) and so there will be only \( 2b \) rationals of this form contained in \( I \).) Given this, the set
        \[
            A = \left\{ \abs{x - \tfrac{a}{b}} : \tfrac{a}{b} \in I, \tfrac{a}{b} \neq x, 1 \leq b \leq n \right\}
        \]
        is non-empty and finite. Thus we can set \( \delta := \min A \); we have \( \delta > 0 \) since each element of \( A \) is strictly positive. It follows that \( V_{\delta}(x) \) can contain only rationals \( \tfrac{a}{b} \) with denominators \( b > n \), other than possibly \( x \) itself. That there exist such rationals is clear from the density of \( \Q \) in \( \R \). \qed
        
        Now we can prove that \( \lim_{x \to 1} t(x) = 0 \). Let \( \epsilon > 0 \) be given and let \( n \in \N \) be such that \( \tfrac{1}{n} < \epsilon \). By Lemma 1, there exists a \( \delta > 0 \) such that if \( \tfrac{a}{b} \neq 1 \) is a rational number contained in \( V_{\delta}(1) \), then \( b > n \). Suppose \( x \in V_{\delta}(1) \). If \( x \) is irrational, then \( t(x) = 0 \in V_{\epsilon}(0) \). If \( x = \tfrac{a}{b} \neq 1 \) is rational, then
        \[
            0 \leq t(x) = \tfrac{1}{b} < \tfrac{1}{n} < \epsilon \implies t(x) \in V_{\epsilon}(0).
        \]
        In either case, \( x \in V_{\delta}(1) \setminus \{ 1 \} \) implies that \( t(x) \in V_{\epsilon}(0) \) and thus \( \lim_{x \to 1} t(x) = 0 \).
    \end{enumerate}
\end{solution}

\begin{exercise}
\label{ex:4}
    Consider the reasonable but erroneous claim that
    \[
        \lim_{x \to 10} 1/[[x]] = 1/10.
    \]
    \begin{enumerate}
        \item Find the largest \( \delta \) that represents a proper response to the challenge of \( \epsilon = 1/2 \).

        \item Find the largest \( \delta \) that represents a proper response to \( \epsilon = 1/50 \).

        \item Find the largest \( \epsilon \) challenge for which there is no suitable \( \delta \) response possible.
    \end{enumerate}
\end{exercise}

\begin{solution}
    Let \( f(x) = 1/[[x]] \), which is defined provided \( [[x]] \neq 0 \), which is the case if and only if \( x < 0 \) or \( x \geq 1 \). Thus the domain of \( f \) is \( A = (-\infty, 0) \cup [1, \infty) \).
    \begin{enumerate}
        \item Let \( \delta = 8 \). Then
        \[
            x \in V_{\delta}(10) = (2, 18) \implies f(x) \in [1/17, 1/2] \subseteq (-2/5, 3/5) = V_{1/2}(1/10).
        \]
        Thus \( \delta = 8 \) is a valid response to the challenge of \( \epsilon = 1/2 \). Suppose that \( \delta > 8 \). Then there exists an \( x \in V_{\delta}(10) \) such that \( 1 \leq x < 2 \), which gives \( f(x) = 1 \not\in (-2/5, 3/5) = V_{1/2}(1/10) \). Hence \( \delta = 8 \) is the largest proper response to the challenge of \( \epsilon = 1/2 \).

        \item Let \( \delta = 1 \). Then
        \[
            x \in V_{\delta}(10) = (9, 11) \implies f(x) \in [1/10, 1/9] \subseteq (2/25, 3/25) = V_{1/50}(1/10).
        \]
        Thus \( \delta = 1 \) is a valid response to the challenge of \( \epsilon = 1/50 \). Suppose that \( \delta > 1 \). Then there exists an \( x \in V_{\delta}(10) \) such that \( 8 \leq x < 9 \), which gives \( f(x) = 1/8 \not\in (2/25, 3/25) = V_{1/50}(1/10) \). Hence \( \delta = 1 \) is the largest proper response to the challenge of \( \epsilon = 1/50 \).

        \item Suppose that \( \epsilon = 1/90 \) and \( \delta > 0 \). Then there exists an \( x \in V_{\delta}(10) \) such that \( 9 \leq x < 10 \), which gives \( f(x) = 1/9 \not\in V_{\epsilon}(1/10) \). Thus there is no valid \( \delta \) response to the challenge of \( \epsilon = 1/90 \).
        
        Now suppose \( \epsilon > 1/90 \) and let \( \delta = 1 \). Then
        \[
            x \in V_{\delta}(10) = (9, 11) \implies f(x) \in [1/10, 1/9] \subseteq V_{\epsilon}(1/10).
        \]
        Thus \( \delta = 1 \) is a valid response to this \( \epsilon \).

        We may conclude that \( \epsilon = 1/90 \) is the largest challenge for which there is no suitable \( \delta \) response possible.
    \end{enumerate}
\end{solution}

\begin{exercise}
\label{ex:5}
    Use Definition 4.2.1 to supply a proper proof for the following limit statements.
    \begin{enumerate}
        \item \( \lim_{x \to 2} (3x + 4) = 10 \).

        \item \( \lim_{x \to 0} x^3 = 0 \).

        \item \( \lim_{x \to 2} (x^2 + x - 1) = 5 \).

        \item \( \lim_{x \to 3} 1/x = 1/3 \).
    \end{enumerate}
\end{exercise}

\begin{solution}
    \begin{enumerate}
        \item Let \( \epsilon > 0 \) be given and set \( \delta = \tfrac{\epsilon}{3} \). Then if \( x \in \R \) is such that \( 0 < \abs{x - 2} < \delta \), we have
        \[
            \abs{(3x + 4) - 10} = \abs{3x - 6} = 3 \abs{x - 2} < 3 \delta = \epsilon.
        \]
        Thus \( \lim_{x \to 2} (3x + 4) = 10 \).

        \item Let \( \epsilon > 0 \) be given and set \( \delta = \epsilon^{1/3} \). Then if \( x \in \R \) is such that \( 0 < \abs{x} < \delta \), we have
        \[
            \abs{x^3} = \abs{x}^3 < \delta^3 = \epsilon.
        \]
        Thus \( \lim_{x \to 0} x^3 = 0 \).

        \item Let \( \epsilon > 0 \) be given. Observe that if \( \abs{x - 2} < 1 \), i.e.\ \( x \in (1, 3) \), then \( x + 3 \in (4, 7) \). Set \( \delta = \min \left\{ \tfrac{\epsilon}{7}, 1 \right\} \). Then if \( x \in \R \) is such that \( 0 < \abs{x - 2} < \delta \), we have
        \[
            \abs{x^2 + x - 1 - 5} = \abs{x + 3} \abs{x - 2} < 7 \delta \leq \epsilon.
        \]
        Thus \( \lim_{x \to 2} (x^2 + x - 1) = 5 \).

        \item Let \( \epsilon > 0 \) be given. Observe that if \( \abs{x - 3} < 1 \), i.e.\ \( x \in (2, 4) \), then \( \tfrac{1}{3x} \in \left( \tfrac{1}{12}, \tfrac{1}{6} \right) \). Set \( \delta = \min \{ 6 \epsilon, 1 \} \). Then if \( x \in \R \setminus \{ 0 \} \) is such that \( 0 < \abs{x - 3} < \delta \), we have
        \[
            \abs{\frac{1}{x} - \frac{1}{3}} = \frac{\abs{x - 3}}{\abs{3x}} < \frac{\delta}{6} \leq \epsilon.
        \]
        Thus \( \lim_{x \to 3} 1/x = 1/3 \).
    \end{enumerate}
\end{solution}

\begin{exercise}
\label{ex:6}
    Decide if the following claims are true or false, and give short justifications of each conclusion.
    \begin{enumerate}
        \item If a particular \( \delta \) has been constructed as a suitable response to a particular \( \epsilon \) challenge, then any smaller positive \( \delta \) will also suffice.

        \item If \( \lim_{x \to a} f(x) = L \) and \( a \) happens to be in the domain of \( f \), then \( L = f(a) \).

        \item If \( \lim_{x \to a} f(x) = L \), then \( \lim_{x \to a} 3[f(x) - 2]^2 = 3(L - 2)^2 \).

        \item If \( \lim_{x \to a} f(x) = 0 \), then \( \lim_{x \to a} f(x) g(x) = 0 \) for any function \( g \) (with domain equal to the domain of \( f \).)
    \end{enumerate}
\end{exercise}

\begin{solution}
    \begin{enumerate}
        \item This is true, since if \( \delta > \delta' > 0 \) then \( V_{\delta'}(c) \subseteq V_{\delta}(c) \) for any \( c \in \R \).

        \item This is false. For a counterexample, consider Thomae's function \( t \). In \Cref{ex:3}, we showed that \( \lim_{x \to 1} t(x) = 0 \), but \( t(1) = 1 \).

        \item This is true and follows from several applications of Corollary 4.2.4 (Algebraic Limit Theorem for Functional Limits).

        \item This is false. Define \( f, g : \R \setminus \{ 0 \} \to \R \) by \( f(x) = x \) and \( g(x) = 1/x \). It is not hard to see that \( \lim_{x \to 0} f(x) = 0 \), but \( \lim_{x \to 0} f(x) g(x) = \lim_{x \to 0} 1 = 1 \).
    \end{enumerate}
\end{solution}

\begin{exercise}
\label{ex:7}
    Let \( g : A \to \R \) and assume that \( f \) is a bounded function on \( A \) in the sense that there exists \( M > 0 \) satisfying \( \abs{f(x)} \leq M \) for all \( x \in A \).

    Show that if \( \lim_{x \to c} g(x) = 0 \), then \( \lim_{x \to c} g(x) f(x) = 0 \) as well.
\end{exercise}

\begin{solution}
    Let \( \epsilon > 0 \) be given. Since \( \lim_{x \to c} g(x) = 0 \), there is a \( \delta > 0 \) such that \( 0 < \abs{x - c} < \delta \) and \( x \in A \) implies that \( \abs{g(x)} < \tfrac{\epsilon}{M} \). Observe that for such \( x \), we then have
    \[
        \abs{f(x) g(x)} = \abs{f(x)} \abs{g(x)} < M \tfrac{\epsilon}{M} = \epsilon.
    \]
    Thus \( \lim_{x \to c} g(x) f(x) = 0 \).
\end{solution}

\begin{exercise}
\label{ex:8}
    Compute each limit or state that it does not exist. Use the tools developed in this section to justify each conclusion.
    \begin{enumerate}
        \item \( \lim_{x \to 2} \tfrac{\abs{x-2}}{x-2} \)

        \item \( \lim_{x \to 7/4} \tfrac{\abs{x-2}}{x-2} \)

        \item \( \lim_{x \to 0} (-1)^{[[1/x]]} \)

        \item \( \lim_{x \to 0} \sqrt[3]{x} (-1)^{[[1/x]]} \)
    \end{enumerate}
\end{exercise}

\begin{solution}
    \begin{enumerate}
        \item Let \( f : \R \setminus \{ 2 \} \to \R \) be given by \( f(x) = \tfrac{\abs{x-2}}{x-2} \). Observe that
        \[
            f(x) = \begin{cases}
                1 & \text{if } x > 2, \\
                -1 & \text{if } x < 2.
            \end{cases}
        \]
        We claim that \( \lim_{x \to 2} f(x) \) does not exist. To see this, consider the sequences \( (x_n) \) and \( (y_n) \) given by \( x_n = 2 + \tfrac{1}{n} \) and \( y_n = 2 - \tfrac{1}{n} \). Then \( \lim x_n = \lim y_n = 2 \), but
        \[
            \lim f(x_n) = \lim 1 = 1 \neq -1 = \lim -1 = \lim f(y_n).
        \]
        Our claim now follows from Corollary 4.2.5.

        \item We claim that \( \lim_{x \to 7/4} f(x) = -1 \). To see this, let \( \epsilon > 0 \) be given. If \( x \in \R \setminus \{ 2 \} \) is such that \( 0 < \abs{x - 7/4} < 1/4 \), i.e.\ \( x \in (3/2, 2) \), then
        \[
            \abs{f(x) - (-1)} = \abs{-1 + 1} = 0 < \epsilon.
        \]
        Thus \( \lim_{x \to 7/4} f(x) = -1 \).

        \item We claim that \( \lim_{x \to 0} (-1)^{[[1/x]]} \) does not exist. To see this, consider the sequences \( (x_n) \) and \( (y_n) \) given by \( x_n = \tfrac{1}{2n} \) and \( y_n = \tfrac{1}{2n+1} \). Then \( \lim x_n = \lim y_n = 0 \), but
        \[
            \lim (-1)^{[[1/x_n]]} = \lim (-1)^{[[2n]]} = \lim 1 = 1 \neq -1 = \lim -1 = \lim (-1)^{[[2n+1]]} = \lim (-1)^{[[1/y_n]]}.
        \]
        Our claim now follows from Corollary 4.2.5.

        \item First, we claim that \( \lim_{x \to 0} \sqrt[3]{x} = 0 \). Let \( \epsilon > 0 \) be given and set \( \delta = \epsilon^3 \). Then if \( x \in \R \) is such that \( 0 < \abs{x} < \delta \), we have
        \[
            \abs{\sqrt[3]{x}} = \sqrt[3]{\abs{x}} < \sqrt[3]{\delta} = \epsilon.
        \]
        Thus \( \lim_{x \to 0} \sqrt[3]{x} = 0 \). Since the function \( (-1)^{[[1/x]]} \) is evidently bounded, we may apply \Cref{ex:7} to conclude that \( \lim_{x \to 0} \sqrt[3]{x} (-1)^{[[1/x]]} = 0 \).
    \end{enumerate}
\end{solution}

\begin{exercise}[Infinite Limits]
\label{ex:9}
    The statement \( \lim_{x \to 0} 1/x^2 = \infty \) certainly makes intuitive sense. To construct a rigorous definition in the challenge-response style of Definition 4.2.1 for an infinite limit statement of this form, we replace the (arbitrarily small) \( \epsilon > 0 \) challenge with an (arbitrarily large) \( M > 0 \) challenge:

    \textit{Definition}: \( \lim_{x \to c} f(x) = \infty \) means that for all \( M > 0 \) we can find a \( \delta > 0 \) such that whenever \( 0 < \abs{x - c} < \delta \), it follows that \( f(x) > M \).
    \begin{enumerate}
        \item Show \( \lim_{x \to 0} 1/x^2 = \infty \) in the sense described in the previous definition.

        \item Now, construct a definition for the statement \( \lim_{x \to \infty} f(x) = L \). Show \( \lim_{x \to \infty} 1/x = 0 \).

        \item What would a rigorous definition for \( \lim_{x \to \infty} f(x) = \infty \) look like? Give an example of such a limit.
    \end{enumerate}
\end{exercise}

\begin{solution}
    \begin{enumerate}
        \item Let \( M > 0 \) be given. Set \( \delta = \tfrac{1}{\sqrt{M}} > 0 \). Then if \( x \in \R \) is such that \( 0 < \abs{x} < \delta \), observe that
        \[
            \frac{1}{\abs{x}} > \sqrt{M} > 0 \implies \frac{1}{x^2} > M.
        \]
        It follows that \( \lim_{x \to 0} 1/x^2 = \infty \).

        \item The statement \( \lim_{x \to \infty} f(x) = L \) means that for all \( \epsilon > 0 \) we can find an \( M > 0 \) such that whenever \( x > M \), it follows that \( \abs{f(x) - L} < \epsilon \).

        To see that \( \lim_{x \to \infty} 1/x = 0 \), let \( \epsilon > 0 \) be given and set \( M = 1/\epsilon \). Then
        \[
            x > M = 1/\epsilon \implies 1/x < \epsilon.
        \]

        \item The statement \( \lim_{x \to \infty} f(x) = \infty \) means that for all \( M > 0 \) we can find a \( K > 0 \) such that whenever \( x > K \), it follows that \( f(x) > M \). It is not hard to see that \( \lim_{x \to \infty} x = \infty \) is an example of such a limit.
    \end{enumerate}
\end{solution}

\begin{exercise}[Right and Left Limits]
\label{ex:10}
    Introductory calculus courses typically refer to the \textit{right-hand limit} of a function as the limit obtained by ``letting \( x \) approach \( a \) from the right-hand side.''
    \begin{enumerate}
        \item Give a proper definition in the style of Definition 4.2.1 for the right-hand and left-hand limit statements:
        \[
            \lim_{x \to a^+} f(x) = L \quand \lim_{x \to a^-} f(x) = M.
        \]

        \item Prove that \( \lim_{x \to a} f(x) = L \) if and only if both the right and left-hand limits equal \( L \).
    \end{enumerate}
\end{exercise}

\begin{solution}
    \begin{enumerate}
        \item Suppose we have a function \( f : A \to \R \) and \( a \in \R \) is a limit point of \( A \). We say that \( \lim_{x \to a^+} f(x) = L \) provided that, for all \( \epsilon > 0 \), there exists a \( \delta > 0 \) such that whenever \( a < x < a + \delta \) and \( x \in A \), it follows that \( \abs{f(x) - L} < \epsilon \). Similarly, we say that \( \lim_{x \to a^-} f(x) = M \) provided that, for all \( \epsilon > 0 \), there exists a \( \delta > 0 \) such that whenever \( a - \delta < x < a \) and \( x \in A \), it follows that \( \abs{f(x) - M} < \epsilon \).

        \item If \( \lim_{x \to a} f(x) = L \), then certainly \( \lim_{x \to a^+} f(x) = \lim_{x \to a^-} f(x) = L \), since both \( a < x < a + \delta \) and \( a - \delta < x < a \) imply that \( 0 < \abs{x - a} < \delta \). Suppose therefore that \( \lim_{x \to a^+} f(x) = \lim_{x \to a^-} f(x) = L \) and let \( \epsilon > 0 \) be given. There are positive real numbers \( \delta_1 \) and \( \delta_2 \) such that
        \[
            a < x < a + \delta_1 \implies \abs{f(x) - L} < \epsilon \quand a - \delta_2 < x < a \implies \abs{f(x) - L} < \epsilon.
        \]
        Set \( \delta = \min \{ \delta_1, \delta_2 \} \) and suppose \( x \) is such that \( 0 < \abs{x - a} < \delta \). Then either
        \begin{gather*}
            a < x < a + \delta < a + \delta_1 \implies \abs{f(x) - L} < \epsilon, \text{ or} \\[2mm]
            a - \delta_2 < a - \delta < x < a \implies \abs{f(x) - L} < \epsilon.
        \end{gather*}
        In either case we have \( \abs{f(x) - L} < \epsilon \) and hence \( \lim_{x \to a} f(x) = L \).
    \end{enumerate}
\end{solution}

\begin{exercise}[Squeeze Theorem]
\label{ex:11}
    Let \( f, g, \) and \( h \) satisfy \( f(x) \leq g(x) \leq h(x) \) for all \( x \) in some common domain \( A \). If \( \lim_{x \to c} f(x) = L \) and \( \lim_{x \to c} h(x) = L \) at some limit point \( c \) of \( A \), show \( \lim_{x \to c} g(x) = L \).
\end{exercise}

\begin{solution}
    Suppose \( (x_n) \) is a sequence contained in \( A \) satisfying \( x_n \neq c \) and \( \lim x_n = c \). By assumption, we have \( f(x_n) \leq g(x_n) \leq h(x_n) \) for all \( n \in \N \), and Theorem 4.2.3 guarantees that \( \lim f(x_n) = \lim h(x_n) = L \). We may now apply the Squeeze Theorem for sequences (see \href{https://lew98.github.io/Mathematics/UA_Section_2_3_Exercises.pdf}{Exercise 2.3.3}) to see that \( \lim g(x_n) = L \). Theorem 4.2.3 allows us to conclude that \( \lim_{x \to c} g(x) = L \).
\end{solution}

\noindent \hrulefill

\noindent \hypertarget{ua}{\textcolor{blue}{[UA]} Abbott, S. (2015) \textit{Understanding Analysis.} 2\ts{nd} edition.}

\end{document}