\documentclass[12pt]{article}
\usepackage[utf8]{inputenc}
\usepackage[utf8]{inputenc}
\usepackage{amsmath}
\usepackage{amsthm}
\usepackage{geometry}
\usepackage{amsfonts}
\usepackage{mathrsfs}
\usepackage{bm}
\usepackage{hyperref}
\usepackage[dvipsnames]{xcolor}
\usepackage{enumitem}
\usepackage{mathtools}
\usepackage{changepage}
\usepackage{lipsum}
\usepackage{tikz}
\usetikzlibrary{matrix}
\usepackage{tikz-cd}
\usepackage[nameinlink]{cleveref}
\geometry{
headheight=15pt,
left=60pt,
right=60pt
}
\usepackage{fancyhdr}
\pagestyle{fancy}
\fancyhf{}
\lhead{}
\chead{Section 2.B Exercises}
\rhead{\thepage}
\hypersetup{
    colorlinks=true,
    linkcolor=blue,
    urlcolor=blue
}

\theoremstyle{definition}
\newtheorem*{remark}{Remark}

\newtheoremstyle{exercise}
    {}
    {}
    {}
    {}
    {\bfseries}
    {.}
    { }
    {\thmname{#1}\thmnumber{#2}\thmnote{ (#3)}}
\theoremstyle{exercise}
\newtheorem{exercise}{Exercise 2.B.}

\newtheoremstyle{solution}
    {}
    {}
    {}
    {}
    {\itshape\color{magenta}}
    {.}
    { }
    {\thmname{#1}\thmnote{ #3}}
\theoremstyle{solution}
\newtheorem*{solution}{Solution}

\Crefformat{exercise}{#2Exercise 2.B.#1#3}

\newcommand{\ts}{\textsuperscript}
\newcommand{\Span}{\text{span}}
\newcommand{\setcomp}[1]{#1^{\mathsf{c}}}
\newcommand{\N}{\mathbf{N}}
\newcommand{\Z}{\mathbf{Z}}
\newcommand{\Q}{\mathbf{Q}}
\newcommand{\R}{\mathbf{R}}
\newcommand{\C}{\mathbf{C}}
\newcommand{\F}{\mathbf{F}}

\DeclarePairedDelimiter\abs{\lvert}{\rvert}
% Swap the definition of \abs* and \norm*, so that \abs
% and \norm resizes the size of the brackets, and the 
% starred version does not.
\makeatletter
\let\oldabs\abs
\def\abs{\@ifstar{\oldabs}{\oldabs*}}
%
\let\oldnorm\norm
\def\norm{\@ifstar{\oldnorm}{\oldnorm*}}
\makeatother

\setlist[enumerate,1]{label={(\alph*)}}

\begin{document}

\section{Section 2.B Exercises}

Exercises with solutions from Section 2.B of \hyperlink{ladr}{[LADR]}.

\begin{exercise}
\label{ex:1}
    Find all vector spaces that have exactly one basis.
\end{exercise}

\begin{solution}
    We will consider only finite-dimensional vector spaces over \( \R \) or \( \C \).

    First consider the trivial vector space \( \{ 0 \} \). There are two possible lists of vectors: the empty list \( (\,) \) and the list \( 0 \). Since any list containing the zero vector is linearly dependent, the list \( 0 \) cannot be a basis for \( \{ 0 \} \). By definition the empty list is linearly independent and satisfies \( \Span (\,) = \{ 0 \} \), so we see that the empty list is a basis for \( \{ 0 \} \). Thus the trivial vector space has exactly one basis.

    Now suppose that \( V \neq \{ 0 \} \). By (2.32), \( V \) has a basis \( v_1, \ldots, v_m \). Since \( V \neq \{ 0 \} \), this basis is not the empty list, so \( v_1 \) exists and is non-zero. It follows that \( B := 2 v_1, \ldots, 2 v_m \) is distinct from \( v_1, \ldots, v_m \). By \href{https://lew98.github.io/Mathematics/LADR_Section_2_A_Exercises.pdf}{Exercise 2.A.8}, \( B \) is linearly independent. Furthermore, we claim that \( \Span\,B = V \). To see this, let \( v \in V \) be given. Since \( v_1, \ldots, v_m \) is a basis, there are scalars \( a_1, \ldots, a_m \) such that \( v = \sum_{j=1}^m a_j v_j \). This is equivalent to
    \[
        v = \sum_{j=1}^m \left( \tfrac{1}{2} a_j \right) (2 v_j),
    \]
    whence \( v \in \Span\,B \). It follows that \( V = \Span\,B \) and hence that \( B \) is a basis for \( V \), distinct from the original basis \( v_1, \ldots, v_m \). We may conclude that the trivial vector space is the only vector space which has exactly one basis.
\end{solution}

\begin{exercise}
\label{ex:2}
    Verify all the assertions in Example 2.28.
\end{exercise}

\begin{solution}
    \begin{enumerate}
        \item The assertion is that the list \( B := (1, 0, \ldots, 0), (0, 1, 0, \ldots, 0), \ldots, (0, \ldots, 0, 1) \) is a basis of \( \F^n \). Since any \( (x_1, x_2, \ldots, x_n) \in \F^n \) can be written as
        \[
            x_1 (1, 0, \ldots, 0) + x_2 (0, 1, 0, \ldots, 0) + \cdots + x_n (0, \ldots, 0, 1),
        \]
        it is clear that \( B \) spans \( \F^n \) and that \( B \) is linearly independent.

        \item The assertion is that the list \( B:= (1, 2), (3, 5) \) is a basis of \( \F^2 \). Solving the two equations \( x + 3y = 0 \) and \( 3x + 5y = 0 \) gives \( x = y = 0 \), demonstrating that \( B \) is linearly independent. If \( (a, b) \in \F^2 \), then observe that
        \[
            (-5a + 3b) (1, 2) + (2a - b) (3, 5) = (a, b).
        \]
        Hence \( \Span\,B = V \) and we may conclude that \( B \) is a basis of \( \F^2 \).

        \item The assertion is that the list \( B := (1, 2, -4), (7, -5, 6) \) is linearly independent in \( \F^3 \) but is not a basis of \( \F^3 \) because it does not span \( \F^3 \). Solving the equations \( x + 7y = 0 \) and \( 2x - 5y = 0 \) gives \( x = y = 0 \), demonstrating that \( B \) is linearly independent. However, since the list \( (1, 0, 0), (0, 1, 0), (0, 0, 1) \) of length 3 is linearly independent in \( \F^3 \) (see (a)), (2.23) implies that \( B \) cannot span \( \F^3 \).

        \item The assertion is that the list \( B := (1, 2), (3, 5), (4, 13) \) spans \( \F^2 \) but is not a basis of \( \F^2 \) because it is not linearly independent. Part (b) shows that \( B \) spans \( \F^2 \) and that \( (4, 13) \) lies in the span of \( (1, 2) \) and \( (3, 5) \), so that \( B \) is linearly dependent.

        \item The assertion is that the list \( B:= (1, 1, 0), (0, 0, 1) \) is a basis of \( U := \{ (x, x, y) \in \F^3 : x, y \in \F \} \). \( \Span\,U = B \) since \( x(1, 1, 0) + y(0, 0, 1) = (x, x, y) \), and \( B \) is linearly independent since \( (x, x, y) = (0, 0, 0) \) forces \( x = y = 0 \).

        \item The assertion is that the list \( B := (1, -1, 0), (1, 0, -1) \) is a basis of
        \[
            U := \{ (x, y, z) \in \F^3 : x + y + z = 0 \} = \{ (-y - z, y, z) \in \F^3 : y, z \in \F \}.
        \]
        \( B \) is linearly independent since
        \[
            y(1, -1, 0) + z(1, 0, -1) = (y + z, -y, -z) = (0, 0, 0)
        \]
        gives \( y = z = 0 \), and \( B \) spans \( U \) since
        \[
            (-y - z, y, z) = (-y)(1, -1, 0) + (-z)(1, 0, -1).
        \]

        \item The assertion is that the list \( B := 1, z, \ldots, z^m \) is a basis of \( \mathcal{P}_m (\F) \). \( B \) is linearly independent by \href{https://lew98.github.io/Mathematics/LADR_Section_2_A_Exercises.pdf}{Exercise 2.A.2 (d)}, and \( B \) spans \( \mathcal{P}_m (\F) \) since any polynomial in \( \mathcal{P}_m (\F) \) is of the form
        \[
            a_0 + a_1 z + \cdots + a_m z^m
        \]
        for scalars \( a_0, a_1, \ldots, a_m \).
    \end{enumerate}
\end{solution}

\begin{exercise}
\label{ex:3}
    \begin{enumerate}
        \item Let \( U \) be the subspace of \( \R^5 \) defined by
        \[
            U = \{ (x_1, x_2, x_3, x_4, x_5) \in \R^5 : x_1 = 3 x_2 \text{ and } x_3 = 7 x_4 \}.
        \]
        Find a basis of \( U \).

        \item Extend the basis in part (a) to a basis of \( \R^5 \).

        \item Find a subspace \( W \) of \( \R^5 \) such that \( \R^5 = U \oplus W \).
    \end{enumerate}
\end{exercise}

\begin{solution}
    \begin{enumerate}
        \item \( U \) is the subspace
        \[
            U = \{ (3 x_2, x_2, 7 x_4, x_4, x_5) \in \R^5 : x_2, x_4, x_5 \in \R \}.
        \]
        Let \( v_1 = (3, 1, 0, 0, 0), v_2 = (0, 0, 7, 1, 0), v_3 = (0, 0, 0, 0, 1) \), and \( B = v_1, v_2, v_3 \). Then since
        \[
            x_2 v_1 + x_4 v_2 + x_5 v_3 = (3 x_2, x_2, 7 x_4, x_4, x_5),
        \]
        it is clear that \( B \) spans \( U \) and that \( B \) is linearly independent.

        \item Denote the \( i \)\ts{th} standard basis vector of \( \R^5 \) by \( e_i \). Following the procedure outlined in (2.31) and (2.33), we adjoin the five standard basis vectors to \( B \) to obtain the spanning list
        \[
            v_1, v_2, v_3, e_1, e_2, e_3, e_4, e_5.
        \]
        \begin{itemize}
            \item \( e_1 \) does not belong to \( \Span(v_1, v_2, v_3) \), so we do not delete it.

            \item Note that \( e_2 = v_1 - 3 e_1 \), so we delete \( e_2 \) from the list.

            \item \( e_3 \) does not belong to \( \Span(v_1, v_2, v_3, e_1) \), so we do not delete it.

            \item Note that \( e_4 = v_2 - 7 e_3 \), so we delete \( e_4 \) from the list.

            \item Since \( e_5 = v_3 \), we delete \( e_5 \) from the list.
        \end{itemize}
        We are left with the list \( v_1, v_2, v_3, e_1, e_3 \); as the proof of (2.33) shows, this must be a basis of \( \R^5 \).

        \item As shown in the proof of (2.34), if we let
        \[
            W = \Span (e_1, e_3) = \{ (x_1, 0, x_3, 0, 0) \in \R^5 : x_1, x_3 \in \R \},
        \]
        then \( \R^5 = U \oplus W \).
    \end{enumerate}
\end{solution}

\begin{exercise}
\label{ex:4}
    \begin{enumerate}
        \item Let \( U \) be the subspace of \( \C^5 \) defined by
        \[
            U = \{ (z_1, z_2, z_3, z_4, z_5) \in \C^5 : 6 z_1 = z_2 \text{ and } z_3 + 2 z_4 + 3 z_5 = 0 \}.
        \]
        Find a basis of \( U \).

        \item Extend the basis in part (a) to a basis of \( \C^5 \).

        \item Find a subspace \( W \) of \( \C^5 \) such that \( \C^5 = U \oplus W \).
    \end{enumerate}
\end{exercise}

\begin{solution}
    \begin{enumerate}
        \item \( U \) is the subspace
        \[
            \{ (z_1, 6 z_1, -2 z_4 - 3 z_5, z_4, z_5) \in \C^5 : z_1, z_4, z_5 \in \C \}.
        \]
        Let \( v_1 = (1, 6, 0, 0, 0), v_2 = (0, 0, -2, 1, 0), v_3 = (0, 0, -3, 0, 1), \) and \( B = v_1, v_2, v_3 \). Then since
        \[
            z_1 v_1 + z_4 v_2 + z_5 v_3 = (z_1, 6 z_1, -2 z_4 - 3 z_5, z_4, z_5),
        \]
        it is clear that \( B \) spans \( U \) and that \( B \) is linearly independent.

        \item Denote the \( i \)\ts{th} standard basis vector of \( \C^5 \) by \( e_i \). Following the procedure outlined in (2.31) and (2.33), we adjoin the five standard basis vectors to \( B \) to obtain the spanning list
        \[
            v_1, v_2, v_3, e_1, e_2, e_3, e_4, e_5.
        \]
        \begin{itemize}
            \item \( e_1 \) does not belong to \( \Span(v_1, v_2, v_3) \), so we do not delete it.

            \item Note that \( e_2 = \tfrac{1}{6} (v_1 - e_1) \), so we delete \( e_2 \) from the list.

            \item \( e_3 \) does not belong to \( \Span(v_1, v_2, v_3, e_1) \), so we do not delete it.

            \item Note that \( e_4 = v_2 + 2 e_3 \), so we delete \( e_4 \) from the list.

            \item Note that \( e_5 = v_3 + 3 e_3 \), so we delete \( e_5 \) from the list.
        \end{itemize}
        We are left with the list \( v_1, v_2, v_3, e_1, e_3 \); as the proof of (2.33) shows, this must be a basis of \( \C^5 \).

        \item As shown in the proof of (2.34), if we let
        \[
            W = \Span (e_1, e_3) = \{ (z_1, 0, z_3, 0, 0) \in \C^5 : z_1, z_3 \in \C \},
        \]
        then \( \C^5 = U \oplus W \).
    \end{enumerate}
\end{solution}

\begin{exercise}
\label{ex:5}
    Prove or disprove: there exists a basis \( p_0, p_1, p_2, p_3 \) of \( \mathcal{P}_3 (\F) \) such that none of the polynomials \( p_0, p_1, p_2, p_3 \) has degree 2.
\end{exercise}

\begin{solution}
    This is true. Consider \( B = 1, x, x^2 + x^3, x^3 \); none of the polynomials in this list has degree 2. Suppose \( a_0, a_1, a_2, a_3 \) are scalars such that
    \[
        a_0 + a_1 x + a_2 (x^2 + x^3) + a_3 x^3 = a_0 + a_1 x + a_2 x^2 + (a_2 + a_3) x^3 = 0
    \]
    for all \( x \in \F \). This implies that \( a_0 = a_1 = a_2 = a_2 + a_3 = 0 \), which in turn gives \( a_3 = 0 \). It follows that \( B \) is linearly independent. Now suppose that \( p = a_0 + a_1 x + a_2 x^2 + a_3 x^3 \in \mathcal{P}_3 (\F) \) is given. Observe that
    \[
        a_0 + a_1 x + a_2 (x^2 + x^3) + (a_3 - a_2) x^3 = p,
    \]
    so that \( p \in \Span\,B \). Thus \( B \) is a basis for \( \mathcal{P}_3 (\F) \).
\end{solution}

\begin{exercise}
\label{ex:6}
    Suppose \( v_1, v_2, v_3, v_4 \) is a basis of \( V \). Prove that
    \[
        v_1 + v_2, v_2 + v_3, v_3 + v_4, v_4
    \]
    is also a basis of \( V \).
\end{exercise}

\begin{solution}
    Suppose there are scalars \( a_1, a_2, a_3, a_4 \) such that
    \[
        a_1 (v_1 + v_2) + a_2 (v_2 + v_3) + a_3 (v_3 + v_4) + a_4 v_4 = a_1 v_1 + (a_1 + a_2) v_2 + (a_2 + a_3) v_3 + (a_3 + a_4) v_4 = 0.
    \]
    Since \( v_1, v_2, v_3, v_4 \) is a basis, this implies that \( a_1 = a_1 + a_2 = a_2 + a_3 = a_3 + a_4 = 0 \), which in turn gives \( a_1 = a_2 = a_3 = a_4 = 0 \). Hence the list \( v_1 + v_2, v_2 + v_3, v_3 + v_4, v_4 \) is linearly independent. Now let \( v \in V \) be given. Since \( v_1, v_2, v_3, v_4 \) is a basis, there are scalars \( a_1, a_2, a_3, a_4 \) such that \( v = a_1 v_1 + a_2 v_2 + a_3 v_3 + a_4 v_4 \). Then observe that
    \begin{multline*}
        a_1 (v_1 + v_2) + (a_2 - a_1)(v_2 + v_3) + (a_3 - a_2 + a_1)(v_3 + v_4) + (a_4 - a_3 + a_2 - a_1)v_4 \\
        = a_1 v_1 + a_2 v_2 + a_3 v_3 + a_4 v_4 = v.
    \end{multline*}
    It follows that \( v_1 + v_2, v_2 + v_3, v_3 + v_4, v_4 \) spans \( V \) and hence that \( v_1 + v_2, v_2 + v_3, v_3 + v_4, v_4 \) is a basis for \( V \).
\end{solution}

\begin{exercise}
\label{ex:7}
    Prove or give a counterexample: If \( v_1, v_2, v_3, v_4 \) is a basis of \( V \) and \( U \) is a subspace of \( V \) such that \( v_1, v_2 \in U \) and \( v_3 \not\in U \) and \( v_4 \not\in U \), then \( v_1, v_2 \) is a basis of \( U \).
\end{exercise}

\begin{solution}
    For a counterexample, consider \( V = \R^4 \), let \( v_1 = (1, 0, 0, 0), v_2 = (0, 1, 0, 0) , v_3 = (0, 0, 1, 1), v_4 = (1, 0, 0, 1) \), and \( U = \Span(v_1, v_2, (0, 0, 1, 0)) \). Suppose we have scalars \( a_1, a_2, a_3, a_4 \) such that
    \[
        a_1 v_1 + a_2 v_2 + a_3 v_3 + a_4 v_4 = (a_1 + a_4, a_2, a_3, a_3 + a_4) = (0, 0, 0, 0).
    \]
    It is easy to see that this gives \( a_1 = a_2 = a_3 = a_4 = 0 \); it follows that the list \( v_1, v_2, v_3, v_4 \) is linearly independent. Suppose that \( (a_1, a_2, a_3, a_4) \in \R^4 \) is given. Then
    \[
        (a_1 + a_3 - a_4) v_1 + a_2 v_2 + a_3 v_3 + (a_4 - a_3) v_4 = (a_1, a_2, a_3, a_4).
    \]
    Thus \( \R^4 = \Span(v_1, v_2, v_3, v_4) \), and so \( v_1, v_2, v_3, v_4 \) is a basis for \( \R^4 \). Clearly, \( v_1, v_2 \in U \). Since each vector \( (a_1, a_2, a_3, a_4) \) in \( U \) must satisfy \( a_4 = 0 \), we have \( v_3, v_4 \not\in U \). However, \( v_1, v_2 \) is not a basis for \( U \): since \( v_1, v_2, \) and \( (0, 0, 1, 0) \) are evidently linearly independent, any spanning list for \( U \) must contain at least three vectors.
\end{solution}

\begin{exercise}
\label{ex:8}
    Suppose \( U \) and \( W \) are subspaces of \( V \) such that \( V = U \oplus W \). Suppose also that \( u_1, \ldots, u_m \) is a basis of \( U \) and \( w_1, \ldots, w_n \) is a basis of \( W \). Prove that
    \[
        u_1, \ldots, u_m, w_1, \ldots, w_n
    \]
    is a basis of \( V \).
\end{exercise}

\begin{solution}
    Let \( v \in V \) be given. Since \( V = U \oplus W \), there are unique vectors \( u \in U \) and \( w \in W \) such that \( v = u + w \). Since \( u_1, \ldots, u_m \) is a basis for \( U \), by (2.29) there are unique scalars \( a_1, \ldots, a_m \) such that \( u = a_1 u_1 + \cdots + a_m u_m \). Similarly, there are unique scalars \( b_1, \ldots, b_n \) such that \( w = b_1 w_1 + \cdots + b_n w_n \). It follows that \( v \) can be uniquely represented as
    \[
        v = a_1 u_1 + \cdots + a_m u_m + b_1 w_1 + \cdots b_n w_n.
    \]
    Hence by (2.29), \( u_1, \ldots, u_m, w_1, \ldots, w_n \) is a basis for \( V \).
\end{solution}

\noindent \hrulefill

\noindent \hypertarget{ladr}{\textcolor{blue}{[LADR]} Axler, S. (2015) \textit{Linear Algebra Done Right.} 3rd edn.}

\end{document}