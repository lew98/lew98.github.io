\documentclass[12pt]{article}
\usepackage[utf8]{inputenc}
\usepackage[utf8]{inputenc}
\usepackage{amsmath}
\usepackage{amsthm}
\usepackage{geometry}
\usepackage{amsfonts}
\usepackage{mathrsfs}
\usepackage{bm}
\usepackage{hyperref}
\usepackage[dvipsnames]{xcolor}
\usepackage{enumitem}
\usepackage{mathtools}
\usepackage{changepage}
\usepackage{lipsum}
\usepackage{tikz}
\usetikzlibrary{matrix}
\usepackage{tikz-cd}
\usepackage[nameinlink]{cleveref}
\geometry{
headheight=15pt,
left=60pt,
right=60pt
}
\usepackage{fancyhdr}
\pagestyle{fancy}
\fancyhf{}
\lhead{}
\chead{Section 2.2 Exercises}
\rhead{\thepage}
\hypersetup{
    colorlinks=true,
    linkcolor=blue,
    urlcolor=blue
}

\theoremstyle{definition}
\newtheorem*{remark}{Remark}

\newtheoremstyle{exercise}
    {}
    {}
    {}
    {}
    {\bfseries}
    {.}
    { }
    {\thmname{#1}\thmnumber{#2}\thmnote{ (#3)}}
\theoremstyle{exercise}
\newtheorem{exercise}{Exercise 2.2.}

\newtheoremstyle{solution}
    {}
    {}
    {}
    {}
    {\itshape\color{magenta}}
    {.}
    { }
    {\thmname{#1}\thmnote{ #3}}
\theoremstyle{solution}
\newtheorem*{solution}{Solution}

\Crefformat{exercise}{#2Exercise 2.2.#1#3}

\newcommand{\setcomp}[1]{#1^{\mathsf{c}}}
\newcommand{\N}{\mathbf{N}}
\newcommand{\Z}{\mathbf{Z}}
\newcommand{\Q}{\mathbf{Q}}
\newcommand{\R}{\mathbf{R}}
\newcommand{\C}{\mathbf{C}}

\DeclarePairedDelimiter\abs{\lvert}{\rvert}
% Swap the definition of \abs* and \norm*, so that \abs
% and \norm resizes the size of the brackets, and the 
% starred version does not.
\makeatletter
\let\oldabs\abs
\def\abs{\@ifstar{\oldabs}{\oldabs*}}
%
\let\oldnorm\norm
\def\norm{\@ifstar{\oldnorm}{\oldnorm*}}
\makeatother

\setlist[enumerate,1]{label={(\alph*)}}

\begin{document}

\section{Section 2.2 Exercises}

Exercises with solutions from Section 2.2 of \hyperlink{ua}{[UA]}.

\begin{exercise}
\label{ex:1}
    What happens if we reverse the order of the quantifiers in Definition 2.2.3?

    \textit{Definition:} A sequence \( (x_n) \) \textit{verconges} to \( x \) if \textit{there exists} an \( \epsilon > 0 \) such that \textit{for all} \( N \in \N \) it is true that \( n \geq N \) implies \( |x_n - x| < \epsilon \).

    Give an example of a vercongent sequence. Is there an example of a vercongent sequence that is divergent? Can a sequence verconge to two different values? What exactly is being described in this strange definition?
\end{exercise}

\begin{solution}
    First observe that the statement
    \[
        \text{for all } N \in \N,\, n \geq N \implies |x_n - x| < \epsilon
    \]
    is equivalent to
    \[
        \text{for all } n \in \N,\, |x_n - x| < \epsilon.
    \]
    So a sequence verconges to \( x \) if there exists an \( \epsilon > 0 \) such that \( |x_n - x| < \epsilon \), or equivalently such that \( x_n \in (x - \epsilon, x + \epsilon) \), for all \( n \in \N \). Such a sequence is then bounded; conversely, if a sequence is bounded then it must verconge to some \( x \).

    For an example of a vercongent sequence, take \( (x_n) = (1, 1, 1, 1, \ldots) \). This sequence verconges to 1 since \( |x_n - 1| = 0 < \epsilon \) for all \( n \in \N \), for any choice of \( \epsilon \) we make; \( \epsilon = 1 \) will do. It is clear that this sequence also converges to 1.

    A vercongent sequence can also be divergent. For an example, consider \( (x_n) = (1, 0, 1, 0, \ldots) \). This sequence verconges to \( \tfrac{1}{2} \) since \( |x_n - \tfrac{1}{2}| = \tfrac{1}{2} < 1 \) for all \( n \in \N \). This sequence also diverges. To see this, suppose there was some \( x \) such that \( \lim x_n = x \). Then there must exist some \( N \in \N \) such that \( n \geq N \) implies that \( |x_n - x| < \tfrac{1}{2} \). Observe that
    \[
        1 = |x_N - x_{N+1}| \leq |x_N - x| + |x_{N+1} - x| < \tfrac{1}{2} + \tfrac{1}{2} = 1,
    \]
    i.e.\ \( 1 < 1 \), which is a contradiction.

    A sequence can verconge to two different values; take \( (x_n) = (1, 1, 1, 1, \ldots) \) again. Then \( (x_n) \) verconges to 1 and also to 0, since \( |x_n - 0| = 1 < 2 \) for all \( n \in \N \).
\end{solution}

\begin{exercise}
\label{ex:2}
    Verify, using the definition of convergence of a sequence, that the following sequences converge to the proposed limit.
    \begin{enumerate}
        \item \( \lim \tfrac{2n + 1}{5n + 4} = \tfrac{2}{5} \).

        \item \( \lim \tfrac{2n^2}{n^3 + 3} = 0 \).

        \item \( \lim \tfrac{\sin(n^2)}{\sqrt[3]{n}} = 0 \).
    \end{enumerate}
\end{exercise}

\begin{solution}
    \begin{enumerate}
        \item Let \( \epsilon > 0 \) be given. Choose \( N \in \N \) such that \( N > \tfrac{3}{25 \epsilon} \) and observe that for \( n \geq N \) we have
        \[
            \abs{\frac{2n + 1}{5n + 4} - \frac{2}{5}} = \frac{3}{25n + 20} < \frac{3}{25n} < \epsilon.
        \]
        It follows that \( \lim \tfrac{2n + 1}{5n + 4} = \tfrac{2}{5} \).

        \item Let \( \epsilon > 0 \) be given. Choose \( N \in \N \) such that \( N > \tfrac{2}{\epsilon} \) and observe that for \( n \geq N \) we have
        \[
            \abs{\frac{2n^2}{n^3 + 3} - 0} = \frac{2n^2}{n^3 + 3} < \frac{2n^2}{n^3} = \frac{2}{n} < \epsilon.
        \]
        It follows that \( \lim \tfrac{2n^2}{n^3 + 3} = 0 \).

        \item Let \( \epsilon > 0 \) be given. Choose \( N \in \N \) such that \( N > \tfrac{1}{\epsilon^3} \) and observe that for \( n \geq N \) we have
        \[
            \abs{\frac{\sin(n^2)}{\sqrt[3]{n}} - 0} = \frac{\abs{\sin(n^2)}}{\sqrt[3]{n}} \leq \frac{1}{\sqrt[3]{n}} < \epsilon.
        \]
        It follows that \( \lim \tfrac{\sin(n^2)}{\sqrt[3]{n}} = 0 \).
    \end{enumerate}
\end{solution}

\begin{exercise}
\label{ex:3}
    Describe what we would have to demonstrate in order to disprove each of the following statements.
    \begin{enumerate}
        \item At every college in the United States, there is a student who is at least seven feet tall.

        \item For all colleges in the United States, there exists a professor who gives every student a grade of either A or B.

        \item There exists a college in the United States where every student is at least six feet tall.
    \end{enumerate}
\end{exercise}

\begin{solution}
    \begin{enumerate}
        \item We would have to find a college in the United States where every student is less than seven feet tall.

        \item We would have to find a college in the United States where each professor gives at least one student a grade of C or worse.

        \item We would have to show that every college in the United States has a student who is less than six feet tall.
    \end{enumerate}
\end{solution}

\begin{exercise}
\label{ex:4}
    Give an example of each or state that the request is impossible. For any that are impossible, give a compelling argument for why that is the case.
    \begin{enumerate}
        \item A sequence with an infinite number of ones that does not converge to one.

        \item A sequence with an infinite number of ones that converges to a limit not equal to one.

        \item A divergent sequence such that for every \( n \in \N \) it is possible to find \( n \) consecutive ones somewhere in the sequence.
    \end{enumerate}
\end{exercise}

\begin{solution}
    \begin{enumerate}
        \item Consider \( (x_n) = (1, 0, 1, 0, \ldots) \). This sequence has an infinite number of ones but, as shown in \Cref{ex:1}, diverges.

        \item This is impossible. Suppose \( (x_n) \) is such a sequence with \( \lim x_n = x \neq 1 \). Then there must exist some \( N \in \N \) such that for all \( n \geq N \) we have \( |x_n - x| < |1 - x| \). Since this sequence contains infinitely many ones, it must be the case that there is some \( m \geq N \) such that \( x_m = 1 \). This implies that \( |x_m - x| = |1 - x| < |1 - x| \), which is a contradiction.

        \item Consider the sequence
        \[
            (x_n) = (1, 0, 1, 1, 0, 1, 1, 1, 0, 1, 1, 1, 1, 0, \ldots).
        \]
        Clearly, for each \( n \in \N \) we can find \( n \) consecutive ones somewhere in the sequence. Furthermore, the sequence is divergent. To see this, suppose there was some \( x \) such that \( \lim x_n = x \). Then there must exist some \( N \in \N \) such that \( n \geq N \) implies that \( |x_n - x| < \tfrac{1}{2} \). Since the sequence contains infinitely many ones and zeros, we can find indices \( k, l \geq N \) such that \( x_k = 1 \) and \( x_l = 0 \). Then
        \[
            1 = |x_k - x_l| \leq |x_k - x| + |x_l - x| < \tfrac{1}{2} + \tfrac{1}{2} = 1,
        \]
        i.e.\ \( 1 < 1 \), which is a contradiction.
    \end{enumerate}
\end{solution}

\begin{exercise}
\label{ex:5}
    Let \( [[x]] \) be the greatest integer less than or equal to \( x \). For example, \( [[\pi]] = 3 \) and \( [[3]] = 3 \). For each sequence, find \( \lim a_n \) and verify it with the definition of convergence.
    \begin{enumerate}
        \item \( a_n = [[5/n]] \),

        \item \( a_n = [[(12 + 4n)/3n]] \).
    \end{enumerate}
    Reflecting on these examples, comment on the statement following Definition 2.2.3 that ``the smaller the \(\epsilon\)-neighborhood, the larger \( N \) may have to be."
\end{exercise}

\begin{solution}
    \begin{enumerate}
        \item We claim that \( \lim a_n = 0 \). Let \( \epsilon > 0 \) be given and observe that if \( n \geq 6 \), then \( 0 < 5/n < 1 \implies [[5/n]] = 0 \). So if we take \( N \geq 6 \), then \( n \geq N \) implies that \( |[[5/n]] - 0| = 0 < \epsilon \).

        \item We claim that \( \lim a_n = 1 \). Let \( \epsilon > 0 \) be given and observe that if \( n \geq 7 \), then
        \[
            \tfrac{1}{n} < \tfrac{1}{6} \iff \tfrac{4}{n} < \tfrac{2}{3} \iff \tfrac{4}{n} + \tfrac{1}{3} < 1.
        \]
        Hence for \( n \geq 7 \) we have \( 0 < 4/n + 1/3 < 1 \implies [[4/n + 1/3]] = 0 \). So if we take \( N \geq 7 \), then \( n \geq N \) implies that
        \[
            \left[ \left[ \frac{12 + 4n}{3n} - 1 \right] \right] = \left[ \left[ \frac{4}{n} + \frac{1}{3} \right] \right] = 0 < \epsilon.
        \]
    \end{enumerate}
    These examples demonstrate that taking smaller \(\epsilon\)-neighborhoods may not require us to take larger values of \( N \); the same value of \( N \) in each example works for every \(\epsilon\)-neighborhood that we choose.
\end{solution}

\begin{exercise}
\label{ex:6}
    Prove Theorem 2.2.7. To get started, assume \( (a_n) \to a \) and \( (a_n) \to b \). Now argue \( a = b \).
\end{exercise}

\begin{solution}
    Let \( \epsilon > 0 \) be given. Then there are positive integers \( N_1 \) and \( N_2 \) such that
    \[
        n \geq N_1 \implies |a_n - a| < \tfrac{\epsilon}{2} \quad \text{and} \quad n \geq N_2 \implies |a_n - b| < \tfrac{\epsilon}{2}.
    \]
    Let \( N = \max \{ N_1, N_2 \} \) and observe that for \( n \geq N \) we have
    \[
        |a - b| = |a - a_n + a_n - b| \leq |a_n - a| + |a_n - b| < \tfrac{\epsilon}{2} + \tfrac{\epsilon}{2} = \epsilon.
    \]
    So we have shown that \( |a - b| < \epsilon \) for any \( \epsilon > 0 \). It follows that \( a = b \).
\end{solution}

\begin{exercise}
\label{ex:7}
    Here are two useful definitions:
    \begin{enumerate}[label = (\roman*)]
        \item A sequence \( (a_n) \) is \textit{eventually} in a set \( A \subseteq \R \) if there exists an \( N \in \N \) such that \( a_n \in A \) for all \( n \geq N \).

        \item A sequence \( (a_n) \) is \textit{frequently} in a set \( A \subseteq \R \) if, for every \( N \in \N \), there exists an \( n \geq N \) such that \( a_n \in A \).
        \begin{enumerate}
            \item Is the sequence \( (-1)^n \) eventually or frequently in the set \( \{ 1 \} \)?

            \item Which definition is stronger? Does frequently imply eventually or does eventually imply frequently?

            \item Give an alternate rephrasing of Definition 2.2.3B using either frequently or eventually. Which is the term we want?

            \item Suppose an infinite number of terms of a sequence \( (x_n) \) are equal to 2. Is \( (x_n) \) necessarily eventually in the interval \( (1.9, 2.1) \)? Is it frequently in \( (1.9, 2.1) \)?
        \end{enumerate}
    \end{enumerate}
\end{exercise}

\begin{solution}
    \begin{enumerate}
        \item The sequence \( (-1)^n \) is frequently but not eventually in the set \( \{ 1 \} \). To see this, let \( N \in \N \) be given. If \( N \) is even, then \( (-1)^N \in \{ 1 \} \) and \( (-1)^{N+1} \not\in \{ 1 \} \), and if \( N \) is odd then \( (-1)^N \not\in \{ 1 \} \) and \( (-1)^{N+1} \in \{ 1 \} \). In any case, we can always find indices \( m, n \geq N \) such that \( (-1)^m \not\in \{ 1 \} \) (this says that the sequence is not eventually in \( \{ 1 \} \)) and such that \( (-1)^n \in \{ 1 \} \) (this says that the sequence is frequently in \( \{ 1 \} \)).

        \item Eventually is the stronger definition. Frequently does not imply eventually, as part (a) shows, but eventually does imply frequently. To see this, suppose that \( (a_n) \) is eventually in a set \( A \), i.e.\ there is an \( N \in \N \) such that \( a_n \in A \) for all \( n \geq N \). Let \( M \in \N \) be given. Set \( n = \max \{ M, N \} \) and observe that \( n \geq M \) and \( n \geq N \implies a_n \in A \). Hence \( (a_n) \) is frequently in \( A \).

        \item The term we want is eventually. Here is a rephrasing of Definition 2.2.3B. A sequence \( (a_n) \) converges to \( a \) if, given any \( \epsilon > 0 \), the sequence \( (a_n) \) is eventually in the \(\epsilon\)-neighborhood \( V_{\epsilon}(a) \) of \( a \).

        \item Such a sequence is not necessarily eventually in \( (1.9, 2.1) \); consider the sequence \( (x_n) = (2, 0, 2, 0, 2, \ldots) \) for example. For any \( N \in \N \), we can always find an index \( n \geq N \) (either \( n = N \) or \( n = N + 1 \)) such that \( x_n = 0 \not\in (1.9, 2.1) \). However, such a sequence must be frequently in \( (1.9, 2.1) \). To see this, let \( N \in \N \) be given. Then there must exist an index \( n \geq N \) such that \( x_n = 2 \in (1.9, 2.1) \) (otherwise there would be only finitely many twos in the sequence).
    \end{enumerate}
\end{solution}

\begin{exercise}
\label{ex:8}
    For some additional practice with nested quantifiers, consider the following invented defintion:

    Let's call a sequence \( (x_n) \) \textit{zero-heavy} if there exists \( M \in \N \) such that for all \( N \in \N \) there exists \( n \) satisfying \( N \leq n \leq N + M \) where \( x_n = 0 \).
    \begin{enumerate}
        \item Is the sequence \( (0, 1, 0, 1, 0, 1, \ldots) \) zero-heavy?

        \item If a sequence is zero-heavy does it necessarily contain an infinite number of zeros? If not, provide a counterexample.

        \item If a sequence contains an infinite number of zeros, is it necessarily zero-heavy? If not, provide a counterexample.

        \item Form the logical negation of the above definition. That is, complete the sentence: A sequence is \textit{not} zero-heavy if ....
    \end{enumerate}
\end{exercise}

\begin{solution}
    \begin{enumerate}
        \item This sequence is zero-heavy; \( M = 1 \) works. Indeed, let \( N \in \N \) be given. If \( N \) is odd then let \( n = N \) and if \( N \) is even then let \( n = N + 1 \). In either case, we have \( N \leq n \leq N + 1 \) and \( x_n = 0 \).

        \item A zero-heavy sequence must contain an infinite number of zeros. To see this, suppose \( (x_n) \) is a sequence with a finite number of zeros, i.e.\ there is an \( N \in \N \) such that \( x_n \neq 0 \) for all \( n \geq N \). Then no matter which \( M \) we choose, we will never be able to find \( n \in \N \) with \( N \leq n \leq N + M \) and \( x_n = 0 \). Hence the sequence \( (x_n) \) is not zero-heavy.

        \item A sequence with an infinite number of zeros is not necessarily zero-heavy. For a counterexample, consider the sequence
        \[
            (x_n) = (1, 0, 1, 1, 0, 1, 1, 1, 0, 1, 1, 1, 1, 0, \ldots).
        \]
        This sequence contains infinitely many zeros, but is not zero-heavy. To see this, let \( M \in \N \) be given. Then it is always possible to find \( M \) consecutive ones in the sequence \( (x_n) \); suppose this string of ones starts at \( x_N = 1 \). Then for each \( n \in \N \) satisfying \( N \leq n \leq N + M \), we have \( x_n = 1 \neq 0 \).

        \item A sequence is not zero-heavy if for every \( M \in \N \) there exists an \( N \in \N \) such that \( x_n \neq 0 \) for each \( n \in \N \) satisfying \( N \leq n \leq N + M \).
    \end{enumerate}
\end{solution}

\noindent \hrulefill

\noindent \hypertarget{ua}{\textcolor{blue}{[UA]} Abbott, S. (2015) \textit{Understanding Analysis.} 2nd edn.}

\end{document}
