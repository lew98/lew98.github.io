\documentclass[12pt]{article}
\usepackage[utf8]{inputenc}
\usepackage[utf8]{inputenc}
\usepackage{amsmath}
\usepackage{amsthm}
\usepackage{amssymb}
\usepackage{geometry}
\usepackage{amsfonts}
\usepackage{mathrsfs}
\usepackage{bm}
\usepackage{hyperref}
\usepackage{float}
\usepackage[dvipsnames]{xcolor}
\usepackage[inline]{enumitem}
\usepackage{mathtools}
\usepackage{changepage}
\usepackage{graphicx}
\usepackage{caption}
\usepackage{subcaption}
\usepackage{lipsum}
\usepackage{tikz}
\usetikzlibrary{matrix, patterns, decorations.pathreplacing, calligraphy}
\usepackage{tikz-cd}
\usepackage[nameinlink]{cleveref}
\geometry{
headheight=15pt,
left=60pt,
right=60pt
}
\setlength{\emergencystretch}{20pt}
\usepackage{fancyhdr}
\pagestyle{fancy}
\fancyhf{}
\lhead{}
\chead{Section 6.5 Exercises}
\rhead{\thepage}
\hypersetup{
    colorlinks=true,
    linkcolor=blue,
    urlcolor=blue
}

\theoremstyle{definition}
\newtheorem*{remark}{Remark}

\newtheoremstyle{exercise}
    {}
    {}
    {}
    {}
    {\bfseries}
    {.}
    { }
    {\thmname{#1}\thmnumber{#2}\thmnote{ (#3)}}
\theoremstyle{exercise}
\newtheorem{exercise}{Exercise 6.5.}

\newtheoremstyle{solution}
    {}
    {}
    {}
    {}
    {\itshape\color{magenta}}
    {.}
    { }
    {\thmname{#1}\thmnote{ #3}}
\theoremstyle{solution}
\newtheorem*{solution}{Solution}

\Crefformat{exercise}{#2Exercise 6.5.#1#3}

\newcommand{\interior}[1]{%
  {\kern0pt#1}^{\mathrm{o}}%
}
\newcommand{\ts}{\textsuperscript}
\newcommand{\setcomp}[1]{#1^{\mathsf{c}}}
\newcommand{\quand}{\quad \text{and} \quad}
\newcommand{\quimplies}{\quad \implies \quad}
\newcommand{\quiff}{\quad \iff \quad}
\newcommand{\N}{\mathbf{N}}
\newcommand{\Z}{\mathbf{Z}}
\newcommand{\Q}{\mathbf{Q}}
\newcommand{\I}{\mathbf{I}}
\newcommand{\R}{\mathbf{R}}
\newcommand{\C}{\mathbf{C}}

\DeclarePairedDelimiter\abs{\lvert}{\rvert}
% Swap the definition of \abs* and \norm*, so that \abs
% and \norm resizes the size of the brackets, and the 
% starred version does not.
\makeatletter
\let\oldabs\abs
\def\abs{\@ifstar{\oldabs}{\oldabs*}}
%
\let\oldnorm\norm
\def\norm{\@ifstar{\oldnorm}{\oldnorm*}}
\makeatother

\DeclarePairedDelimiter\paren{(}{)}
\makeatletter
\let\oldparen\paren
\def\paren{\@ifstar{\oldparen}{\oldparen*}}
\makeatother

\DeclarePairedDelimiter\bkt{[}{]}
\makeatletter
\let\oldbkt\bkt
\def\bkt{\@ifstar{\oldbkt}{\oldbkt*}}
\makeatother

\DeclarePairedDelimiter\set{\{}{\}}
\makeatletter
\let\oldset\set
\def\set{\@ifstar{\oldset}{\oldset*}}
\makeatother

\setlist[enumerate,1]{label={(\alph*)}}

\begin{document}

\section{Section 6.5 Exercises}

Exercises with solutions from Section 6.5 of \hyperlink{ua}{[UA]}.

\begin{exercise}
\label{ex:1}
    Consider the function \( g \) defined by the power series
    \[
        g(x) = x - \frac{x^2}{2} + \frac{x^3}{3} - \frac{x^4}{4} + \frac{x^5}{5} - \cdots \, .
    \]
    \begin{enumerate}
        \item Is \( g \) defined on \( (-1, 1) \)? Is it continuous on this set? Is \( g \) defined on \( (-1, 1] \)? Is it continuous on this set? What happens on \( [-1, 1] \)? Can the power series for \( g(x) \) possibly converge for any other points \( \abs{x} > 1 \)? Explain.

        \item For what values of \( x \) is \( g'(x) \) defined? Find a formula for \( g' \).
    \end{enumerate}
\end{exercise}

\begin{solution}
    \begin{enumerate}
        \item Observe that
        \[
            g(1) = 1 - \frac{1}{2} + \frac{1}{3} - \frac{1}{4} + \cdots
        \]
        is convergent by the Alternating Series Test (Theorem 2.7.7). It follows from Theorem 6.5.1 that \( g \) converges absolutely on \( (-1, 1) \) and hence \( g \) is defined on \( (-1, 1] \). Theorem 6.5.7 then implies that \( g \) is continuous on \( (-1, 1] \). Note that
        \[
            g(-1) = -1 - \frac{1}{2} - \frac{1}{3} - \frac{1}{4} - \cdots
        \]
        is the (negated) divergent harmonic series. We claim that \( g(x) \) cannot possibly converge for any other points \( \abs{x} > 1 \). To see this, suppose that \( g(x) \) does converge for some \( x \in \R \) such that \( \abs{x} > 1 \) and let \( r \in \R \) be such that \( \abs{x} > r > 1 \). It follows from Theorem 6.5.1 that \( g(r) \) converges absolutely; but
        \[
            r + \frac{r^2}{2} + \frac{r^3}{3} + \frac{r^4}{4} + \cdots
        \]
        diverges by comparison with the harmonic series.

        \item Theorem 6.5.7 guarantees that \( g \) is differentable on \( (-1, 1) \) and the derivative is given by
        \[
            g'(x) = 1 - x + x^2 - x^3 + x^4 - \cdots \, .
        \]
        Note that this series does not converge at \( x = 1 \), despite \( g(1) \) converging.
    \end{enumerate}
\end{solution}

\begin{exercise}
\label{ex:2}
    Find suitable coefficients \( (a_n) \) so that the resulting power series \( \sum a_n x^n \) has the given properties, or explain why such a request is impossible.
    \begin{enumerate}
        \item Converges for every value of \( x \in \R \).

        \item Diverges for every value of \( x \in \R \).

        \item Converges absolutely for all \( x \in [-1, 1] \) and diverges off of this set.

        \item Converges conditionally at \( x = -1 \) and converges absolutely at \( x = 1 \).

        \item Converges conditionally at both \( x = -1 \) and \( x = 1 \).
    \end{enumerate}
\end{exercise}

\begin{solution}
    \begin{enumerate}
        \item Take \( a_n = 0 \) for every \( n \geq 0 \).

        \item This is impossible; any power series converges to zero at \( x = 0 \).

        \item Take \( a_0 = 0 \) and \( a_n = \tfrac{1}{n^2} \) for each \( n \in \N \). Then for \( \abs{x} \leq 1 \) we have
        \[
            \abs{\frac{x^n}{n^2}} \leq \frac{1}{n^2}
        \]
        and thus \( \sum_{n=0}^{\infty} a_n x^n \) converges absolutely. If \( \abs{x} > 1 \) then
        \[
            \frac{x^n}{n^2} \not\to 0
        \]
        and thus \( \sum_{n=0}^{\infty} a_n x^n \) diverges.

        \item This is impossible. Note that
        \[
            \sum_{n=0}^{\infty} \abs{a_n (-1)^n} = \sum_{n=0}^{\infty} \abs{a_n} = \sum_{n=0}^{\infty} \abs{a_n 1^n},
        \]
        so that a power series converges absolutely at \( x = 1 \) if and only if it converges absolutely at \( x = -1 \).

        \item Take
        \[
            a_n = \begin{cases}
                0 & \text{if } n = 0 \text{ or } n \text{ is odd}, \\
                \frac{2(-1)^{1 + n/2}}{n} & \text{if } n \text{ is even},
            \end{cases}
        \]
        so that
        \[
            \sum_{n=0}^{\infty} a_n x^n = x^2 - \frac{x^4}{2} + \frac{x^6}{3} - \frac{x^8}{4} + \cdots \, .
        \]
        Then
        \[
            \sum_{n=0}^{\infty} a_n = \sum_{n=0}^{\infty} a_n (-1)^n = 1 - \frac{1}{2} + \frac{1}{3} - \frac{1}{4} + \cdots
        \]
        are both conditionally convergent series.
    \end{enumerate}
\end{solution}

\begin{exercise}
\label{ex:3}
    Use the Weierstrass M-Test to prove Theorem 6.5.2.
\end{exercise}

\begin{solution}
    Note that for any \( x \in \R \) such that \( \abs{x} \leq \abs{x_0} \), we have
    \[
        \abs{a_n x^n} = \abs{a_n} \abs{x}^n \leq \abs{a_n} \abs{x_0}^n.
    \]
    The series \( \sum_{n=0}^{\infty} \abs{a_n} \abs{x_0}^n \) is convergent by assumption so the Weierstrass M-Test implies that \( \sum_{n=0}^{\infty} a_n x^n \) converges uniformly on \( [-c, c] \), where \( c = \abs{x_0} \).
\end{solution}

\begin{exercise}[Term-by-term Antidifferentiation]
\label{ex:4}
    Assume \( f(x) = \sum_{n=0}^{\infty} a_n x^n \) converges on \( (-R, R) \).
    \begin{enumerate}
        \item Show
        \[
            F(x) = \sum_{n=0}^{\infty} \frac{a_n}{n + 1} x^{n+1}
        \]
        is defined on \( (-R, R) \) and satisfies \( F'(x) = f(x) \).

        \item Antiderivatives are not unique. If \( g \) is a arbitrary function satisfying \( g'(x) = f(x) \) on \( (-R, R) \), find a power series representation for \( g \).
    \end{enumerate}
\end{exercise}

\begin{solution}
    \begin{enumerate}
        \item Let \( x \in (-R, R) \) be given. Theorem 6.5.1 implies that the series \( \sum_{n=0}^{\infty} \abs{a_n} \abs{x}^n \) is convergent, which implies that the series \( \sum_{n=0}^{\infty} \abs{a_n} \abs{x}^{n+1} \) is convergent. Observe that
        \[
            \abs{\frac{a_n}{n + 1} x^{n+1}} = \frac{\abs{a_n}}{n+1} \abs{x}^{n+1} \leq \abs{a_n} \abs{x}^{n+1}
        \]
        for each \( n \geq 0 \) and thus \( F(x) \) is absolutely convergent by the Comparison Test (Theorem 2.7.4). It follows that \( F \) is defined on the open interval \( (-R, R) \) and it is then immediate from Theorem 6.5.7 that \( F'(x) = f(x) \) on this interval.

        \item Corollary 5.3.4 implies that \( g(x) = k + F(x) \) on \( (-R, R) \) for some constant \( k \in \R \). Thus
        \[
            g(x) = k + \sum_{n=0}^{\infty} \frac{a_n}{n + 1} x^{n+1} = k + a_0 x + \frac{a_1}{2} x^2 + \frac{a_2}{3} x^3 + \cdots \, .
        \]
    \end{enumerate}
\end{solution}

\begin{exercise}
\label{ex:5}
    \begin{enumerate}
        \item If \( s \) satisfies \( 0 < s < 1 \), show \( ns^{n-1} \) is bounded for all \( n \geq 1 \).

        \item Given an arbitrary \( x \in (-R, R) \), pick \( t \) to satisfy \( \abs{x} < t < R \). Use this start to construct a proof for Theorem 6.5.6.
    \end{enumerate}
\end{exercise}

\begin{solution}
    \begin{enumerate}
        \item Clearly \( 0 < ns^{n-1} \) for each \( n \geq 1 \). Let \( N \in \N \) be such that \( s \leq 1 - \tfrac{1}{N+1} \). For \( n \geq N \) we then have
        \[
            s \leq 1 - \frac{1}{n+1} \quiff (n + 1)s \leq n \quiff (n + 1) s^n \leq n s^{n-1}.
        \]
        Thus the sequence \( \paren{ns^{n-1}}_{n=1}^{\infty} \) is bounded below and eventually decreasing. It follows from the Monotone Convergence Theorem that this sequence is convergent and hence bounded.

        \item From part (a), there is an \( M > 0 \) such that
        \[
            n \abs{\frac{x}{t}}^{n-1} \leq M
        \]
        for all \( n \in \N \). Since \( t \in (-R, R) \), Theorem 6.5.1 implies that the series \( \sum_{n=0}^{\infty} a_n t^n \) is absolutely convergent. It follows that the series
        \[
            \sum_{n=1}^{\infty} M \abs{a_n} t^{n-1}
        \]
        is convergent. Now observe that
        \[
            \abs{n a_n x^{n-1}} = n \abs{a_n} \abs{x}^{n-1} = n \abs{\frac{x}{t}}^{n-1} \abs{a_n} t^{n-1} \leq M \abs{a_n} t^{n-1}
        \]
        for each \( n \in \N \). By comparison with the convergent series \( \sum_{n=1}^{\infty} M \abs{a_n} t^{n-1}\) we see that the series \( \sum_{n=1}^{\infty} n a_n x^{n-1} \) is absolutely convergent. It follows that the power series \( \sum_{n=1}^{\infty} n a_n x^{n-1} \) converges on the open interval \( (-R, R) \).
    \end{enumerate}
\end{solution}

\begin{exercise}
\label{ex:6}
    Previous work on geometric series (Example 2.7.5) justifies the formula
    \[
        \frac{1}{1 - x} = 1 + x + x^2 + x^3 + x^4 + \cdots \, , \quad \text{for all } \abs{x} < 1.
    \]
    Use the results about power series proved in this section to find values for \( \sum_{n=1}^{\infty} n / 2^n \) and \( \sum_{n=1}^{\infty} n^2 / 2^n \). The discussion in Section 6.1 may be helpful.
\end{exercise}

\begin{solution}
    The power series
    \[
        \frac{1}{1 - x} = \sum_{n=0}^{\infty} x^n
    \]
    has radius of convergence \( R = 1 \). Theorem 6.5.6 then implies that the formula
    \[
        \frac{1}{(1 - x)^2} = \sum_{n=1}^{\infty} n x^{n-1}
    \]
    is valid on \( (-1, 1) \), from which we obtain
    \[
        \frac{x}{(1 - x)^2} = \sum_{n=1}^{\infty} n x^n \tag{1}
    \]
    for all \( x \in (-1, 1) \). Substituting \( x = \tfrac{1}{2} \) gives us
    \[
        2 = \sum_{n=1}^{\infty} \frac{n}{2^n}.
    \]
    Differentiating the power series (1) term-by-term gives us
    \[
        \frac{1 + x}{(1 - x)^3} = \sum_{n=1}^{\infty} n^2 x^{n-1},
    \]
    valid on \( (-1, 1) \), from which we obtain
    \[
        \frac{x(1 + x)}{(1 - x)^3} = \sum_{n=1}^{\infty} n^2 x^n,
    \]
    valid on \( (-1, 1) \). Substituting \( x = \tfrac{1}{2} \) gives us
    \[
        6 = \sum_{n=1}^{\infty} \frac{n^2}{2^n}.
    \]
\end{solution}

\begin{exercise}
\label{ex:7}
    Let \( \sum a_n x^n \) be a power series with \( a_n \neq 0 \), and assume
    \[
        L = \lim_{n \to \infty} \abs{\frac{a_{n+1}}{a_n}}
    \]
    exists.
    \begin{enumerate}
        \item Show that if \( L \neq 0 \), then the series converges for all \( x \in (-1/L, 1/L) \). (The advice in \href{https://lew98.github.io/Mathematics/UA_Section_2_7_Exercises.pdf}{Exercise 2.7.9} may be helpful.)
        
        \item Show that if \( L = 0 \), then the series converges for all \( x \in \R \).

        \item Show that (a) and (b) continue to hold if \( L \) is replaced by the limit
        \[
            L' = \lim_{n \to \infty} s_n \quad \text{where} \quad s_n = \sup \set{\abs{\frac{a_{k+1}}{a_k}} : k \geq n}.
        \]
        (General properties of the \textit{limit superior} are discusseed in \href{https://lew98.github.io/Mathematics/UA_Section_2_4_Exercises.pdf}{Exercise 2.4.7}.)
    \end{enumerate}
\end{exercise}

\begin{solution}
    \begin{enumerate}
        \item Clearly the power series converges if \( x = 0 \), so suppose that \( 0 < \abs{x} < \tfrac{1}{L} \). It follows that
        \[
            \lim_{n \to \infty} \abs{\frac{a_{n+1} x^{n+1}}{a_n x^n}} = L \abs{x} < 1
        \]
        and hence the series \( \sum_{n=0}^{\infty} a_n x^n \) is absolutely convergent by the Ratio Test (\href{https://lew98.github.io/Mathematics/UA_Section_2_7_Exercises.pdf}{Exercise 2.7.9}).

        \item Clearly the power series converges if \( x = 0 \), so suppose that \( x \neq 0 \). It follows that
        \[
            \lim_{n \to \infty} \abs{\frac{a_{n+1} x^{n+1}}{a_n x^n}} = 0
        \]
        and hence the series \( \sum_{n=0}^{\infty} a_n x^n \) is absolutely convergent by the Ratio Test (\href{https://lew98.github.io/Mathematics/UA_Section_2_7_Exercises.pdf}{Exercise 2.7.9}).

        \item Let us refine the result in \href{https://lew98.github.io/Mathematics/UA_Section_2_7_Exercises.pdf}{Exercise 2.7.9} as follows. Given a series \( \sum_{n=1}^{\infty} a_n \) with \( a_n \neq 0 \), if
        \[
            \lim_{n \to \infty} s_n = r < 1 \quad \text{where} \quad s_n = \sup \set{\abs{\frac{a_{k+1}}{a_k}} : k \geq n},
        \]
        then the series \( \sum_{n=1}^{\infty} a_n \) converges absolutely. To see this, let \( r' \) be such that \( r < r' < 1 \). Since \( \lim_{n \to \infty} s_n = r \), there is an \( N \in \N \) such that
        \[
            \abs{s_N - r} = s_N - r < r' - r \quimplies s_N < r';
        \]
        for the first equality we have used that the sequence \( (s_n) \) decreases to \( r \) (see \href{https://lew98.github.io/Mathematics/UA_Section_2_4_Exercises.pdf}{Exercise 2.4.7}). It follows from this inequality that
        \[
            n \geq N \quimplies \abs{\frac{a_{n+1}}{a_n}} \leq s_N < r' \quimplies \abs{a_{n+1}} < \abs{a_n} r'.
        \]
        We may now argue as in \href{https://lew98.github.io/Mathematics/UA_Section_2_7_Exercises.pdf}{Exercise 2.7.9} to conclude the proof of this refined ratio test. Using this refined test, the desired results about power series follow as in parts (a) and (b).
    \end{enumerate}
\end{solution}

\begin{exercise}
\label{ex:8}
    \begin{enumerate}
        \item Show that power series representations are unique. If we have
        \[
            \sum_{n=0}^{\infty} a_n x^n = \sum_{n=0}^{\infty} b_n x^n
        \]
        for all \( x \) in an interval \( (-R, R) \), prove that \( a_n = b_n \) for all \( n = 0, 1, 2, \ldots \, \).

        \item Let \( f(x) = \sum_{n=0}^{\infty} a_n x^n \) converge on \( (-R, R) \), and assume \( f'(x) = f(x) \) for all \( x \in (-R, R) \) and \( f(0) = 1 \). Deduce the values of \( a_n \).
    \end{enumerate}
\end{exercise}

\begin{solution}
    \begin{enumerate}
        \item Let us show that if a power series \( h(x) = \sum_{n=0}^{\infty} a_n x^n \) satisfies \( h(x) = 0 \) for all \( x \in (-R, R) \), then \( a_n = 0 \) for all \( n \geq 0 \). Theorem 6.5.7 implies that
        \[
            h^{(k)}(x) = \sum_{n=k}^{\infty} n (n - 1) \cdots (n - k + 1) a_n x^{n - k}
        \]
        for all \( x \in (-R, R) \) and all \( k \geq 0 \), where \( h^{(k)} \) is the \( k \)\ts{th} derivative of \( h \). Since \( h \) is identically zero on \( (-R, R) \) it must be the case that \( h^{(k)} \) is identically zero on \( (-R, R) \) and thus
        \[
            0 = h^{(k)}(0) = k! a_k \quiff a_k = 0
        \]
        for each \( k \geq 0 \).

        Now suppose that
        \[
            \sum_{n=0}^{\infty} a_n x^n = \sum_{n=0}^{\infty} b_n x^n
        \]
        for all \( x \) in an interval \( (-R, R) \). Then
        \[
            \sum_{n=0}^{\infty} (a_n - b_n) x^n = 0
        \]
        for all \( x \in (-R, R) \) and our previous discussion shows that we must have \( a_n - b_n = 0 \) for each \( n \geq 0 \).

        \item Theorem 6.5.7 gives us
        \[
            f(x) = \sum_{n=0}^{\infty} a_n x^n = \sum_{n=0}^{\infty} (n + 1) a_{n+1} x^n = f'(x)
        \]
        for all \( x \in (-R, R) \). It follows from part (a) that
        \[
            a_{n+1} = \frac{a_n}{n + 1}
        \]
        for all \( n \geq 0 \). From \( f(0) = 1 \) we obtain \( a_0 = 1 \) and hence \( a_1 = 1, a_2 = \tfrac{1}{2}, a_3 = \tfrac{1}{6} \), and, in general,
        \[
            a_n = \frac{1}{n!}.
        \]
    \end{enumerate}
\end{solution}

\begin{exercise}
\label{ex:9}
    Review the definitions and results from Section 2.8 concerning products of series and Cauchy products in particular. At the end of Section 2.9, we mentioned the following result: If both \( \sum a_n \) and \( \sum b_n \) converge conditionally to \( A \) and \( B \) respectively, then it is possible for the Cauchy product,
    \[
        \sum d_n \quad \text{where} \quad d_n = a_0 b_n + a_1 b_{n-1} + \cdots + a_n b_0,
    \]
    to diverge. However, if \( \sum d_n \) does converge, then it must converge to \( AB \). To prove this, set
    \[
        f(x) = \sum a_n x^n, \quad g(x) = \sum b_n x^n, \quand h(x) = \sum d_n x^n.
    \]
    Use Abel's Theorem and the result in \href{https://lew98.github.io/Mathematics/UA_Section_2_8_Exercises.pdf}{Exercise 2.8.7} to establish this result.
\end{exercise}

\begin{solution}
    Our hypothesis is that \( f, g, \) and \( h \) all converge at \( x = 1 \). It follows from Theorem 6.5.1 that \( f \) and \( g \) converge absolutely for any \( x \in (-1, 1) \) and hence by \href{https://lew98.github.io/Mathematics/UA_Section_2_8_Exercises.pdf}{Exercise 2.8.7} we have
    \[
        h(x) = \sum_{n=0}^{\infty} d_n x^n = \paren{ \sum_{n=0}^{\infty} a_n x^n } \paren{ \sum_{n=0}^{\infty} b_n x^n } = f(x) g(x) \quad \text{for all } x \in (-1, 1). \tag{1}
    \]
    Abel's Theorem (Theorem 6.5.4) implies that \( f, g, \) and \( h \) converge uniformly and hence are continuous on \( [0, 1] \). The continuity at \( x = 1 \) allows us to extend the equality in (1) to all \( x \in (-1, 1] \), which gives us
    \[
        h(1) = \sum_{n=0}^{\infty} d_n = \paren{ \sum_{n=0}^{\infty} a_n } \paren{ \sum_{n=0}^{\infty} b_n } = f(1) g(1) = AB.
    \]
\end{solution}

\begin{exercise}
\label{ex:10}
    Let \( g(x) = \sum_{n=0}^{\infty} b_n x^n \) converge on \( (-R, R) \), and assume \( (x_n) \to 0 \) with \( x_n \neq 0 \). If \( g(x_n) = 0 \) for all \( n \in \N \), show that \( g(x) \) must be identically zero on all of \( (-R, R) \).
\end{exercise}

\begin{solution}
    Theorem 6.5.7 implies that \( g \) is continuous at zero. It follows that
    \[
        b_0 = g(0) = g \paren{ \lim_{k \to \infty} x_k } = \lim_{k \to \infty} g(x_k) = 0.
    \]
    Theorem 6.5.7 also allows us to differentiate \( g \) term-by-term, obtaining the power series
    \[
        g'(x) = \sum_{n=1}^{\infty} n b_n x^{n-1},
    \]
    valid on \( (-R, R) \). It follows that
    \[
        b_1 = g'(0) = \lim_{k \to \infty} \frac{g(x_k) - g(0)}{x_k} = 0.
    \]
    We can continue in this manner to see that \( b_0 = b_1 = \cdots = 0 \), which by \Cref{ex:8} implies that \( g \) is identically zero on \( (-R, R) \).
\end{solution}

\begin{exercise}
\label{ex:11}
    A series \( \sum_{n=0}^{\infty} a_n \) is said to be \textit{Abel-summable to} \( L \) if the power series
    \[
        f(x) = \sum_{n=0}^{\infty} a_n x^n
    \]
    converges for all \( x \in [0, 1) \) and \( L = \lim_{x \to 1^-} f(x) \).
    \begin{enumerate}
        \item Show that any series that converges to a limit \( L \) is also Abel-summable to \( L \).

        \item Show that \( \sum_{n=0}^{\infty} (-1)^n \) is Abel-summable and find the sum.
    \end{enumerate} 
\end{exercise}

\begin{solution}
    \begin{enumerate}
        \item Suppose \( \sum_{n=0}^{\infty} a_n \) converges to \( L \). In other words, the power series \( f(x) = \sum_{n=0}^{\infty} a_n x^n \) converges to \( L \) at \( x = 1 \); Abel's Theorem then implies that the power series is uniformly convergent and hence continuous on \( [0, 1] \). It follows that
        \[
            \lim_{x \to 1^-} f(x) = f \paren{ \lim_{x \to 1^-} x } = f(1) = \sum_{n=0}^{\infty} a_n = L.
        \]

        \item Let
        \[
            f(x) = \sum_{n=0}^{\infty} (-1)^n x^n = \sum_{n=0}^{\infty} (-x)^n = \frac{1}{1 + x};
        \]
        this is valid for \( \abs{x} < 1 \) (see \Cref{ex:6}). It follows that
        \[
            \lim_{x \to 1^-} f(x) = \lim_{x \to 1^-} \frac{1}{1 + x} = \frac{1}{2}
        \]
        and hence \( \sum_{n=0}^{\infty} (-1)^n \) is Abel-summable to \( \tfrac{1}{2} \).
    \end{enumerate}
\end{solution}

\noindent \hrulefill

\noindent \hypertarget{ua}{\textcolor{blue}{[UA]} Abbott, S. (2015) \textit{Understanding Analysis.} 2\ts{nd} edition.}

\end{document}