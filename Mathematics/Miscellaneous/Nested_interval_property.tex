\documentclass[12pt]{article}
\title{Nested interval property}
\author{}
\date{\vspace{-24mm}}
\usepackage[utf8]{inputenc}
\usepackage{amsmath}
\usepackage{amsthm}
\usepackage{geometry}
\usepackage{amsfonts}
\usepackage{mathrsfs}
\usepackage{bm}
\usepackage{hyperref}
\usepackage{xcolor}
\usepackage{enumitem}
\usepackage{changepage}
\usepackage{tikz-cd}
\usepackage[nameinlink]{cleveref}
\geometry{
headheight=15pt,
left=60pt,
right=60pt
}
\usepackage{fancyhdr}
\pagestyle{fancy}
\fancyhf{}
\lhead{}
\chead{Nested interval property}
\rhead{\thepage}

% \setlength{\parindent}{0pt}
\hypersetup{
    colorlinks=true,
    linkcolor=blue,
    urlcolor=blue
}

\newcommand{\newp}{\vspace{5mm}}

\theoremstyle{definition}
\newtheorem{definition}{Definition}
\newtheorem{proposition}[definition]{Proposition}

\newtheorem*{remark}{Remark}

\begin{document}

\begin{remark}
    \( \mathbb{N} = \{ 1, 2, 3, \ldots \} \).
\end{remark}

\noindent Let \( F \) be an \href{https://en.wikipedia.org/wiki/Ordered_field}{ordered field} (for example, \( \mathbb{Q} \) and \( \mathbb{R} \) are ordered fields). Suppose we have a sequence \( (I_n)_{n \in \mathbb{N}} \) of closed bounded intervals
\[
    I_n = [a_n, b_n] = \{ x \in F : a_n \leq x \leq b_n \}.
\]
Consider the following two properties.
\begin{enumerate}[label = (\roman*)]
    \item For all \( n \in \mathbb{N}, I_{n+1} \subseteq I_n \).

    \item For all \( \varepsilon > 0 \) there exists an \( N \in \mathbb{N} \) such that \( |I_n| := b_n - a_n < \varepsilon \).
\end{enumerate}

If \( (I_n)_{n \in \mathbb{N}} \) satisfies property (i), then \( (I_n)_{n \in \mathbb{N}} \) is known as a sequence of \textbf{nested intervals}. If \( (I_n)_{n \in \mathbb{N}} \) satisfies both properties (i) and (ii), then \( (I_n)_{n \in \mathbb{N}} \) is known as a sequence of \textbf{shrinking nested intervals}.

If \( F \) is such that any sequence \( (I_n)_{n \in \mathbb{N}} \) of shrinking nested intervals has a singleton intersection, i.e.\ \( \bigcap_{n=1}^{\infty} I_n = \{ x \} \) for some \( x \in F \), then \( F \) is said to have the \textbf{nested interval property}.

Let us first show that if \( F \) has the least-upper-bound property, then \( F \) has the nested interval property. Since \( \mathbb{R} \) is the unique ordered field with the least-upper-bound property, the proposition we want to prove is the following.

\begin{proposition}
\label{prop:R_has_nip}
    \( \mathbb{R} \) has the nested interval property.
\end{proposition}

\begin{proof}
    Let \( (I_n)_{n \in \mathbb{N}} \) be a sequence of shrinking nested intervals, where \( I_n = [a_n, b_n] \), and let \( A \) be the set of left endpoints, i.e.\ \( A = \{ a_n : n \in \mathbb{N} \} \). Note that for any \( m \in \mathbb{N} \), \( b_m \) is an upper bound of \( A \):
    \begin{itemize}
        \item if \( n \leq m\), then \( a_n \leq a_m \leq b_m \);
        \item if \( m < n \), then \( a_n \leq b_n \leq b_m \).
    \end{itemize}
    (The sequence of left endpoints is non-decreasing and the sequence of right endpoints is non-increasing since the intervals are nested.) Hence \( x := \sup A \) exists in \( \mathbb{R} \) and satisfies \( a_n \leq x \leq b_n \) for each \( n \in \mathbb{N} \), i.e.\ \( x \in \bigcap_{n=1}^{\infty} I_n \). To see that this \( x \) is unique, suppose that \( x_1, x_2 \in \bigcap_{n=1}^{\infty} I_n \) and without loss of generality suppose that \( x_1 \leq x_2 \). Let \( \varepsilon > 0 \) be given. Then there exists an \( N \in \mathbb{N} \) such that \( b_N - a_N < \varepsilon \). Since \( x_1, x_2 \in I_n \) for each \( n \in \mathbb{N} \), we have
    \[
        a_N \leq x_1, x_2 \leq b_N \implies x_2 - x_1 \leq b_N - a_N < \varepsilon.
    \]
    Hence \( 0 \leq x_2 - x_1 < \varepsilon \) for any \( \varepsilon > 0 \); it follows that \( x_1 = x_2 \).
\end{proof}

Note that the proof just given shows that if \( (I_n)_{n \in \mathbb{N}} \) is only a sequence of nested intervals in \( \mathbb{R} \), not necessarily shrinking, then \( \bigcap_{n=1}^{\infty} I_n \) is non-empty.

Next, let us show that if \( F \) has the nested interval property, then \( F \) has the least-upper-bound property, i.e.\ \( F = \mathbb{R} \).

\begin{proposition}
\label{prop:nip_implies_lub}
    If \( F \) has the nested interval property, then \( F \) has the least-upper-bound property.
\end{proposition}

\begin{proof}
    Let \( E \subseteq F \) be non-empty and bounded above by some \( b_1 \in F \). If \( E \) has a maximum \( x \), then \( \sup E = x \) and we are done. Otherwise, we shall use an induction argument to construct a sequence \( (I_n)_{n \in \mathbb{N}} \) of shrinking nested intervals. Pick some \( a_1 \in E \); it must be the case that \( a_1 \) is not an upper bound of \( E \) since \( E \) has no maximum. Let \( I_1 = [a_1, b_1] \). Then:
    \begin{itemize}
        \item \( a_1 \) is not an upper bound of \( E \);
        \item \( b_1 \) is an upper bound of \( E \);
        \item \( |I_1| = 2^0 (b_1 - a_1) \).
    \end{itemize}
    Suppose that after \( N \) steps we have chosen intervals \( I_n = [a_n, b_n] \), \( 1 \leq n \leq N \), such that
    \begin{itemize}
        \item \( a_1 \leq \cdots \leq a_N \) are not upper bounds of \( E \);
        \item \( b_N \leq \cdots \leq b_1 \) are upper bounds of \( E \);
        \item \( |I_n| = 2^{-(n-1)}(b_1 - a_1) \) for \( 1 \leq n \leq N \).
    \end{itemize}
    Let \( m = \tfrac{a_N + b_N}{2} \), the midpoint of the interval \( I_N \). If \( m \) is not an upper bound of \( E \), set
    \[
        a_{N+1} = m, b_{N+1} = b_N, \text{and } I_{N+1} = [a_{N+1}, b_{N+1}].
    \]
    If \( m \) is an upper bound of \( E \), set
    \[
        a_{N+1} = a_N, b_{N+1} = m, \text{and } I_{N+1} = [a_{N+1}, b_{N+1}].
    \]
    In either case, we have chosen intervals \( I_n = [a_n, b_n] \), \( 1 \leq n \leq N + 1 \), such that
    \begin{itemize}
        \item \( a_1 \leq \cdots \leq a_{N+1} \) are not upper bounds of \( E \);
        \item \( b_{N+1} \leq \cdots \leq b_1 \) are upper bounds of \( E \);
        \item \( |I_n| = 2^{-(n-1)}(b_1 - a_1) \) for \( 1 \leq n \leq N + 1 \).
    \end{itemize}
    In this way we obtain a sequence \( (I_n)_{n \in \mathbb{N}} \) of intervals \( I_n = [a_n, b_n] \) such that
    \begin{itemize}
        \item \( a_1 \leq \cdots \leq a_n \leq \cdots \) are not upper bounds of \( E \);
        \item \( \cdots \leq b_n \leq \cdots \leq b_1 \) are upper bounds of \( E \);
        \item \( |I_n| = 2^{-(n-1)}(b_1 - a_1) \) for all \( n \in \mathbb{N} \).
    \end{itemize}
    Hence \( (I_n)_{n \in \mathbb{N}} \) is a sequence of shrinking nested intervals. By assumption, \( F \) has the nested interval property, so there exists an \( x \in F \) such that \( \bigcap_{n=1}^{\infty} I_n = \{ x \} \). We claim that \( x = \sup E \). For \( y \in E \), suppose \( x < y \). Then there is an \( N \in \mathbb{N} \) such that
    \[
        b_N - a_N < y - x \implies x + (b_N - a_N) < y.
    \]
    Since \( x \in \bigcap_{n=1}^{\infty} I_n \), we have
    \[
        a_N \leq x \implies 0 \leq x - a_N \implies b_N \leq x + (b_N - a_N) < y.
    \]
    This is a contradiction since \( b_N \) is an upper bound of \( E \). It follows that \( y \leq x \), so that \( x \)  is an upper bound of \( E \). Suppose that \( z \in F \) is such that \( z < x \). There is an \( N \in \mathbb{N} \) such that
    \[
        b_N - a_N < x - z \implies z < x - (b_N - a_N).
    \]
    Since \( x \in \bigcap_{n=1}^{\infty} I_n \), we have
    \[
        x \leq b_N \implies x - b_N \leq 0 \implies x - (b_N - a_N) \leq a_N \implies z < a_N.
    \]
    It follows that \( z \) is not an upper bound of \( E \) since \( a_N \) is not an upper bound of \( E \). We may conclude that \( x \) is the least upper bound of \( E \), i.e.\ \( x = \sup E \).
\end{proof}

So for ordered fields, the nested interval property and the least-upper-bound property are equivalent. In light of the fact that \( \mathbb{R} \) is the unique ordered field with the least-upper-bound property, we see that \( \mathbb{R} \) is the unique ordered field with the nested interval property.

\end{document}
