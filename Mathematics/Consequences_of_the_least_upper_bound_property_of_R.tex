\documentclass[12pt]{article}
\title{Consequences of the least-upper-bound property of \texorpdfstring{\( \mathbb{R} \)}{}}
\author{}
\date{\vspace{-24mm}}
\usepackage[utf8]{inputenc}
\usepackage{amsmath}
\usepackage{amsthm}
\usepackage{geometry}
\usepackage{amsfonts}
\usepackage{mathrsfs}
\usepackage{bm}
\usepackage{hyperref}
\usepackage{xcolor}
\usepackage{enumitem}
\usepackage{changepage}
\usepackage{tikz}
\usetikzlibrary{matrix}
\usepackage{tikz-cd}
\usepackage[nameinlink]{cleveref}
\geometry{
headheight=15pt,
left=60pt,
right=60pt
}
\usepackage{fancyhdr}
\pagestyle{fancy}
\fancyhf{}
\lhead{}
\chead{Consequences of the least-upper-bound property of \texorpdfstring{\( \mathbb{R} \)}{}}
\rhead{\thepage}

\hypersetup{
    colorlinks=true,
    linkcolor=blue,
    urlcolor=blue
}

\theoremstyle{definition}
\newtheorem{theorem}{Theorem}
\newtheorem{lemma}[theorem]{Lemma}
\newtheorem{corollary}[theorem]{Corollary}

\newtheorem*{remark}{Remark}

\begin{document}

\maketitle

\tableofcontents

\newpage

\noindent The following is mostly paraphrased from Chapter 1 of \hyperlink{pma}{[PMA]}.

\begin{theorem}
\label{thm:R_exists}
    There exists an ordered field \( \mathbb{R} \) which has the least-upper-bound property. Moreover, \( \mathbb{R} \) contains \( \mathbb{Q} \) as a subfield.
\end{theorem}

For a proof of \Cref{thm:R_exists}, see \href{https://lew98.github.io/Mathematics/Construction_of_R_from_Q_via_Dedekind_cuts.pdf}{here}. In this document, we shall present some consequences of \Cref{thm:R_exists}.

\section{The Archimedean property of \texorpdfstring{\( \mathbb{R} \)}{}}
\label{sec:archimedean_property_of_R}

\begin{theorem}[Archimedean property of \( \mathbb{R} \)]
\label{thm:archimedean_property_of_R}
    Let \( x > 0 \) and \( y \) be real numbers. Then there exists a positive integer \( n \) such that \( nx > y \).
\end{theorem}

\begin{proof}
    Suppose to the contrary that for all positive integers \( n \) we have \( nx \leq y \). Then the set \( A = \{ nx : n \in \mathbb{N} \} \) is non-empty and bounded above, so by the least-upper-bound property of \( \mathbb{R} \) the supremum \( \alpha = \sup A \) exists in \( \mathbb{R} \). Since \( x > 0 \), we have \( \alpha - x < \alpha \) so that \( \alpha - x \) is not an upper bound of \( A \). Hence there exists a positive integer \( m \) such that \( \alpha - x < mx \), which gives \( \alpha < (m+1)x \); but this contradicts the fact that \( \alpha \) is the supremum of \( A \).
\end{proof}

\section{Density of \texorpdfstring{\( \mathbb{Q} \)}{} and \texorpdfstring{\( \mathbb{Q}^{\mathsf{C}} \)}{} in \texorpdfstring{\( \mathbb{R} \)}{}}

\begin{lemma}
\label{lem:real_number_lies_between_consecutive_integers}
    Any real number lies between two consecutive integers. That is, for any \( x \in \mathbb{R} \) there exists an \( m \in \mathbb{Z} \) such that \( m - 1 \leq x < m \).
\end{lemma}

\begin{proof}
    By the Archimedean property, there exist positive integers \( m_1, m_2 \) such that \( m_1 > x \) and \( m_2 > - x \), which gives \( -m_2 < x < m_1 \). This implies that the set \( A = \{ n \in \mathbb{Z} : x < n \} \) is non-empty (\( m_1 \in A \)) and bounded below (by \( -m_2 \)). Then by the \href{https://en.wikipedia.org/wiki/Well-ordering_principle}{well-ordering principle}, \( A \) has a least element; call it \( m \). Since this is the least element of \( A \), we must have \( m - 1 \not\in A \) and so \( m - 1 \leq x < m \).
\end{proof}

\begin{theorem}
\label{thm:density_of_Q_in_R}
    Between any two real numbers there exists a rational number. That is, for any \( x, y \in \mathbb{R} \) with \( x < y \) there exists a \( p \in \mathbb{Q} \) such that \( x < p < y \).
\end{theorem}

\begin{proof}
    By the Archimedean property, there exists a positive integer \( n \) such that \( n(y - x) > 1 \). By \Cref{lem:real_number_lies_between_consecutive_integers}, there exists an integer \( m \) such that \( m - 1 \leq nx < m \). Combining these inequalities gives \( nx < m \leq 1 + nx < ny \), which implies that \( x < \frac{m}{n} < y \). So the desired rational is \( p = \frac{m}{n} \).
\end{proof}

\begin{corollary}
\label{cor:density_of_Qc_in_R}
    Between any two real numbers there exists an irrational number. That is, for any \( x, y \in \mathbb{R} \) with \( x < y \) there exists a \( z \in \mathbb{Q}^{\mathsf{C}} \) such that \( x < z < y \).
\end{corollary}

\begin{proof}
    By \Cref{thm:density_of_Q_in_R}, there exists a rational number \( p \) such that \( x - \sqrt{2} < p < y - \sqrt{2} \), which gives \( x < p + \sqrt{2} < y \). So the desired irrational number is \( z = p + \sqrt{2} \).
\end{proof}

\section{Existence of \textit{n}th roots in \texorpdfstring{\(\mathbb{R}\)}{}}
\label{sec:existence_of_nth_roots_in_R}

First, a useful inequality. Suppose \( n \) is a positive integer and \( a, b \) are real numbers such that \( 0 < a < b \). This implies that \( 0 < b^{n-2}a < b^{n-1} \). Furthermore, we have \( 0 < a^2 < b^2 \), which gives \( 0 < b^{n-3}a^2 < b^{n-1} \), and so on. Combining this with the equality
\[
    b^n - a^n = (b - a)(b^{n-1} + b^{n-2}a + \cdots + a^{n-1})
\]
gives us the inequality
\[
    b^n - a^n < (b - a)nb^{n-1}. \tag{1}
\]

\begin{theorem}
\label{thm:existence_of_nth_roots_in_R}
    For every real \( x > 0 \) and every positive integer \( n \) there is exactly one positive real \( y \) such that \( y^n = x \).
\end{theorem}

\begin{proof}
    Suppose \( y_1 \) and \( y_2 \) are positive real numbers such that \( y_1 \neq y_2 \). Without loss of generality, assume \( 0 < y_1 < y_2 \). Then \( 0 < y_1^n < y_2^n \), so that \( y_1^n \neq y_2^n \). Hence by the contrapositive, \( y_1^n = y_2^n \) implies that \( y_1 = y_2 \). This gives us the uniqueness of any such \( y \) in Theorem 1.
    
    For existence, let \( E = \{ t \in \mathbb{R} : t > 0, t^n < x \} \). Observe that \( t = \frac{x}{1 + x} \) satisfies \( t < x \) and \( 0 < t < 1 \), which gives \( 0 < t^n < t < x \). Hence \( t \in E \) and so \( E \) is non-empty. Now suppose \( t \geq 1 + x > 1 \), so that \( t^n > t \geq 1 + x > x \). Then by the contrapositive, \( t^n < x \) implies that \( t < 1 + x \), and we see that \( E \) is bounded above by \( 1 + x \). We may now invoke the least-upper-bound property of \( \mathbb{R} \) and set \( y = \sup E \). Note that \( y \) must be positive, since \( \frac{x}{1 + x} \) belongs to \( E \). To show that \( y^n = x \), we will show that both of the assumptions \( y^n < x \) and \( y^n > x \) lead to contradictions.
    
    First, assume that \( y^n < x \). Using the Archimedean property, choose \( h \) such that \( 0 < h < 1 \) and \( h < \frac{x - y^n}{n (y+1)^{n-1}} \). Now take \( a = y \) and \( b = y + h \) in inequality (1) to obtain
    \[
    (y + h)^n - y^n < h n (y + h)^{n-1} < h n (y + 1)^{n-1} < x - y^n,
    \]
    whence \( (y + h)^n < x \) and so \( y + h \in E \); but this contradicts the fact that \( y \) is the supremum of \( E \), since \( y + h > y \).
    
    Next, assume that \( y^n > x \) and set \( k = \frac{y^n - x}{ny^{n-1}} < y \). Take \( a = y - k \) and \( b = y \) in inequality (1) to obtain
    \[
    y^n - (y - k)^n < kny^{n-1} = y^n - x,
    \]
    whence \( (y - k)^n \geq x \). Then \( t \geq y - k \) implies that \( t^n \geq x \); the contrapositive of this shows that \( y - k \) is an upper bound for \( E \). This contradicts the fact that \( y \) is the least upper bound of \( E \), since \( y - k < y \).
\end{proof}

\begin{corollary}
\label{cor:exponent_rule}
    Let \( a \) and \( b \) be positive real numbers and \( n \) a positive integer. Then
    \[
    \sqrt[n]{ab} = \sqrt[n]{a}\sqrt[n]{b}.
    \]
\end{corollary}

\begin{proof}
    Let \( \alpha = \sqrt[n]{a} \) and \( \beta = \sqrt[n]{b} \). Then by the commutativity of multiplication, we have
    \[
    (\alpha \beta)^n = \alpha^n \beta^n = ab.
    \]
    The uniqueness part of \Cref{thm:existence_of_nth_roots_in_R} then implies that \( \sqrt[n]{ab} = \alpha \beta = \sqrt[n]{a} \sqrt[n]{b} \).
\end{proof}

\noindent \hrulefill

\noindent \hypertarget{pma}{\textcolor{blue}{[PMA]} Rudin, W. (1976) \textit{Principles of Mathematical Analysis.} 3rd edn.}

\end{document}
