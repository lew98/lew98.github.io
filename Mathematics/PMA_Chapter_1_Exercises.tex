\documentclass[12pt]{article}
\usepackage[utf8]{inputenc}
\usepackage{amsmath}
\usepackage{amsthm}
\usepackage{geometry}
\usepackage{amsfonts}
\usepackage{bm}
\usepackage{hyperref}
\usepackage{xcolor}
\usepackage{enumitem}
\usepackage{mathtools}
\usepackage{graphicx}
\usepackage{changepage}
\geometry{
headheight=15pt,
left=60pt,
right=60pt
}
\usepackage{fancyhdr}
\pagestyle{fancy}
\fancyhf{}
\lhead{}
\chead{Chapter 1 Exercises}
\rhead{\thepage}

\pagenumbering{arabic}
\setlength{\parindent}{0pt}
\hypersetup{
    colorlinks=true,
    linkcolor=blue,
    urlcolor=blue
}

\newcommand{\newp}{\vspace{5mm}}
\DeclarePairedDelimiter\abs{\lvert}{\rvert}
% Swap the definition of \abs* and \norm*, so that \abs
% and \norm resizes the size of the brackets, and the 
% starred version does not.
\makeatletter
\let\oldabs\abs
\def\abs{\@ifstar{\oldabs}{\oldabs*}}
%
\let\oldnorm\norm
\def\norm{\@ifstar{\oldnorm}{\oldnorm*}}
\makeatother

\theoremstyle{definition}
\newtheorem{theorem}{Theorem}
\newtheorem{lemma}{Lemma}

\newtheorem*{remark}{Remark}

\begin{document}

\section{Chapter 1 Exercises}

Exercises with solutions from Chapter 1 of \hyperlink{pma}{[PMA]}.

\newp

\textbf{1.} If \( r \) is rational (\( r \neq 0 \)) and \( x \) is irrational, prove that \( r + x \) and \( rx \) are irrational.

\newp

\textit{\textcolor{magenta}{Solution.}} In both cases we shall prove the contrapositive. Suppose \( r + x = p \in \mathbb{Q} \). Then \( x = p - r \), a rational number. Similarly, \( rx = q \in \mathbb{Q} \) implies that \( x = \frac{q}{r} \), a rational number.

\newp

\textbf{2.} Prove that there is no rational number whose square is 12.

\newp

\textit{\textcolor{magenta}{Solution.}} Taking as given that each positive integer has a unique prime factorization, we shall prove the following stronger result:
\vspace{6px}
\begin{adjustwidth}{16px}{16px}
    Let \( n \) be a positive integer. If \( n \) is not the square of an integer then there is no rational number whose square is \( n \).
\end{adjustwidth}
\vspace{6px}
In fact, we shall prove the contrapositive. Suppose there is a rational \( \frac{a}{b} \) such that \( a^2 = n b^2 \). The primes in the factorizations of \( a^2 \) and \( b^2 \) must appear to even powers; unique prime factorization implies that the primes in the factorization of \( nb^2 \) must be the same as those in the factorization of \( a^2 \). It follows that the factorization of \( n \) contains only primes raised to even powers, else the factorization of \( n b^2 \) would contain a prime raised to an odd power. Hence \( n \) is the square of an integer.

\newp

\textbf{3.} Prove Proposition 1.15.

\newp

\textbf{1.15 Proposition} \textit{The axioms for multiplication imply the following statements.}

\begin{enumerate}[label = (\alph*)]
    \item \textit{If \( x \neq 0 \) and \( xy = xz \) then \( y = z \)}.
    \item \textit{If \( x \neq 0 \) and \( xy = x \) then \( y = 1 \)}.
    \item \textit{If \( x \neq 0 \) and \( xy = 1 \) then \( y = x^{-1} \)}.
    \item \textit{If \( x \neq 0 \) then \( \left(x^{-1}\right)^{-1} = x \)}.
\end{enumerate}

\textit{\textcolor{magenta}{Solution.}} For (a), we have
\[
    y = 1y = (x^{-1} x) y = x^{-1} (xy) = x^{-1} (xz) = (x^{-1} x) z = 1z = z.
\]
Take \( z = 1 \) in (a) to obtain (b), \( z = x^{-1} \) in (a) to obtain (c), and for (d), replace \( x \) with \( x^{-1} \) and \( y \) with \( x \) in (c).

\newp

\textbf{4.} Let \( E \) be a non-empty subset of an ordered set; suppose \( \alpha \) is a lower bound of \( E \) and \( \beta \) is an upper bound of \( E \). Prove that \( \alpha \leq \beta \).

\newp

\textit{\textcolor{magenta}{Solution.}} Since \( E \) is non-empty, there exists \( x \in E \). Then \( \alpha \leq x \leq \beta \).

\newp

\textbf{5.} Let \( A \) be a non-empty set of real numbers which is bounded below. Let \( -A \) be the set of all numbers \( -x \), where \( x \in A \). Prove that
\[
\inf A = -\sup (-A).
\]

\textit{\textcolor{magenta}{Solution.}} \( -A \) is non-empty since \( A \) is non-empty and \( -A \) is bounded above since \( A \) is bounded below (\( x \geq y \) implies \( -x \leq -y \) in an ordered field). Hence \( \sup (-A) \) exists in \( \mathbb{R} \). Let \( x \in A \) be given. Then \( -x \leq \sup (-A) \), which gives \( x \geq -\sup (-A) \), so that \( -\sup (-A) \) is a lower bound of \( A \). Now suppose \( y > -\sup (-A) \). Then \( -y < \sup (-A) \), so that \( -y \) is not an upper bound for \( -A \), i.e.\ there exists \( x \in A \) such that \( -y < -x \). This gives \( y > x \), demonstrating that \( y \) is not a lower bound of \( A \). It follows that \( - \sup(-A) \) is the infimum of \( A \).

\newp

\textbf{6.} Fix \( b > 1 \).

\textbf{(a)} If \( m, n, p, q \) are integers, \( n > 0, q > 0 \), and \( r = m/n = p/q \), prove that
\[
(b^m)^{1/n} = (b^p)^{1/q}.
\]
Hence it makes sense to define \( b^r = (b^m)^{1/n} \).
\newp

\textit{\textcolor{magenta}{Solution.}} Let \( x = (b^m)^{1/n} \) and \( y = (b^p)^{1/q} \). Then observe that
\[
b^{np} = b^{mq} \iff (b^p)^n = (b^m)^q \iff (y^q)^n = (x^n)^q \iff y^{nq} = x^{nq}.
\]
Since \( x, y, n, q \) are all positive, this implies that \( x = y \).

\newp

\textbf{(b)} Prove that \( b^{r + s} = b^r b^s \) if \( r \) and \( s \) are rational.

\newp

\textit{\textcolor{magenta}{Solution.}} The result is certainly true if \( r \) and \( s \) are integers; we shall use this freely. Suppose \( r = m/n \) and \( s = p/q \). Then
\[
b^{r + s} = b^{\frac{mq + np}{nq}} = (b^{mq + np})^{1/{nq}} = (b^{mq} b^{np})^{1/nq} = (b^{mq})^{1/nq} (b^{np})^{1/nq} = b^{mq/nq} b^{np/nq} = b^{m/n} b^{p/q} = b^r b^s,
\]
where we have used the corollary to Theorem 1.21 of \hyperlink{pma}{[PMA]}.

\newp

\textbf{(c)} If \( x \) is real, define \( B(x) \) to be the set of all numbers \( b^t \), where \( t \) is rational and \( t \leq x \). Prove that
\[
b^r = \sup B(r)
\]
when \( r \) is rational. Hence it makes sense to define
\[
b^x = \sup B(x)
\]
for every real \( x \).

\newp

\textit{\textcolor{magenta}{Solution.}} First, let us work from the field axioms and Theorem 1.21 of \hyperlink{pma}{[PMA]} to prove some useful lemmas.
\begin{lemma}
Suppose we have a real number \( b > 1 \) and integers \( m, n \). Then \( m \leq n \) if and only if \( b^m \leq b^n \).
\end{lemma}

\begin{proof}
Since \( b > 1 \), induction on \( k \) shows that \( b^k \geq 1 \) for any non-negative integer \( k \), with equality exactly when \( k = 0 \). Suppose \( m \leq n \); then \( 1 \leq b^{n-m} \). Multiply both sides of this inequality by the positive quantity \( b^m \) to obtain \( b^m \leq b^n \). Now suppose \( m > n \). Then \( b^{m-n} > 1 \), and \( b^m > b^n \) follows since \( b^n \) is positive.
\end{proof}

\begin{lemma}
Suppose we have positive real numbers \( x, y \) and a positive integer \( n \). Then \( x \leq y \) if and only if \( x^{1/n} \leq y^{1/n} \).
\end{lemma}

\begin{proof}
It follows from the uniqueness part of Theorem 1.21 of \hyperlink{pma}{[PMA]} that \( x = (x^n)^{1/n} = (x^{1/n})^n \). Given this, the result of the lemma is equivalent to \( x \leq y \iff x^n \leq y^n \). Both of the implications \( x \leq y \implies x^n \leq y^n \) and \( x > y \implies x^n > y^n \) follow quickly from the field axioms and induction on \( n \).
\end{proof}

Lemmas 1 and 2 give us the following result.

\begin{lemma}
Suppose we have a real number \( b > 1 \) and rationals \( r = m/n, t = p/q \), where \( n, q > 0 \). Then \( r\leq t \) if and only if \( b^r \leq b^t \).
\end{lemma}

\begin{proof}
\[
\begin{array}{rcll}
r \leq t & \iff & mq \leq np & \\
& \iff & b^{mq} \leq b^{np} & \text{(Lemma 1)} \\
& \iff & b^m \leq (b^{p/q})^n & \text{(Lemma 2)} \\
& \iff & b^{m/n} \leq b^{p/q} & \text{(Lemma 2).}
\end{array}
\]
\end{proof}

Now, returning to the exercise, let us show that
\[
    B(x) = \{ b^t : t \in \mathbb{Q}, t \leq x \}
\]
is non-empty and bounded above for any real \( x \). There are certainly rational numbers less than \( x \), so \( B(x) \) is non-empty, and there are certainly rational numbers greater than \( x \), so by Lemma 3 \( B(x) \) is bounded above by \( b \) to the power of any such rational. Hence \( \sup B(x) \) always exists in \( \mathbb{R} \).

\newp

\textbf{Remark.} Since any element of \( B(x) \) is positive, \( \sup B(x) \) is also positive, i.e.\ \( b^x \) is positive for any \( b > 1 \) and \( x \in \mathbb{R} \).

\newp

Finally, let us show that \( b^r = \sup B(r) \) for a rational \( r \). It follows from Lemma 3 that \( b^r \) is an upper bound for \( B(r) \), and since \( b^r \) belongs to \( B(r) \) it must be the supremum of \( B(r) \).

\newp

\textbf{(d)} Prove that \( b^{x + y} = b^x b^y \) for all real \( x \) and \( y \).

\newp

\textit{\textcolor{magenta}{Solution.}} To prove this, we will show that both of the assumptions \( b^{x+y} < b^x b^y \) and \( b^x b^y < b^{x+y} \) lead to contradictions. First, suppose that \( b^{x+y} < b^x b^y \), i.e.\ \( \sup B(x + y) < \sup B(x) \cdot \sup B(y) \). This assumption is equivalent to \( \frac{\sup B(x + y)}{\sup B(y)} < \sup B(x) \), so that \( \frac{\sup B(x + y)}{\sup B(y)} \) is not an upper bound for \( B(x) \). Then there must exist some rational \( r \) such that \( r \leq x \) and
\[
    \frac{\sup B(x + y)}{\sup B(y)} < b^r \iff \frac{\sup B(x + y)}{b^r} < \sup B(y).
\]
This demonstrates that \( \frac{\sup B(x + y)}{b^r} \) is not an upper bound for \( B(y) \), so there must exist a rational \( s \) such that \( s \leq y \) and
\[
    \frac{\sup B(x + y)}{b^r} < b^s \iff \sup B(x + y) < b^r b^s = b^{r + s}.
\]
This is a contradiction since
\[
    r + s \leq x + y \implies b^{r+s} \in B(x + y) \implies b^{r+s} \leq \sup B(x + y).
\]
Now suppose that \( b^x b^y < b^{x+y} \). We shall make use of the following inequality:
\[
    \forall n \in \mathbb{N} \quad\quad b^{1/n} \leq 1 + \frac{b-1}{n}.
\]
This is proved in Exercise 7 (a) and (b). By assumption \( b^{x+y} - b^x b^y > 0 \), so by invoking the Archimedean property of \( \mathbb{R} \) we may obtain a positive integer \( n \) such that
\[
    n(b^{x+y} - b^x b^y) > (b - 1) b^x b^y \implies \frac{b^{x+y}}{b^x b^y} > 1 + \frac{b-1}{n} \geq b^{1/n} \implies b^x b^y b^{1/n} < b^{x+y}.
\]
The density of \( \mathbb{Q} \) in \( \mathbb{R} \) implies that there exist rational numbers \( r \) and \( s \) such that \( x - \frac{1}{2n} < r \leq x \) and \( y - \frac{1}{2n} < s \leq y \), which implies that \( x + y < r + s + \frac{1}{n} \). It follows that
\[
    b^{x+y} \leq b^{r+s+1/n} = b^r b^s b^{1/n} \leq b^x b^y b^{1/n} < b^{x+y},
\]
i.e.\ \( b^{x+y} < b^{x+y} \), a contradiction.

\newp

\textbf{7.} Fix \( b > 1, y > 0 \), and prove that there exists a unique real \( x \) such that \( b^x = y \), by completing the following outline. (This \( x \) is called the \textit{logarithm of y to the base b}.)

\textbf{(a)} For any positive integer \( n \), \( b^n - 1 \geq n(b - 1) \).

\newp

\textit{\textcolor{magenta}{Solution.}} Observe that
\[
    \textstyle{b^n - 1 = (\sum_{j=0}^{n-1} b^j)}(b - 1).
\]
The desired result follows since there are \( n \) terms in the sum \( \sum_{j=0}^{n-1} b^j\) and \( b > 1 \implies b^j > 1 \).

\newp

\textbf{(b)} Hence \( b - 1 \geq n(b^{1/n} - 1) \).

\newp

\textit{\textcolor{magenta}{Solution.}} By Lemma 2, \( b > 1 \implies b^{1/n} > 1 \). So this result follows by replacing \( b \) with \( b^{1/n} \) in the inequality of part (a).

\newp

\textbf{(c)} If \( t > 1 \) and \( n > (b - 1)/(t - 1) \), then \( b^{1/n} < t \).

\newp

\textit{\textcolor{magenta}{Solution.}} By part (b), we have \( b^{1/n} - 1 \leq (b - 1)/n < t - 1 \); the result follows.

\newp

\textbf{(d)} If \( w \) is such that \( b^w < y \), then \( b^{w + (1/n)} < y \) for sufficiently large \( n \); to see this, apply part (c) with \( t = y b^{-w} \).

\newp

\textit{\textcolor{magenta}{Solution.}} Since \( y > b^w \), we have \( t = b^{-w}y > 1 \). Take \( n \) large enough so that \( n > (b - 1)/(t - 1) \) and apply part (c) to obtain \( b^{1/n} < b^{-w}y \), from which the result follows.

\newp

\textbf{(e)} If \( b^w > y \), then \( b^{w - (1/n)} > y \) for sufficiently large \( n \).

\newp

\textit{\textcolor{magenta}{Solution.}} Similarly, we take \( t = b^w y^{-1} > 1 \), \( n \) large enough so that \( n > (b - 1)/(t - 1) \), and apply part (c) to obtain \( b^{1/n} < b^w y^{-1} \); the result follows.

\newp

\textbf{(f)} Let \( A \) be the set of all \( w \) such that \( b^w < y \), and show that \( x = \sup A \) satisfies \( b^x = y \).

\newp

\textit{\textcolor{magenta}{Solution.}} First, a lemma which generalises Lemma 3 above.

\begin{lemma}
Suppose we have real numbers \( b, x, y \) such that \( b > 1 \). Then \( x \leq y \) if and only if \( b^x \leq b^y \).
\end{lemma}

\begin{proof}
If \( x \leq y \) then
\[
B(x) = \{ b^t : t \in \mathbb{Q}, t \leq x \} \subseteq B(y) = \{ b^t : t \in \mathbb{Q}, t \leq y \},
\]
whence \( b^x = \sup B(x) \leq \sup B(y) = b^y \); the implication \( x > y \implies b^x > b^y \) follows similarly.
\end{proof}

Now let us show that \( A = \{ w \in \mathbb{R} : b^w < y \} \) is non-empty and bounded above. Since \( b - 1 > 0 \), there is a positive integer \( n \) such that \( n(b - 1) > y^{-1} - 1 \). Part (a) then gives us \( b^n - 1 > y^{-1} - 1 \), so that \( b^{-n} < y \). Hence \( -n \in A \). Similarly, there is a positive integer \( N \) such that \( b^N > y \). Then for any \( w \in A \) we have \( b^w < y < b^N \) and an application of Lemma 4 gives us \( w < N \), so that \( N \) is an upper bound for \( A \). Hence \( x = \sup A \) exists in \( \mathbb{R} \).

\newp

To show that \( b^x = y \), we will show that both of the assumptions \( b^x < y \) and \( b^x > y \) lead to contradictions. If \( b^x < y \), then by part (d) there is a positive integer \( n \) such that \( b^{x + (1/n)} < y \); but then \( x + (1/n) \in A \), contradicting that \( x \) is the supremum of \( A \). If \( b^x > y \), then by part (e) there is a positive integer \( n \) such that \( b^{x - (1/n)} > y \); but then for any \( w \in A \) we have
\[
    b^w < y < b^{x - (1/n)} \implies w < x - (1/n),
\]
where we have used Lemma 4. This says that \( x - (1/n) \) is an upper bound for \( A \), contradicting that \( x \) is the supremum of \( A \).

\newp

\textbf{(g)} Prove that this \( x \) is unique.

\newp

\textit{\textcolor{magenta}{Solution.}} This follows from Lemma 4.

\newp

\textbf{8.} Prove that no order can be defined in the complex field that turns it into an ordered field. \textit{Hint:} -1 is a square.

\newp

\textit{\textcolor{magenta}{Solution.}} Suppose there was such an order \( < \). Then by Proposition 1.18 of \hyperlink{pma}{[PMA]} we must have \( i^2 = -1 > 0 \), which contradicts the very same proposition.

\newp

\textbf{9.} Suppose \( z = a + bi, w = c + di \). Define \( z \prec w \) if \( a < c \), and also if \( a = c \) but \( b < d \). Prove that this turns the set of all complex numbers into an ordered set. (This type of order relation is called a \textit{dictionary order}, or \textit{lexicographic order}, for obvious reasons.) Does this ordered set have the least-upper-bound property?

\newp

\textit{\textcolor{magenta}{Solution.}} Consider the following cases.

\begin{enumerate}[label = Case \arabic*., font=\bfseries, wide=0pt]
    \item \( a < c \). Then \( z \prec w \).
    \item \( a = c \).
    \begin{enumerate}[label* = \arabic*., font=\bfseries, wide=10pt]
        \item \( b < d \). Then \( z \prec w \).
        \item \( b = d \). Then \( z = w \).
        \item \( b > d \). Then \( z \succ w \).
    \end{enumerate}
    \item \( a > c \). Then \( z \succ w \).
\end{enumerate}

These cases are exclusive and exhaustive since \( < \) is an order on \( \mathbb{R} \). So exactly one of \( z \prec w, z = w, \) or \( z \succ w \) always holds. Suppose, for \( u = x + yi \), we have \( z \prec w \) and \( w \prec u \). Then there are four cases:

\begin{enumerate}[label = Case \arabic*., font=\bfseries, wide=0pt]
    \item \( a < c \).
    \begin{enumerate}[label* = \arabic*., font=\bfseries, wide=10pt]
        \item \( c < x \). Then \( a < x \), so that \( z \prec u \).
        \item \( c = x \) and \( d < y \). Then \( a < x \), so that \( z \prec u \).
    \end{enumerate}
    \item \( a = c \) and \( b < d \).
    \begin{enumerate}[label* = \arabic*., font=\bfseries, wide=10pt]
        \item \( c < x \). Then \( a < x \), so that \( z \prec u \).
        \item \( c = x \) and \( d < y \). Then \( a = x \) and \( b < y \), so that \( z \prec u \).
    \end{enumerate}
\end{enumerate}

In any case, we have transitivity and hence have shown that \( \prec \) is an order on \( \mathbb{C} \).

\newp

Now we claim that \( (\mathbb{C}, \prec) \) does not have the least-upper-bound property. To see this, consider the set \( E = \{ 0 + yi : y \in \mathbb{R} \} \); this is clearly non-empty and bounded above by, for example, any number of the form \( x + 0i \) for a positive real number \( x \). Suppose \( a + bi \) is an upper bound for \( E \). It follows that \( a > 0 \), for if \( a \leq 0 \) then \( (b + 1)i \succ a + bi \). But now \( \frac{a}{2} + bi \) is also an upper bound for \( E \) and \( \frac{a}{2} + bi \prec a + bi \). Hence \( E \) has no least upper bound.

\newp

\textbf{10.} Suppose \( z = a + bi, w = u + vi \), and
\[
a = \left( \frac{|w| + u}{2} \right)^{1/2}, \qquad b = \left( \frac{|w| - u}{2} \right)^{1/2}.
\]
Prove that \( z^2 = w \) if \( v \geq 0 \) and that \( (\overline{z})^2 = w \) if \( v \leq 0 \). Conclude that every complex number (with one exception!) has two complex square roots.

\newp

\textit{\textcolor{magenta}{Solution.}} First, suppose that \( v \geq 0 \). Then
\begin{align*}
z^2 &= a^2 - b^2 + 2abi \\
&= \frac{|w| + u}{2} - \frac{|w| - u}{2} + 2 \left( \frac{|w| + u}{2} \right)^{1/2} \left( \frac{|w| - u}{2} \right)^{1/2} i \\
&= u + (|w|^2 - u^2)^{1/2} i \\
&= u + (v^2)^{1/2} i \\
&= u + vi.
\end{align*}
So \( z^2 = w \), which also gives us \( (-z)^2 = w \). Hence \( w \) has two complex square roots, \( z \) and \( -z \). These are distinct precisely when \( z \neq 0 \iff z^2 = w \neq 0 \). Now suppose that \( v \leq 0 \). Then
\begin{align*}
(\overline{z})^2 &= a^2 - b^2 - 2abi \\
&= u - (v^2)^{1/2} i \\
&= u - (-v)i \\
&= u + vi.
\end{align*}
So \( (\overline{z})^2 = w \), which also gives us \( (-\overline{z})^2 = w \). Hence \( w \) has two complex square roots, \( \overline{z} \) and \( -\overline{z} \). Similarly, these are distinct precisely when \( \overline{z} \neq 0 \iff \overline{z}^2 = w \neq 0 \). We conclude that every complex number other than 0 has (at least) two distinct complex square roots; 0 has itself as its unique square root.

\newp

\textbf{11.} If \( z \) is a complex number, prove that there exists an \( r \geq 0 \) and a complex number \( w \) with \( |w| = 1 \) such that \( z = rw \). Are \( w \) and \( r \) always uniquely determined by \( z \)?

\newp

\textit{\textcolor{magenta}{Solution.}} If \( z \neq 0 \), take \( r = |z| \) and \( w = \frac{z}{|z|} \). These choices are unique, since \( z = rw \) implies that \( |z| = |rw| = r \), which in turn gives \( w = \frac{z}{|z|} \). If \( z = 0 \), then \( r = 0 \) and any \( w \) with \( |w| = 1 \) will do. In this case, \( r = 0 \) is uniquely determined, but there any infinitely many choices of \( w \) which satisfy the equation.

\newp

\textbf{12.} If \( z_1, \ldots, z_n \) are complex, prove that
\[
|z_1 + z_2 + \cdots + z_n | \leq |z_1| + |z_2| + \cdots + |z_n|.
\]

\newp

\textit{\textcolor{magenta}{Solution.}} This follows from Theorem 1.33 (e) of \hyperlink{pma}{[PMA]} and induction on \( n \).

\newp

\textbf{13.} If \( x, y \) are complex, prove that
\[
||x| - |y|| \leq |x - y|.
\]

\newp

\textit{\textcolor{magenta}{Solution.}} Observe that
\[
|x| = |x - y + y| \leq |x - y| + |y| \implies |x| - |y| \leq |x - y|,
\]
\[
|y| = |x - y - x| \leq |x - y| + |x| \implies -(|x| - |y|) \leq |x - y|.
\]
Since \( ||x| - |y|| = |x| - |y| \) or \( -(|x| - |y|) \), the result follows.

\newp

\textbf{14.} If \( z \) is a complex number such that \( |z| = 1 \), that is, such that \( z \overline{z} = 1 \), compute
\[
|1 + z|^2 + |1 - z|^2.
\]
\textit{\textcolor{magenta}{Solution.}} This is a quick computation:
\begin{align*}
|1 + z|^2 + |1 - z|^2 &= (1 + z)(1 + \overline{z}) + (1 - z)(1 - \overline{z}) \\
&= 1 + z + \overline{z} + z \overline{z} + 1 - z - \overline{z} + z \overline{z} \\
&= 4.
\end{align*}

\newp

\textbf{15.} Under what conditions does equality hold in the Schwarz inequality?

\newp

\textit{\textcolor{magenta}{Solution.}} Let \( \bm{a} = (a_1, \ldots, a_n) \) and \( \bm{b} = (b_1, \ldots, b_n) \) be vectors in \( \mathbb{C}^n \). We will show that equality holds in the Schwarz inequality
\[
    \abs{ \sum_{j=1}^n a_j \overline{b_j} \, }^2 \leq \left( \sum_{j=1}^n \abs{a_j}^2 \right) \left( \sum_{j=1}^n \abs{b_j}^2 \right) \tag{\( \dagger \)}
\]
if and only if \( \bm{a} \) and \( \bm{b} \) are linearly dependent.

\newp

First, suppose that \( \bm{a} \) and \( \bm{b} \) are linearly dependent. Then one is a complex multiple of the other and equality in \( (\dagger) \) is easily verified. Conversely, suppose that equality holds in \( (\dagger) \). As in the proof of the Schwarz inequality in \hyperlink{pma}{[PMA]}, put \( A = \sum_{j=1}^n \abs{a_j}^2, B = \sum_{j=1}^n \abs{b_j}^2 \), and \( C = \sum_{j=1}^n a_j \overline{b_j} \), so that
\[
\sum_{j=1}^n \abs{B a_j - C b_j}^2 = B(AB - \abs{C}^2). \tag{\( * \)}
\]
Equality in \( (\dagger) \) implies that the above quantity is zero, which gives us \( B a_j = C b_j \) for each \( j \). If \( \bm{b} \) is the zero vector, then certainly \( \bm{a} \) and \( \bm{b} \) are linearly dependent. So assume that \( \bm{b} \) is not the zero vector, which is the case precisely when \( B > 0 \). Then we have \( a_j = \frac{C}{B} b_j \) for each \( j \), i.e.\ \( \bm{a} = \frac{C}{B} \bm{b} \), which demonstrates the linear dependence of \( \bm{a} \) and \( \bm{b} \).

\newp

\textbf{18.} If \( k \geq 2 \) and \( \bm{x} \in \mathbb{R}^k \), prove that there exists \( \bm{y} \in \mathbb{R}^k \) such that \( \bm{y} \neq \bm{0} \) but \( \bm{x} \cdot \bm{y} = 0 \). Is this also true if \( k = 1 \)?

\newp

(Exercise 18 has been shifted up since the solution will be useful for Exercise 16 below.)

\newp

\textit{\textcolor{magenta}{Solution.}} If \( \bm{x} = \bm{0} \), any \( \bm{y} \in \mathbb{R}^k \) will do. Assume therefore that \( \bm{x} \neq \bm{0} \), so that there is some \( 1 \leq i \leq k \) such that \( x_i \neq 0 \). Choose any real numbers, other than all zeros, for the components of \( \bm{y} \) other than \( y_i \) (there is at least one such component since \( k \geq 2 \)). Now set
\[
y_i = \frac{-(x_1 y_1 + \cdots + x_{i-1} y_{i-1} + x_{i+1} y_{i+1} + \cdots + x_k y_k)}{x_i}.
\]
It then follows that: \( \bm{y} \neq \bm{0} \) since we chose at least one non-zero component for \( \bm{y} \), \( \bm{x} \cdot \bm{y} = 0 \), and, given the choices for the components of \( \bm{y} \) other than \( y_i \), only this choice of \( y_i \) will yield a \( \bm{y} \) satisfying \( \bm{x} \cdot \bm{y} = 0 \); indeed
\[
\bm{x} \cdot \bm{y} = 0 \iff \bm{x} \cdot \bm{y} - x_i y_i + x_i y_i = 0 \iff y_i = \frac{-1}{x_i}(\bm{x} \cdot \bm{y} - x_i y_i).
\]
The result is no longer true if \( k = 1 \). It is still the case that if \( x = 0 \) then any \( y \in \mathbb{R} \) will do, however for \( x \neq 0 \) there are no non-zero solutions for \( y \); this would violate the field axioms (see Proposition 1.16 (b) of \hyperlink{pma}{[PMA]}).

\newp

\textbf{16.} Suppose \( k \geq 3 \), \( \bm{x}, \bm{y} \in \mathbb{R}^k, \abs{\bm{x} - \bm{y}} = d > 0 \), and \( r > 0 \). Prove:

\textbf{(a)} If \( 2r > d \), there are infinitely many \( \bm{z} \in \mathbb{R}^k \) such that
\[
\abs{\bm{z} - \bm{x}} = \abs{\bm{z} - \bm{y}} = r.
\]

\textit{\textcolor{magenta}{Solution.}} Suppose \( \bm{w} \in \mathbb{R}^k \) satisfies the following two conditions:

\begin{enumerate}[label = (\arabic*)]
    \item \( \bm{w} \cdot (\bm{x} - \bm{y}) = 0 \);
    \item \( \abs{\bm{w}} = \sqrt{r^2 - \frac{d^2}{4}} \).
\end{enumerate}

(The quantity \( \sqrt{r^2 - \frac{d^2}{4}} \) is a positive real number since \( 2r > d \).) Set \( \bm{z} = \bm{w} + \frac{1}{2} \bm{x} + \frac{1}{2} \bm{y} \). Then
\begin{align*}
\abs{\bm{z} - \bm{x}}^2 &= \abs{\bm{w} - \left(\tfrac{1}{2} \bm{x} - \tfrac{1}{2} \bm{y}\right)}^2 \\
&= \tfrac{1}{4} \abs{2 \bm{w} - (\bm{x} - \bm{y})}^2 \\
&= \tfrac{1}{4} (2 \bm{w} - (\bm{x} - \bm{y})) \cdot (2 \bm{w} - (\bm{x} - \bm{y})) \\
&= \bm{w} \cdot \bm{w} - \bm{w} \cdot (\bm{x} - \bm{y}) + \tfrac{1}{4} (\bm{x} - \bm{y}) \cdot (\bm{x} - \bm{y}) \\
&= \abs{\bm{w}}^2 + \tfrac{1}{4} \abs{\bm{x} - \bm{y}}^2 \\
&= r^2,
\end{align*}
so that \( \abs{\bm{z} - \bm{x}} = r \). Similarly, one sees that \( \abs{\bm{z} - \bm{y}} = r \). In fact, all solutions are obtained in this way. That is, if \( \bm{z} \in \mathbb{R}^k \) is such that \( \abs{\bm{z} - \bm{x}} = \abs{\bm{z} - \bm{y}} = r \), then \( \bm{w} = \bm{z} - \tfrac{1}{2} \bm{x} - \tfrac{1}{2} \bm{y} \) satisfies conditions (1) and (2). Indeed,
\begin{align*}
    (\bm{z} - \tfrac{1}{2} \bm{x} - \tfrac{1}{2} \bm{y}) \cdot (\bm{x} - \bm{y}) &= \tfrac{1}{2} (\bm{z} - \bm{y} + \bm{z} - \bm{x}) \cdot (\bm{z} - \bm{y} - (\bm{z} - \bm{x})) \\
    &= \tfrac{1}{2} \left( \abs{\bm{z} - \bm{y}}^2 - \abs{\bm{z} - \bm{x}}^2 \right) \\
    &= 0,
\end{align*}
whence \( \bm{w} \) satisfies condition (1), i.e.\ \( \bm{w} \) is orthogonal to \( \bm{x} - \bm{y} \). Then \( \bm{w} \) is also orthogonal to \( \tfrac{1}{2} \bm{x} - \tfrac{1}{2} \bm{y} \), which has length \( \tfrac{d}{2} \). An application of the Pythagorean theorem now shows that \( \bm{w} \) satisfies condition (2). Given that \( \bm{w} \mapsto \bm{w} + \tfrac{1}{2} \bm{x} + \tfrac{1}{2} \bm{y} \) is a bijection of \( \mathbb{R}^k \), we have now classified all solutions to the given problem; note that this classification did not depend on \( k \).

\newp

Focus now on the case \( k \geq 3 \). To show that there are infinitely many such \( \bm{z} \), it will suffice to show that there are infinitely many \( \bm{w} \in \mathbb{R}^k \) satisfying conditions (1) and (2). Since \( \abs{\bm{x} - \bm{y}} \neq 0 \), \( \bm{x} - \bm{y} \) is not the zero vector and so has at least one non-zero component, say \( x_1 - y_1 \neq 0 \) (a similar discussion holds for other non-zero components). As shown in the solution to Exercise 18, if we choose any real numbers, not all zero, for the components \( u_2, \ldots, u_k \) of a vector \( \bm{u} \), then setting
\[
    u_1 = \frac{-(u_2(x_2 - y_2) + \cdots + u_k (x_k - y_k))}{x_1 - y_1}
\]
will yield a non-zero \( \bm{u} \) satisfying condition (1). Any scalar multiple of \( \bm{u} \) will also satisfy condition (1), so taking
\[
\bm{w} = \frac{\sqrt{r^2 - \frac{d^2}{4}}}{\abs{\bm{u}}} \bm{u}
\]
gives us a non-zero \( \bm{w} \) satisfying conditions (1) and (2). There are infinitely many choices for the components \( u_2, \ldots, u_k \), yielding infinitely many distinct vectors \( \bm{u} \). However, not all of these choices give us distinct vectors \( \bm{w} \) (some of the choices for \( \bm{u} \) give us vectors lying on the same line). To surmount this, suppose we have chosen \( u_3, \ldots, u_k \) such that \( u_3 \neq 0 \) (note that this requires \( k \geq 3 \)). Then observe that the ratio \( \tfrac{u_2}{u_3} = \tfrac{w_2}{w_3} \), since \( \bm{w} \) is a scalar multiple of \( \bm{u} \). There are infinitely many choices of \( u_2 \) yielding a distinct ratio; it follows that each choice yields a distinct \( \bm{w} \).

\newp

\textbf{(b)} If \( 2r = d \), there is exactly one such \( \bm{z} \).

\newp

\textit{\textcolor{magenta}{Solution.}} It is quickly verified that \( \bm{z} = \tfrac{1}{2} \bm{x} + \tfrac{1}{2} \bm{y} \) has the desired properties. To see that this is the only solution, note that \( 2r = d \) implies equality in the triangle inequality:
\[
\abs{\bm{x} - \bm{y}} = \abs{\bm{x} - \bm{z}} + \abs{\bm{z} - \bm{y}}.
\]
By studying the proof of the triangle inequality on page 17 of \hyperlink{pma}{[PMA]}, one sees that equality occurs precisely when one has equality in the Schwarz inequality. By Exercise 15, this is the case exactly when \( \bm{x} - \bm{z} \) and \( \bm{z} - \bm{y} \) are linearly dependent, say \( \bm{x} - \bm{z} = \lambda (\bm{z} - \bm{y}) \). Taking absolute values and using that \( r > 0 \), it follows that \( \lambda = \pm 1 \). \( \lambda = -1 \) gives \( \bm{x} = \bm{y} \), which is not the case since \( \abs{\bm{x} - \bm{y}} = d > 0 \), so it must be that \( \lambda = 1 \), which in turn gives \( \bm{z} = \tfrac{1}{2} \bm{x} + \tfrac{1}{2} \bm{y} \).

\newp

\textbf{(c)} If \( 2r < d \), there is no such \( \bm{z} \).

\newp

\textit{\textcolor{magenta}{Solution.}} The existence of such a \( \bm{z} \) would violate the triangle inequality
\[
    \abs{\bm{x} - \bm{y}} \leq \abs{\bm{x} - \bm{z}} + \abs{\bm{z} - \bm{y}},
\]
which in this case says \( d \leq 2r \).

\newp

How must these statements be modified if \( k \) is 2 or 1?

\newp

\textit{\textcolor{magenta}{Solution.}} The statements in \textbf{(b)} and \textbf{(c)} need no modification; the solutions there do not depend on \( k \). We shall modify part \textbf{(a)} as follows.

\newp

\( k = 2 \). If \( 2r > d \), there are exactly two \( \bm{z} \in \mathbb{R}^2 \) such that
\[
\abs{\bm{z} - \bm{x}} = \abs{\bm{z} - \bm{y}} = r.
\]
By the discussion in part \textbf{(a)}, it will suffice to show that there are exactly two \( \bm{w} = (w_1, w_2) \in \mathbb{R}^2 \) such that
\begin{enumerate}[label = (\arabic*)]
    \item \( \bm{w} \cdot (\bm{x} - \bm{y}) = 0 \);
    \item \( \abs{\bm{w}} = \sqrt{r^2 - \frac{d^2}{4}} \).
\end{enumerate}
Let us assume once again that \( x_1 - y_1 \neq 0 \), and for convenience let \( D = \sqrt{r^2 - \frac{d^2}{4}} \). For any given \( w_2 \), it is necessary to set
\[
    w_1 = \frac{-w_2(x_2 - y_2)}{x_1 - y_1}
\]
in order to satisfy condition (1). Substituting this expression for \( w_1 \) into condition (2) then constrains \( w_2 \):
\begin{align*}
    & \frac{w_2^2(x_2 - y_2)^2}{(x_1 - y_1)^2} + w_2^2 = D^2 \\
    \iff & w_2^2 \left( \frac{(x_2 - y_2)^2}{(x_1 - y_1)^2} + 1 \right) = D^2 \\
    \iff & w_2^2 \left( \frac{d^2}{(x_1 - y_1)^2} \right) = D^2 \\
    \iff & w_2 = \frac{\pm D(x_1 - y_1)}{d}.
\end{align*}
This gives us exactly two values for \( w_2 \), and hence for \( \bm{w} \), since \( D(x_1 - y_1) \neq 0 \).

\newp

\( k = 1 \). If \( 2r > d \), there are no \( z \in \mathbb{R} \) such that
\[
\abs{z - x} = \abs{z - y} = r.
\]
In this case, one is forced to take \( w = 0 \) to satisfy condition (1); but then condition (2) cannot possibly be satisfied.

\newp

\textbf{17.} Prove that
\[
    \abs{\bm{x} + \bm{y}}^2 + \abs{\bm{x} - \bm{y}}^2 = 2 \, \abs{\bm{x}}^2 + 2 \, \abs{\bm{y}}^2
\]
if \( \bm{x} \in \mathbb{R}^k \) and \( \bm{y} \in \mathbb{R}^k \). Interpret this geometrically, as a statement about parallelograms.

\newp

\textit{\textcolor{magenta}{Solution.}} This is a quick computation:
\begin{align*}
    \abs{\bm{x} + \bm{y}}^2 + \abs{\bm{x} - \bm{y}}^2 &= (\bm{x} + \bm{y}) \cdot (\bm{x} + \bm{y}) + (\bm{x} - \bm{y}) \cdot (\bm{x} - \bm{y}) \\
    &= \bm{x} \cdot \bm{x} + 2 \, \bm{x} \cdot \bm{y} + \bm{y} \cdot \bm{y} + \bm{x} \cdot \bm{x} - 2 \, \bm{x} \cdot \bm{y} + \bm{y} \cdot \bm{y} \\
    &= 2 \, \bm{x} \cdot \bm{x} + 2 \, \bm{y} \cdot \bm{y} \\
    &= 2 \, \abs{\bm{x}}^2 + 2 \, \abs{\bm{y}}^2.
\end{align*}

Geometrically, this result says that the sum of the squares of the lengths of the two diagonals of a parallelogram is equal to twice the sum of the squares of the lengths of the two sides; see \href{https://en.wikipedia.org/wiki/Parallelogram_law}{here}.

\newp 

\textbf{19.} Suppose \( \bm{a} \in \mathbb{R}^k, \bm{b} \in \mathbb{R}^k \). Find \( \bm{c} \in \mathbb{R}^k \) and \( r > 0 \) such that
\[
    \abs{\bm{x} - \bm{a}} = 2 \abs{\bm{x} - \bm{b}}
\]
if and only if \( \abs{\bm{x} - \bm{c}} = r \).

(\textit{Solution:} \( 3 \bm{c} = 4 \bm{b} - \bm{a}, 3r = 2 \abs{\bm{b} - \bm{a}}. \))

\newp

\textit{\textcolor{magenta}{Solution.}} Rudin gives us the solution; to derive it ourselves we perform the following computation:
\begin{align*}
    \abs{\bm{x} - \bm{a}} = 2 \abs{\bm{x} - \bm{b}} & \iff \left[ \sum_{j=1}^k (x_j - a_j)^2 \right]^{1/2} = 2 \left[ \sum_{j=1}^k (x_j - b_j)^2 \right]^{1/2} \\
    & \iff \sum_{j=1}^k (x_j - a_j)^2 = 4 \sum_{j=1}^k (x_j - b_j)^2 \\
    & \iff \sum_{j=1}^k x_j^2 - 2 x_j a_j + a_j^2 = \sum_{j=1}^k 4 x_j^2 - 8 x_j b_j + 4 b_j^2 \\
    & \iff \sum_{j=1}^k 3 x_j^2 - (8 b_j - 2 a_j)x_j = \sum_{j=1}^k a_j^2 - 4 b_j^2 \\
    & \iff \sum_{j=1}^k x_j^2 - \tfrac{1}{3} (8 b_j - 2 a_j)x_j = \frac{1}{3} \sum_{j=1}^k a_j^2 - 4 b_j^2 \\
    \text{(complete the square)} & \iff \sum_{j=1}^k (x_j - \tfrac{1}{3}(4 b_j - a_j))^2 - \tfrac{1}{9} (a_j - 4 b_j)^2 = \frac{1}{9} \sum_{j=1}^k 3 a_j^2 - 12 b_j^2 \\
    & \iff \sum_{j=1}^k (x_j - \tfrac{1}{3}(4 b_j - a_j))^2 = \frac{1}{9} \sum_{j=1}^k 3 a_j^2 - 12 b_j^2 + (a_j - 4 b_j)^2 \\
    & \iff \sum_{j=1}^k (x_j - \tfrac{1}{3}(4 b_j - a_j))^2 = \frac{1}{9} \sum_{j=1}^k 4 a_j^2 - 8 a_j b_j + 4 b_j^2 \\
    & \iff \sum_{j=1}^k (x_j - \tfrac{1}{3}(4 b_j - a_j))^2 = \frac{4}{9} \sum_{j=1}^k (a_j - b_j)^2 \\
    & \iff \left[ \sum_{j=1}^k (x_j - \tfrac{1}{3}(4 b_j - a_j))^2 \right]^{1/2} = \frac{2}{3} \left[ \sum_{j=1}^k (a_j - b_j)^2 \right]^{1/2} \\
    & \iff \abs{\bm{x} - \tfrac{1}{3} (4 \bm{b} - \bm{a})} = \tfrac{2}{3} \abs{\bm{b} - \bm{a}}.
\end{align*}

This exercise concerns the circles of Apollonius; see \href{https://en.wikipedia.org/wiki/Circles_of_Apollonius}{here}. It demonstrates that one may specify a sphere either by giving a centre and a radius (here, \( \bm{c} \) and \( r \)), or by giving two distinct points (here, \( \bm{a} \) and \( \bm{b} \); Rudin should really specify \( \bm{a} \neq \bm{b} \)), known as the foci, and a ratio for the distances of a point on the sphere to the two foci (here, 2).

\newp

\textbf{20.} With reference to the Appendix, suppose that property (III) were omitted from the definition of a cut. Keep the same definitions of order and addition. Show that the resulting ordered set has the least-upper-bound property, that addition satisfies axioms (A1) to (A4) (with a slightly different zero-element!) but that (A5) fails.

\newp

\textit{\textcolor{magenta}{Solution.}} (I will instead make reference to \href{https://lew98.github.io/Mathematics/Construction_of_R_from_Q_via_Dedekind_cuts.pdf}{my own write-up of the Appendix}, where I have relabeled property (III) as property (IV).) Let us call this resulting ordered set \( \Tilde{\mathbb{R}} \). By examining the linked document, we see that property (IV) is not used at all when defining the order on \( \mathbb{R} \), and is only used in the section on the least-upper-bound property to show that the proposed supremum has property (IV); so we lose nothing here by omitting property (IV). The same is true for axioms (A1) - (A3) in the section on addition, however we must modify axiom (A4) for \( \Tilde{\mathbb{R}} \) as follows.

\begin{enumerate}[label = (A\arabic*), start = 4]
    \item There exists an element \( 0 \in \Tilde{\mathbb{R}} \) such that \( A + 0 = A \) (\textbf{additive identity}). We shall use a slightly different zero element; \( 0^* = \{ p \in \mathbb{Q} : p \leq 0 \} \). It is clear that \( 0^* \) satisfies properties (I) - (III). We claim that \( 0^* \) is the additive identity in \( \Tilde{\mathbb{R}} \). For the inclusion \( A + 0^* \subseteq A \), suppose \( r \in A \) and \( s \in 0^* \), i.e.\ \( s \in \mathbb{Q} \) with \( s \leq 0 \). Then either \( r + s < r \) and so property (III) implies that \( r + s \in A \), or \( r + s = r \in A \). For the reverse inclusion \( A \subseteq A + 0^* \), simply observe that any \( r \in A \) can be written as \( r + 0 \in A + 0^* \). Hence \( A \subseteq A + 0^* \) and we conclude that \( A + 0^* = A \).
\end{enumerate}

Now we will show that axiom (A5) fails, by considering the set \( 1^* = \{ p \in \mathbb{Q} : p < 1 \} \in \Tilde{\mathbb{R}} \). Suppose there exists some \( A \in \Tilde{\mathbb{R}} \) such that \( 1^* + A = 0^* \). Since \( 0 \in 0^* \), we must be able to write \( 0 = r + s \) for some \( r \in 1^* \) and \( s = -r \in A \). Since \( r < 1 \), we have \( \tfrac{1 + r}{2} \in 1^* \). It follows that
\[
    \tfrac{1 + r}{2} - r = \tfrac{1 - r}{2} \in 1^* + A \implies \tfrac{1 - r}{2} \in 0^* \implies \tfrac{1 - r}{2} \leq 0.
\]
However, this is a contradiction:
\[
    r < 1 \implies 0 < \tfrac{1 - r}{2}.
\]

\hrulefill

\hypertarget{pma}{\textcolor{blue}{[PMA]} Rudin, W. (1976) \textit{Principles of Mathematical Analysis.} 3rd edn.}

\end{document}
