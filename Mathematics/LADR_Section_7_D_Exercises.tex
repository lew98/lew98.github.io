\documentclass[12pt]{article}
\usepackage[utf8]{inputenc}
\usepackage[utf8]{inputenc}
\usepackage{amsmath}
\usepackage{amsthm}
\usepackage{tabularray}
\usepackage{geometry}
\usepackage{amsfonts}
\usepackage{mathrsfs}
\usepackage{bm}
\usepackage{hyperref}
\usepackage[dvipsnames]{xcolor}
\usepackage{enumitem}
\usepackage{mathtools}
\usepackage{changepage}
\usepackage{lipsum}
\usepackage{float}
\usepackage{tikz}
\usetikzlibrary{matrix}
\usepackage{tikz-cd}
\usepackage[nameinlink]{cleveref}
\geometry{
headheight=15pt,
left=60pt,
right=60pt
}
\setlength{\emergencystretch}{20pt}
\usepackage{fancyhdr}
\pagestyle{fancy}
\fancyhf{}
\lhead{}
\chead{Section 7.D Exercises}
\rhead{\thepage}
\hypersetup{
    colorlinks=true,
    linkcolor=blue,
    urlcolor=blue
}

\theoremstyle{definition}
\newtheorem*{remark}{Remark}

\newtheoremstyle{exercise}
    {}
    {}
    {}
    {}
    {\bfseries}
    {.}
    { }
    {\thmname{#1}\thmnumber{#2}\thmnote{ (#3)}}
\theoremstyle{exercise}
\newtheorem{exercise}{Exercise 7.D.}

\newtheoremstyle{solution}
    {}
    {}
    {}
    {}
    {\itshape\color{magenta}}
    {.}
    { }
    {\thmname{#1}\thmnote{ #3}}
\theoremstyle{solution}
\newtheorem*{solution}{Solution}

\Crefformat{exercise}{#2Exercise 7.D.#1#3}

\newcommand{\upd}{\,\text{d}}
\newcommand{\re}{\text{Re}\,}
\newcommand{\im}{\text{Im}\,}
\newcommand{\poly}{\mathcal{P}}
\newcommand{\lmap}{\mathcal{L}}
\newcommand{\mat}{\mathcal{M}}
\newcommand{\ts}{\textsuperscript}
\newcommand{\Span}{\text{span}}
\newcommand{\Null}{\text{null\,}}
\newcommand{\Range}{\text{range\,}}
\newcommand{\Rank}{\text{rank\,}}
\newcommand{\quand}{\quad \text{and} \quad}
\newcommand{\quimplies}{\quad \implies \quad}
\newcommand{\quiff}{\quad \iff \quad}
\newcommand{\ipanon}{\langle \cdot, \cdot \rangle}
\newcommand{\normanon}{\lVert \, \cdot \, \rVert}
\newcommand{\setcomp}[1]{#1^{\mathsf{c}}}
\newcommand{\tpose}[1]{#1^{\text{t}}}
\newcommand{\ocomp}[1]{#1^{\perp}}
\newcommand{\N}{\mathbf{N}}
\newcommand{\Z}{\mathbf{Z}}
\newcommand{\Q}{\mathbf{Q}}
\newcommand{\R}{\mathbf{R}}
\newcommand{\C}{\mathbf{C}}
\newcommand{\F}{\mathbf{F}}

\DeclarePairedDelimiter\abs{\lvert}{\rvert}
% Swap the definition of \abs* and \norm*, so that \abs
% and \norm resizes the size of the brackets, and the 
% starred version does not.
\makeatletter
\let\oldabs\abs
\def\abs{\@ifstar{\oldabs}{\oldabs*}}

\DeclarePairedDelimiter\norm{\lVert}{\rVert}
\makeatletter
\let\oldnorm\norm
\def\norm{\@ifstar{\oldnorm}{\oldnorm*}}
\makeatother

\DeclarePairedDelimiter\paren{(}{)}
\makeatletter
\let\oldparen\paren
\def\paren{\@ifstar{\oldparen}{\oldparen*}}
\makeatother

\DeclarePairedDelimiter\bkt{[}{]}
\makeatletter
\let\oldbkt\bkt
\def\bkt{\@ifstar{\oldbkt}{\oldbkt*}}
\makeatother

\DeclarePairedDelimiter\Set{\{}{\}}
\makeatletter
\let\oldSet\Set
\def\Set{\@ifstar{\oldSet}{\oldSet*}}
\makeatother

\DeclarePairedDelimiter\ip{\langle}{\rangle}
\makeatletter
\let\oldip\ip
\def\set{\@ifstar{\oldip}{\oldip*}}
\makeatother

\setlist[enumerate,1]{label={(\alph*)}}

\begin{document}

\section{Section 7.D Exercises}

Exercises with solutions from Section 7.D of \hyperlink{ladr}{[LADR]}.

\begin{exercise}
\label{ex:1}
    Fix \( u, x \in V \) with \( u \neq 0 \). Define \( T \in \lmap(V) \) by
    \[
        Tv = \ip{v, u} x
    \]
    for every \( v \in V \). Prove that
    \[
        \sqrt{T^* T} v = \frac{\norm{x}}{\norm{u}} \ip{v, u} u
    \]
    for every \( v \in V \).
\end{exercise}

\begin{solution}
    It follows from Example 7.4 that
    \[
        T^* T v = \norm{x}^2 \ip{v, u} u
    \]
    for \( v \in V \). Let \( R \in \lmap(V) \) be given by
    \[
        Rv = \frac{\norm{x}}{\norm{u}} \ip{v, u} u.
    \]
    Note that
    \[
        R^2 v = \frac{\norm{x}}{\norm{u}} \ip*{\frac{\norm{x}}{\norm{u}} \ip{v, u} u, u} u = \norm{x}^2 \ip{v, u} u = T^* T v
    \]
    and that
    \[
        \ip{Rv, v} = \ip*{\frac{\norm{x}}{\norm{u}} \ip{v, u} u, v} = \frac{\norm{x}}{\norm{u}} \abs{\ip{v, u}}^2 \geq 0
    \]
    for any \( v \in V \). It follows that \( R \) is the unique positive square root of \( T^* T \), i.e.\ \( R = \sqrt{T^* T} \).
\end{solution}

\begin{exercise}
\label{ex:2}
    Give an example of \( T \in \lmap(\C^2) \) such that 0 is the only eigenvalue of \( T \) and the singular values of \( T \) are 5, 0.
\end{exercise}

\begin{solution}
    Let \( T \in \lmap(\C^2) \) be the operator whose matrix with respect to the standard basis of \( \C^2 \) is
    \[
        \begin{pmatrix}
            0 & \sqrt{5} \\
            0 & 0
        \end{pmatrix},
    \]
    so that 0 is the only eigenvalue of \( T \). The matrix of \( T^* T \) is then
    \[
        \begin{pmatrix}
            0 & 0 \\
            \sqrt{5} & 0
        \end{pmatrix}
        \begin{pmatrix}
            0 & \sqrt{5} \\
            0 & 0
        \end{pmatrix}
        =
        \begin{pmatrix}
            0 & 0 \\
            0 & 5
        \end{pmatrix},
    \]
    from which we see that the singular values of \( T \) are 5, 0.
\end{solution}

\begin{exercise}
\label{ex:3}
    Suppose \( T \in \lmap(V) \). Prove that there exists an isometry \( S \in \lmap(V) \) such that \( T = \sqrt{T T^*} S \).
\end{exercise}

\begin{solution}
    The Polar Decomposition (7.45) implies that there exists an isometry \( R \in \lmap(V) \) such that \( T^* = R \sqrt{T T^*} \). Taking adjoints of both sides of this equation gives \( T = \sqrt{T T^*} R^* \); here we are using that \( \sqrt{T T^*} \) is a positive operator and hence is self-adjoint. Observe that \( R^* \) is an isometry by 7.42 and thus the desired isometry is \( S = R^* \).
\end{solution}

\begin{exercise}
\label{ex:4}
    Suppose \( T \in \lmap(V) \) and \( s \) is a singular value of \( T \). Prove that there exists a vector \( v \in V \) such that \( \norm{v} = 1 \) and \( \norm{Tv} = s \).
\end{exercise}

\begin{solution}
    Suppose \( T \) has singular values \( s_1, \ldots, s_n \), with \( s = s_j \) for some \( 1 \leq j \leq n \). The Singular Value Decomposition (7.51) implies that there are orthonormal bases \( e_1, \ldots, e_n \) and \( f_1, \ldots, f_n \) of \( V \) such that
    \[
        Tv = s_1 \ip{v, e_1} f_1 + \cdots + s_n \ip{v, e_n} f_n
    \]
    for every \( v \in V \). In particular, we have
    \[
        T e_j = s_1 \ip{e_j, e_1} f_1 + \cdots + s_j \ip{e_j, e_j} f_j + \cdots + s_n \ip{e_j, e_n} f_n = s f_j.
    \]
    Thus \( \norm{e_j} = 1 \) and \( \norm{T e_j} = \norm{s f_j} = \abs{s} \norm{f_j} = s \), where we have used that singular values are non-negative and that \( \norm{f_j} = 1 \).
\end{solution}

\begin{exercise}
\label{ex:5}
    Suppose \( T \in \lmap(\C^2) \) is defined by \( T(x, y) = (-4y, x) \). Find the singular values of \( T \).
\end{exercise}

\begin{solution}
    The matrix of \( T \) with respect to the standard basis of \( \C^2 \) is
    \[
        A = \begin{pmatrix}
            0 & -4 \\
            1 & 0
        \end{pmatrix},
    \]
    from which we find
    \[
        A^* A = \begin{pmatrix}
            0 & 1 \\
            -4 & 0
        \end{pmatrix}
        \begin{pmatrix}
            0 & -4 \\
            1 & 0
        \end{pmatrix}
        =
        \begin{pmatrix}
            1 & 0 \\
            0 & 16
        \end{pmatrix}.
    \]
    It then follows from 7.52 that the singular values of \( T \) are 1 and 4.
\end{solution}

\begin{exercise}
\label{ex:6}
    Find the singular values of the differentiation operator \( D \in \lmap(\poly_2(\R)) \) defined by \( Dp = p' \), where the inner product on \( \poly_2(\R) \) is as in Example 6.33. (See \href{https://linear.axler.net/LADRErrataThird.html}{errata}.)
\end{exercise}

\begin{solution}
    As shown in Example 6.33, the list
    \[
        e_1 = \frac{1}{\sqrt{2}}, \quad e_2 = \sqrt{\frac{3}{2}} x, \quad e_3 = \sqrt{\frac{45}{8}} \paren{x^2 - \frac{1}{3}}
    \]
    is an orthonormal basis of \( \poly_2(\R) \) with respect to the inner product given in Example 6.33. Note that
    \[
        D e_1 = 0, \quad D e_2 = \sqrt{\frac{3}{2}} = \sqrt{3} e_1, \quand D e_3 = \sqrt{\frac{45}{8}} \paren{2x} = \sqrt{15} e_2.
    \]
    It follows that the matrix of \( D \) with respect to \( e_1, e_2, e_3 \) is
    \[
        A = \begin{pmatrix}
            0 & \sqrt{3} & 0 \\
            0 & 0 & \sqrt{15} \\
            0 & 0 & 0
        \end{pmatrix},
    \]
    from which we find that
    \[
        A^* A = \begin{pmatrix}
            0 & 0 & 0 \\
            \sqrt{3} & 0 & 0 \\
            0 & \sqrt{15} & 0
        \end{pmatrix}
        \begin{pmatrix}
            0 & \sqrt{3} & 0 \\
            0 & 0 & \sqrt{15} \\
            0 & 0 & 0
        \end{pmatrix}
        =
        \begin{pmatrix}
            0 & 0 & 0 \\
            0 & 3 & 0 \\
            0 & 0 & 15
        \end{pmatrix}.
    \]
    We can now use 7.52 to see that the singular values of \( D \) are \( \sqrt{3} \) and \( \sqrt{15} \).
\end{solution}

\begin{exercise}
\label{ex:7}
    Define \( T \in \lmap(\F^3) \) by
    \[
        T(z_1, z_2, z_3) = (z_3, 2 z_1, 3 z_2).
    \]
    Find (explicitly) an isometry \( S \in \lmap(\F^3) \) such that \( T = S \sqrt{T^* T} \).
\end{exercise}

\begin{solution}
    A straightforward calculation shows that \( \sqrt{T^* T} \) is given by
    \[
        \sqrt{T^* T}(z_1, z_2, z_3) = (2 z_1, 3 z_2, z_3).
    \]
    Thus we should take \( S \) to be the map \( S(z_1, z_2, z_3) = (z_3, z_1, z_2) \), which is evidently an isometry.
\end{solution}

\begin{exercise}
\label{ex:8}
    Suppose \( T \in \lmap(V), S \in \lmap(V) \) is an isometry, and \( R \in \lmap(V) \) is a positive operator such that \( T = SR \). Prove that \( R = \sqrt{T^* T} \).

    \noindent [\textit{The exercise above shows that if we write \( T \) as the product of an isometry and a positive operator (as in the Polar Decomposition 7.45), then the positive operator equals \( \sqrt{T^* T} \)}.]
\end{exercise}

\begin{solution}
    Since \( R \) is given as a positive operator, it will suffice to show that \( R \) is a square root of \( T^* T \); it will then follow that \( R \) is the unique positive square root of \( T^* T \), i.e.\ \( R = \sqrt{T^* T} \). Taking the adjoint of both sides of the equation \( T = SR \) gives us \( T^* = R S^* \) (we have used that \( R \) is self-adjoint), from which we see that
    \[
        T^* T = R S^* S R = R^2;
    \]
    here we have used that \( S^* = S^{-1} \) since \( S \) is an isometry (7.42).
\end{solution}

\begin{exercise}
\label{ex:9}
    Suppose \( T \in \lmap(V) \). Prove that \( T \) is invertible if and only if there exists a unique isometry \( S \) in \( \lmap(V) \) such that \( T = S \sqrt{T^* T} \).
\end{exercise}

\begin{solution}
    Suppose that \( T \) is invertible. The Polar Decomposition (7.45) implies that there always exists some isometry \( S \) such that \( T = S \sqrt{T^* T} \); suppose that there are two isometries \( S \) and \( U \) such that
    \[
        T = S \sqrt{T^* T} = U \sqrt{T^* T}.  
    \]
    Since \( T \) and \( S \) are invertible, \href{https://lew98.github.io/Mathematics/LADR_Section_3_D_Exercises.pdf}{Exercise 3.D.9} implies that \( \sqrt{T^* T} \) is invertible and thus
    \[
        S = U = T \paren{ \sqrt{T^* T} }^{-1}.  
    \]
    Now suppose that \( T \) is not invertible and consider the proof of the Polar Decomposition (7.45). Since \( T \) is not invertible, it must be the case that \( \dim \Range T < \dim V \) and hence
    \[
        \dim \ocomp{ \paren{ \Range \sqrt{T^* T} } } = \dim \ocomp{(\Range T)} = m \geq 1.
    \]
    The next step of the proof involves choosing an orthonormal basis \( e_1, \ldots, e_m \) of \( \ocomp{ \paren{ \Range \sqrt{T^* T} } } \), an orthonormal basis \( f_1, \ldots, f_m \) of \( \ocomp{(\Range T)} \), and then obtaining an isometry \( S \in \lmap(V) \) such that
    \[
        S(a_1 e_1 + \cdots + a_m e_m) = a_1 f_1 + \cdots + a_m f_m
    \]
    and such that \( T = S \sqrt{T^* T} \). Observe that the list \( -f_1, f_2, \ldots, f_m \) is also an orthonormal basis of \( \ocomp{(\Range T)} \). Define an isometry \( S' \in \lmap(V) \) as in the proof and note that \( T = S' \sqrt{T^* T} \), however
    \[
        S' e_1 = -f_1 \neq f_1 = S e_1;
    \]
    here we are crucially using that \( m \geq 1 \). Thus the isometry \( S \) is not unique.
\end{solution}

\begin{exercise}
\label{ex:10}
    Suppose \( T \in \lmap(V) \) is self-adjoint. Prove that the singular values of \( T \) equal the absolute values of the eigenvalues of \( T \), repeated appropriately.
\end{exercise}

\begin{solution}
    The relevant Spectral Theorem (7.24 or 7.29) implies that there is an orthonormal basis \( e_1, \ldots, e_n \) of \( V \) such that \( T e_j = \lambda_j e_j \) for each \( 1 \leq j \leq n \), where \( \lambda_1, \ldots, \lambda_n \) are the eigenvalues of \( T \). It follows that the adjoint of \( T \) is given by \( T^* e_j = \overline{\lambda_j} e_j \), from which we see that \( T^* T \) is given by \( T^* T e_j = \abs{\lambda_j}^2 e_j \), so that the eigenvalues of \( T^* T \) are \( \abs{\lambda_1}^2, \ldots, \abs{\lambda_n}^2 \). It now follows from 7.52 that the singular values of \( T \) are \( \abs{\lambda_1}, \ldots, \abs{\lambda_n} \).
\end{solution}

\begin{exercise}
\label{ex:11}
    Suppose \( T \in \lmap(V) \). Prove that \( T \) and \( T^* \) have the same singular values.
\end{exercise}

\begin{solution}
    By 7.52, it will suffice to show that \( T^* T \) and \( T T^* \) have the same eigenvalues. Suppose that \( \lambda \) is an eigenvalue of \( T^* T \), so that \( T^* T v = \lambda v \) for some \( v \neq 0 \). If \( Tv \neq 0 \), then observe that
    \[
        T T^* (Tv) = T (\lambda v) = \lambda Tv,
    \]
    so that \( \lambda \) is an eigenvalue of \( T T^* \) with a corresponding eigenvector \( Tv \). If \( Tv = 0 \), then it must be the case that \( \lambda = 0 \). This implies that \( T^* T \) is not invertible, which in turn implies that \( T T^* \) is not invertible (\href{https://lew98.github.io/Mathematics/LADR_Section_3_D_Exercises.pdf}{Exercise 3.D.10}). It follows that 0 is also an eigenvalue of \( T T^* \). A similar argument, reversing the roles of \( T \) and \( T^* \), shows that \( \lambda \) is an eigenvalue of \( T^* T \) if \( \lambda \) is an eigenvalue of \( T T^* \).
\end{solution}

\begin{exercise}
\label{ex:12}
    Prove or give a counterexample: if \( T \in \lmap(V) \), then the singular values of \( T^2 \) equal the squares of the singular values of \( T \).
\end{exercise}

\begin{solution}
    This is false. For a counterexample, consider the operator \( T \in \lmap(\F^2) \) whose matrix with respect to the standard basis of \( \F^2 \) is
    \[
        \begin{pmatrix}
            0 & 1 \\
            0 & 0
        \end{pmatrix}.
    \]
    A straightforward calculation shows that the singular values of \( T \) are 0, 1. However, since \( T^2 = 0 \), the singular values of \( T^2 \) are 0, 0.
\end{solution}

\begin{exercise}
\label{ex:13}
    Suppose \( T \in \lmap(V) \). Prove that \( T \) is invertible if and only if 0 is not a singular value of \( T \).
\end{exercise}

\begin{solution}
    Observe that
    \begin{align*}
        0 \text{ is not a singular value of } T &\iff 0 \text{ is not an eigenvalue of } \sqrt{T^* T} \\[2mm]
        &\iff \sqrt{T^* T} \text{ is invertible} \\[2mm]
        &\iff T^* T \text{ is invertible} \tag{\href{https://lew98.github.io/Mathematics/LADR_Section_3_D_Exercises.pdf}{Exercise 3.D.9}} \\[2mm]
        &\iff T^* \text{ and } T \text{ are invertible} \tag{\href{https://lew98.github.io/Mathematics/LADR_Section_3_D_Exercises.pdf}{Exercise 3.D.9}} \\[2mm]
        &\iff T \text{ is invertible}. \tag{\href{https://lew98.github.io/Mathematics/LADR_Section_7_A_Exercises.pdf}{Exercise 7.A.4}}
    \end{align*}
\end{solution}

\begin{exercise}
\label{ex:14}
    Suppose \( T \in \lmap(V) \). Prove that \( \dim \Range T \) equals the number of nonzero singular values of \( T \).
\end{exercise}

\begin{solution}
    The operator \( \sqrt{T^* T} \) is self-adjoint and hence the relevant Spectral Theorem (7.24 or 7.29) implies that \( \sqrt{T^* T} \) is diagonalizable. Letting \( s_1, \ldots, s_n \) be the non-zero singular values of \( T \), it then follows from 3.22 and 5.41 that
    \begin{multline*}
        \dim \Null \sqrt{T^* T} + \dim \Range \sqrt{T^* T} = \dim E \paren{ 0, \sqrt{T^* T} } + \dim \Range \sqrt{T^* T} \\[2mm]
        = \dim V = \dim E \paren{ 0, \sqrt{T^* T} } + \dim E \paren{ s_1, \sqrt{T^* T} } + \cdots + \dim E \paren{ s_n, \sqrt{T^* T} },
    \end{multline*}
    from which we see that
    \[
        \dim \Range \sqrt{T^* T} = \dim E \paren{ s_1, \sqrt{T^* T} } + \cdots + \dim E \paren{ s_n, \sqrt{T^* T} };
    \]
    note that the integer on the right-hand side is the number of non-zero singular values of \( T \). Finally, as the proof of the Polar Decomposition (7.45) shows, we have
    \[
        \dim \Range T = \dim \Range \sqrt{T^* T}.
    \]
\end{solution}

\begin{exercise}
\label{ex:15}
    Suppose \( S \in \lmap(V) \). Prove that \( S \) is an isometry if and only if all the singular values of \( S \) equal 1.
\end{exercise}

\begin{solution}
    Suppose that \( S \) is an isometry. By 7.42 we then have \( S^* = S^{-1} \) and thus \( \sqrt{S^* S} = I \), from which it is clear that all the singular values of \( S \) equal 1. Now suppose that each singular value of \( S \) is 1. The Singular Value Decomposition (7.51) implies that there exist orthonormal bases \( e_1, \ldots, e_n \) and \( f_1, \ldots, f_n \) of \( V \) such that
    \[
        Sv = \ip{v, e_1} f_1 + \cdots + \ip{v, e_n} f_n
    \]
    for any \( v \in V \). By 6.25 and 6.30 we then have
    \[
        \norm{Sv}^2 = \norm{v}^2 = \abs{\ip{v, e_1}}^2 + \cdots + \abs{\ip{v, e_n}}^2,
    \]
    from which we see that \( S \) is an isometry.
\end{solution}

\begin{exercise}
\label{ex:16}
    Suppose \( T_1, T_2 \in \lmap(V) \). Prove that \( T_1 \) and \( T_2 \) have the same singular values if and only if there exist isometries \( S_1, S_2 \in \lmap(V) \) such that \( T_1 = S_1 T_2 S_2 \).
\end{exercise}

\begin{solution}
    Suppose there exist such isometries \( S_1 \) and \( S_2 \). In light of 7.52, it will suffice to show that \( T_1^* T_1 \) and \( T_2^* T_2 \) have the same eigenvalues. Observe that
    \[
        T_1^* T_1 = (S_1 T_2 S_2)^* (S_1 T_2 S_2) = S_2^* T_2^* S_1^* S_1 T_2 S_2 = S_2^* T_2^* T_2 S_2,
    \]
    where we have used that \( S_1^* = S_1^{-1} \) (7.42). It now follows from \href{https://lew98.github.io/Mathematics/LADR_Section_5_A_Exercises.pdf}{Exercise 5.A.15} that \( T_1^* T_1 \) and \( T_2^* T_2 \) have the same eigenvalues.

    Now suppose that \( T_1 \) and \( T_2 \) have the same singular values \( s_1, \ldots, s_n \). The Polar Decomposition (7.45) implies that there exist orthonormal bases
    \[
        e_1, \ldots, e_n, \quad f_1, \ldots, f_n, \quad u_1, \ldots, u_n, \quand v_1, \ldots, v_n
    \]
    such that
    \[
        T_1 v = s_1 \ip{v, e_1} f_1 + \cdots + s_n \ip{v, e_n} f_n \quand T_2 v = s_1 \ip{v, u_1} v_1 + \cdots + s_n \ip{v, u_n} v_n
    \]
    for any \( v \in V \). Define two linear maps \( S_1 \) and \( S_2 \) by \( S_1 v_j = f_j \) and \( S_2 e_j = u_j \); it follows from the equivalence of (a) and (d) in 7.42 that \( S_1 \) and \( S_2 \) are isometries. Furthermore, observe that
    \[
        S_1 T_2 S_2 e_j = S_1 T_2 u_j = s_j S_1 v_j = s_j f_j = T_1 e_j.
    \]
    Thus \( T_1 = S_1 T_2 S_2 \).
\end{solution}

\begin{exercise}
\label{ex:17}
    Suppose \( T \in \lmap(V) \) has singular value decomposition given by
    \[
        Tv = s_1 \ip{v, e_1} f_1 + \cdots + s_n \ip{v, e_n} f_n
    \]
    for every \( v \in V \), where \( s_1, \ldots, s_n \) are the singular values of \( T \) and \( e_1, \ldots, e_n \) and \( f_1, \ldots, f_n \) are orthonormal bases of \( V \).
    \begin{enumerate}
        \item Prove that if \( v \in V \), then
        \[
            T^* v = s_1 \ip{v, f_1} e_1 + \cdots + s_n \ip{v, f_n} e_n.
        \]

        \item Prove that if \( v \in V \), then
        \[
            T^* T v = s_1^2 \ip{v, e_1} e_1 + \cdots + s_n^2 \ip{v, e_n} e_n.
        \]

        \item Prove that if \( v \in V \), then
        \[
            \sqrt{T^* T} v = s_1 \ip{v, e_1} e_1 + \cdots + s_n \ip{v, e_n} e_n.
        \]

        \item Suppose \( T \) is invertible. Prove that if \( v \in V \), then
        \[
            T^{-1} v = \frac{\ip{v, f_1} e_1}{s_1} + \cdots + \frac{\ip{v, f_n} e_n}{s_n}
        \]
        for every \( v \in V \).
    \end{enumerate}
\end{exercise}

\begin{solution}
    \begin{enumerate}
        \item The given singular value decomposition implies that
        \[
            \mat(T, (e_1, \ldots, e_n), (f_1, \ldots, f_n)) = \begin{pmatrix}
                s_1 & \cdots & 0 \\
                \vdots & \ddots & \vdots \\
                0 & \cdots & s_n
            \end{pmatrix}.
        \]
        It follows from 7.10 that
        \[
            \mat(T^*, (f_1, \ldots, f_n), (e_1, \ldots, e_n)) = \begin{pmatrix}
                \overline{s_1} & \cdots & 0 \\
                \vdots & \ddots & \vdots \\
                0 & \cdots & \overline{s_n}
            \end{pmatrix}
            =
            \begin{pmatrix}
                s_1 & \cdots & 0 \\
                \vdots & \ddots & \vdots \\
                0 & \cdots & s_n
            \end{pmatrix},
        \]
        where we have used that singular values are necessarily non-negative real numbers, being the eigenvalues of a positive operator. This matrix implies the desired result.

        \item We have by 3.43 and part (a) that
        \begin{multline*}
            \mat(T^* T, (e_1, \ldots, e_n), (e_1, \ldots, e_n)) \\[2mm]
            = \mat(T^*, (f_1, \ldots, f_n), (e_1, \ldots, e_n)) \mat(T, (e_1, \ldots, e_n), (f_1, \ldots, f_n)) \\[2mm]
            = \begin{pmatrix}
                s_1 & \cdots & 0 \\
                \vdots & \ddots & \vdots \\
                0 & \cdots & s_n
            \end{pmatrix}^2
            =
            \begin{pmatrix}
                s_1^2 & \cdots & 0 \\
                \vdots & \ddots & \vdots \\
                0 & \cdots & s_n^2
            \end{pmatrix},
        \end{multline*}
        which implies the desired result.

        \item Define an operator \( R \in \lmap(V) \) by
        \[
            Rv = s_1 \ip{v, e_1} e_1 + \cdots + s_n \ip{v, e_n} e_n,
        \]
        so that
        \[
            \mat(R, (e_1, \ldots, e_n)) = \begin{pmatrix}
                s_1 & \cdots & 0 \\
                \vdots & \ddots & \vdots \\
                0 & \cdots & s_n
            \end{pmatrix}.
        \]
        It is then clear, given part (b), that \( R^2 = T^* T \). Furthermore, since each \( s_j \) is a non-negative real number, we see that \( R \) is self-adjoint and that each eigenvalue of \( R \) is a non-negative real number. It follows that \( R \) is a positive operator (7.35) and hence that \( R = \sqrt{T^* T} \) (7.36).

        \item Define an operator \( S \in \lmap(V) \) by
        \[
            Sv = \frac{\ip{v, f_1} e_1}{s_1} + \cdots + \frac{\ip{v, f_n} e_n}{s_n}
        \]
        and observe that
        \[
            ST e_j = s_j S f_j = e_j.
        \]
        It follows that \( ST = I \), which is the case if and only if \( TS = I \) (\href{https://lew98.github.io/Mathematics/LADR_Section_3_D_Exercises.pdf}{Exercise 3.D.10}), and hence by the uniqueness of the inverse (3.54) we have \( S = T^{-1} \).
    \end{enumerate}
\end{solution}

\begin{exercise}
\label{ex:18}
    Suppose \( T \in \lmap(V) \). Let \( \hat{s} \) denote the smallest singular value of \( T \), and let \( s \) denote the largest singular value of \( T \).
    \begin{enumerate}
        \item Prove that \( \hat{s} \norm{v} \leq \norm{Tv} \leq s \norm{v} \) for every \( v \in V \).

        \item Suppose \( \lambda \) is an eigenvalue of \( T \). Prove that \( \hat{s} \leq \abs{\lambda} \leq s \).
    \end{enumerate}
\end{exercise}

\begin{solution}
    \begin{enumerate}
        \item Let \( s_1, \ldots, s_n \) be the singular values of \( T \). The Singular Value Decomposition (7.51) implies that there exist orthonormal bases \( e_1, \ldots, e_n \) and \( f_1, \ldots, f_n \) of \( V \) such that
        \[
            Tv = s_1 \ip{v, e_1} f_1 + \cdots + s_n \ip{v, e_n} f_n
        \]
        for every \( v \in V \), from which we see that
        \[
            \norm{Tv}^2 = s_1^2 \abs{\ip{v, e_1}}^2 + \cdots + s_n^2 \abs{\ip{v, e_n}}^2.
        \]
        It is then clear that
        \[
            \hat{s}^2 \abs{\ip{v, e_1}}^2 + \cdots + \hat{s}^2 \abs{\ip{v, e_n}}^2 = \hat{s}^2 \norm{v}^2 \leq \norm{Tv}^2 \leq s^2 \norm{v}^2 = s^2 \abs{\ip{v, e_1}}^2 + \cdots + s^2 \abs{\ip{v, e_n}}^2,
        \]
        which implies the desired result upon taking square roots.

        \item Let \( v \) be an eigenvector associated with \( \lambda \); by replacing \( v \) with \( \tfrac{v}{\norm{v}} \) if necessary, we may assume that \( \norm{v} = 1 \). Part (a) then gives
        \[
            \hat{s} \leq \norm{Tv} = \norm{\lambda v} = \abs{\lambda} \leq s.
        \]
    \end{enumerate}
\end{solution}

\begin{exercise}
\label{ex:19}
    Suppose \( T \in \lmap(V) \). Show that \( T \) is uniformly continuous with respect to the metric \( d \) on \( V \) defined by \( d(u, v) = \norm{u - v} \).
\end{exercise}

\begin{solution}
    As we showed in the solution to \href{https://lew98.github.io/Mathematics/LADR_Section_6_B_Exercises.pdf}{Exercise 6.B.16}, there exists a non-negative constant \( C \in \R \) such that \( \norm{Tv} \leq C \norm{v} \) for all \( v \in V \). Let \( \epsilon > 0 \) be given and suppose \( u, v \in V \) are such that \( \norm{u - v} < \tfrac{\epsilon}{1 + C} \). Then
    \[
        \norm{Tu - Tv} = \norm{T(u - v)} \leq C \norm{u - v} < \frac{C \epsilon}{1 + C} < \epsilon
    \]
    and thus \( T \) is uniformly continuous.
\end{solution}

\begin{exercise}
\label{ex:20}
    Suppose \( S, T \in \lmap(V) \). Let \( s \) denote the largest singular value of \( S \), let \( t \) denote the largest singular value of \( T \), and let \( r \) denote the largest singular value of \( S + T \). Prove that \( r \leq s + t \).
\end{exercise}

\begin{solution}
    Let \( R = S + T \). Since \( r \) is a singular value of \( R \), there exists an eigenvector \( v \in V \) such that \( \norm{v} = 1 \) and \( \sqrt{R^* R} v = rv \). It follows that
    \[
        r = \norm{rv} = \norm{\sqrt{R^* R} v} = \norm{Rv} = \norm{Sv + Tv} \leq \norm{Sv} + \norm{Tv} \leq s + t;
    \]
    the third equality follows from 7.46 and the last inequality follows from \Cref{ex:18}.
\end{solution}

\noindent \hrulefill

\noindent \hypertarget{ladr}{\textcolor{blue}{[LADR]} Axler, S. (2015) \textit{Linear Algebra Done Right.} 3\ts{rd} edition.}

\end{document}