\documentclass[12pt]{article}
\usepackage[utf8]{inputenc}
\usepackage[utf8]{inputenc}
\usepackage{amsmath}
\usepackage{amsthm}
\usepackage{amssymb}
\usepackage{geometry}
\usepackage{amsfonts}
\usepackage{mathrsfs}
\usepackage{bm}
\usepackage{hyperref}
\usepackage{float}
\usepackage[dvipsnames]{xcolor}
\usepackage[inline]{enumitem}
\usepackage{mathtools}
\usepackage{changepage}
\usepackage{graphicx}
\usepackage{caption}
\usepackage{subcaption}
\usepackage{lipsum}
\usepackage{tikz}
\usetikzlibrary{matrix, patterns, decorations.pathreplacing, calligraphy}
\usepackage{tikz-cd}
\usepackage[nameinlink]{cleveref}
\geometry{
headheight=15pt,
left=60pt,
right=60pt
}
\setlength{\emergencystretch}{20pt}
\usepackage{fancyhdr}
\pagestyle{fancy}
\fancyhf{}
\lhead{}
\chead{Section 6.4 Exercises}
\rhead{\thepage}
\hypersetup{
    colorlinks=true,
    linkcolor=blue,
    urlcolor=blue
}

\theoremstyle{definition}
\newtheorem*{remark}{Remark}

\newtheoremstyle{exercise}
    {}
    {}
    {}
    {}
    {\bfseries}
    {.}
    { }
    {\thmname{#1}\thmnumber{#2}\thmnote{ (#3)}}
\theoremstyle{exercise}
\newtheorem{exercise}{Exercise 6.4.}

\newtheoremstyle{solution}
    {}
    {}
    {}
    {}
    {\itshape\color{magenta}}
    {.}
    { }
    {\thmname{#1}\thmnote{ #3}}
\theoremstyle{solution}
\newtheorem*{solution}{Solution}

\Crefformat{exercise}{#2Exercise 6.4.#1#3}

\newcommand{\interior}[1]{%
  {\kern0pt#1}^{\mathrm{o}}%
}
\newcommand{\ts}{\textsuperscript}
\newcommand{\setcomp}[1]{#1^{\mathsf{c}}}
\newcommand{\quand}{\quad \text{and} \quad}
\newcommand{\quimplies}{\quad \implies \quad}
\newcommand{\N}{\mathbf{N}}
\newcommand{\Z}{\mathbf{Z}}
\newcommand{\Q}{\mathbf{Q}}
\newcommand{\I}{\mathbf{I}}
\newcommand{\R}{\mathbf{R}}
\newcommand{\C}{\mathbf{C}}

\DeclarePairedDelimiter\abs{\lvert}{\rvert}
% Swap the definition of \abs* and \norm*, so that \abs
% and \norm resizes the size of the brackets, and the 
% starred version does not.
\makeatletter
\let\oldabs\abs
\def\abs{\@ifstar{\oldabs}{\oldabs*}}
%
\let\oldnorm\norm
\def\norm{\@ifstar{\oldnorm}{\oldnorm*}}
\makeatother

\DeclarePairedDelimiter\paren{(}{)}
\makeatletter
\let\oldparen\paren
\def\paren{\@ifstar{\oldparen}{\oldparen*}}
\makeatother

\DeclarePairedDelimiter\bkt{[}{]}
\makeatletter
\let\oldbkt\bkt
\def\bkt{\@ifstar{\oldbkt}{\oldbkt*}}
\makeatother

\DeclarePairedDelimiter\set{\{}{\}}
\makeatletter
\let\oldset\set
\def\set{\@ifstar{\oldset}{\oldset*}}
\makeatother

\setlist[enumerate,1]{label={(\alph*)}}

\begin{document}

\section{Section 6.4 Exercises}

Exercises with solutions from Section 6.4 of \hyperlink{ua}{[UA]}.

\begin{exercise}
\label{ex:1}
    Supply the details for the proof of the Weierstrass M-Test (Corollary 6.4.5).
\end{exercise}

\begin{solution}
    Let \( \epsilon > 0 \) be given. Since the series \( \sum_{n=1}^{\infty} M_n \) is convergent, its sequence of partial sums is a Cauchy sequence. Consequently, there exists an \( N \in \N \) such that
    \[
        n > m \geq N \quad \implies \quad M_{m+1} + \cdots + M_n < \epsilon.
    \]
    Suppose \( x \in A \) and \( n > m \geq N \). Then
    \[
        \abs{f_{m+1}(x) + \cdots + f_n(x)} \leq \abs{f_{m+1}(x)} + \cdots + \abs{f_n(x)} \leq M_{m+1} + \cdots + M_n < \epsilon.
    \]
    It follows from Theorem 6.4.4 that the series \( \sum_{n=1}^{\infty} f_n \) converges uniformly on \( A \).
\end{solution}

\begin{exercise}
\label{ex:2}
    Decide whether each proposition is true or false, providing a short justification or counterexample as appropriate.
    \begin{enumerate}
        \item If \( \sum_{n=1}^{\infty} g_n \) converges uniformly, then \( (g_n) \) converges uniformly to zero.

        \item If \( 0 \leq f_n(x) \leq g_n(x) \) and \( \sum_{n=1}^{\infty} g_n \) converges uniformly, then \( \sum_{n=1}^{\infty} f_n \) converges uniformly.

        \item If \( \sum_{n=1}^{\infty} f_n \) converges uniformly on \( A \), then there exist constants \( M_n \) such that \( \abs{f_n(x)} \leq M_n \) for all \( x \in A \) and \( \sum_{n=1}^{\infty} M_n \) converges.
    \end{enumerate}
\end{exercise}

\begin{solution}
    \begin{enumerate}
        \item This is true. Suppose that each \( g_n \) is defined on some domain \( A \subseteq \R \). Note that Theorem 6.4.4 implies in particular that for any \( \epsilon > 0 \) there is an \( N \in N \) such that
        \[
            x \in A \text{ and } n \geq N \quad \implies \quad \abs{g_n(x)} \leq \epsilon.
        \]
        Thus \( g_n \) converges uniformly to the zero function.

        \item This is true. Suppose that each \( f_n \) and each \( g_n \) is defined on some domain \( A \subseteq \R \). Theorem 6.4.4 implies that for any \( \epsilon > 0 \) there is an \( N \in \N \) such that
        \[
            x \in A \text{ and } n > m \geq N \quad \implies \quad g_{m+1}(x) + \cdots + g_n(x) < \epsilon;
        \]
        note that we have used the non-negativity of each \( g_n \) here. Suppose \( x \in A \) and \( n > m \geq N \). By hypothesis we have
        \[
            f_{m+1}(x) + \cdots + f_n(x) \leq g_{m+1}(x) + \cdots + g_n(x) < \epsilon.
        \]
        Combining the above inequality with the non-negativity of each \( f_n \) and Theorem 6.4.4, we see that the series \( \sum_{n=1}^{\infty} f_n \) converges uniformly on \( A \).

        \item This is false. For each \( n \in \N \) define a function \( f_n : \R \to \R \) by
        \[
            f_n(x) = \begin{cases}
                \tfrac{1}{x} & \text{if } x = n, \\
                0 & \text{otherwise},
            \end{cases}
        \]
        and let \( f : \R \to \R \) be given by
        \[
            f(x) = \begin{cases}
                \tfrac{1}{x} & \text{if } x \in \N, \\
                0 & \text{otherwise}.
            \end{cases}
        \]
        We claim that \( \sum_{n=1}^{\infty} f_n \) converges to \( f \) uniformly on \( \R \). Observe that the partial sum function is
        \[
            s_n(x) = f_1(x) + \cdots + f_n(x) = \begin{cases}
                \tfrac{1}{x} & \text{if } x = 1, \ldots, n, \\
                0 & \text{otherwise}.
            \end{cases}
        \]
        It follows that for any \( x \in \R \)
        \[
            \abs{s_n(x) - f(x)} \leq \tfrac{1}{n + 1};
        \]
        since this bound converges to zero and does not depend on \( x \) our claim follows.
        
        Clearly, \( \sup_{x \in \R} \abs{f_n(x)} = \tfrac{1}{n} \). Since the harmonic series diverges, we see that the converse of the Weierstrass M-Test does not hold.
    \end{enumerate}
\end{solution}

\begin{exercise}
\label{ex:3}
    \begin{enumerate}
        \item Show that
        \[
            g(x) = \sum_{n=0}^{\infty} \frac{\cos (2^n x)}{2^n}
        \]
        is continuous on all of \( \R \).

        \item The function \( g \) was cited in Section 5.4 as an example of a continuous nowhere differentiable function. What happens if we try to use Theorem 6.4.3 to explore whether \( g \) is differentiable?
    \end{enumerate}
\end{exercise}

\begin{solution}
    \begin{enumerate}
        \item Observe that
        \[
            \abs{\frac{\cos (2^n x)}{2^n}} \leq \frac{1}{2^n}
        \]
        for every \( x \in \R \). Since the series \( \sum_{n=0}^{\infty} 2^{-n} \) is convergent, the Weierstrass M-Test implies that \( g(x) = \sum_{n=0}^{\infty} 2^{-n} \cos(2^n x) \) converges uniformly on \( \R \). Each \( 2^{-n} \cos(2^n x) \) is continuous on \( \R \), so Theorem 6.4.2 implies that \( g \) is continuous on \( \R \).

        \item To use Theorem 6.4.3, we should show that the series
        \[
            \sum_{n=0}^{\infty} \paren{\frac{\cos (2^n x)}{2^n}}' = -\sum_{n=0}^{\infty} \sin(2^n x). 
        \]
        converges uniformly on \( \R \). However, this series does not even converge pointwise on \( \R \). For example, consider the series of real numbers
        \[
            \sum_{n=0}^{\infty} \sin(2^n).
        \]
        To show that this series is divergent, we will show that the sequence \( (\sin(2^n)) \) does not converge to zero. To see this, consider the following two cases.
        \begin{description}
            \item[Case 1.] If there exists an \( N \in \N \) such that \( \abs{\sin(2^{n+1})} > \abs{\sin(2^n)} \) for all \( n \geq N \), then it must be the case that \( \sin(2^n) \not\to 0 \).

            \item[Case 2.] If there does not exist such an \( N \), then there must be infinitely many natural numbers \( n \) such that \( \abs{\sin{(2^{n+1})}} \leq \abs{\sin(2^n)} \). Consider such an \( n \). Using the identity
            \[
                \sin(2^{n+1}) = 2 \sin(2^n) \cos(2^n)
            \]
            and the fact that \( \sin(2^n) \neq 0 \) for any \( n \in \N \), we see that \( \abs{\cos(2^n)} \leq \tfrac{1}{2} \). The Pythagorean identity then implies that \( \abs{\sin(2^n)} \geq \tfrac{\sqrt{3}}{2} \). So in this case, the sequence \( (\sin(2^n)) \) satisfies \( \abs{\sin(2^n)} \geq \tfrac{\sqrt{3}}{2} \) infinitely often and hence \( \sin(2^n) \not\to 0 \).
        \end{description}
        So Theorem 6.4.3 does not allow us to conclude anything about the differentiability of \( g \).
    \end{enumerate}
\end{solution}

\begin{exercise}
\label{ex:4}
    Define
    \[
        g(x) = \sum_{n=0}^{\infty} \frac{x^{2n}}{(1 + x^{2n})}.
    \]
    Find the values of \( x \) where the series converges and show that we get a continuous function on this set.
\end{exercise}

\begin{solution}
    For \( \abs{x} = 1 \) we have \( g(x) = \sum_{n=0}^{\infty} \frac{1}{2} \), which diverges. For \( \abs{x} > 1 \) we have
    \[
        \frac{x^{2n}}{1 + x^{2n}} = \frac{1}{x^{-2n} + 1} \to 1 \text{ as } n \to \infty
    \]
    and thus \( g(x) \) diverges.

    Now suppose that \( r > 0 \) is such that \( 0 \leq r^2 < 1 \). Observe that for all \( x \in [-r, r] \) we have
    \[
        0 \leq \frac{x^{2n}}{1 + x^{2n}} \leq x^{2n} \leq r^{2n}.
    \]
    Since \( \sum_{n=0}^{\infty} r^{2n} \) is a convergent geometric series, the Weierstrass M-Test implies that \( g \) converges uniformly on \( [-r, r] \). Since any \( x \in (-1, 1) \) is contained inside an interval of this form, we may conclude that \( g \) converges and is continuous at each \( x \in (-1, 1) \) (Theorem 6.4.2). Combining this with our previous discussion, we see that \( g \) converges pointwise precisely on the open interval \( (-1, 1) \).
\end{solution}

\begin{exercise}
\label{ex:5}
    \begin{enumerate}
        \item Prove that
        \[
            h(x) = \sum_{n=1}^{\infty} \frac{x^n}{n^2} = x + \frac{x^2}{4} + \frac{x^3}{9} + \frac{x^4}{16} + \cdots
        \]
        is continuous on \( [-1, 1] \).

        \item The series
        \[
            f(x) = \sum_{n=1}^{\infty} \frac{x^n}{n} = x + \frac{x^2}{2} + \frac{x^3}{3} + \frac{x^4}{4} + \cdots
        \]
        converges for every \( x \) in the half-open interval \( [-1, 1) \) but does not converge when \( x = 1 \). For a fixed \( x_0 \in (-1, 1) \), explain how we can still use the Weierstrass M-Test to prove that \( f \) is continuous at \( x_0 \).
    \end{enumerate}
\end{exercise}

\begin{solution}
    \begin{enumerate}
        \item For any \( x \in [-1, 1] \) we have
        \[
            \abs{\frac{x^n}{n^2}} \leq \frac{1}{n^2}.
        \]
        Since the series \( \sum_{n=1}^{\infty} \tfrac{1}{n^2} \) is convergent, the Weierstrass M-Test allows us to conclude that \( h \) converges uniformly on \( [-1, 1] \) and Theorem 6.4.2 then implies that \( h \) is continuous on \( [-1, 1] \), since each \( \tfrac{x^n}{n^2} \) is continuous here.

        \item Observe that
        \[
            \abs{\frac{x^n}{n}} \leq \abs{x_0}^n
        \]
        for every \( x \in [-x_0, x_0] \). Since \( \sum_{n=1}^{\infty} \abs{x_0}^n \) is a convergent geometric series, the Weierstrass M-Test implies that \( f \) converges uniformly on \( [-x_0, x_0] \). Theorem 6.4.2 allows us to conclude that \( f \) is continuous on \( [-x_0, x_0] \), since each \( \tfrac{x^n}{n} \) is continuous here.
    \end{enumerate}
\end{solution}

\begin{exercise}
\label{ex:6}
    Let
    \[
        f(x) = \frac{1}{x} - \frac{1}{x + 1} + \frac{1}{x + 2} - \frac{1}{x + 3} + \frac{1}{x + 4} - \cdots \, .
    \]
    Show \( f \) is defined for all \( x > 0 \). Is \( f \) continuous on \( (0, \infty) \)? How about differentiable?
\end{exercise}

\begin{solution}
    Observe that
    \[
        f(x) = \sum_{n=0}^{\infty} \frac{(-1)^n}{x + n}.
    \]
    The term-by-term differentiated series is
    \[
        -\frac{1}{x^2} + \sum_{n=1}^{\infty} \frac{(-1)^{n+1}}{(x + n)^2}.
    \]
    Note that
    \[
        \abs{\frac{(-1)^{n+1}}{(x + n)^2}} \leq \frac{1}{n^2}
    \]
    for any \( x \in (0, \infty) \) and \( n \in \N \). Since the series \( \sum_{n=1}^{\infty} \frac{1}{n^2} \) is convergent, the Weierstrass M-Test implies that the series \( \sum_{n=1}^{\infty} \frac{(-1)^{n+1}}{(x + n)^2} \) converges uniformly on \( (0, \infty) \). It follows that the term-by-term differentiated series converges uniformly on \( (0, \infty) \) and hence we may invoke Theorem 6.4.3 to see that the series \( \sum_{n=0}^{\infty} \frac{(-1)^n}{x + n} \) converges uniformly on \( (0, \infty) \) to a differentiable function \( f \); this also implies that \( f \) is defined and continuous on \( (0, \infty) \).
\end{solution}

\begin{exercise}
\label{ex:7}
    Let
    \[
        f(x) = \sum_{k=1}^{\infty} \frac{\sin(kx)}{k^3}.
    \]
    \begin{enumerate}
        \item Show that \( f(x) \) is differentiable and that the derivative \( f'(x) \) is continuous.

        \item Can we determine if \( f \) is twice-differentiable?
    \end{enumerate}
\end{exercise}

\begin{solution}
    \begin{enumerate}
        \item Let \( f_k : \R \to \R \) be given by \( f_k(x) = \tfrac{\sin(kx)}{k^3} \). Observe that
        \[
            \abs{f_k'(x)} = \abs{\frac{\cos(kx)}{k^2}} \leq \frac{1}{k^2}
        \]
        for all \( x \in \R \). The Weierstrass M-Test then implies that the series
        \[
            \sum_{k=1}^{\infty} f_k'(x) = \sum_{k=1}^{\infty} \frac{\cos(kx)}{k^2}
        \]
        converges uniformly on \( \R \); since each \( f_k' \) is continuous on \( \R \), Theorem 6.4.2 implies that \( \sum_{k=1}^{\infty} f_k'(x) \) is also continuous on \( \R \). Combining our previous discussion with Theorem 6.4.3 and the fact that \( f(0) = 0 \), we see that \( \sum_{k=1}^{\infty} \frac{\sin(kx)}{k^3} \) converges uniformly on \( \R \) to a differentiable function \( f \), that
        \[
            f'(x) = \sum_{k=1}^{\infty} f_k'(x) = \sum_{k=1}^{\infty} \frac{\cos(kx)}{k^2},
        \]
        and that \( f' \) is continuous on \( \R \).

        \item We will show that Theorem 6.4.3 cannot be used to determine if \( f \) is twice-differentiable on \( \R \), by showing that the series of second derivatives
        \[
            \sum_{k=1}^{\infty} f_k''(x) = - \sum_{k=1}^{\infty} \frac{\sin(kx)}{k}
        \]
        does not converge uniformly on \( \R \). To see this, we will use the negation of Theorem 6.4.4. Let \( N \in \N \) be given and set \( x = \tfrac{\pi}{4N} \). For any \( N + 1 \leq k \leq 2N \), we then have \( \tfrac{\pi}{4} \leq kx \leq \tfrac{\pi}{2} \) and hence \( \sin(kx) \geq \tfrac{1}{\sqrt{2}} \). Now observe that
        \[
            \abs{\sum_{k=N+1}^{2N} \frac{\sin(kx)}{k}} \geq \frac{1}{\sqrt{2}} \sum_{k=N+1}^{2N} \frac{1}{k} \geq \frac{1}{\sqrt{2}} \sum_{k=N+1}^{2N} \frac{1}{2N} = \frac{1}{2 \sqrt{2}}.
        \]
        It follows from Theorem 6.4.4 that the convergence of the series \( \sum_{k=1}^{\infty} \tfrac{\sin(kx)}{k} \) is not uniform on \( \R \). Consequently, we may not use Theorem 6.4.3 to conclude anything about the twice-differentiability of \( f \).
    \end{enumerate}
\end{solution}

\begin{exercise}
\label{ex:8}
    Consider the function
    \[
        f(x) = \sum_{k=1}^{\infty} \frac{\sin(x/k)}{k}.
    \]
    Where is \( f \) defined? Continuous? Differentiable? Twice-differentiable?
\end{exercise}

\begin{solution}
    Let \( f_k : \R \to \R \) be given by \( f_k(x) = \frac{\sin(x/k)}{k} \), so that \( f(x) = \sum_{k=1}^{\infty} f_k(x) \). Observe that
    \[
        f_k'(x) = \frac{\cos \paren{\tfrac{x}{k}}}{k^2} \quand f_k''(x) = -\frac{\sin \paren{\tfrac{x}{k}}}{k^3}.
    \]
    The bound \( \abs{f_k''(x)} \leq \tfrac{1}{k^3} \) for all \( x \in \R \) combined with the Weierstrass M-Test shows that the series \( \sum_{k=1}^{\infty} f_k''(x) \) converges uniformly on \( \R \). Since
    \[
        f(0) = 0 \quand \sum_{k=1}^{\infty} f_k'(0) = \sum_{k=1}^{\infty} \frac{1}{k^2}
    \]
    are both convergent, two applications of Theorem 6.4.3 show that \( \sum_{k=1}^{\infty} f_k'(x) \) and \( \sum_{k=1}^{\infty} f_k(x) \) converge uniformly on \( \R \). Furthermore,
    \[
        f'(x) = \sum_{k=1}^{\infty} f_k'(x) \quand f''(x) = \sum_{k=1}^{\infty} f_k''(x).
    \]
    In particular, \( f \) is defined and continuous on \( \R \).
\end{solution}

\begin{exercise}
\label{ex:9}
    Let
    \[
        h(x) = \sum_{n=1}^{\infty} \frac{1}{x^2 + n^2}.
    \]
    \begin{enumerate}
        \item Show that \( h \) is a continuous function defined on all of \( \R \).

        \item Is \( h \) differentiable? If so, is the derivative function \( h' \) continuous?
    \end{enumerate}
\end{exercise}

\begin{solution}
    \begin{enumerate}
        \item We have the bound
        \[
            \frac{1}{x^2 + n^2} \leq \frac{1}{n^2}
        \]
        for all \( x \in \R \); the Weierstrass M-Test now implies that the series \( \sum_{n=1}^{\infty} \tfrac{1}{x^2 + n^2} \) converges uniformly on \( \R \). Since each \( \tfrac{1}{x^2 + n^2} \) is continuous on \( \R \), Theorem 6.4.2 allows us to conclude that \( h \) is also continuous on \( \R \).

        \item The term-by-term differentiated series is
        \[
            - \sum_{n=1}^{\infty} \frac{2x}{(x^2 + n^2)^2}.
        \]
        Note that
        \[
            \abs{x} \leq 1 \text{ and } n \geq 2 \quimplies \abs{\frac{2x}{(x^2 + n^2)^2}} \leq \frac{2}{n^4} \leq \frac{1}{n^2}
        \]
        and that
        \[
            \abs{x} > 1 \quimplies \abs{\frac{2x}{(x^2 + n^2)^2}} = \frac{2 \abs{x}}{x^4 + 2 x^2 n^2 + n^4} \leq \frac{1}{\abs{x} n^2} \leq \frac{1}{n^2}.
        \]
        Since the series \( \sum_{n=1}^{\infty} \tfrac{1}{n^2} \) is convergent and each summand \( \tfrac{2x}{(x^2 + n^2)^2} \) is continuous on \( \R \), the Weierstrass M-Test and Theorem 6.4.2 imply that the series \( \sum_{n=1}^{\infty} \tfrac{2x}{(x^2 + n^2)^2} \) converges uniformly on \( \R \) to a continuous function. We showed in part (a) that \( h \) converges uniformly on \( \R \) and thus by Theorem 6.4.3 we have
        \[
            h'(x) = - \sum_{n=1}^{\infty} \frac{2x}{(x^2 + n^2)^2}.
        \]
    \end{enumerate}
\end{solution}

\begin{exercise}
\label{ex:10}
    Let \( \{ r_1, r_2, r_3, \ldots \} \) be an enumeration of the set of rational numbers. For each \( r_n \in \Q \), define
    \[
        u_n(x) = \begin{cases}
            1/2^n & \text{for } x > r_n \\
            0 & \text{for } x \leq r_n.
        \end{cases}
    \]
    Now, let \( h(x) = \sum_{n=1}^{\infty} u_n(x) \). Prove that \( h \) is a monotone function defined on all of \( \R \) that is continuous at every irrational point.
\end{exercise}

\begin{solution}
    Observe that \( \abs{u_n(x)} \leq 2^{-n} \) for all \( x \in \R \). Since \( \sum_{n=1}^{\infty} 2^{-n} \) is a convergent geometric series, the Weierstrass M-Test implies that \( h \) converges uniformly on \( \R \). To see that \( h \) is strictly increasing, let \( x < y \) be given real numbers. There are a countable infinity of rational numbers contained in \( (x, y] \), which we can enumerate as a subsequence \( \{ r_{n_1}, r_{n_2}, r_{n_3}, \ldots \} \) of the sequence \( \{ r_1, r_2, r_3, \ldots \} \). It follows that
    \[
        h(y) = h(x) + \sum_{k=1}^{\infty} 2^{-n_k} > h(x).
    \]
    To see that \( h \) is continuous at every irrational point, let us first show that each \( u_n \) is continuous at every irrational point. Fix \( n \in \N, y \in \I \) (recall that \( \I \) is the set of irrational numbers), and set \( \delta := \abs{y - r_n} \); \( \delta \) must be positive since \( y \) is not rational. There are two cases:
    \begin{description}
        \item[Case 1.] If \( y < r_n \), then \( u_n(x) = 0 \) for all \( x \in (y - \delta, y + \delta) \) and hence \( u_n \) is continuous at \( y \).

        \item[Case 2.] If \( y > r_n \), then \( u_n(x) = 2^{-n} \) for all \( x \in (y - \delta, y + \delta) \) and hence \( u_n \) is continuous at \( y \).
    \end{description}
    So each summand \( u_n \) is continuous on \( \I \), and we showed earlier that \( h \) converges uniformly on \( \R \) and so in particular uniformly on \( \I \); Theorem 6.4.2 allows us to conclude that \( h \) is also continuous on \( \I \).
\end{solution}

\noindent \hrulefill

\noindent \hypertarget{ua}{\textcolor{blue}{[UA]} Abbott, S. (2015) \textit{Understanding Analysis.} 2\ts{nd} edition.}

\end{document}