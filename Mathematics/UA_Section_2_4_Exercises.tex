\documentclass[12pt]{article}
\usepackage[utf8]{inputenc}
\usepackage[utf8]{inputenc}
\usepackage{amsmath}
\usepackage{amsthm}
\usepackage{geometry}
\usepackage{amsfonts}
\usepackage{mathrsfs}
\usepackage{bm}
\usepackage{hyperref}
\usepackage[dvipsnames]{xcolor}
\usepackage{enumitem}
\usepackage{mathtools}
\usepackage{changepage}
\usepackage{lipsum}
\usepackage{tikz}
\usetikzlibrary{matrix}
\usepackage{tikz-cd}
\usepackage[nameinlink]{cleveref}
\geometry{
headheight=15pt,
left=60pt,
right=60pt
}
\usepackage{fancyhdr}
\pagestyle{fancy}
\fancyhf{}
\lhead{}
\chead{Section 2.4 Exercises}
\rhead{\thepage}
\hypersetup{
    colorlinks=true,
    linkcolor=blue,
    urlcolor=blue
}

\theoremstyle{definition}
\newtheorem*{remark}{Remark}

\newtheoremstyle{exercise}
    {}
    {}
    {}
    {}
    {\bfseries}
    {.}
    { }
    {\thmname{#1}\thmnumber{#2}\thmnote{ (#3)}}
\theoremstyle{exercise}
\newtheorem{exercise}{Exercise 2.4.}

\newtheoremstyle{solution}
    {}
    {}
    {}
    {}
    {\itshape\color{magenta}}
    {.}
    { }
    {\thmname{#1}\thmnote{ #3}}
\theoremstyle{solution}
\newtheorem*{solution}{Solution}

\Crefformat{exercise}{#2Exercise 2.4.#1#3}

\newcommand{\setcomp}[1]{#1^{\mathsf{c}}}
\newcommand{\N}{\mathbf{N}}
\newcommand{\Z}{\mathbf{Z}}
\newcommand{\Q}{\mathbf{Q}}
\newcommand{\R}{\mathbf{R}}
\newcommand{\C}{\mathbf{C}}

\DeclarePairedDelimiter\abs{\lvert}{\rvert}
% Swap the definition of \abs* and \norm*, so that \abs
% and \norm resizes the size of the brackets, and the 
% starred version does not.
\makeatletter
\let\oldabs\abs
\def\abs{\@ifstar{\oldabs}{\oldabs*}}
%
\let\oldnorm\norm
\def\norm{\@ifstar{\oldnorm}{\oldnorm*}}
\makeatother

\setlist[enumerate,1]{label={(\alph*)}}

\begin{document}

\section{Section 2.4 Exercises}

Exercises with solutions from Section 2.4 of \hyperlink{ua}{[UA]}.

\begin{exercise}
\label{ex:1}
    \begin{enumerate}
        \item Prove that the sequence defined by \( x_1 = 3 \) and
        \[
            x_{n+1} = \frac{1}{4 - x_n}
        \]
        converges.

        \item Now that we know \( \lim x_n \) exists, explain why \( \lim x_{n+1} \) must also exist and equal the same value.

        \item Take the limit of each side of the recursive equation in part (a) to explicitly compute \( \lim x_n \).
    \end{enumerate}
\end{exercise}

\begin{solution}
    \begin{enumerate}
        \item Let us use (strong) induction to show that the sequence \( (x_n) \) is decreasing, i.e.\ \( x_{n+1} \leq x_n \) for all \( n \in \N \). Since \( x_1 = 3 \) and \( x_2 = 1 \), the truth of this claim for \( n = 1 \) is clear. Suppose that \( x_{n+1} \leq x_n \leq \cdots \leq x_1 = 3 \) for some \( n \in \N \). Then observe that
        \[
            x_{n+1} \leq x_n \implies 4 - x_n \leq 4 - x_{n+1} \implies \frac{1}{4 - x_{n+1}} \leq \frac{1}{4 - x_n},
        \]
        i.e. \( x_{n+2} \leq x_{n+1} \). This completes the induction step and hence we may conclude that the sequence is decreasing.

        Now we will use induction to show that the sequence \( (x_n) \) is bounded below: we claim that \( x_n > \tfrac{1}{10} \) for all \( n \in \N \). The base case \( n = 1 \) is clear, so suppose that \( x_n > \tfrac{1}{10} \) for some \( n \in \N \). Then observe that
        \[
            x_n > \frac{1}{10} \implies x_n > -6 \implies 4 - x_n < 10 \implies \frac{1}{4 - x_n} > \frac{1}{10},
        \]
        i.e.\ \( x_{n+1} > \tfrac{1}{10} \). Our claim now follows by induction.

        We have shown that the sequence \( (x_n) \) is bounded below and decreasing, so the Monotone Convergence Theorem implies that the sequence is convergent.

        \item If \( (x_n) \) is any convergent sequence with \( \lim x_n = x \), then the sequence \( (y_n) \) given by \( y_n = x_{n + k} \) for any \( k \in \N \) is also convergent with \( \lim y_n = x \). To see this, let \( \epsilon > 0 \) be given. Then there exists an \( N \in \N \) such that \( n \geq N \implies \abs{x_n - x} < \epsilon \). Suppose \( n \geq \max \{ N - k, 1 \} \). Then \( n + k \geq N \), so
        \[
            \abs{y_n - x} = \abs{x_{n + k} - x} < \epsilon.
        \]
        It follows that \( \lim y_n = x \).

        \item By parts (a) and (b), we have \( \lim x_n = \lim x_{n+1} = x \) for some \( x \in \R \). Then by the Algebraic Limit Theorem, we have
        \[
            x_{n+1} = \frac{1}{4 - x_n} \implies \lim x_{n+1} = \frac{1}{4 - \lim x_n} \iff x = \frac{1}{4 - x} \iff x^2 - 4x + 1 = 0.
        \]
        This quadratic equation has solutions \( x = 2 \pm \sqrt{3} \). Since \( (x_n) \) is decreasing and \( x_2 = 1 \), it must be the case that \( \lim x_n = x \leq 1 \). So we may discard the solution \( x = 2 + \sqrt{3} \) and conclude that \( \lim x_n = 2 - \sqrt{3} \).
    \end{enumerate}
\end{solution}

\begin{exercise}
\label{ex:2}
    \begin{enumerate}
        \item Consider the recursively defined sequence \( y_1 = 1 \),
        \[
            y_{n+1} = 3 - y_n,
        \]
        and set \( y = \lim y_n \). Because \( (y_n) \) and \( (y_{n+1}) \) have the same limit, taking the limit across the recursive equation gives \( y = 3 - y \). Solving for \( y \), we conclude \( \lim y_n = 3/2 \).

        What is wrong with this argument?

        \item This time set \( y_1 = 1 \) and \( y_{n+1} = 3 - \frac{1}{y_n} \). Can the strategy in (a) be applied to compute the limit of this sequence?
    \end{enumerate}
\end{exercise}

\begin{solution}
    \begin{enumerate}
        \item The problem is we have assumed that \( \lim y_n \) exists. Looking at the first few terms of the sequence \( y_1 = 1, y_2 = 2, y_3 = 1, y_4 = 2, \ldots \), we see that in fact the sequence oscillates and does not converge.

        \item The strategy works this time. Let us use (strong) induction to show that the sequence is increasing. The base case \( n = 1 \) is clear since \( y_1 = 1 \) and \( y_2 = 2 \). Suppose that \( y_{n+1} \geq y_n \geq \cdots \geq y_1 = 1 \) for some \( n \in \N \). Then
        \[
            y_{n+1} \geq y_n \iff -y_{n+1} \leq -y_n \iff -\frac{1}{y_{n+1}} \geq -\frac{1}{y_n} \iff 3 - \frac{1}{y_{n+1}} \geq 3 - \frac{1}{y_n},
        \]
        i.e.\ \( y_{n+2} \geq y_{n+1} \). This completes the induction step and hence we may conclude that the sequence is increasing.

        Now let us use induction to show that the sequence \( (y_n) \) is bounded above by \( \tfrac{27}{10} \). The base case \( n = 1 \) is clear. Suppose that \( y_n \leq \tfrac{27}{10} \) for some \( n \in \N \). Then observe that
        \[
            y_n \leq \frac{27}{10} < 3 < \frac{10}{3} \implies \frac{1}{y_n} > \frac{3}{10} \implies - \frac{1}{y_n} < - \frac{3}{10} \implies 3 - \frac{1}{y_n} < \frac{27}{10},
        \]
        i.e.\ \( y_{n+1} \leq \tfrac{27}{10} \); our claim follows by induction.

        We have now shown that \( (y_n) \) is bounded above and increasing, so by the Monotone Convergence Theorem we have \( \lim y_n = y \) for some \( y \in \R \). Given this, the following manipulations are valid:
        \[
            y_{n+1} = 3 - \frac{1}{y_n} \implies y = 3 - \frac{1}{y} \implies y^2 - 3y + 1 = 0.
        \]
        This quadratic equation has solutions \( \tfrac{3}{2} \pm \tfrac{1}{2} \sqrt{5} \). Since \( (y_n) \) is increasing and \( y_2 = 2 \), we must have \( y \geq 2 \). So we may discard the solution \( y = \tfrac{3}{2} - \tfrac{1}{2} \sqrt{5} \) and conclude that \( \lim y_n = \tfrac{3}{2} + \tfrac{1}{2} \sqrt{5} \).
    \end{enumerate}
\end{solution}

\begin{exercise}
\label{ex:3}
    \begin{enumerate}
        \item Show that
        \[
            \sqrt{2}, \sqrt{2 + \sqrt{2}}, \sqrt{2 + \sqrt{2 + \sqrt{2}}}, \ldots
        \]
        converges and find the limit.

        \item Does the sequence
        \[
            \sqrt{2}, \sqrt{2 \sqrt{2}}, \sqrt{2 \sqrt{2 \sqrt{2}}}, \ldots
        \]
        converge? If so, find the limit.
    \end{enumerate}
\end{exercise}

\begin{solution}
    \begin{enumerate}
        \item Let \( x_1 = \sqrt{2} \) and
        \[
            x_{n+1} = \sqrt{2 + x_n}.
        \]
        Let us use (strong) induction to show that the sequence \( (x_n) \) is increasing. Since \( 2 + \sqrt{2} > 2 \), we have \( x_2 > x_1 \), so the base case \( n = 1 \) is satisfied. Suppose that \( x_{n+1} \geq x_n \geq \cdots \geq x_1 = \sqrt{2} \) for some \( n \in \N \). Then
        \[
            x_{n+1} \geq x_n \implies 2 + x_{n+1} \geq 2 + x_n \implies \sqrt{2 + x_{n+1}} \geq \sqrt{2 + x_n},
        \]
        i.e\ \( x_{n+2} \geq x_{n+1} \). This completes the induction step and hence we may conclude that the sequence \( (x_n) \) is increasing.
        
        Now we will use induction to show that the sequence \( (x_n) \) is bounded above by 2. The base case \( n = 1 \) is clear. Suppose that \( x_n \leq 2 \) for some \( n \in \N \). Then
        \[
            x_n \leq 2 \implies 2 + x_n \leq 4 \implies \sqrt{2 + x_n} \leq 2,
        \]
        i.e.\ \( x_{n+1} \leq 2 \). It follows by induction that \( x_n \leq 2 \) for all \( n \in \N \).

        We have shown that the sequence \( (x_n) \) is bounded above and increasing, so by the Monotone Convergence Theorem we have \( \lim x_n = x \) for some \( x \in \R \). We may now take the limit on both sides of the recursive equation:
        \[
            x_{n+1} = \sqrt{2 + x_n} \implies x = \sqrt{2 + x} \implies x^2 - x - 2 = 0 \iff (x - 2)(x + 1) = 0.
        \]
        So \( x = 2 \) or \( x = -1 \). Since the sequence is increasing and \( x_1 = \sqrt{2} \), we must have \( x \geq \sqrt{2} \). It follows that \( \lim x_n = 2 \).

        \item The sequence does converge. Let \( x_1 = \sqrt{2} \) and
        \[
            x_{n+1} = \sqrt{2 x_n}.
        \]
        Let us use (strong) induction to show that the sequence \( (x_n) \) is increasing. Since \( 2 \sqrt{2} > 2 \), we have \( x_2 > x_1 \), so the base case \( n = 1 \) is satisfied. Suppose that \( x_{n+1} \geq x_n \geq \cdots \geq x_1 = \sqrt{2} \) for some \( n \in \N \). Then
        \[
            x_{n+1} \geq x_n \implies 2 x_{n+1} \geq 2 x_n \implies \sqrt{2 x_{n+1}} \geq \sqrt{2 x_n},
        \]
        i.e\ \( x_{n+2} \geq x_{n+1} \). So by induction the sequence \( (x_n) \) is increasing.
        
        Now we will use induction to show that the sequence \( (x_n) \) is bounded above by 2. The base case \( n = 1 \) is clear. Suppose that \( x_n \leq 2 \) for some \( n \in \N \). Then
        \[
            x_n \leq 2 \implies 2 x_n \leq 4 \implies \sqrt{2 x_n} \leq 2,
        \]
        i.e.\ \( x_{n+1} \leq 2 \). This completes the induction step and hence we may conclude that \( x_n \leq 2 \) for all \( n \in \N \).

        We have shown that the sequence \( (x_n) \) is bounded above and increasing, so by the Monotone Convergence Theorem we have \( \lim x_n = x \) for some \( x \in \R \). We may now take the limit on both sides of the recursive equation:
        \[
            x_{n+1} = \sqrt{2 x_n} \implies x = \sqrt{2 x} \implies x^2 - 2x = 0 \iff x(x - 2) = 0.
        \]
        So \( x = 2 \) or \( x = 0 \). Since the sequence is increasing and \( x_1 = \sqrt{2} \), we must have \( x \geq \sqrt{2} \). It follows that \( \lim x_n = 2 \).
    \end{enumerate}
\end{solution}

\begin{exercise}
\label{ex:4}
    \begin{enumerate}
        \item In Section 1.4 we used the Axiom of Completeness (AoC) to prove the Archimedean Property of \( \R \) (Theorem 1.4.2). Show that the Monotone Convergence Theorem can also be used to prove the Archimedean Property without making any use of AoC.

        \item Use the Monotone Convergence Theorem to supply a proof for the Nested Interval Property (Theorem 1.4.1) that doesn't make use of AoC.

        These two results suggest that we could have used the Monotone Convergence Theorem in place of AoC as our starting axiom for building a proper theory of the real numbers.
    \end{enumerate}
\end{exercise}

\begin{solution}
    \begin{enumerate}
        \item Assuming that any bounded monotone sequence converges, we want to prove that for any \( x \in \R \), there exists an \( n \in \N \) satisfying \( n > x \). Part (ii) of Theorem 1.4.2 will then follow as in the proof given there. Let \( x \) be given. Seeking a contradiction, suppose that \( n \leq x \) for each \( n \in \N \). Then the sequence \( (n) \) is bounded and clearly monotone increasing, so by assumption this sequence converges, say \( \lim n = y \) for some \( y \in \R \). Then there exists \( N \in \N \) such that \( n \geq N \implies \abs{n - y} < \tfrac{1}{2} \). Observe that
        \[
            1 = \abs{N + 1 - y + y - N} \leq \abs{N + 1 - y} + \abs{N - y} < \tfrac{1}{2} + \tfrac{1}{2} = 1,
        \]
        i.e.\ \( 1 < 1 \), which is a contradiction. We may conclude that there exists some \( n \in \N \) such that \( n > x \).

        \item Assuming that any bounded monotone sequence converges, we want to prove that any sequence of nested intervals \( I_n = [a_n, b_n] \) has non-empty intersection. Consider the sequence \( (a_n) \) of left-hand endpoints of the nested intervals. This is an increasing sequence which is bounded above by any right-hand endpoint, so by assumption this sequence converges, say \( \lim a_n = x \) for some \( x \in \R \). Let \( n \in \N \) be given. Then \( a_n \leq b_n \), and \( a_n \leq a_{n+1} \) since \( (a_n) \) is increasing. The Order Limit Theorem then implies that \( x = \lim a_n \leq b_n\) and \( a_n \leq \lim a_{n+1} = x \); it follows that \( a_n \leq x \leq b_n \) for all \( n \in \N \), i.e. \( x \in \bigcap_{n=1}^{\infty} I_n \).
    \end{enumerate}
\end{solution}

\begin{exercise}[Calculating Square Roots]
\label{ex:5}
    Let \( x_1 = 2 \), and define
    \[
        x_{n+1} = \frac{1}{2} \left( x_n + \frac{2}{x_n}. \right)
    \]
    \begin{enumerate}
        \item Show that \( x_n^2 \) is always greater than or equal to 2, and then use this to prove that \( x_n - x_{n+1} \geq 0 \). Conclude that \( \lim x_n = \sqrt{2} \).

        \item Modify the sequence \( (x_n) \) so that it converges to \( \sqrt{c} \).
    \end{enumerate}
\end{exercise}

\begin{solution}
    \begin{enumerate}
        \item First, let us use induction to show that the sequence is always positive. The base case \( n = 1 \) is clear. Suppose that \( x_n > 0 \) for some \( n \in \N \). Then
        \[
            x_n > 0 \implies x_n + \frac{2}{x_n} > 0 \implies \frac{1}{2} \left( x_n + \frac{2}{x_n} \right) > 0 \iff x_{n+1} > 0.
        \]
        This completes the induction step. So \( x_n > 0 \) for all \( n \in \N \); it follows that \( x_n^2 \geq 2 \) if and only if \( x_n \geq \sqrt{2} \). We will use induction again to show that \( x_n \geq \sqrt{2} \) for all \( n \in \N \). The base case \( n = 1 \) is clear. Suppose that \( x_n \geq \sqrt{2} \) for some \( n \in \N \). Then
        \[
            (x_n - \sqrt{2})^2 \geq 0 \iff x_n^2 - 2 \sqrt{2} x_n + 2 \geq 0 \iff x_n - 2 \sqrt{2} + \frac{2}{x_n} \geq 0 \iff \frac{1}{2} \left( x_n + \frac{2}{x_n} \right) \geq \sqrt{2},
        \]
        i.e.\ \( x_{n+1} \geq \sqrt{2} \). Hence by induction we have \( x_n \geq \sqrt {2} \) for all \( n \in \N \). Given this, we have
        \[
            x_n^2 - 2 \geq 0 \iff x_n - \frac{2}{x_n} \geq 0 \iff x_n - \frac{1}{2} \left( x_n + \frac{2}{x_n} \right) \geq 0 \iff x_n - x_{n+1} \geq 0.
        \]
        It follows that the sequence \( (x_n) \) satisfies \( x_{n+1} \leq x_n \) for all \( n \in \N \).

        We have now shown that the sequence \( (x_n) \) is decreasing and bounded below. The Monotone Convergence Theorem then implies that \( \lim x_n \) exists, say \( \lim x_n = x \) for some \( x \in \R \). Given this, we are justified in taking the limit across the recursive equation:
        \[
            x_{n+1} = \frac{1}{2} \left( x_n + \frac{2}{x_n} \right) \implies x = \frac{1}{2} \left( x + \frac{2}{x} \right) \implies x^2 = 2.
        \]
        It follows that \( x = \pm \sqrt{2} \). We showed above that \( x_n \geq \sqrt{2} \) for all \( n \in \N \); the Order Limit Theorem then implies that \( x \geq \sqrt{2} \), so we may conclude that \( x = \sqrt{2} \).

        \item For \( c > 0 \), let \( x_1 = 1 + c \) and define
        \[
            x_{n+1} = \frac{1}{2} \left( x_n + \frac{c}{x_n} \right).
        \]
        Repeating the argument given in part (a), replacing 2 with \( c \) where appropriate, shows that \( \lim x_n = \sqrt{c} \). Note that \( x_1 = 1 + c \), so that \( 1 + c > 1 \implies 1 + c > \sqrt{1 + c} > \sqrt{c} \).
    \end{enumerate}
\end{solution}

\begin{exercise}[Arithmetic-Geometric Mean]
\label{ex:6}
    \begin{enumerate}
        \item Explain why \( \sqrt{xy} \leq (x + y)/2 \) for any two positive real numbers \( x \) and \( y \). (The geometric mean is always less than the arithmetic mean.)

        \item Now let \( 0 \leq x_1 \leq y_1 \) and define
        \[
            x_{n+1} = \sqrt{x_n y_n} \quad \text{and} \quad y_{n+1} = \frac{x_n + y_n}{2}.
        \]
        Show \( \lim x_n \) and \( \lim y_n \) both exist and are equal.
    \end{enumerate}
\end{exercise}

\begin{solution}
    \begin{enumerate}
        \item Observe that
        \[
            0 \leq (x - y)^2 \iff 0 \leq x^2 - 2xy + y^2 \iff 4xy \leq x^2 + 2xy + y^2 \iff 4xy \leq (x + y)^2.
        \]
        Since \( x \) and \( y \) are both positive, this implies that \( \sqrt{xy} \leq (x + y)/2 \).

        \item By part (a), we have \( x_n \leq y_n \) for all \( n \in \N \). Then
        \[
            y_{n+1} = \frac{x_n + y_n}{2} \leq \frac{y_n + y_n}{2} = y_n \quad \text{and} \quad x_{n+1} = \sqrt{x_n y_n} \geq \sqrt{x_n^2} = x_n.
        \]
        So \( (x_n) \) is increasing and \( (y_n) \) is decreasing. Furthermore, \( (y_n) \) is bounded below: for any \( n \in \N \), we have
        \[
            y_n \geq x_n \geq \cdots \geq x_1.
        \]
        Hence the Monotone Convergence Theorem implies that \( \lim y_n = y \) for some \( y \in \R \), and so the Algebraic Limit Theorem gives
        \[
            x_n = 2 y_{n+1} - y_n \implies \lim x_n = 2 \lim y_{n+1} - \lim y_n = 2 y - y = y.
        \]
    \end{enumerate}
\end{solution}

\begin{exercise}[Limit Superior]
\label{ex:7}
    Let \( (a_n) \) be a bounded sequence.
    \begin{enumerate}
        \item Prove that the sequence defined by \( y_n = \sup \{ a_k : k \geq n \} \) converges.

        \item The \textit{limit superior} of \( (a_n) \), or \( \limsup a_n \), is defined by
        \[
            \limsup a_n = \lim y_n,
        \]
        where \( y_n \) is the sequence from part (a) of this exercise. Provide a reasonable definition for \( \liminf a_n \) and briefly explain why it always exists for any bounded sequence.

        \item Prove that \( \liminf a_n \leq \limsup a_n \) for every bounded sequence, and give an example of a sequence for which the inequality is strict.

        \item Show that \( \liminf a_n = \limsup a_n \) if and only if \( \lim a_n \) exists. In this case, all three share the same value.
    \end{enumerate}
\end{exercise}

\begin{solution}
    \begin{enumerate}
        \item Suppose \( M > 0 \) is the bound for \( (a_n) \), i.e.\ \( \abs{a_n} \leq M \) for all \( n \in \N \). Then for any \( n \in \N \) we have \( y_n \geq a_n \geq -M \), so that the sequence \( (y_n) \) is bounded below. Furthermore, for any \( n \in \N \) we have
        \[
            \{ a_k : k \geq n + 1 \} \subseteq \{ a_k : k \geq n \} \implies \sup \{ a_k : k \geq n + 1 \} \leq \sup \{ a_k : k \geq n \} \iff y_{n+1} \leq y_n,
        \]
        i.e.\ the sequence \( (y_n) \) is decreasing. We may now invoke the Monotone Convergence Theorem to conclude that \( (y_n) \) converges.

        \item Let \( z_n = \inf \{ a_k : k \geq n \} \). Similarly to part (a), this sequence is bounded above and increasing and hence convergent. Then the limit inferior is defined by \( \liminf a_n = \lim z_n \).

        \item The infimum of a set is always less than or equal to the supremum of that set, so we have \( z_n \leq y_n \) for each \( n \in \N \). The Order Limit Theorem then implies that \( \lim z_n \leq \lim y_n \), i.e.\ \( \liminf a_n \leq \limsup a_n \).

        Consider the sequence \( a_n = (-1)^n \). It is not hard to see that for this sequence we have \( y_n = (1, 1, 1, \ldots) \) and \( z_n = (-1, -1, -1, \ldots ) \), so that \( \liminf a_n = -1 < 1 = \limsup a_n \).

        \item Suppose \( \liminf a_n = \limsup a_n \). Since \( z_n \leq a_n \leq y_n \) for all \( n \in \N \), the Squeeze Theorem implies that \( (a_n) \) converges and that \( \liminf a_n = \limsup a_n = \lim a_n \). Now suppose that \( \lim a_n = a \) for some \( a \in \R \). Let \( \epsilon > 0 \) be given. There is an \( N \in \N \) such that
        \[
            n \geq N \implies \abs{a_n - a} < \tfrac{\epsilon}{2} \iff a - \tfrac{\epsilon}{2} < a_n < a + \tfrac{\epsilon}{2}.
        \]
        This implies that \( a - \tfrac{\epsilon}{2} \) is a lower bound for \( \{ a_k : k \geq N \} \) and that \( a + \tfrac{\epsilon}{2} \) is an upper bound for \( \{ a_k : k \geq N \} \). It follows that \( a - \tfrac{\epsilon}{2} \leq z_N \leq a_N \leq y_N \leq a + \tfrac{\epsilon}{2} \). Since \( (z_n) \) is increasing and \( (y_n) \) is decreasing, we then have
        \[
            n \geq N \implies a - \epsilon < a - \tfrac{\epsilon}{2} \leq z_n \leq a_n \leq y_n \leq a + \tfrac{\epsilon}{2} < a + \epsilon,
        \]
        i.e.\ \( n \geq N \implies \abs{z_n - a} < \epsilon \) and \( n \geq N \implies \abs{y_n - a} < \epsilon \). It follows that \( \liminf a_n = \limsup a_n = \lim a_n = a \).
    \end{enumerate}
\end{solution}

\begin{exercise}
\label{ex:8}
    For each series, find an explicit formula for the sequence of partial sums and determine if the series converges.
    \[
        \text{(a) } \sum_{n=1}^{\infty} \frac{1}{2^n} \qquad \text{(b) } \sum_{n=1}^{\infty} \frac{1}{n(n+1)} \qquad \text{(c) } \sum_{n=1}^{\infty} \log \left( \frac{n+1}{n} \right)
    \]
    (In (c), \( \log(x) \) refers to the natural logarithm function from calculus.)
\end{exercise}

\begin{solution}
    For each series, let \( (s_m) \) be its sequence of partial sums.
    \begin{enumerate}
        \item Here we have
        \begin{align*}
            s_m &= \frac{1}{2} + \frac{1}{2^2} + \cdots + \frac{1}{2^{m-1}} + \frac{1}{2^m} \\[6pt]
            \implies 2 s_m &= 1 + \frac{1}{2} + \cdots + \frac{1}{2^m} + \frac{1}{2^{m+1}} \\[6pt]
            \implies 2 s_m &= \frac{1 - (1/2)^{m+2}}{1 - (1/2)} \\[6pt]
            \implies s_m &= 1 - \frac{1}{2^{m+2}},
        \end{align*}
        where we have used the formula \( (1 - x)(1 + x + x^2 + \cdots + x^n) = 1 - x^{n+1} \). The set \( \{ 2^k : k \in \N \} \) is unbounded in \( \R \), so we have \( \lim s_m = 1 - \lim \left( \frac{1}{2^{m+2}} \right) = 1 - 0 = 1. \)

        \item For this series,
        \begin{multline*}
            s_m = \sum_{n=1}^m \frac{1}{n(n+1)} = \sum_{n=1}^m \left( \frac{1}{n} - \frac{1}{n+1} \right) \\ = \left( 1 - \frac{1}{2} \right) + \left( \frac{1}{2} - \frac{1}{3} \right) + \cdots + \left( \frac{1}{m} - \frac{1}{m+1} \right) = 1 - \frac{1}{m+1}.
        \end{multline*}
        It follows that \( \lim s_m = 1 \).

        \item We have
        \begin{multline*}
            s_m = \sum_{n=1}^m \log \left( \frac{n+1}{n} \right) = \sum_{n=1}^m (\log(n+1) - \log(n)) \\ = (\log(2) - \log(1)) + (\log(3) - \log(2)) + \cdots + (\log(m+1) - \log(m)) = \log(m+1).
        \end{multline*}
        So \( s_m = \log(m+1) \), which is unbounded and hence not convergent.
    \end{enumerate}
\end{solution}

\begin{exercise}
\label{ex:9}
    Complete the proof of Theorem 2.4.6 by showing that if the series \( \sum_{n=0}^{\infty} 2^n b_{2^n} \) diverges, then so does \( \sum_{n=1}^{\infty} b_n \). Example 2.4.5 may be a useful reference.
\end{exercise}

\begin{solution}
    Define the sequences of partial sums
    \[
        s_m = b_1 + b_2 + \cdots + b_m \qquad \text{and} \qquad t_m = b_1 + 2 b_2 + \cdots + 2^m b_{2^m}.
    \]
    Since each \( b_n \) is non-negative, both sequences of partial sums are non-decreasing. Then by the Monotone Convergence Theorem, the convergence of each series is equivalent to the boundedness of the respective sequence of partial sums. Given this, we want to show that if \( (t_m) \) is unbounded, then so is \( (s_m) \). Suppose therefore that \( (t_m) \) is unbounded and let \( M > 0 \) be given. There exists an \( m \in \N \) such that \( t_m \geq 2M \). Observe that
    \begin{align*}
        t_m &= b_1 + 2 b_2 + 4 b_4 + \cdots + 2^m b_{2^m} \\
        &= (b_1 + b_2) + (b_2 + b_4 + b_4 + b_4) + \cdots + (b_{2^{m-1}} + b_{2^m} + \cdots + b_{2^m} + b_{2^m}) + b_{2^m} \\
        &\leq (b_1 + b_1) + (b_2 + b_2 + b_3 + b_3) + \cdots + (b_{2^{m-1}} + b_{2^{m-1}} + \cdots + b_{2^m - 1} + b_{2^m - 1}) + b_{2^m} + b_{2^m} \\
        &= 2 s_{2^m}.
    \end{align*}
    It follows that \( s_{2^m} \geq M \) and hence that \( (s_m) \) is unbounded.
\end{solution}

\begin{exercise}[Infinite Products]
\label{ex:10}
    A close relative of infinite series is the \textit{infinite product}
    \[
        \prod_{n=1}^{\infty} b_n = b_1 b_2 b_3 \cdots
    \]
    which is understood in terms of its sequence of \textit{partial products}
    \[
        p_m = \prod_{n=1}^m b_n = b_1 b_2 b_3 \cdots b_m.
    \]
    Consider the special class of infinite products of the form
    \[
        \prod_{n=1}^{\infty} (1 + a_n) = (1 + a_1)(1 + a_2)(1 + a_3) \cdots, \quad \text{where } a_n \geq 0.
    \]
    \begin{enumerate}
        \item Find an explicit formula for the sequence of partial products in the case where \( a_n = 1/n \) and decide whether the sequence converges. Write out the first few terms in the sequence of partial products in the case where \( a_n = 1/n^2 \) and make a conjecture about the convergence of this sequence.

        \item Show, in general, that the sequence of partial products converges if and only if \( \sum_{n=1}^{\infty} a_n \) converges. (The inequality \( 1 + x \leq 3^x \) for positive \( x \) will be useful in one direction.)
    \end{enumerate}
\end{exercise}

\begin{solution}
    \begin{enumerate}
        \item We will use induction to show that \( p_m = m + 1 \). The base case \( m = 1 \) is clear. Suppose \( p_m = m + 1 \) for some \( m \in \N \). Then
        \[
            p_{m+1} = p_m \left( 1 + \frac{1}{m+1} \right) = (m + 1) \frac{m+2}{m+1} = m + 2.
        \]
        This completes the induction step and we may conclude that \( p_m = m + 1 \) for all \( m \in \N \); it follows that the sequence \( (p_m) \) does not converge. For \( a_n = 1/n^2 \), the first few partial products are
        \begin{align*}
            p_1 &= 2, \\
            p_2 &= 2(1 + 1/4) = 5/2 = 2.5, \\
            p_3 &= (5/2)(1 + 1/9) = 25/9 \approx 2.778, \\
            p_4  &= (25/9)(1 + 1/16) = 425/144 \approx 2.951.
        \end{align*}
        It looks like the partial products could be bounded; we conjecture that this infinite product converges.

        \item Let \( s_m = \sum_{n=1}^m a_n \). Since \( a_n \geq 0 \) for all \( n \in \N \), the sequence of partial sums and the sequence of partial products are both non-negative and increasing. Then by the Monotone Convergence Theorem, the convergence of each sequence is equivalent to the boundedness of that sequence. By multiplying out the terms in the partial product \( p_m \), we would obtain the sum of \( s_m \) and some other non-negative terms; it follows that \( s_m \leq p_m \). The hint gives us
        \[
            p_m = \prod_{n=1}^m (1 + a_n) \leq \prod_{n=1}^m 3^{a_n} = 3^{\sum_{n=1}^m a_n} = 3^{s_m}.
        \]
        So we have the inequalities \( s_m \leq p_m \leq 3^{s_m} \). It is then clear that \( (s_m) \) is bounded if \( (p_m) \) is bounded, and furthermore if \( (s_m) \) is bounded by some \( M > 0 \), then \( (p_m) \) is bounded by \( 3^M \). Hence for this special case, \( \prod_{n=1}^{\infty} (1 + a_n) \) converges if and only if \( \sum_{n=1}^{\infty} a_n \) converges.
    \end{enumerate}
\end{solution}

\noindent \hrulefill

\noindent \hypertarget{ua}{\textcolor{blue}{[UA]} Abbott, S. (2015) \textit{Understanding Analysis.} 2nd edn.}

\end{document}
