\documentclass[12pt]{article}
\usepackage[utf8]{inputenc}
\usepackage[utf8]{inputenc}
\usepackage{amsmath}
\usepackage{amsthm}
\usepackage{geometry}
\usepackage{amsfonts}
\usepackage{mathrsfs}
\usepackage{bm}
\usepackage{hyperref}
\usepackage[dvipsnames]{xcolor}
\usepackage{enumitem}
\usepackage{changepage}
\usepackage{lipsum}
\usepackage{tikz}
\usetikzlibrary{matrix}
\usepackage{tikz-cd}
\usepackage[nameinlink]{cleveref}
\geometry{
headheight=15pt,
left=60pt,
right=60pt
}
\usepackage{fancyhdr}
\pagestyle{fancy}
\fancyhf{}
\lhead{}
\chead{Section 1.4 Exercises}
\rhead{\thepage}
\hypersetup{
    colorlinks=true,
    linkcolor=blue,
    urlcolor=blue
}

\theoremstyle{definition}

\newtheorem*{remark}{Remark}

\newtheoremstyle{exercise}
    {}
    {}
    {}
    {}
    {\bfseries}
    {.}
    { }
    {\thmname{#1}\thmnumber{#2}\thmnote{ (#3)}}
\theoremstyle{exercise}
\newtheorem{exercise}{Exercise 1.4.}

\newtheoremstyle{solution}
    {}
    {}
    {}
    {}
    {\itshape\color{magenta}}
    {.}
    { }
    {\thmname{#1}\thmnote{ #3}}
\theoremstyle{solution}
\newtheorem*{solution}{Solution}

\Crefformat{exercise}{#2Exercise 1.4.#1#3}

\newcommand{\setcomp}[1]{#1^{\mathsf{c}}}
\newcommand{\N}{\mathbf{N}}
\newcommand{\Z}{\mathbf{Z}}
\newcommand{\Q}{\mathbf{Q}}
\newcommand{\R}{\mathbf{R}}
\newcommand{\C}{\mathbf{C}}

\setlist[enumerate,1]{label={(\alph*)}}

\begin{document}

\section{Section 1.4 Exercises}

Exercises with solutions from Section 1.4 of \hyperlink{ua}{[UA]}.

\begin{exercise}
\label{ex:1}
    Recall that \( \mathbf{I} \) stands for the set of irrational numbers.
    \begin{enumerate}
        \item Show that if \( a, b \in \Q \), then \( ab \) and \( a + b \) are elements of \( \Q \) as well.

        \item Show that if \( a \in \Q \) and \( t \in \mathbf{I} \), then \( a + t \in \mathbf{I} \) and \( at \in \mathbf{I} \) as long as \( a \neq 0 \).

        \item Part (a) can be summarized by saying that \( \Q \) is closed under addition and multiplication. Is \( \mathbf{I} \) closed under addition and multiplication? Given two irrational numbers \( s \) and \( t \), what can we say about \( s + t \) and \( st \)?
    \end{enumerate}
\end{exercise}

\begin{solution}
    \begin{enumerate}
        \item Suppose \( a = \tfrac{k}{l} \) and \( b = \tfrac{m}{n} \). Then
        \[
            ab = \tfrac{km}{ln} \quad \text{and} \quad a + b = \tfrac{kn + lm}{ln},
        \]
        which are rational numbers.

        \item Let \( a \in \Q \) be fixed. We want to prove that
        \[
            t \in \mathbf{I} \implies a + t \in \mathbf{I}.
        \]
        To do this, we will prove the contrapositive statement
        \[
            a + t \in \Q \implies t \in \Q.
        \]
        Simply observe that \( t = (a + t) - a \). Then by part (a), \( t \in \Q \).

        Similarly, let \( a \in \Q \) be non-zero. We can show that
        \[
            at \in \Q \implies t \in \Q
        \]
        by observing that \( t = a^{-1}(at) \) and appealing to part (a) to conclude that \( t \in \Q \).

        \item \( \mathbf{I} \) is not closed under addition or multiplication. For example, \( -\sqrt{2} \) and \( \sqrt{2} \) are irrational numbers, but their sum is the rational number 0 and their product is the rational number \( -2 \). The sum or product of two irrational numbers may be irrational; for example, it can be shown that \( \sqrt{2} + \sqrt{3} \) and \( \sqrt{2} \sqrt{3} \) are irrational. So in general, we cannot say anything about the sum or product of two irrational numbers without more information.
    \end{enumerate}
\end{solution}

\begin{exercise}
\label{ex:2}
    Let \( A \subseteq \R \) be nonempty and bounded above, and let \( s \in \R \) have the property that for all \( n \in \N \), \( s + \tfrac{1}{n} \) is an upper bound for \( A \) and \( s - \tfrac{1}{n} \) is not an upper bound for \( A \). Show \( s = \sup A \).
\end{exercise}

\begin{solution}
    First, let us show that \( s \) is an upper bound for \( A \). Seeking a contradiction, suppose this is not the case. Then there must exist some \( x \in A \) such that \( s < x \). By the Archimedean property of \( \R \), there exists a natural number \( n \) such that \( \tfrac{1}{n} < x - s \iff s + \tfrac{1}{n} < x \). This implies that \( s + \tfrac{1}{n} \) is not an upper bound for \( A \), which contradicts our hypotheses. Hence it must be that \( s \) is an upper bound for \( A \).

    Now let \( \epsilon > 0 \) be given and using the Archimedean property, pick a natural number \( n \) such that \( \tfrac{1}{n} < \epsilon \). By assumption, \( s - \tfrac{1}{n} \) is not an upper bound for \( A \), so there must exist some \( x \in A \) such that \( s - \tfrac{1}{n} < x \); this implies that \( s - \epsilon < x \) since \( \tfrac{1}{n} < \epsilon \). Hence by Lemma 1.3.8, \( s = \sup A \).
\end{solution}

\begin{exercise}
\label{ex:3}
    Prove that \( \bigcap_{n=1}^{\infty} (0, 1/n) = \emptyset \). Notice that this demonstrates that the intervals in the Nested Interval Property must be closed for the conclusion of the theorem to hold.
\end{exercise}

\begin{solution}
    Suppose there was some \( x \in \bigcap_{n=1}^{\infty} (0, 1/n) \), i.e.\ some \( x \in \R \) such that for all \( n \in \N \), \( 0 < x < 1/n \). This directly contradicts the Archimedean property of \( \R \), which says that there must exist an \( N \in \N \) such that \( 1/N < x \). We may conclude that \( \bigcap_{n=1}^{\infty} (0, 1/n) = \emptyset \).
\end{solution}

\begin{exercise}
\label{ex:4}
    Let \( a < b \) be real numbers and consider the set \( T = \Q \cap [a, b] \). Show \( \sup T = b \).
\end{exercise}

\begin{solution}
    It is clear that \( b \) is an upper bound for \( T \). Let \( \epsilon > 0 \) be given. By the density of \( \Q \) in \( \R \), there exists a rational number \( p \) satisfying \( b - \epsilon < p < b \) and a rational number \( q \) satisfying \( a < q < b \). Let \( r = \max \{p , q\} \). Then \( a < r < b \), so that \( r \in T \), and \( b - \epsilon < r \). Hence by Lemma 1.3.8, \( \sup T = b \).
\end{solution}

\begin{exercise}
\label{ex:5}
    Using \Cref{ex:1}, supply a proof for Corollary 1.4.4 by considering the real numbers \( a - \sqrt{2} \) and \( b - \sqrt{2} \).
\end{exercise}

\begin{solution}
    By the density of \( \Q \) in \( \R \), there exists a rational number \( p \) satisfying \( a - \sqrt{2} < p < b - \sqrt{2} \), which gives \( a < p + \sqrt{2} < b \). Since \( p + \sqrt{2} \) is irrational by \Cref{ex:1}, the corollary is proved.
\end{solution}

\begin{exercise}
\label{ex:6}
    Recall that a set \( B \) is \textit{dense} in \( \R \) if an element of \( B \) can be found between any two real numbers \( a < b \). Which of the following sets are dense in \( \R \)? Take \( p \in \Z \) and \( q \in \N \) in every case.
    \begin{enumerate}
        \item The set of all rational numbers \( p/q \) with \( q \leq 10 \).

        \item The set of all rational numbers \( p/q \) with \( q \) a power of 2.

        \item The set of all rational numbers \( p/q \) with \( 10|p| \geq q \).
    \end{enumerate}
\end{exercise}

\begin{solution}
    \begin{enumerate}
        \item This set is not dense in \( \R \). Observe that
        \[
            q \leq 10 \iff \tfrac{1}{q} \geq \tfrac{1}{10}.
        \]
        Then if \( p > 0 \) we have \( p/q \geq 1/10 \), if \( p < 0 \) we have \( p/q \leq -1/10 \), and of course if \( p = 0 \) we have \( p/q = 0 \). So there is no element of this set between the real numbers \( 1/1000 \) and \( 1/100 \), for example.

        \item This set is dense in \( \R \). To see this, let us first prove that for all \( x \in \R \), there exists a \( k \in \N \) such that \( 2^k > x \). Seeking a contradiction, suppose this is not the case. Then the set \( K = \{ 2^k : k \in \N \} \) is non-empty and bounded above, so by the Axiom of Completeness \( \alpha := \sup K \) exists in \( \R \). Since \( 2^1 = 2 \) belongs to \( K \), it must be the case that \( \alpha \) is positive. It follows that \( \tfrac{\alpha}{2} < \alpha \), so that \( \tfrac{\alpha}{2} \) cannot be an upper bound for \( K \). Then there exists some \( k \in \N \) such that \( \tfrac{\alpha}{2} < 2^k \), which implies \( \alpha < 2^{k+1} \); but this contradicts the fact that \( \alpha \) is the supremum of \( K \). Hence it must be the case that for all \( x \in \R \), there exists a \( k \in \N \) such that \( 2^k > x \).

        Let \( a < b \) be real numbers. By the previous paragraph, there exists a \( k \in \N \) such that \( 2^k > \tfrac{1}{b-a} \), which implies that \( \tfrac{1}{2^k} < b - a \). Now let \( p \) be the smallest integer greater than \( 2^k a \), so that \( p - 1 \leq 2^k a < p \). Then observe that
        \[
            2^k a < p \leq 1 + 2^k a < 2^k b \implies a < \tfrac{p}{2^k} < b.
        \]

        \item This set is not dense in \( \R \). If \( p > 0 \) then
        \[
            10|p| \geq q \iff 10p \geq q \iff p/q \geq 1/10,
        \]
        and if \( p < 0 \) then
        \[
            10|p| \geq q \iff -10p \geq q \iff p/q \leq -1/10.
        \]
        We cannot have \( p = 0 \) since \( q \) is a positive integer. So as in part (a), there is no element of this set between the real numbers \( 1/1000 \) and \( 1/100 \), for example.
    \end{enumerate}
\end{solution}

\begin{exercise}
\label{ex:7}
    Finish the proof of Theorem 1.4.5 by showing that the assumption \( \alpha^2 > 2 \) leads to a contradiction of the fact that \( \alpha = \sup T \).
\end{exercise}

\begin{solution}
    By the Archimedean property of \( \R \), there exists an \( n \in \N \) such that \( \tfrac{2 \alpha}{n} < \alpha^2 - 2 \iff 2 < \alpha^2 - \tfrac{2 \alpha}{n} \). Let \( b = \alpha - \tfrac{1}{n} \). Note that since \( 1 \in T \), we have \( \alpha \geq 1 \) and hence \( b \geq 0 \). So if \( t \in T \) is such that \( t < 0 \) then \( t \leq b \). Now observe that
    \[
        b^2 = \left( \alpha - \tfrac{1}{n} \right)^2 = \alpha^2 - \tfrac{2 \alpha}{n} + \tfrac{1}{n^2} > \alpha^2 - \tfrac{2 \alpha}{n} > 2,
    \]
    so that for any \( t \in T \) we have \( t^2 < 2 < b^2 \). It follows that if \( t \geq 0 \), we have \( t \leq b \). We have now shown that \( t \leq b \) for all \( t \in T \), i.e.\ \( b \) is an upper bound for \( T \); but this contradicts the fact that \( \alpha \) is the supremum of \( T \) since \( b < \alpha \).
\end{solution}

\begin{exercise}
\label{ex:8}
    Give an example of each or state that the request is impossible. When a request is impossible, provide a compelling argument for why this is the case.
    \begin{enumerate}
        \item Two sets \( A \) and \( B \) with \( A \cap B = \emptyset, \sup A = \sup B, \sup A \not\in A \) and \( \sup B \not\in B \).

        \item A sequence of nested open intervals \( J_1 \supseteq J_2 \supseteq J_3 \supseteq \cdots \) with \( \bigcap_{n=1}^{\infty} J_n \) nonempty but containing only a finite number of elements.

        \item A sequence of nested unbounded closed intervals \( L_1 \supseteq L_2 \supseteq L_3 \supseteq \cdots \) with \( \bigcap_{n=1}^{\infty} L_n = \emptyset \). (An unbounded closed interval has the form \( [a, \infty) = \{ x \in \R : x \geq a \} \).)

        \item A sequence of closed bounded (not necessarily nested) intervals \( I_1, I_2, I_3, \ldots \) with the property that \( \bigcap_{n=1}^N I_n \neq \emptyset \) for all \( N \in \N \), but \( \bigcap_{n=1}^{\infty} I_n = \emptyset \).
    \end{enumerate}
\end{exercise}

\begin{solution}
    \begin{enumerate}
        \item Take \( A = \left\{ -\tfrac{1}{2n} : n \in \N \right\} \) and \( B = \left\{ -\tfrac{1}{2n - 1} : n \in \N \right\} \). Then \( A \cap B = \emptyset, \sup A = \sup B = 0 \) and \( 0 \) belongs to neither \( A \) nor \( B \).

        \item Take \( J_n = (-1/n, 1/n) \). Then \( \bigcap_{n=1}^{\infty} J_n = \{ 0 \} \).

        \item Take \( L_n = [n, \infty) \).

        \item This is impossible. To see this, let \( J_N = \bigcap_{n=1}^N I_n \) for \( N \in \N \) and note that any finite intersection of closed bounded intervals is a (possibly empty) closed bounded interval. So: each \( J_N \) is a closed bounded interval; these intervals are nonempty and nested \( J_1 \supseteq J_2 \supseteq J_3 \supseteq \cdots \); and \( \bigcap_{n=1}^{\infty} I_n = \bigcap_{N=1}^{\infty} J_N \). Then by the Nested Interval Property of \( \R \), we must have that \( \bigcap_{n=1}^{\infty} I_n = \bigcap_{N=1}^{\infty} J_N \) is non-empty.
    \end{enumerate}
\end{solution}

\noindent \hrulefill

\noindent \hypertarget{ua}{\textcolor{blue}{[UA]} Abbott, S. (2015) \textit{Understanding Analysis.} 2nd edn.}

\end{document}
