\documentclass[12pt]{article}
\usepackage[utf8]{inputenc}
\usepackage[utf8]{inputenc}
\usepackage{amsmath}
\usepackage{amsthm}
\usepackage{geometry}
\usepackage{amsfonts}
\usepackage{mathrsfs}
\usepackage{bm}
\usepackage{hyperref}
\usepackage[dvipsnames]{xcolor}
\usepackage{enumitem}
\usepackage{mathtools}
\usepackage{changepage}
\usepackage{lipsum}
\usepackage{tikz}
\usetikzlibrary{matrix}
\usepackage{tikz-cd}
\usepackage[nameinlink]{cleveref}
\geometry{
headheight=15pt,
left=60pt,
right=60pt
}
\usepackage{fancyhdr}
\pagestyle{fancy}
\fancyhf{}
\lhead{}
\chead{Section 1.B Exercises}
\rhead{\thepage}
\hypersetup{
    colorlinks=true,
    linkcolor=blue,
    urlcolor=blue
}

\theoremstyle{definition}
\newtheorem*{remark}{Remark}

\newtheoremstyle{exercise}
    {}
    {}
    {}
    {}
    {\bfseries}
    {.}
    { }
    {\thmname{#1}\thmnumber{#2}\thmnote{ (#3)}}
\theoremstyle{exercise}
\newtheorem{exercise}{Exercise 1.B.}

\newtheoremstyle{solution}
    {}
    {}
    {}
    {}
    {\itshape\color{magenta}}
    {.}
    { }
    {\thmname{#1}\thmnote{ #3}}
\theoremstyle{solution}
\newtheorem*{solution}{Solution}

\Crefformat{exercise}{#2Exercise 1.B.#1#3}

\newcommand{\setcomp}[1]{#1^{\mathsf{c}}}
\newcommand{\N}{\mathbf{N}}
\newcommand{\Z}{\mathbf{Z}}
\newcommand{\Q}{\mathbf{Q}}
\newcommand{\R}{\mathbf{R}}
\newcommand{\C}{\mathbf{C}}
\newcommand{\F}{\mathbf{F}}

\DeclarePairedDelimiter\abs{\lvert}{\rvert}
% Swap the definition of \abs* and \norm*, so that \abs
% and \norm resizes the size of the brackets, and the 
% starred version does not.
\makeatletter
\let\oldabs\abs
\def\abs{\@ifstar{\oldabs}{\oldabs*}}
%
\let\oldnorm\norm
\def\norm{\@ifstar{\oldnorm}{\oldnorm*}}
\makeatother

\setlist[enumerate,1]{label={(\alph*)}}

\begin{document}

\section{Section 1.B Exercises}

Exercises with solutions from Section 1.B of \hyperlink{ladr}{[LADR]}.

\begin{exercise}
\label{ex:1}
    Prove that \( -(-v) = v \) for every \( v \in V \).
\end{exercise}

\begin{solution}
    Since \( v + (-v) = 0 \) and additive inverses are unique (1.26), it must be the case that \( -(-v) = v \).
\end{solution}

\begin{exercise}
\label{ex:2}
    Suppose \( a \in \F, v \in V \), and \( av = 0 \). Prove that \( a = 0 \) or \( v = 0 \).
\end{exercise}

\begin{solution}
    It will suffice to show that if \( av = 0 \) and \( a \neq 0 \), then \( v = 0 \). Since \( a \neq 0 \), \( a^{-1} \) exists. Then
    \[
        av = 0 \implies a^{-1}(av) = a^{-1} 0 \implies (a^{-1} a)v = 0 \implies 1v = 0 \implies v = 0.
    \]
\end{solution}

\begin{exercise}
\label{ex:3}
    Suppose \( v, w \in V \). Explain why there exists a unique \( x \in V \) such that \( v + 3x = w \).
\end{exercise}

\begin{solution}
    If \( v, w, x \in V \) are such that \( v + 3x = w \), then
    \[
        v + 3x = w \implies 3x = w - v \implies x = \tfrac{1}{3}(w - v);
    \]
    conversely, it is easily verified that \( x = \tfrac{1}{3}(w - v) \) satisfies the equation \( v + 3x = w \).
\end{solution}

\begin{exercise}
\label{ex:4}
    The empty set is not a vector space. The empty set fails to satisfy only one of the requirements listed in 1.19. Which one?
\end{exercise}

\begin{solution}
    The empty set does not contain an additive identity.
\end{solution}

\begin{exercise}
\label{ex:5}
    Show that in the definition of a vector space (1.19), the additive inverse condition can be replaced with the condition that
    \[
        0v = 0 \text{ for all } v \in V.
    \]
    Here the \( 0 \) on the left side is the number \( 0 \), and the \( 0 \) on the right side is the additive identity of \( V \). (The phrase ``a condition can be replaced'' in a definition means that the collection of objects satisfying the definition is unchanged if the original condition is replaced with the new condition.)
\end{exercise}

\begin{solution}
    If \( V \) satisfies all of the conditions in (1.19), then as shown in (1.26) we have \( 0v = 0 \) for all \( v \in V \). Now suppose that \( V \) satisfies all of the conditions in (1.19), except we have replaced the additive inverse condition with the condition that \( 0v = 0 \) for all \( v \in V \). We want to show that for each \( v \in V \), there exists an element \( w \) such that \( v + w = 0 \). Let \( v \in V \) be given and set \( w = (-1)v \). Then
    \[
        v + w = 1v + (-1)v = (1 - 1)v = 0v = 0.
    \]
\end{solution}

\begin{exercise}
\label{ex:6}
    Let \( \infty \) and \( -\infty \) denote two distinct objects, neither of which is in \( \R \). Define an addition and scalar multiplication on \( \R \cup \{ \infty \} \cup \{ -\infty \} \) as you could guess from the notation. Specifically, the sum and product of two real numbers is as usual, and for \( t \in \R \) define
    \begin{gather*}
        t \, \infty = \begin{cases}
            -\infty & \text{if } t < 0, \\
            0 & \text{if } t = 0, \\
            \infty & \text{if } t > 0,
        \end{cases}
        \qquad
        t (-\infty) = \begin{cases}
            \infty & \text{if } t < 0, \\
            0 & \text{if } t = 0, \\
            -\infty & \text{if } t > 0,
        \end{cases} \\
        t + \infty = \infty + t = \infty, \qquad t + (-\infty) = (-\infty) + t = -\infty, \\
        \infty + \infty = \infty, \quad (-\infty) + (-\infty) = -\infty, \quad \infty + (-\infty) = 0.
    \end{gather*}
    Is \( \R \cup \{ \infty \} \cup \{ -\infty \} \) a vector space over \( \R \)? Explain.
\end{exercise}

\begin{solution}
    This is not a vector space. By (1.25), the additive identity in a vector space must be unique. However, we have \( \infty + t = \infty \) for all \( t \in \R \).
\end{solution}

\noindent \hrulefill

\noindent \hypertarget{ladr}{\textcolor{blue}{[LADR]} Axler, S. (2015) \textit{Linear Algebra Done Right.} 3rd edn.}

\end{document}