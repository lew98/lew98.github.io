\documentclass[12pt]{article}
\usepackage[utf8]{inputenc}
\usepackage[utf8]{inputenc}
\usepackage{amsmath}
\usepackage{amsthm}
\usepackage{geometry}
\usepackage{amsfonts}
\usepackage{mathrsfs}
\usepackage{bm}
\usepackage{hyperref}
\usepackage[dvipsnames]{xcolor}
\usepackage{enumitem}
\usepackage{mathtools}
\usepackage{changepage}
\usepackage{lipsum}
\usepackage{tikz}
\usetikzlibrary{matrix}
\usepackage{tikz-cd}
\usepackage[nameinlink]{cleveref}
\geometry{
headheight=15pt,
left=60pt,
right=60pt
}
\setlength{\emergencystretch}{20pt}
\usepackage{fancyhdr}
\pagestyle{fancy}
\fancyhf{}
\lhead{}
\chead{Section 3.B Exercises}
\rhead{\thepage}
\hypersetup{
    colorlinks=true,
    linkcolor=blue,
    urlcolor=blue
}

\theoremstyle{definition}
\newtheorem*{remark}{Remark}

\newtheoremstyle{exercise}
    {}
    {}
    {}
    {}
    {\bfseries}
    {.}
    { }
    {\thmname{#1}\thmnumber{#2}\thmnote{ (#3)}}
\theoremstyle{exercise}
\newtheorem{exercise}{Exercise 3.B.}

\newtheoremstyle{solution}
    {}
    {}
    {}
    {}
    {\itshape\color{magenta}}
    {.}
    { }
    {\thmname{#1}\thmnote{ #3}}
\theoremstyle{solution}
\newtheorem*{solution}{Solution}

\Crefformat{exercise}{#2Exercise 3.B.#1#3}

\newcommand{\poly}{\mathcal{P}}
\newcommand{\lmap}{\mathcal{L}}
\newcommand{\ts}{\textsuperscript}
\newcommand{\Span}{\text{span}}
\newcommand{\Null}{\text{null\,}}
\newcommand{\Range}{\text{range\,}}
\newcommand{\quand}{\quad \text{and} \quad}
\newcommand{\setcomp}[1]{#1^{\mathsf{c}}}
\newcommand{\N}{\mathbf{N}}
\newcommand{\Z}{\mathbf{Z}}
\newcommand{\Q}{\mathbf{Q}}
\newcommand{\R}{\mathbf{R}}
\newcommand{\C}{\mathbf{C}}
\newcommand{\F}{\mathbf{F}}

\DeclarePairedDelimiter\abs{\lvert}{\rvert}
% Swap the definition of \abs* and \norm*, so that \abs
% and \norm resizes the size of the brackets, and the 
% starred version does not.
\makeatletter
\let\oldabs\abs
\def\abs{\@ifstar{\oldabs}{\oldabs*}}
%
\let\oldnorm\norm
\def\norm{\@ifstar{\oldnorm}{\oldnorm*}}
\makeatother

\setlist[enumerate,1]{label={(\alph*)}}

\begin{document}

\section{Section 3.B Exercises}

Exercises with solutions from Section 3.B of \hyperlink{ladr}{[LADR]}.

\begin{exercise}
\label{ex:1}
    Give an example of a linear map \( T \) such that \( \dim \Null T = 3 \) and \( \dim \Range T = 2 \).
\end{exercise}

\begin{solution}
    Let \( T : \R^5 \to \R^2 \) be given by
    \[
        T(x_1, x_2, x_3, x_4, x_5) = (x_1, x_2).
    \]
    Then
    \[
        \Null T = \{ (0, 0, x_3, x_4, x_5) \in \R^5 : x_3, x_4, x_5 \in \R \} \quand \Range T = \R^2.
    \]
    Thus \( \dim \Null T = 3 \) and \( \dim \Range T = 2 \).
\end{solution}

\begin{exercise}
\label{ex:2}
    Suppose \( V \) is a vector space and \( S, T \in \lmap(V, V) \) are such that
    \[
        \Range S \subset \Null T.
    \]
    Prove that \( (ST)^2 = 0 \).
\end{exercise}

\begin{solution}
    Let \( v \in V \) be given. Then \( S(Tv) \in \Range S \subseteq \Null T \), so \( T(S(Tv)) = 0 \). It follows that
    \[
        (ST)^2 (v) = S(T(S(Tv))) = S(0) = 0.
    \]
    Thus \( (ST)^2 = 0 \).
\end{solution}

\begin{exercise}
\label{ex:3}
    Suppose \( v_1, \ldots, v_m \) is a list of vectors in \( V \). Define \( T \in \lmap(\F^m, V) \) by
    \[
        T(z_1, \ldots, z_m) = z_1 v_1 + \cdots + z_m v_m.
    \]
    \begin{enumerate}
        \item What property of \( T \) corresponds to \( v_1, \ldots, v_m \) spanning \( V \)?

        \item What property of \( T \) corresponds to \( v_1, \ldots, v_m \) being linearly independent?
    \end{enumerate}
\end{exercise}

\begin{solution}
    \begin{enumerate}
        \item The surjectivity of \( T \) corresponds to \( v_1, \ldots, v_m \) spanning \( V \), i.e.\ \( v_1, \ldots, v_m \) spans \( V \) if and only if \( T \) is surjective. To see this, observe that \( T \) is surjective if and only if for every \( v \in V \) there exists \( (z_1, \ldots, z_m) \in \F^m \) such that \( T(z_1, \ldots, z_m) = z_1 v_1 + \cdots + z_m v_m = v \). This is the case if and only if \( V = \Span(v_1, \ldots, v_m) \).

        \item The injectivity of \( T \) corresponds to \( v_1, \ldots, v_m \) being linearly independent, i.e.\ \( v_1, \ldots, v_m \) is linearly independent if and only if \( T \) is injective. To see this, observe that by 3.16, \( T \) is injective if and only if \( \Null T = \{ 0 \} \), i.e.\ if and only if the only choice of \( (z_1, \ldots, z_m) \in \F^m \) which gives \( z_1 v_1 + \cdots + z_m v_m = 0 \) is \( (0, \ldots, 0) \); this is the definition of linear independence.
    \end{enumerate}
\end{solution}

\begin{exercise}
\label{ex:4}
    Show that
    \[
        \{ T \in \lmap(\R^5, \R^4) : \dim \Null T > 2 \}
    \]
    is not a subspace of \( \lmap(\R^5, \R^4) \).
\end{exercise}

\begin{solution}
    Let \( W = \{ T \in \lmap(\R^5, \R^4) : \dim \Null T > 2 \} \). Define \( S, T \in \lmap(\R^5, \R^4) \) by
    \[
        S(x_1, x_2, x_3, x_4, x_5) = (x_1, x_2, 0, 0) \quand T(x_1, x_2, x_3, x_4, x_5) = (0, 0, x_3, x_4).
    \]
    Then
    \begin{multline*}
        \Null S = \{ (0, 0, x_3, x_4, x_5) \in \R^5 : x_3, x_4, x_5 \in \R \} \\ \text{and } \Null T = \{ (x_1, x_2, 0, 0, x_5) \in \R^5 : x_1, x_2, x_5 \in \R \},
    \end{multline*}
    so that \( \dim \Null S = \dim \Null T = 3 \) and thus \( S, T \in W \). Observe that
    \[
        (S + T)(x_1, x_2, x_3, x_4, x_5) = (x_1, x_2, x_3, x_4)
    \]
    and so
    \[
        \Null (S + T) = \{ (0, 0, 0, 0, x_5) \in \R^5 : x_5 \in \R \}.
    \]
    Then \( \dim \Null (S + T) = 1 \), so \( S + T \not\in W \). This shows that \( W \) is not closed under addition and hence is not a subspace of \( \lmap(\R^5, \R^4) \).
\end{solution}

\begin{exercise}
\label{ex:5}
    Give an example of a linear map \( T : \R^4 \to \R^4 \) such that
    \[
        \Range T = \Null T.
    \]
\end{exercise}

\begin{solution}
    Let \( T \in \lmap(\R^4, \R^4) \) be given by
    \[
        T(x_1, x_2, x_3, x_4) = (x_3, x_4, 0, 0).
    \]
    Then
    \[
        \Range T = \{ (x_3, x_4, 0, 0) \in \R^4 : x_3, x_4 \in \R \} \quand \Null T = \{ (x_1, x_2, 0, 0) \in \R^4 : x_1, x_2 \in \R \},
    \]
    which are the same subspace of \( \R^4 \).
\end{solution}

\begin{exercise}
\label{ex:6}
    Prove that there does not exist a linear map \( T : \R^5 \to \R^5 \) such that
    \[
        \Range T = \Null T.
    \]
\end{exercise}

\begin{solution}
    If \( V \) is a finite-dimensional vector space and \( T : V \to W \) is a linear map such that \( \Range T = \Null T \), then by the Fundamental Theorem of Linear Maps (3.22), we have
    \[
        \dim V = \dim \Null T + \dim \Range T = 2 \, \dim \Null T.
    \]
    Thus \( \dim V \) must be a non-negative even integer. Since \( \dim \R^5 = 5 \), there can be no linear map \( T : \R^5 \to \R^5 \) satisfying \( \Range T = \Null T \).
\end{solution}

\begin{exercise}
\label{ex:7}
    Suppose \( V \) and \( W \) are finite-dimensional with \( 2 \leq \dim V \leq \dim W \). Show that \( \{ T \in \lmap(V, W) : T \text{ is not injective} \} \) is not a subspace of \( \lmap(V, W) \).
\end{exercise}

\begin{solution}
    Let \( X = \{ T \in \lmap(V, W) : T \text{ is not injective} \} \). By 3.16 we have
    \[
        X = \{ T \in \lmap(V, W) : \Null T \neq \{ 0 \} \}.
    \]
    Let \( v_1, \ldots, v_m \) be a basis for \( V \) and let \( w_1, \ldots, w_n \) be a basis for \( W \); by assumption, we have \( 2 \leq m \leq n \). We will define two linear maps \( S, T \in \lmap(V, W) \) by specifying their effect on the basis vectors \( v_1, \ldots, v_m \) and appealing to 3.5. Let
    \begin{gather*}
        Sv_1 = 0, \quad Sv_2 = w_2, \quad Sv_j = \tfrac{1}{2} w_j \text{ for } 3 \leq j \leq m \text{ if } m \geq 3, \\[2mm]
        Tv_1 = w_1, \quad Tv_2 = 0, \quad Tv_j = \tfrac{1}{2} w_j \text{ for } 3 \leq j \leq m \text{ if } m \geq 3.
    \end{gather*}
    \( S \) and \( T \) are not injective since \( 0 \neq v_1 \in \Null S \) and \( 0 \neq v_2 \in \Null T \), so \( S \) and \( T \) belong to \( X \). Let \( L \) be the the map \( S + T \). Then \( L \) is given by
    \[
        L v_j = w_j \text{ for } 1 \leq j \leq m.
    \]
    We claim that \( L \) is injective. To see this, suppose that \( Lv = 0 \) for some \( v \in V \). There are scalars \( a_1, \ldots, a_m \) such that \( v = a_1 v_1 + \cdots + a_m v_m \). Then
    \[
        0 = Lv = L(a_1 v_1 + \cdots + a_m v_m) = a_1 Lv_1 + \cdots + a_m Lv_m = a_1 w_1 + \cdots + a_m w_m.
    \]
    Since the list \( w_1, \ldots, w_m \) is linearly independent, we see that \( a_1 = \cdots = a_m = 0 \) and thus \( v = 0 \). It follows that \( L \) is injective and hence that \( X \) is not closed under addition and so cannot be a subspace.
\end{solution}

\begin{exercise}
\label{ex:8}
    Suppose \( V \) and \( W \) are finite-dimensional with \( \dim V \geq \dim W \geq 2 \). Show that \( \{ T \in \lmap(V, W) : T \text{ is not surjective} \} \) is not a subspace of \( \lmap(V, W) \).
\end{exercise}

\begin{solution}
    The solution is similar to \Cref{ex:7}. Let \( X = \{ T \in \lmap(V, W) : T \text{ is not surjective} \} \). Suppose \( v_1, \ldots, v_m \) is a basis for \( V \) and \( w_1, \ldots, w_n \) is a basis for \( W \); by assumption, we have \( m \geq n \geq 2 \). Define \( S, T \in \lmap(V, W) \) by
    \[
        Sv_j = \begin{cases}
            0 & \text{if } j = 1 \text{ or } n < j \leq m \text{ if } m > n, \\
            w_2 & \text{if } j = 2, \\
            \tfrac{1}{2} w_j & \text{if } 3 \leq j \leq n \text{ if } n \geq 3.
        \end{cases}
        \quad
        Tv_j = \begin{cases}
            w_1 & \text{if } j = 1, \\
            0 & \text{if } j = 2 \text{ or } n < j \leq m \text{ if } m > n, \\
            \tfrac{1}{2} w_j & \text{if } 3 \leq j \leq n \text{ if } n \geq 3.
        \end{cases}
    \]
    We claim that \( S \) is not surjective. To see this, we will show that \( w_1 \not\in \Range S \). Suppose by way of contradiction that there exists \( v \in V \) such that \( Sv = w_1 \). Then there are scalars \( a_1, \ldots, a_m \) such that \( a_1 v_1 + \cdots + a_m v_m = v \), which gives
    \[
        w_1 = Sv = S(a_1 v_1 + \cdots + a_m v_m) = a_1 S v_1 + \cdots + a_m S v_m = a_2 w_2 + \cdots + \tfrac{1}{2} a_n w_n.
    \]
    Thus \( w_1 \in \Span(w_2, \ldots, w_n) \), contradicting the linear independence of the basis \( w_1, \ldots, w_n \). It follows that \( w_1 \not\in \Range S \), so that \( S \) is not surjective. Similarly, we see that \( T \) is not surjective, since \( w_2 \not\in \Range T \). Hence \( S \) and \( T \) belong to \( X \). Let \( L \) be the map \( S + T \). Then \( L \) is given by
    \[
        Lv_j = \begin{cases}
            w_j & \text{if } 1 \leq j \leq n, \\
            0 & \text{if } n < j \leq m \text{ if } m > n.
        \end{cases}
    \]
    We claim that \( L \) is surjective. To see this, let \( w \in W \) be given. Then there are scalars \( a_1, \ldots, a_n \) such that \( w = a_1 w_1 + \cdots + a_n w_n \). Observe that
    \[
        L(a_1 v_1 + \cdots + a_n v_n) = a_1 Lv_1 + \cdots + a_n Lv_n = a_1 w_1 + \cdots + a_n w_n = w.
    \]
    It follows that \( L \) is surjective and hence that \( X \) is not closed under addition and so cannot be a subspace.
\end{solution}

\begin{exercise}
\label{ex:9}
    Suppose \( T \in \lmap(V, W) \) is injective and \( v_1, \ldots, v_n \) is linearly independent in \( V \). Prove that \( Tv_1, \ldots, Tv_n \) is linearly independent in \( W \).
\end{exercise}

\begin{solution}
    Suppose we have scalars \( a_1, \ldots, a_n \) such that \( a_1 Tv_1 + \cdots + a_n Tv_n = 0 \). By linearity, this is equivalent to \( T(a_1 v_1 + \cdots + a_n v_n) = 0 \). Then since \( T \) is injective, we have by 3.16 that \( a_1 v_1 + \cdots + a_n v_n = 0 \). The linear independence of \( v_1, \ldots, v_n \) then implies that \( a_1 = \cdots = a_n = 0 \) and hence that \( Tv_1, \ldots, Tv_n \) is linearly independent.
\end{solution}

\begin{exercise}
\label{ex:10}
    Suppose \( v_1, \ldots, v_n \) spans \( V \) and \( T \in \lmap(V, W) \). Prove that the list \( Tv_1, \ldots, Tv_n \) spans \( \Range T \).
\end{exercise}

\begin{solution}
    Let \( w \in \Range T \) be given, so that \( w = Tv \) for some \( v \in V \). Since \( v_1, \ldots, v_n \) spans \( V \), there are scalars \( a_1, \ldots, a_n \) such that \( v = a_1 v_1 + \cdots + a_n v_n \). Then:
    \[
        a_1 Tv_1 + \cdots + a_n Tv_n = T(a_1 v_1 + \cdots + a_n v_n) = Tv = w.
    \]
    Thus \( Tv_1, \ldots, Tv_n \) spans \( \Range T \).
\end{solution}

\begin{exercise}
\label{ex:11}
    Suppose \( S_1, \ldots, S_n \) are injective linear maps such that \( S_1 S_2 \cdots S_n \) makes sense. Prove that \( S_1 S_2 \cdots S_n \) is injective.
\end{exercise}

\begin{solution}
    We will prove this by induction. Let \( P(n) \) be the statement that for any collection of \( n \) injective linear maps \( S_1, \ldots, S_n \) such that \( S_1 S_2 \cdots S_n \) makes sense, we have that \( S_1 S_2 \cdots S_n \) is injective. The base case \( P(1) \) is clear. Suppose that \( P(n) \) is true for some \( n \in \N \), and suppose we have \( n + 1 \) linear maps \( S_1, \ldots, S_{n+1} \) such that \( S_1 S_2 \cdots S_{n+1} \) makes sense. Let \( v \) be a vector in the domain of \( S_{n+1} \) such that
    \[
        (S_1 S_2 \cdots S_{n+1})(v) = S_1((S_2 \cdots S_{n+1})(v)) = 0.
    \]
    Since \( S_1 \) is injective, 3.16 implies that \( (S_2 \cdots S_{n+1})(v) = 0 \). Our induction hypothesis guarantees that \( S_2 \cdots S_{n+1} \) is injective, so again by 3.16 we have that \( v = 0 \). It follows that \( \Null (S_1 S_2 \cdots S_{n+1}) = \{ 0 \} \) and hence by 3.16 the linear map \( S_1 S_2 \cdots S_{n+1} \) is injective. This completes the induction step and the proof.
\end{solution}

\begin{exercise}
\label{ex:12}
    Suppose that \( V \) is finite-dimensional and that \( T \in \lmap(V, W) \). Prove that there exists a subspace \( U \) of \( V \) such that \( U \cap \Null T = \{ 0 \} \) and \( \Range T = \{ Tu : u \in U \} \).
\end{exercise}

\begin{solution}
    Since \( \Null T \) is a subspace of \( V \), 2.34 implies that there exists a subspace \( U \) of \( V \) such that \( V = U \oplus \Null T \); 1.45 then gives us \( U \cap \Null T = \{ 0 \} \). Suppose that \( w \in \Range T \), so that \( w = Tv \) for some \( v \in V \). Since \( V = U \oplus \Null T \), there are unique vectors \( u \in U \) and \( x \in \Null T \) such that \( v = u + x \). Then
    \[
        w = Tv = T(u + x) = Tu + Tx = Tu + 0 = Tu.
    \]
    Thus \( \Range T \subseteq \{ Tu : u \in U \} \), and since the reverse inclusion is clear, we may conclude that \( \Range T = \{ Tu : u \in U \} \).
\end{solution}

\begin{exercise}
\label{ex:13}
    Suppose \( T \) is a linear map from \( \F^4 \) to \( \F^2 \) such that
    \[
        \Null T = \{ (x_1, x_2, x_3, x_4) \in \F^4 : x_1 = 5 x_2 \text{ and } x_3 = 7 x_4 \}.
    \]
    Prove that \( T \) is surjective.
\end{exercise}

\begin{solution}
    It is not hard to see that \( (5, 1, 0, 0), (0, 0, 7, 1) \) is a basis for \( \Null T \), so that \( \dim \Null T = 2 \). Then by the Fundamental Theorem of Linear Maps (3.22), we have
    \[
        \dim \F^4 = \dim \Null T + \dim \Range T, \text{ i.e.\ } 4 = 2 + \dim \Range T.
    \]
    Thus \( \dim \Range T = 2 = \dim \F^2 \). Since \( \Range T \) is a subspace of \( \F^2 \), \href{https://lew98.github.io/Mathematics/LADR_Section_2_C_Exercises.pdf}{Exercise 2.C.1} allows us to conclude that \( \Range T = \F^2 \) and hence that \( T \) is surjective.
\end{solution}

\begin{exercise}
\label{ex:14}
    Suppose \( U \) is a 3-dimensional subspace of \( \R^8 \) and that \( T \) is a linear map from \( \R^8 \) to \( \R^5 \) such that \( \Null T = U \). Prove that \( T \) is surjective.
\end{exercise}

\begin{solution}
    Since \( \dim U = 3 \), we also have \( \dim \Null T = 3 \). The Fundamental Theorem of Linear Maps (3.22) gives
    \[
        \dim \R^8 = \dim \Null T + \dim \Range T, \text{ i.e.\ } 8 = 3 + \dim \Range T.
    \]
    Thus \( \dim \Range T = 5 = \dim \R^5 \). \href{https://lew98.github.io/Mathematics/LADR_Section_2_C_Exercises.pdf}{Exercise 2.C.1} allows us to conclude that \( \Range T = \R^5 \) and hence that \( T \) is surjective.
\end{solution}

\begin{exercise}
\label{ex:15}
    Prove that there does not exist a linear map from \( \F^5 \) to \( \F^2 \) whose null space equals
    \[
        \{ (x_1, x_2, x_3, x_4, x_5) \in \F^5 : x_1 = 3 x_2 \text{ and } x_3 = x_4 = x_5 \}.  
    \]
\end{exercise}

\begin{solution}
    Let \( U = \{ (x_1, x_2, x_3, x_4, x_5) \in \F^5 : x_1 = 3 x_2 \text{ and } x_3 = x_4 = x_5 \}. \) It is not hard to see that \( (3, 1, 0, 0, 0), (0, 0, 1, 1, 1) \) is a basis for \( U \), so that \( \dim U = 2 \). Let \( T \) be a linear map from \( \F^5 \) to \( \F^2 \). The Fundamental Theorem of Linear Maps (3.22) implies that
    \[
        \dim \F^5 = \dim \Null T + \dim \Range T, \text{ i.e.\ } 5 = \dim \Null T + \dim \Range T.
    \]
    Since \( \Range T \) is a subspace of \( \F^2 \), we must have \( \dim \Range T \leq \dim \F^2 = 2 \). Combining this with the equality \( 5 = \dim \Null T + \dim \Range T \), we see that \( \dim \Null T \geq 3 \). It follows that \( U \) cannot be the null space of \( T \).
\end{solution}

\begin{exercise}
\label{ex:16}
    Suppose there exists a linear map on \( V \) whose null space and range are both finite-dimensional. Prove that \( V \) is finite-dimensional.
\end{exercise}

\begin{solution}
    Let \( T : V \to V \) be the linear map in question. There is a basis \( u_1, \ldots, u_m \) for \( \Range T \) and a basis \( w_1, \ldots, w_n \) for \( \Null T \). Since each \( u_i \in \Range T \), there exists a \( v_i \in V \) such that \( Tv_i = u_i \). We claim that the list \( v_1, \ldots, v_m, w_1, \ldots, w_n \) spans \( V \). To see this, let \( v \in V \) be given. Then \( Tv \in \Range T \), so there are scalars \( a_1, \ldots, a_m \) such that
    \begin{align*}
        & Tv = a_1 u_1 + \cdots + a_m u_m = a_1 Tv_1 + \cdots + a_m Tv_m = T(a_1 v_1 + \cdots + a_m v_m) \\
        \implies & T(v - (a_1 v_1 + \cdots + a_m v_m)) = 0 \\
        \implies & v - (a_1 v_1 + \cdots + a_m v_m) \in \Null T.
    \end{align*}
    Hence there are scalars \( b_1, \ldots, b_n \) such that
    \begin{align*}
        & v - (a_1 v_1 + \cdots + a_m v_m) = b_1 w_1 + \cdots b_n w_n \\
        \implies & v = a_1 v_1 + \cdots + a_m v_m + b_1 w_1 + \cdots b_n w_n.
    \end{align*}
    Thus the list \( v_1, \ldots, v_m, w_1, \ldots, w_n \) spans \( V \). We may conclude that \( V \) is finite-dimensional.
\end{solution}

\begin{exercise}
\label{ex:17}
    Suppose \( V \) and \( W \) are both finite-dimensional. Prove that there exists an injective linear map from \( V \) to \( W \) if and only if \( \dim V \leq \dim W \).
\end{exercise}

\begin{solution}
    If \( \dim V > \dim W \), then 3.23 guarantees that no linear map from \( V \) to \( W \) is injective. Suppose therefore that \( \dim V \leq \dim W \). Then there is a basis \( v_1, \ldots, v_m \) for \( V \) and a basis \( w_1, \ldots, w_n \) for \( W \), where \( m \leq n \). Define a linear map \( T : V \to W \) by \( Tv_j = w_j \). As shown in \Cref{ex:7} (with the map \( L \)), such a map is injective.
\end{solution}

\begin{exercise}
\label{ex:18}
    Suppose \( V \) and \( W \) are both finite-dimensional. Prove that there exists a surjective linear map from \( V \) onto \( W \) if and only if \( \dim V \geq \dim W \).
\end{exercise}

\begin{solution}
    If \( \dim V < \dim W \), then 3.24 guarantees that no linear map from \( V \) to \( W \) is surjective. Suppose therefore that \( \dim V \geq \dim W \). Then there is a basis \( v_1, \ldots, v_m \) for \( V \) and a basis \( w_1, \ldots, w_n \) for \( W \), where \( m \geq n \). Define a linear map \( T : V \to W \) by \( Tv_j = w_j \) for \( 1 \leq j \leq n \), and \( Tv_j = 0 \) for \( n < j \leq m \), if \( m > n \). As shown in \Cref{ex:8} (with the map \( L \)), such a map is surjective.
\end{solution}

\begin{exercise}
\label{ex:19}
    Suppose \( V \) and \( W \) are finite-dimensional and that \( U \) is a subspace of \( V \). Prove that there exists \( T \in \lmap(V, W) \) such that \( \Null T = U \) if and only if \( \dim U \geq \dim V - \dim W \).
\end{exercise}

\begin{solution}
    Suppose that there exists \( T \in \lmap(V, W) \) such that \( \Null T = U \). The Fundamental Theorem of Linear Maps (3.22) implies that
    \[
        \dim V = \dim \Null T + \dim \Range T = \dim U + \dim \Range T.
    \]
    Since \( \Range T \) is a subspace of \( W \), we have \( \dim \Range T \leq \dim W \). Combining this with the equality \( \dim V - \dim U = \dim \Range T \), we see that \( \dim U \geq \dim V - \dim W \).

    Now suppose that \( \dim U \geq \dim V - \dim W \). Let \( u_1, \ldots, u_m \) be a basis of \( U \), which we extend to a basis \( u_1, \ldots, u_m, v_1, \ldots, v_n \) of \( V \), and let \( w_1, \ldots, w_k \) be a basis of \( W \). By assumption, we have \( m \geq m + n - k \), or \( k \geq n \). Define a map \( T : V \to W \) by \( Tu_i = 0 \) for \( 1 \leq i \leq m \), and \( Tv_i = w_i \) for \( 1 \leq i \leq n \); this is possible precisely because \( k \geq n \), i.e.\ there are enough \( w_i \)'s to define this map. It is not hard to see that \( U \subseteq \Null T \). Suppose that \( v \in \Null T \). There are scalars \( a_1, \ldots, a_m, b_1, \ldots, b_n \) such that \( v = a_1 u_1 + \cdots + a_m u_m + b_1 v_1 + \cdots + b_n v_n \). Then
    \begin{multline*}
        0 = Tv = T(a_1 u_1 + \cdots + a_m u_m + b_1 v_1 + \cdots + b_n v_n) \\ = a_1 Tu_1 + \cdots + a_m Tu_m + b_1 Tv_1 + \cdots + b_n Tv_n = b_1 w_1 + \cdots b_n w_n.
    \end{multline*}
    The linear independence of \( w_1, \ldots, w_n \) then implies that \( b_1 = \cdots = b_n = 0 \), so that \( v = a_1 u_1 + \cdots + a_m u_m \). Hence \( v \in U \) and we may conclude that \( \Null T = U \).
\end{solution}

\begin{exercise}
\label{ex:20}
    Suppose \( W \) is finite-dimensional and \( T \in \lmap(V, W) \). Prove that \( T \) is injective if and only if there exists \( S \in \lmap(W, V) \) such that \( ST \) is the identity map on \( V \).
\end{exercise}

\begin{solution}
    Suppose there exists such a map \( S \) and suppose that \( v \in V \) is such that \( Tv = 0 \). Then
    \[
        (ST)(v) = S(0) = 0 \implies v = 0,
    \]
    since \( ST \) is the identity map on \( V \). Thus \( \Null T = \{ 0 \} \), which is the case if and only if \( T \) is injective.

    Now suppose that \( T \) is injective. Let \( u_1, \ldots, u_m \) be a basis for \( \Range T \), which we extend to a basis \( u_1, \ldots, u_m, w_1, \ldots, w_n \) for \( W \). Since each \( u_i \in \Range T \), there is a \( v_i \in V \) such that \( u_i = Tv_i \). Define a linear map \( S : W \to V \) by \( Su_i = v_i \) and \( Sw_i = 0 \). We claim that \( ST \) is the identity map on \( V \). To see this, let \( v \in V \) be given. Then \( Tv \in \Range T \), so there are scalars \( a_1, \ldots, a_m \) such that \( Tv = a_1 u_1 + \cdots + a_m u_m \), which gives
    \[
        Tv = a_1 Tv_1 + \cdots + a_m Tv_m = T(a_1 v_1 + \cdots + a_m v_m).
    \]
    Since \( T \) is injective, we must then have \( v = a_1 v_1 + \cdots + a_m v_m \). Applying \( S \) to both sides of \( Tv = a_1 u_1 + \cdots + a_m u_m \) gives
    \[
        (ST)(v) = a_1 Su_1 + \cdots + a_m Su_m = a_1 v_1 + \cdots + a_m v_m = v.
    \]
    Thus \( ST \) is the identity map on \( V \).
\end{solution}

\begin{exercise}
\label{ex:21}
    Suppose \( V \) is finite-dimensional and \( T \in \lmap(V, W) \). Prove that \( T \) is surjective if and only if there exists \( S \in \lmap(W, V) \) such that \( TS \) is the identity map on \( W \).
\end{exercise}

\begin{solution}
    Suppose there exists such a map \( S \) and let \( w \in W \) be given. Then \( T(Sw) = w \) and thus \( w \in \Range T \). It follows that \( T \) is surjective.

    Now suppose that \( T \) is surjective, i.e.\ that \( \Range T = W \). Then by the Fundamental Theorem of Linear Maps (3.22), \( W \) is finite-dimensional, so let \( w_1, \ldots, w_n \) be a basis of \( W \). Since \( \Range T = W \), there are vectors \( v_1, \ldots, v_n \) such that \( Tv_i = w_i \). Define a linear map \( S : W \to V \) by \( Sw_i = v_i \). We claim that \( TS \) is the identity map on \( W \). To see this, let \( w \) be given. There are scalars \( a_1, \ldots, a_n \) such that \( w = a_1 w_1 + \cdots + a_n w_n \). Then
    \begin{multline*}
        (TS)(w) = (TS)(a_1 w_1 + \cdots + a_n w_n) = a_1 (TS)(w_1) + \cdots + a_n (TS)(w_n) \\ = a_1 Tv_1 + \cdots + a_n Tv_n = a_1 w_1 + \cdots + a_n w_n = w.
    \end{multline*}
    Thus \( TS \) is the identity map on \( W \).
\end{solution}

\begin{exercise}
\label{ex:22}
    Suppose \( U \) and \( V \) are finite-dimensional vector spaces and \( S \in \lmap(V, W) \) and \( T \in \lmap(U, V) \). Prove that
    \[
        \dim \Null ST \leq \dim \Null S + \dim \Null T.
    \]
\end{exercise}

\begin{solution}
    If \( u \in U \) is such that \( Tu = 0 \), then \( (ST)(u) = S(0) = 0 \). It follows that \( \Null T \) is a subspace of \( \Null ST \). Thus if we let \( u_1, \ldots, u_m \) be a basis of \( \Null T \), then we can extend this to a basis \( u_1, \ldots, u_m, x_1, \ldots, x_n \) of \( \Null ST \). Letting \( X = \Span(x_1, \ldots, x_n) \), we then have \( \Null ST = \Null T \oplus X \). Let \( v_1, \ldots, v_k \) be a basis for \( \Null S \). Proving that \( \dim \Null ST \leq \dim \Null S + \dim \Null T \) is then equivalent to showing that \( m + n \leq m + k \), i.e.\ \( n \leq k \).
    
    First, we claim that the list \( Tx_1, \ldots, Tx_n \) is linearly independent. To see this, suppose we have scalars \( a_1, \ldots, a_n \) such that
    \[
        a_1 Tx_1 + \cdots + a_n Tx_n = T(a_1 x_1 + \cdots + a_n x_n) = 0.
    \]
    Then \( a_1 x_1 + \cdots + a_n x_n \in \Null T \). Evidently, we have \( a_1 x_1 + \cdots + a_n x_n \in X \). Since the sum \( \Null T \oplus X \) is direct, we have \( \Null T \cap X = \{ 0 \} \); it follows that \( a_1 x_1 + \cdots + a_n x_n = 0 \). The linear independence of the list \( x_1, \ldots, x_n \) then implies that \( a_1 = \cdots = a_n = 0 \) and our claim follows.

    Now, since each \( x_i \in \Null ST \), we have \( S(Tx_i) = 0 \), so that each \( Tx_i \) belongs to \( \Null S \). Since \( \Null S \) has a basis of length \( k \) and we showed that the list \( Tx_1, \ldots, Tx_n \) is linearly independent, 2.23 implies that \( n \leq k \), as desired.
\end{solution}

\begin{exercise}
\label{ex:23}
    Suppose \( U \) and \( V \) are finite-dimensional vector spaces and \( S \in \lmap(V, W) \) and \( T \in \lmap(U, V) \). Prove that
    \[
        \dim \Range ST \leq \min \{ \dim \Range S, \dim \Range T \}.
    \]
\end{exercise}

\begin{solution}
    If \( w \in \Range ST \), then \( w = S(Tu) \) for some \( u \in U \), so that \( w \in \Range S \) also. It follows that \( \Range ST \) is a subspace of \( \Range S \), which implies that \( \dim \Range ST \leq \dim \Range S \).

    Let \( w_1, \ldots, w_m \) be a basis for \( \Range ST \) and let \( v_1, \ldots, v_n \) be a basis for \( \Range T \). We claim that the list \( Sv_1, \ldots, Sv_n \) spans \( \Range ST \). To see this, let \( w \in \Range ST \) be given, so that \( w = S(Tu) \) for some \( u \in U \). Since \( Tu \in \Range T \), there are scalars \( a_1, \ldots, a_n \) such that \( Tu = a_1 v_1 + \cdots + a_n v_n \). Then
    \[
        w = S(Tu) = S(a_1 v_1 + \cdots + a_n v_n) = a_1 Sv_1 + \cdots + a_n Sv_n.
    \]
    Thus \( w \in \Span(Sv_1, \ldots, Sv_n) \) and our claim follows. 2.23 now implies that \( m \leq n \), i.e.\ \( \dim \Range ST \leq \dim \Range T \).

    We now have both the inequalities \( \dim \Range ST \leq \dim \Range S \) and \( \dim \Range ST \leq \dim \Range T \). It follows that
    \[
        \dim \Range ST \leq \min \{ \dim \Range S, \dim \Range T \}.
    \]
\end{solution}

\begin{exercise}
\label{ex:24}
    Suppose \( W \) is finite-dimensional and \( T_1, T_2 \in \lmap(V, W) \). Prove that \( \Null T_1 \subset \Null T_2 \) if and only if there exists \( S \in \lmap(W, W) \) such that \( T_2 = ST_1 \).
\end{exercise}

\begin{solution}
    Suppose that there exists such a map \( S \) and let \( v \in \Null T_1 \). Then \( T_2 v = S(T_1 v) = S(0) = 0 \), so that \( v \in \Null T_2 \) also.

    Now suppose that \( \Null T_1 \subseteq \Null T_2 \). Let \( x_1, \ldots, x_m \) be a basis of \( \Range T_1 \), which we extend to a basis \( x_1, \ldots, x_m, y_1, \ldots, y_n \) of \( W \). Since each \( x_j \in \Range T_1 \), we have \( x_j = T_1 v_j \) for some \( v_j \in V \). Define a linear map \( S : W \to W \) by \( S x_j = T_2 v_j \) for \( 1 \leq j \leq m \) and \( S y_j = 0 \) for \( 1 \leq j \leq n \). We claim that \( T_2 = ST_1 \). To see this, let \( v \in V \) be given. Then \( T_1 v \in \Range T_1 \), so there are scalars \( a_1, \ldots, a_m \) such that
    \begin{align*}
        & T_1 v = a_1 x_1 + \cdots + a_m x_m = a_1 T_1 v_1 + \cdots + a_m T_1 v_m = T_1 (a_1 v_1 + \cdots + a_m v_m) \\
        \implies & T_1 (v - (a_1 v_1 + \cdots + a_m v_m)) = 0 \\
        \implies & v - (a_1 v_1 + \cdots + a_m v_m) \in \Null T_1.
    \end{align*}
    Then since \( \Null T_1 \subseteq \Null T_2 \), we have \( v - (a_1 v_1 + \cdots + a_m v_m) \in \Null T_2 \). Following the algebra above in reverse with \( T_2 \) in place of \( T_1 \) shows that \( T_2 v = a_1 T_2 v_1 + \cdots + a_m T_2 v_m \), and applying \( S \) to both sides of the equality \( T_1 v = a_1 x_1 + \cdots + a_m x_m \) gives us
    \[
        S(T_1 v) = a_1 Sx_1 + \cdots + a_m Sx_m = a_1 T_2 v_1 + \cdots + a_m T_2 v_m = T_2 v.
    \]
    Thus \( T_2 = ST_1 \).
\end{solution}

\begin{exercise}
\label{ex:25}
    Suppose \( V \) is finite-dimensional and \( T_1, T_2 \in \lmap(V, W) \). Prove that \( \Range T_1 \subset \Range T_2 \) if and only if there exists \( S \in \lmap(V, V) \) such that \( T_1 = T_2 S \).
\end{exercise}

\begin{solution}
    Suppose that there exists such a map \( S \) and let \( w \in \Range T_1 \) be given, so that \( w = T_1 v \) for some \( v \in V \). Then
    \[
        w = T_1 v = T_2(Sv) \in \Range T_2.
    \]
    Thus \( \Range T_1 \subseteq \Range T_2 \).

    Now suppose that \( \Range T_1 \subseteq \Range T_2 \). Let \( v_1, \ldots, v_m \) be a basis for \( V \). By assumption, each \( T_1 v_j \) belongs to \( \Range T_2 \), so we have \( T_1 v_j = T_2 u_j \) for some \( u_j \in V \). Define a linear map \( S : V \to V \) by \( Sv_j = u_j \) for \( 1 \leq j \leq m \). We claim that \( T_1 = T_2 S \). To see this, let \( v \in V \) be given. Then there are scalars \( a_1, \ldots, a_m \) such that \( v = a_1 v_1 + \cdots + a_m v_m \). Observe that
    \begin{multline*}
        T_2(Sv) = T_2 (S(a_1 v_1 + \cdots + a_m v_m)) = T_2 (a_1 Sv_1 + \cdots + a_m Sv_m) = T_2 (a_1 u_1 + \cdots + a_m u_m) \\ = a_1 T_2 u_1 + \cdots + a_m T_2 u_m = a_1 T_1 v_1 + \cdots + a_m T_1 v_m = T_1 (a_1 v_1 + \cdots + a_m v_m) = T_1 v.
    \end{multline*}
    Thus \( T_1 = T_2 S \).
\end{solution}

\begin{exercise}
\label{ex:26}
    Suppose \( D \in \lmap(\poly(\R), \poly(\R)) \) is such that \( \deg Dp = (\deg p) - 1 \) for every nonconstant polynomial \( p \in \poly(\R) \). Prove that \( D \) is surjective.

    \noindent [\textit{The notation \( D \) is used above to remind you of the differentiation map that sends a polynomial \( p \) to \( p' \). Without knowing the formula for the derivative of a polynomial (except that it reduces the degree by 1), you can use the exercise above to show that for every polynomial \( q \in \poly(\R) \), there exists a polynomial \( p \in \poly(\R) \) such that \( p' = q \).}]
\end{exercise}

\begin{solution}
    For a non-negative integer \( n \), let \( A(n) \) be the statement that there exists a polynomial \( p_n \in \poly(\R) \) such that \( D(p_n) = x^n \). We will use strong induction to show that \( A(n) \) holds for all non-negative integers \( n \). For the base case \( n = 0 \), note that by assumption the polynomial \( D(x) \) must have degree 0, i.e.\ must be a non-zero constant, say \( D(x) = b \neq 0 \). Define \( p_0 := b^{-1} x \). Then by linearity,
    \[
        D(p_0) = D(b^{-1} x) = b^{-1} D(x) = b^{-1} b = 1.
    \]
    Thus the base case \( A(0) \) holds.

    Now suppose that \( A(0), A(1), \ldots, A(n) \) all hold for some non-negative integer \( n \). By assumption, the polynomial \( D(x^{n+2}) \) must have degree \( n + 1 \), i.e.\ must be of the form
    \[
        D(x^{n+2}) = b_{n+1} x^{n+1} + b_n x^n + \cdots + b_1 x + b_0,
    \]
    where \( b_{n+1} \neq 0 \). Our induction hypothesis guarantees the existence of polynomials \( p_0, p_1, \ldots, p_n \) such that \( D(p_j) = x^j \) for \( 1 \leq j \leq n \). Thus we can write
    \[
        b_{n+1}^{-1} D(x^{n+2}) = x^{n+1} + b_{n+1}^{-1} (b_n D(p_n) + \cdots + b_1 D(p_1) + b_0 D(p_0)),
    \]
    which by the linearity of \( D \) implies
    \[
        x^{n+1} = D(b_{n+1}^{-1}(x^{n+2} - (b_n p_n + \cdots + b_1 p_1 + b_0 p_0))).
    \]
    So if we define \( p_{n+1} := b_{n+1}^{-1}(x^{n+2} - (b_n p_n + \cdots + b_1 p_1 + b_0 p_0)) \), then we have \( D(p_{n+1}) = x^{n+1} \). Thus \( A(n+1) \) holds. This completes the induction step.

    So \( A(n) \) holds for all non-negative integers \( n \). We can now show that \( D \) is surjective. Let \( p \) be an arbitrary polynomial in \( \poly(\R) \) and let \( n = \deg p \). Then \( p = \sum_{j=0}^n a_j x^j \) for some coefficients \( a_0, \ldots, a_n \) (with \( a_n \neq 0 \)). Set \( q = \sum_{j=0}^n a_j p_j \). Then we have
    \[
        D(q) = \sum_{j=0}^n a_j D(p_j) = \sum_{j=0}^n a_j x^j = p.  
    \]
    Thus \( D \) is surjective.
\end{solution}

\begin{exercise}
\label{ex:27}
    Suppose \( p \in \poly(\R) \). Prove that there exists a polynomial \( q \in \poly(\R) \) such that \( 5q'' + 3q' = p \).

    \noindent [\textit{This exercise can be done without linear algebra, but it's more fun to do it using linear algebra.}]
\end{exercise}

\begin{solution}
    Define a map \( D : \poly(\R) \to \poly(\R) \) by \( Dq = 5q'' + 3q' \). It is not hard to see that this map is linear, since differentiation is a linear operation. Suppose \( q \in \poly(\R) \) is a non-constant polynomial of degree \( n \geq 1 \), so that \( q = \sum_{j=0}^n a_j x^j \) where \( a_n \neq 0 \). Some algebra gives
    \[
        Dq = \begin{cases}
            3 a_n & \text{if } n = 1, \\
            3n a_n x^{n-1} + \sum_{j=0}^{n-2} (j+1)[3 a_{j+1} + 5(j+2) a_{j+2}] x^j & \text{if } n \geq 2.
        \end{cases}
    \]
    In either case, since \( a_n \neq 0 \), \( Dq \) is a polynomial of degree \( n - 1 \). Thus the linear map \( D \) satisfies the hypotheses of \Cref{ex:26} and hence must be surjective.
\end{solution}

\begin{exercise}
\label{ex:28}
    Suppose \( T \in \lmap(V, W) \), and \( w_1, \ldots, w_m \) is a basis of \( \Range T \). Prove that there exist \( \varphi_1, \ldots, \varphi_m \in \lmap(V, \F) \) such that
    \[
        Tv = \varphi_1(v) w_1 + \cdots + \varphi_m(v) w_m
    \]
    for every \( v \in V \).
\end{exercise}

\begin{solution}
    Let \( v \in V \) be given. Since \( w_1, \ldots, w_m \) is a basis of \( \Range T \), there are unique scalars \( b_1, \ldots, b_m \) such that \( Tv = b_1 w_1 + \cdots + b_m w_m \). For \( 1 \leq j \leq m \), define \( \varphi_j : V \to \F \) by \( \varphi_j(v) := b_j \); since the scalars \( b_1, \ldots, b_m \) are unique, each \( \varphi_j \) is well-defined. We claim that each \( \varphi_j \) is linear. To see this, let \( u, v \in V \) be given. Then
    \[
        Tu = a_1 w_1 + \cdots + a_m w_m \quand Tv = b_1 w_1 + \cdots + b_m w_m
    \]
    for unique scalars \( a_1, \ldots, a_m, b_1, \ldots, b_m \). Since
    \[
        T(u + v) = Tu + Tv = (a_1 + b_1) w_1 + \cdots + (a_m + b_m) w_m,
    \]
    the scalars \( a_1 + b_1, \ldots, a_m + b_m \) must be the unique coefficients for \( T(u + v) \) as a linear combination of the basis vectors \( w_1, \ldots, w_m \). Given this, for any \( 1 \leq j \leq m \) we have
    \[
        \varphi_j(u) = a_j, \quad \varphi_j(v) = b_j, \quand \varphi_j(u + v) = a_j + b_j.
    \]
    Thus \( \varphi_j(u + v) = \varphi_j(u) + \varphi_j(v) \). Similarly, let \( \lambda \in \F \) be a scalar. Then since
    \[
        T(\lambda u) = \lambda Tu = \lambda a_1 w_1 + \cdots + \lambda a_m w_m,
    \]
    the scalars \( \lambda a_1, \ldots, \lambda a_m \) must be the unique coefficients for \( T(\lambda u) \) as a linear combination of the basis vectors \( w_1, \ldots, w_m \). Given this, for any \( 1 \leq j \leq m \) we have
    \[
        \varphi_j(u) = a_j \quand \varphi_j(\lambda u) = \lambda a_j.
    \]
    Thus \( \varphi_j(\lambda u) = \lambda \varphi_j(u) \) and we see that each \( \varphi_j \) is linear, as claimed. Furthermore, given the definition of each \( \varphi_j \), it is clear that
    \[
        Tv = \varphi_1(v) w_1 + \cdots + \varphi_m(v) w_m
    \]
    for every \( v \in V \).
\end{solution}

\begin{exercise}
\label{ex:29}
    Suppose \( \varphi \in \lmap(V, \F) \). Suppose \( u \in V \) is not in \( \Null \varphi \). Prove that
    \[
        V = \Null \varphi \oplus \{ au : a \in \F \}.
    \]
\end{exercise}

\begin{solution}
    Let \( v \in V \) be given. If \( \varphi(v) = 0 \), then certainly \( v \in \Null \varphi + \{ au : a \in \F \} \). If \( \varphi(v) \neq 0 \), then observe that
    \begin{align*}
        & \varphi \left( \frac{v}{\varphi(v)} \right) = 1 = \varphi \left( \frac{u}{\varphi(u)} \right) \\
        \implies & \varphi \left( \frac{v}{\varphi(v)} - \frac{u}{\varphi(u)} \right) = 0 \\
        \implies & \frac{v}{\varphi(v)} - \frac{u}{\varphi(u)} \in \Null \varphi.
    \end{align*}
    Thus \( \frac{v}{\varphi(v)} - \frac{u}{\varphi(u)} = w \) for some \( w \in \Null \varphi \), which gives
    \[
        v = \varphi(v) w + \frac{\varphi(v)}{\varphi(u)} u \in \Null \varphi + \{ au : a \in \F \}.
    \]
    Hence \( V \) is the sum \( \Null \varphi + \{ au : a \in \F \} \). To see that this sum is direct, suppose that \( v \in \Null \varphi \) and \( v \in \{ au : a \in \F \} \), so that \( v = au \) for some \( a \in \F \). Then
    \[
        0 = \varphi(v) = \varphi(au) = a \varphi(u).
    \]
    Since \( \varphi(u) \neq 0 \), it must be the case that \( a = 0 \) and hence that \( v = 0 \). Thus
    \[
        \Null \varphi \cap \{ au : a \in \F \} = \{ 0 \}
    \]    
    and we see that the sum \( \Null \varphi + \{ au : a \in \F \} \) is direct.
\end{solution}

\begin{exercise}
\label{ex:30}
    Suppose \( \varphi_1 \) and \( \varphi_2 \) are linear maps from \( V \) to \( \F \) that have the same null space. Show that there exists a constant \( c \in \F \) such that \( \varphi_1 = c \varphi_2 \).
\end{exercise}

\begin{solution}
    If \( \Null \varphi_1 = \Null \varphi_2 = V \), then \( \varphi_1 \) and \( \varphi_2 \) are both the map \( v \mapsto 0 \), and so any \( c \in \F \) will do. Suppose therefore that \( \Null \varphi_1 \neq V \), so that there is a \( u \in V \) with \( u \not\in \Null \varphi_1 \) and \( u \not\in \Null \varphi_2 \). Define \( c := \tfrac{\varphi_1 u}{\varphi_2 u} \). We claim that \( \varphi_1 = c \varphi_2 \). To see this, first observe that by \Cref{ex:29}, we have
    \[
        V = \Null \varphi_1 \oplus \{ au : a \in \F \}.
    \]
    Let \( v \in V \) be given, so that \( v = x + au \) for some \( a \in \F \), where \( x \in \Null \varphi_1 = \Null \varphi_2 \). Then
    \[
        c \varphi_2 v = \frac{\varphi_1 u}{\varphi_2 u} \varphi_2(x + au) = \frac{\varphi_1 u}{\varphi_2 u} a \varphi_2 u = a \varphi_1 u = \varphi_1(x + au) = \varphi_1 v.
    \]
    Thus \( \varphi_1 = c \varphi_2 \).
\end{solution}

\begin{exercise}
\label{ex:31}
    Give an example of two linear maps \( T_1 \) and \( T_2 \) from \( \R^5 \) to \( \R^2 \) that have the same null space but are such that \( T_1 \) is not a scalar multiple of \( T_2 \).
\end{exercise}

\begin{solution}
    Let \( T_1 \) and \( T_2 \) be the linear maps given by
    \[
        T_1(x_1, x_2, x_3, x_4, x_5) = (x_4, x_5) \quand T_2(x_1, x_2, x_3, x_4, x_5) = (x_5, x_4).
    \]
    Then
    \[
        \Null T_1 = \Null T_2 = \{ (x_1, x_2, x_3, 0, 0) \in \R^5 : x_1, x_2, x_3 \in \R \}.
    \]
    However, \( T_1 \) is not a scalar multiple of \( T_2 \). To see this, note that
    \[
        T_1(0, 0, 0, 1, 0) = (1, 0) \quand T_2(0, 0, 0, 1, 0) = (0, 1),
    \]
    which are two linearly independent vectors in \( \R^2 \) and thus not scalar multiples of one another.
\end{solution}

\noindent \hrulefill

\noindent \hypertarget{ladr}{\textcolor{blue}{[LADR]} Axler, S. (2015) \textit{Linear Algebra Done Right.} 3\ts{rd} edition.}

\end{document}