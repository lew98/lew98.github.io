\documentclass[12pt]{article}
\usepackage[utf8]{inputenc}
\usepackage{amsmath}
\usepackage{amsthm}
\usepackage{geometry}
\usepackage{amsfonts}
\usepackage{bm}
\geometry{
headheight=15pt,
left=60pt,
right=60pt
}
\usepackage{fancyhdr}
\pagestyle{fancy}
\fancyhf{}
\lhead{}
\chead{Banach fixed-point theorem}
\rhead{}

\pagenumbering{gobble}
\setlength{\parindent}{0pt}

\newcommand{\newp}{\vspace{5mm}}

\theoremstyle{definition}
\newtheorem{theorem}{Theorem}

\newtheorem*{remark}{Remark}

\begin{document}

\section{Banach fixed-point theorem}

The following theorem is known as the \textit{Banach fixed-point theorem}, \textit{contractive mapping theorem}, or some variant thereof.

\begin{theorem}

Let \( (X, d) \) be a metric space and let \( f : X \to X \) be a contraction on \( X \), i.e.\ \( f \) is Lipschitz with Lipschitz constant \( 0 \leq L < 1 \). Then if \( f \) has a fixed point, this fixed point is unique. Furthermore, if \( X \) is non-empty and complete then \( f \) has a fixed point and this fixed point is given by \( \lim_{n\to\infty} x_n \), where \( x_n = f(x_{n-1}) \) for \( n \geq 1 \) and \( x_0 \) is any point in \( X \).

\end{theorem}

\begin{proof}
Suppose that \( x \) and \( y \) are fixed points of \( f \). Since \( f \) is a contraction, we must have
\[
    d(x, y) = d(f(x), f(y)) \leq L \, d(x, y),
\]
where \( 0 \leq L < 1 \). This can only be satisfied if \( d(x, y) = 0 \), i.e.\ if \( x = y \). So any fixed point of \( f \) must be unique.

\newp

Now suppose that \( X \) is non-empty and complete. Let \( x_0 \in X \) be arbitrary and set \( x_n = f(x_{n-1}) \) for \( n \geq 1 \). For any \( n \geq 0 \) we have the inequality
\begin{equation}
\label{ineq_1}
    \boxed{ d(x_{n+1}, x_n) \leq L^n \, d(x_1, x_0)}
\end{equation}
which can be seen by induction on \( n \):
\begin{align*}
    d(x_{n+1}, x_n) &= d(f(x_n), f(x_{n-1})) \\
    &\leq L \, d(x_n, x_{n-1}) \\
    &\cdots \\
    &\leq L^n \, d(x_1, x_0).
\end{align*}

Then for any \( n > m \geq 0 \) we apply inequality \eqref{ineq_1} as follows:
\begin{align*}
    d(x_n, x_m) &\leq d(x_n, x_{n-1}) + \cdots + d(x_{m+1}, x_m) \\
    &\leq (L^{n-1} + \cdots + L^m) \, d(x_1, x_0) \\
    &= L^m \, (L^{n-m-1} + \cdots + 1) \, d(x_1, x_0) \\
    &\leq L^m \, \left( \sum_{i=0}^{\infty} L^i \right) \, d(x_1, x_0) \\
    &= L^m \, \frac{d(x_1, x_0)}{1 - L},
\end{align*}
where we have used that \( 0 \leq L < 1 \). So for any \( n > m \geq 0 \) we have the inequality
\begin{equation}
\label{ineq_2}
    \boxed{ d(x_n, x_m) \leq L^m \, \frac{d(x_1, x_0)}{1 - L}.}
\end{equation}

Now let \( \varepsilon > 0 \) be given. Since \( 0 \leq L < 1 \), there exists a positive integer \( M \) such that
\[
    m \geq M \implies L^m \, \frac{d(x_1, x_0)}{1 - L} < \varepsilon.
\]
Then provided we take \( n > m \geq M \), inequality \eqref{ineq_2} gives us \( d(x_n, x_m) < \varepsilon \), demonstrating that \( (x_n)_{n=0}^{\infty} \) is a Cauchy sequence. By the completeness of \( X \), there then exists some \( x \in X \) such that \( \lim_{n\to\infty} x_n = x \). This \( x \) is the fixed point of \( f \):
\[
    f(x) = f\left( \lim_{n\to\infty} x_n \right) = \lim_{n\to\infty} f(x_n) = \lim_{n\to\infty} x_{n+1} = x. \qedhere
\]
\end{proof}

\section{A corollary}

\begin{theorem}

Let \( (X, d) \) be a non-empty and complete metric space and let \( f : X \to X \) be such that \( f^N \) is a contraction for some \( N \geq 1 \). Then \( f \) has a unique fixed point.

\end{theorem}

\begin{proof}
By Theorem 1, there exists a unique \( x \in X \) such that \( f^N(x) = x \). Observe that
\[
    d(f(x), x) = d(f^{N+1}(x), f^N(x)) \leq L \, d(f(x), x),
\]
where \( 0 \leq L < 1 \) is the Lipschitz constant of \( f^N \). This inequality can only be satisfied if \( d(f(x), x) = 0 \), i.e.\ if \( f(x) = x \). So \( x \) is also a fixed point of \( f \).

\newp

For the uniqueness of \( x \) as a fixed point of \( f \), suppose that \( y \) is a fixed point of \( f \). Then \( y \) must also be a fixed point of \( f^N \):
\[
    f^N(y) = f^{N-1}(f(y)) = f^{N-1}(y) = \cdots = f(y) = y.
\]
The uniqueness of \( x \) as a fixed point of \( f^N \) then implies \( x = y \).
\end{proof}

\end{document}
