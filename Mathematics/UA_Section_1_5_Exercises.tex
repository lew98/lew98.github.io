\documentclass[12pt]{article}
\usepackage[utf8]{inputenc}
\usepackage[utf8]{inputenc}
\usepackage{amsmath}
\usepackage{amsthm}
\usepackage{geometry}
\usepackage{amsfonts}
\usepackage{mathrsfs}
\usepackage{bm}
\usepackage{hyperref}
\usepackage[dvipsnames]{xcolor}
\usepackage{enumitem}
\usepackage{changepage}
\usepackage{lipsum}
\usepackage{tikz}
\usetikzlibrary{matrix}
\usepackage{tikz-cd}
\usepackage[nameinlink]{cleveref}
\geometry{
headheight=15pt,
left=60pt,
right=60pt
}
\usepackage{fancyhdr}
\pagestyle{fancy}
\fancyhf{}
\lhead{}
\chead{Section 1.5 Exercises}
\rhead{\thepage}
\hypersetup{
    colorlinks=true,
    linkcolor=blue,
    urlcolor=blue
}

\theoremstyle{definition}

\newtheorem*{remark}{Remark}

\newtheoremstyle{exercise}
    {}
    {}
    {}
    {}
    {\bfseries}
    {.}
    { }
    {\thmname{#1}\thmnumber{#2}\thmnote{ (#3)}}
\theoremstyle{exercise}
\newtheorem{exercise}{Exercise 1.5.}

\newtheoremstyle{solution}
    {}
    {}
    {}
    {}
    {\itshape\color{magenta}}
    {.}
    { }
    {\thmname{#1}\thmnote{ #3}}
\theoremstyle{solution}
\newtheorem*{solution}{Solution}

\Crefformat{exercise}{#2Exercise 1.5.#1#3}

\newcommand{\setcomp}[1]{#1^{\mathsf{c}}}
\newcommand{\N}{\mathbf{N}}
\newcommand{\Z}{\mathbf{Z}}
\newcommand{\Q}{\mathbf{Q}}
\newcommand{\R}{\mathbf{R}}
\newcommand{\C}{\mathbf{C}}

\setlist[enumerate,1]{label={(\alph*)}}

\begin{document}

\section{Section 1.5 Exercises}

Exercises with solutions from Section 1.5 of \hyperlink{ua}{[UA]}.

\begin{exercise}
\label{ex:1}
    Finish the following proof for Theorem 1.5.7.

    Assume \( B \) is a countable set. Thus, there exists \( f : \N \to B \) which is 1-1 and onto. Let \( A \subseteq B \) be an infinite subset of \( B \). We must show that \( A \) is countable.

    Let \( n_1 = \min \{ n \in \N : f(n) \in A \} \). As a start to a definition of \( g : \N \to A \), set \( g(1) = f(n_1) \). Show how to inductively continue this process to produce a 1-1 function \( g \) from \( \N \) onto \( A \).
\end{exercise}

\begin{solution}
    Set \( n_2 = \min (\{ n \in \N : f(n) \in A \} \setminus \{ n_1 \}), n_3 = \min (\{ n \in \N : f(n) \in A \} \setminus \{ n_1, n_2 \}) \), and in general for \( k \geq 2 \), \( n_k = \min (\{ n \in \N : f(n) \in A \} \setminus \{ n_1, \ldots, n_{k-1} \}) \). Since \( A \) is infinite and \( f \) is onto, the set \( \{ n \in \N : f(n) \in A \} \setminus \{ n_1, \ldots, n_{k-1} \} \) is always non-empty (indeed, it must be infinite), so the minimum exists and each \( n_k \) is well-defined. It is also clear from the definition of this sequence that \( n_1 < n_2 < n_3 < \cdots \). Now set \( g(k) = f(n_k) \) for each \( k \in \N \). Then \( g \) is 1-1 since
    \begin{align*}
        g(l) = g(k) &\iff f(n_l) = f(n_k) \\
        &\iff n_l = n_k \tag{f is 1-1} \\
        &\iff l = k. \tag{sequence is increasing}
    \end{align*}
    To see that \( g \) is onto, let \( a \in A \) be given. Since \( f \) is onto, there is a positive integer \( N \) such that \( f(N) = a \). Suppose that for all \( k \in \N \) we have \( N \neq n_k \). It cannot be the case that \( N < n_1 \), else \( n_1 \) would not be the minimum of \( \{ n \in \N : f(n) \in A \} \), so we must have \( n_1 < N \). Then there must exist some \( l \in \N \) such that \( n_l < N < n_{l+1} \); but this contradicts \( n_{l+1} \) being the minimum of the set \( \{ n \in \N : f(n) \in A \} \setminus \{ n_1, \ldots, n_l \} \). So in fact there must exist a \( k \in \N \) such that \( n_k = N \) and it follows that \( g(k) = f(n_k) = f(N) = a \).
\end{solution}

\begin{exercise}
\label{ex:2}
    Review the proof of Theorem 1.5.6, part (ii) showing that \( \R \) is uncountable, and then find the flaw in the following erroneous proof that \( \Q \) is uncountable:

    Assume, for contradiction, that \( \Q \) is countable. Thus we can write \( \Q = \{ r_1, r_2, r_3, \ldots \} \) and, as before, construct a nested sequence of closed intervals with \( r_n \not\in I_n \). Our construction implies \( \bigcap_{n=1}^{\infty} I_n = \emptyset \) while NIP implies \( \bigcap_{n=1}^{\infty} I_n \neq \emptyset \). This contradiction implies \( \Q \) must therefore be uncountable.
\end{exercise}

\begin{solution}
    The problem with this ``proof" is that \( \Q \) does not have the nested interval property. \href{https://lew98.github.io/Mathematics/Nested_interval_property.pdf}{It can be shown} that for an ordered field, the NIP and the least-upper-bound property are equivalent, so that \( \R \) is the only ordered field with the NIP.
\end{solution}

\begin{exercise}
\label{ex:3}
    Use the following outline to supply proofs for the statements in Theorem 1.5.8.
    \begin{enumerate}
        \item First, prove statement (i) for two countable sets, \( A_1 \) and \( A_2 \). Example 1.5.3 (ii) may be a useful reference. Some technicalities can be avoided by first replacing \( A_2 \) with the set \( B_2 = A_2 \setminus A_1 = \{ x \in A_2 : x \not\in A_1 \} \). The point of this is that the union \( A_1 \cup B_2 \) is equal to \( A_1 \cup A_2 \) and the sets \( A_1 \) and \( B_2 \) are disjoint. (What happens if \( B_2 \) is finite?)

        Now, explain how the more general statement in (i) follows.

        \item Explain why induction \textit{cannot} be used to prove part (ii) of Theorem 1.5.8 from part (i).

        \item Show how arranging \( \N \) into the two-dimensional array
        \[
            \begin{matrix}
            1 & 3 & 6 & 10 & 15 & \cdots \\
            2 & 5 & 9 & 14 & \cdots &  \\
            4 & 8 & 13 & \cdots &  &  \\
            7 & 12 & \cdots &   &  &  \\
            11 & \cdots &  &  &  &  \\
            \vdots &  &  &  &  & 
            \end{matrix}
        \]
        leads to a proof of Theorem 1.5.8 (ii).
    \end{enumerate}
\end{exercise}

\begin{solution}
    \begin{enumerate}
        \item Since \( A_1 \) is countable, there exists a 1-1 and onto function \( f : \N \to A_1 \). It will suffice to show that \( A_1 \cup B_2 \) is countable. First, suppose that \( B_2 \) is empty. Then \( A_1 \cup B_2 = A_1 \) which is countable by assumption. Next, suppose that \( B_2 \) is nonempty and finite, say \( B_2 = \{ x_1, \ldots, x_k \} \) for some \( k \in \N \). Define \( g : \N \to A_1 \cup B_2 \) by
        \[
            g(n) = \begin{cases}
                x_n & \text{if } 1 \leq n \leq k, \\
                f(n - k) & \text{if } k < n.
            \end{cases}
        \]
        That \( g \) is 1-1 follows since \( A_1 \) and \( B_2 \) are disjoint and \( f \) is 1-1. It is clear that \( g \) is onto since \( f \) is onto.

        Finally, suppose that \( B_2 \) is infinite. Since \( B_2 \) is a subset of the countable set \( A_2 \), \Cref{ex:1} implies that \( B_2 \) is countable, i.e.\ there exists a function \( h : \N \to B_2 \) which is 1-1 and onto. Define \( g : \N \to A_1 \cup B_2 \) by
        \[
            g(n) = \begin{cases}
                f \left( \tfrac{n}{2} \right) & \text{if } n \text{ is even}, \\
                h \left( \tfrac{n+1}{2} \right) & \text{if } n \text{ is odd}.
            \end{cases}
        \]
        To see that \( g \) is 1-1, suppose that \( m \neq n \) are positive integers. If both of \( m, n \) are even then \( g(m) \neq g(n) \) since \( f \) is 1-1, if both of \( m, n \) are odd then \( g(m) \neq g(n) \) since \( h \) is 1-1, and if one of \( m, n \) is even and the other is odd then \( g(m) \neq g(n) \) since \( f \) maps into \( A_1 \), \( h \) maps into \( B_2 \), and \( A_1 \cap B_2 = \emptyset \).

        To see that \( g \) is onto, let \( x \in A_1 \cup B_2 \) be given. Since \( A_1 \cap B_2 = \emptyset \), exactly one of \( x \in A_1 \) or \( x \in B_2 \) holds. Suppose \( x \in A_1 \). Then since \( f \) is onto, there is a positive integer \( N \) such that \( f(N) = x \); it follows that \( g(2N) = f(N) = x \). If \( x \in B_2 \), then since \( h \) is onto there exists a positive integer \( N \) such that \( h(N) = x \). Then \( g(2N - 1) = h(N) = x \).
        
        We may conclude that \( g \) is 1-1 and onto and hence that \( A_1 \cup B_2 \) is countable.

        A simple induction argument proves the more general statement in Theorem 1.5.8 (i). Let \( P(n) \) be the statement that for countable sets \( A_1, \ldots, A_n \), the union \( A_1 \cup \cdots \cup A_n \) is countable. The truth of \( P(1) \) is clear. Suppose that \( P(n) \) holds for some \( n \in \N \) and suppose we have countable sets \( A_1, \ldots, A_{n+1} \). Let \( A' = A_1 \cup \cdots \cup A_n \). Then by the induction hypothesis, \( A' \) is countable. Observe that
        \[
            A_1 \cup \cdots \cup A_n \cup A_{n+1} = A' \cup A_{n+1}.
        \]
        Since \( A' \) and \( A_{n+1} \) are countable, the union \( A' \cup A_{n+1} \) is also countable by the previous discussion, i.e.\ \( P(n+1) \) holds. Hence we may conclude that, by induction, \( P(n) \) is true for all \( n \in \N \).

        \item Induction can only be used to show that a particular statement \( P(n) \) holds for each value of \( n \in \N \).

        \item Since each \( A_n \) is countable, for each \( n \in \N \) there exists a function \( f_n : \N \to A_n \) which is 1-1 and onto. Let \( a_{mn} = f_n(m) \) and arrange these into another two-dimensional array like so:
        \[
            \begin{matrix}
            a_{11} & a_{12} & a_{13} & a_{14} & a_{15} & \cdots \\
            a_{21} & a_{22} & a_{23} & a_{24} & \cdots &  \\
            a_{31} & a_{32} & a_{33} & \cdots &  &  \\
            a_{41} & a_{42} & \cdots &   &  &  \\
            a_{51} & \cdots &  &  &  &  \\
            \vdots &  &  &  &  & 
            \end{matrix}
            \qquad
            \begin{matrix}
            1 & 3 & 6 & 10 & 15 & \cdots \\
            2 & 5 & 9 & 14 & \cdots &  \\
            4 & 8 & 13 & \cdots &  &  \\
            7 & 12 & \cdots &   &  &  \\
            11 & \cdots &  &  &  &  \\
            \vdots &  &  &  &  & 
            \end{matrix}
        \]
        Since each \( f_n \) is onto, each element of \( \bigcup_{n=1}^{\infty} A_n \) appears somewhere in the left array. We define a function \( g : \bigcup_{n=1}^{\infty} A_n \to \N \) by working through the grid along the diagonals (first \( a_{11} \), then \( a_{21} \), then \( a_{12} \), then \( a_{31} \), and so on), mapping an element \( a_{mn} \) to the natural number appearing in the corresponding position in the right array. Since the \( A_n \)'s may have elements in common, if we encounter an element \( a_{mn} \) that we have already seen before, we simply skip this element and move on to the next one. In this way, we obtain a 1-1 function \( g \). If we denote the range of \( g \) by \( B \subseteq \N \), then \( g : \bigcup_{n=1}^{\infty} A_n \to B \) is both 1-1 and onto. Since \( A_1 \subseteq \bigcup_{n=1}^{\infty} A_n \), \( A_1 \) is infinite, and \( g \) is 1-1, it follows that \( B \) is infinite. Then by \Cref{ex:1}, \( B \) must be countable, i.e.\ there is a function \( h : \N \to B \) which is 1-1 and onto. Then the function \( g^{-1} \circ h : \N \to \bigcup_{n=1}^{\infty} A_n \) is 1-1 and onto; we may conclude that \( \bigcup_{n=1}^{\infty} A_n \) is countable.
    \end{enumerate}
\end{solution}

\begin{exercise}
\label{ex:4}
    \begin{enumerate}
        \item Show \( (a, b) \sim \R \) for any interval \( (a, b) \).

        \item Show that an unbounded interval like \( (a, \infty) = \{ x : x > a \} \) has the same cardinality as \( \R \) as well.

        \item Using open intervals makes it more convenient to produce the required 1-1, onto functions, but it is not really necessary. Show that \( [0, 1) \sim (0, 1) \) by exhibiting a 1-1 onto function between the two sets.
    \end{enumerate}
\end{exercise}

\begin{solution}
    \begin{enumerate}
        \item Let \( f : (-1, 1) \to \R \) be given by \( f(x) = \tfrac{x}{x^2 - 1} \). Then \( f \) is 1-1 and onto (see Example 1.5.4 in \hyperlink{ua}{[UA]}). Now let \( g : (a, b) \to (-1, 1) \) be given by \( g(x) = \tfrac{2(x - a)}{b - a} - 1 \). It is easily verified that \( g \) is 1-1 and onto. Then \( f \circ g : (a, b) \to \R \) is 1-1 and onto (see \Cref{ex:5}), so that \( (a, b) \sim \R \).

        \item Let \( f : (a, \infty) \to (0, 1) \) be given by \( f(x) = \tfrac{1}{x + 1 - a} \). It is easily verified that \( f \) is 1-1 and onto, so that \( (a, \infty) \sim (0, 1) \). By part (a) we have \( (0, 1) \sim \R \); it follows that \( (a, \infty) \sim \R \) (see \Cref{ex:5}).

        \item It is clear that \( [0, 1) \sim (0, 1] \) via the map \( x \mapsto 1 - x \). Define a function \( f : (0, 1) \to (0, 1] \) by
        \[
            f(x) = \begin{cases}
                \tfrac{1}{n} & \text{if } x = \tfrac{1}{n + 1} \text{ for some } n \in \N, \\
                x & \text{otherwise}.
            \end{cases}
        \]
        This function is 1-1 and onto since it has an inverse \( f^{-1} : (0, 1] \to (0, 1) \) given by
        \[
            f^{-1}(x) = \begin{cases}
                \tfrac{1}{n + 1} & \text{if } x = \tfrac{1}{n} \text{ for some } n \in \N, \\
                x & \text{otherwise}.
            \end{cases}
        \]
        It follows that \( (0, 1) \sim (0, 1] \) and hence that \( (0, 1) \sim [0, 1) \) (see \Cref{ex:5}).
    \end{enumerate}
\end{solution}

\begin{exercise}
\label{ex:5}
    \begin{enumerate}
        \item Why is \( A \sim A \) for every set \( A \)?

        \item Given sets \( A \) and \( B \), explain why \( A \sim B \) is equivalent to asserting \( B \sim A \).

        \item For three sets \( A, B, \) and \( C \), show that \( A \sim B \) and \( B \sim C \) implies \( A \sim C \). These three properties are what is meant by saying that \( \sim \) is an \textit{equivalence relation}.
    \end{enumerate}
\end{exercise}

\begin{solution}
    \begin{enumerate}
        \item The function \( f : A \to A \) given by \( f(x) = x \) is clearly 1-1 and onto.

        \item Since \( A \sim B \), there is a 1-1 and onto function \( f : A \to B \). A function is 1-1 and onto if and only if it has an inverse function \( f^{-1} : B \to A \) which must also be 1-1 and onto. It follows that \( B \sim A \). The equivalence is given by swapping the roles of \( A \) and \( B \).

        \item There are 1-1 and onto functions \( f : A \to B \) and \( g : B \to C \). It follows that the composite function \( g \circ f : A \to C \) is also 1-1 and onto.
    \end{enumerate}
\end{solution}

\begin{exercise}
\label{ex:6}
    \begin{enumerate}
        \item Give an example of a countable collection of disjoint open intervals.

        \item Give an example of an uncountable collection of disjoint open intervals, or argue that no such collection exists.
    \end{enumerate}
\end{exercise}

\begin{solution}
    \begin{enumerate}
        \item Take \( A_n = (n, n+1) \) for \( n \in \N \).

        \item No such collection exists. To see this, suppose there was such a collection \( \{ I_a : a \in A \} \) for some uncountable set \( A \). By the density of \( \Q \) in \( \R \), there exists a rational number \( r_a \in I_a \) for each \( a \in A \). Since the intervals are disjoint, each \( r_a \) must be distinct and hence the collection \( \{ r_a : a \in A \} \) must be an uncountable subset of \( \Q \); but this contradicts \Cref{ex:1}.
    \end{enumerate}
\end{solution}

\begin{exercise}
\label{ex:7}
    Consider the open interval \( (0, 1) \), and let \( S \) be the set of points in the open unit square; that is, \( S = \{ (x, y) : 0 < x, y < 1 \} \).
    \begin{enumerate}
        \item Find a 1-1 function that maps \( (0, 1) \) into, but not necessarily onto, \( S \). (This is easy.)

        \item Use the fact that every real number has a decimal expansion to produce a 1-1 function that maps \( S \) into \( (0, 1) \). Discuss whether the formulated function is onto. (Keep in mind that any terminating decimal expansion such as \( .235 \) represents the same real number as \( .234999 \ldots  \) .)
    \end{enumerate}
    The Schröder-Bernstein theorem discussed in \Cref{ex:11} can now be applied to conclude that \( (0, 1) \sim S \).
\end{exercise}

\begin{solution}
    \begin{enumerate}
        \item Take \( f : (0, 1) \to S \) given by \( f(x) = (x, 1/2) \).

        \item For \( (x, y) \in S \), suppose \( x \) has decimal representation \( 0.x_1 x_2 x_3 \ldots \) and \( y \) has decimal representation \( 0.y_1 y_2 y_3 \ldots  \), where if necessary we choose the decimal representation terminating in 0s. To define \( g : S \to (0, 1) \), let \( g(x, y) = 0.x_1 y_1 x_2 y_2 x_3 y_3 \ldots \) . To see that \( g \) is 1-1, suppose we have \( (x, y) \neq (a, b) \) in \( S \). Then at least one of \( x \neq a \) or \( y \neq b \) holds. Suppose that \( x \neq a \) (the case where \( y \neq b \) is similar), so that there is some index \( n \) such that \( x_n \neq a_n \). Then if \( g(x, y) \) has decimal representation \( 0.s_1 s_2 s_3 \ldots \) and \( g(a, b) \) has decimal representation \( 0.t_1 t_2 t_3 \ldots \), we have \( s_{2n - 1} = x_n \neq a_n = t_{2n - 1} \). This implies that \( g(x, y) \neq g(a, b) \), provided \( g(x, y) \) does not terminate in 0s and \( g(a, b) \) does not terminate in 9s, or vice versa. To rule this out, suppose that \( g(a, b) \) does terminate in 9s (the case where \( g(x, y) \) terminates in 9s is similarly handled). Simply observe that this implies that both \( a \) and \( b \) terminate in 9s; but our construction specifically chooses the decimal representation terminating in 0s if necessary.

        This function \( g \) is not onto; \( g(x, y) = 0.1000\ldots \) implies that \( y = 0.000 \ldots \), but \( (x, 0) \not\in S \) for any \( x \in (0, 1) \).
    \end{enumerate}
\end{solution}

\begin{exercise}
\label{ex:8}
    Let \( B \) be a set of positive real numbers with the property that adding together any finite subset of elements from \( B \) always gives a sum of 2 or less. Show \( B \) must be finite or countable.
\end{exercise}

\begin{solution}
    Suppose \( a \in (0, 1] \). We claim that \( B \cap (a, 2] \) must be a (possibly empty) finite set. By the Archimedean property of \( \R \), there is an \( n \in \N \) such that \( na > 2 \). Suppose that \( B \cap (a, 2] \) contains at least \( n \) elements, say \( \{ b_1, \ldots, b_n \} \). Then since each \( b_i > a \), we have
    \[
        b_1 + \cdots + b_n > na > 2.
    \]
    This contradicts our hypotheses, so it must be the case that \( B \cap (a, 2] \) contains less than \( n \) elements, so that \( B \cap (a, 2] \) is finite.

    Any element of \( B \) must be less than or equal to 2, so \( B \subseteq (0, 2] \). It follows that
    \[
        B = \bigcup_{n=1}^{\infty} (B \cap (1/n, 2]).
    \]
    So we have expressed \( B \) as a countable union of finite sets; this implies that \( B \) is either finite or countable.
\end{solution}

\begin{exercise}
\label{ex:9}
    A real number \( x \in \R \) is called \textit{algebraic} if there exist integers \( a_0, a_1, a_2, \ldots, a_n \in \Z \), not all zero, such that
    \[
        a_n x^n + a_{n-1} x^{n-1} + \cdots + a_1 x + a_0 = 0.
    \]
    Said another way, a real number is algebraic if it is the root of a polynomial with integer coefficients. Real numbers that are not algebraic are called \textit{transcendental} numbers. Reread the last paragraph of Section 1.1. The final question posed here is closely related to the question of whether or not transcendental numbers exist.
    \begin{enumerate}
        \item Show that \( \sqrt{2}, \sqrt[3]{2} \), and \( \sqrt{3} + \sqrt{2} \) are algebraic.

        \item Fix \( n \in \N \), and let \( A_n \) be the algebraic numbers obtained as roots of polynomials with integer coefficients that have degree \( n \). Using the fact that every polynomial has a finite number of roots, show that \( A_n \) is countable.

        \item Now, argue that the set of all algebraic numbers is countable. What may we conclude about the set of transcendental numbers?
    \end{enumerate}
\end{exercise}

\begin{solution}
    \begin{enumerate}
        \item One can verify that \( \sqrt{2} \) is a root of the polynomial \( x^2 - 2 \), \( \sqrt[3]{2} \) is a root of the polynomial \( x^3 - 2 \), and \( \sqrt{3} + \sqrt{2} \) is a root of the polynomial \( x^4 - 10 x^2 + 1 \).

        \item Let \( P_n \) be the collection of polynomials with integer coefficients that have degree \( n \), i.e.\
        \[
            P_n = \{ a_n x^n + a_{n-1} x^{n-1} + \cdots + a_1 x + a_0 : a_n, \ldots, a_0 \in \Z, a_n \neq 0 \}. 
        \]
        It is not hard to see that
        \[
            P_n \sim (\Z \setminus \{ 0 \}) \times \underbrace{\Z \times \cdots \times \Z}_{n \text{ times}}.
        \]
        A corollary of Theorem 1.5.8 (ii) is that finite Cartesian products of countable sets are themselves countable (see Corollary 8 \href{https://lew98.github.io/Mathematics/Cardinality.pdf}{here}); it follows that \( P_n \) is countable. For a polynomial \( p \in P_n \), let \( R_p \) be the set of its roots, i.e.\ \( R_p = \{ x \in \R : p(x) = 0 \} \), and note that \( R_p \) is always a finite set. Now observe that
        \[
            A_n = \bigcup_{p \in P_n} R_p.
        \]
        So we have expressed \( A_n \) as a countable union of finite sets. It follows that \( A_n \) is either finite or countable. Since \( k^{1/n} \in A_n \) for each \( k \in \N \) (it is a root of the polynomial \( x^n - k \)), we see that \( A_n \) must be infinite and hence countable.

        \item If we let \( A \) be the set of all algebraic numbers, then \( A = \bigcup_{n=1}^{\infty} A_n \), a countable union of countable sets. It follows that \( A \) is countable.

        A consequence of this is that the set of transcendental numbers \( \setcomp{A} \) must be uncountable. To see this, simply note that: \( \R = A \cup \setcomp{A} \); the union of two countable sets is countable; and \( \R \) is not countable.
    \end{enumerate}
\end{solution}

\begin{exercise}
\label{ex:10}
    \begin{enumerate}
        \item Let \( C \subseteq [0, 1] \) be uncountable. Show that there exists \( a \in (0, 1) \) such that \( C \cap [a, 1] \) is uncountable.

        \item Now let \( A \) be the set of all \( a \in (0, 1) \) such that \( C \cap [a, 1] \) is uncountable, and let \( \alpha = \sup A \). Is \( C \cap [\alpha, 1] \) an uncountable set?

        \item Does the statement in (a) remain true if ``uncountable" is replaced by ``infinite"?
    \end{enumerate}
\end{exercise}

\begin{solution}
    \begin{enumerate}
        \item Suppose that for each \( a \in (0, 1) \), the set \( C \cap [a, 1] \) is countable. Then we can express \( C \) as a countable union of countable sets:
        \[
            C = \bigcup_{n=2}^{\infty} (C \cap [1/n, 1]),
        \]
        This implies that \( C \) is countable. So if \( C \) is uncountable, there must exist some \( a \in (0, 1) \) such that \( C \cap [a, 1] \) is uncountable. 

        \item Not necessarily. Suppose \( C = [0, 1] \). Then for all \( a \in (0, 1) \), we have that \( C \cap [a, 1] = [a, 1] \) is uncountable, so that \( A = (0, 1) \). Then \( \alpha = \sup A = 1 \), but \( C \cap [\alpha, 1] = \{ 1 \} \) is not uncountable.

        \item The statement is no longer true in general. Consider \( C = \{ 1/n : n \in \N \} \). Then no matter which \( a \in (0, 1) \) we choose, \( C \cap [a, 1] \) is a finite set (since there are finitely many positive integers less than or equal to \( 1/a \)).
    \end{enumerate}
\end{solution}

\begin{exercise}[Schröder-Bernstein Theorem]
\label{ex:11}
    Assume there exists a 1-1 function \( f : X \to Y \) and another 1-1 function \( g : Y \to X \). Follow the steps to show that there exists a 1-1, onto function \( h : X \to Y \) and hence \( X \sim Y \).

    The strategy is to partition \( X \) and \( Y \) into components
    \[
        X = A \cup A' \quad \text{and} \quad Y = B \cup B'
    \]
    with \( A \cap A' = \emptyset \) and \( B \cap B' = \emptyset \), in such a way that \( f \) maps \( A \) onto \( B \), and \( g \) maps \( B' \) onto \( A' \).
    \begin{enumerate}
        \item Explain how achieving this would lead to a proof that \( X \sim Y \).

        \item Set \( A_1 = X \setminus g(Y) = \{ x \in X : x \not\in g(Y) \} \) (what happens if \( A_1 = \emptyset \)?) and inductively define a sequence of sets by letting \( A_{n+1} = g(f(A_n)) \). Show that \( \{ A_n : n \in \N \} \) is a pairwise disjoint collection of subsets of \( X \), while \( \{ f(A_n) : n \in \N \} \) is a similar collection in \( Y \).

        \item Let \( A = \bigcup_{n=1}^{\infty} A_n \) and \( B = \bigcup_{n=1}^{\infty} f(A_n) \). Show that \( f \) maps \( A \) onto \( B \).

        \item Let \( A' = X \setminus A \) and \( B' = Y \setminus B \). Show \( g \) maps \( B' \) onto \( A' \).
    \end{enumerate}
\end{exercise}

\begin{solution}
    \begin{enumerate}
        \item Abusing notation slightly, we have 1-1 and onto functions \( f : A \to B \) and \( g : B' \to A' \), and their inverses \( f^{-1} : B \to A \) and \( g^{-1} : A' \to B' \). Since \( A \cap A' = \emptyset \) and \( B \cap B' = \emptyset \), the functions \( h : X \to Y \) and \( h' : Y \to X \) given by
        \[
            h(x) = \begin{cases}
                f(x) & \text{if } x \in A, \\
                g^{-1}(x) & \text{if } x \in A',
            \end{cases}
            \qquad
            h'(y) = \begin{cases}
                f^{-1}(y) & \text{if } y \in B, \\
                g(y) & \text{if } y \in B'
            \end{cases}
        \]
        are well-defined and mutual inverses. It follows that \( X \sim Y \).

        \item If \( A_1 = \emptyset \), then \( X = g(Y) \) i.e.\ \( g \) is onto. Then since \( g \) is assumed to be 1-1, we could immediately conclude that \( X \sim Y \) via \( g \).

        Let \( P(n) \) be the statement that \( \{ A_1, \ldots, A_n \} \) is a pairwise disjoint collection of sets. The truth of \( P(1) \) is clear. Suppose that \( P(n) \) holds for some \( n \in \N \). Then to show that \( P(n+1) \) is true, we need to show that for all \( 1 \leq k \leq n \), \( A_k \cap A_{n+1} = \emptyset \). It is clear that \( A_1 \cap A_{n+1} = \emptyset \) since \( A_{n+1} \subseteq g(Y) \). Suppose that \( 2 \leq k \leq n \). Then observe that
        \begin{align*}
            A_k \cap A_{n+1} &= g(f(A_{k-1})) \cap g(f(A_n)) \\
            &= g(f(A_{k-1} \cap A_n)) \tag{\( f \) and \( g \) are 1-1} \\
            &= g(f(\emptyset)) \tag{induction hypothesis} \\
            &= \emptyset.
        \end{align*}
        Hence \( P(n+1) \) holds. Then by induction, \( P(n) \) holds for all \( n \in \N \). This implies that \( \{ A_n : n \in \N \} \) is a pairwise disjoint collection of sets. Since \( f \) is 1-1, we immediately have that \( \{ f(A_n) : n \in \N \} \) is also a pairwise disjoint collection of sets.

        \item \( f : A \to B \) is 1-1 since \( f : X \to Y \) is 1-1 and it is clear from the definition of \( A \) and \( B \) that \( f : A \to B \) really does map into \( B \) and in fact is onto.

        \item Again, it is clear that \( g : B' \to A' \) is 1-1. That \( g \) maps \( B' \) into \( A' \) follows since
        \begin{align*}
            b \in B' & \iff \forall n \in \N,\, b \not\in f(A_n) \\
            & \iff \forall n \in \N,\, g(b) \not\in g(f(A_n)) \tag{\( g \) is 1-1} \\
            & \iff \forall n \in \N,\, g(b) \not\in A_{n+1} \\
            & \iff \forall n \geq 2,\, g(b) \not\in A_n.
        \end{align*}
        Since \( A_1 \) is the complement of the image of \( Y \) under \( g \), it follows that \( g(y) \not\in A_1 \) for any \( y \in Y \). Hence
        \[
            b \in B' \iff \forall n \in \N,\, g(b) \not\in A_n \iff g(b) \in A'.
        \]
        Furthermore, \( g : B' \to A' \) is onto since for any \( a \in A' \) we have \( a \not\in A_1 \iff a \in g(Y) \), so that \( a = g(y) \) for some \( y \in Y \). The chain of biconditionals above then shows that \( y \in B' \).
    \end{enumerate}
\end{solution}

\noindent \hrulefill

\noindent \hypertarget{ua}{\textcolor{blue}{[UA]} Abbott, S. (2015) \textit{Understanding Analysis.} 2nd edn.}

\end{document}
