\documentclass[12pt]{article}
\usepackage[utf8]{inputenc}
\usepackage[utf8]{inputenc}
\usepackage{amsmath}
\usepackage{amsthm}
\usepackage{geometry}
\usepackage{amsfonts}
\usepackage{mathrsfs}
\usepackage{bm}
\usepackage{hyperref}
\usepackage[dvipsnames]{xcolor}
\usepackage{enumitem}
\usepackage{mathtools}
\usepackage{changepage}
\usepackage{lipsum}
\usepackage{tikz}
\usetikzlibrary{matrix}
\usepackage{tikz-cd}
\usepackage[nameinlink]{cleveref}
\geometry{
headheight=15pt,
left=60pt,
right=60pt
}
\usepackage{fancyhdr}
\pagestyle{fancy}
\fancyhf{}
\lhead{}
\chead{Section 1.C Exercises}
\rhead{\thepage}
\hypersetup{
    colorlinks=true,
    linkcolor=blue,
    urlcolor=blue
}

\theoremstyle{definition}
\newtheorem*{remark}{Remark}

\newtheoremstyle{exercise}
    {}
    {}
    {}
    {}
    {\bfseries}
    {.}
    { }
    {\thmname{#1}\thmnumber{#2}\thmnote{ (#3)}}
\theoremstyle{exercise}
\newtheorem{exercise}{Exercise 1.C.}

\newtheoremstyle{solution}
    {}
    {}
    {}
    {}
    {\itshape\color{magenta}}
    {.}
    { }
    {\thmname{#1}\thmnote{ #3}}
\theoremstyle{solution}
\newtheorem*{solution}{Solution}

\Crefformat{exercise}{#2Exercise 1.C.#1#3}

\newcommand{\setcomp}[1]{#1^{\mathsf{c}}}
\newcommand{\N}{\mathbf{N}}
\newcommand{\Z}{\mathbf{Z}}
\newcommand{\Q}{\mathbf{Q}}
\newcommand{\R}{\mathbf{R}}
\newcommand{\C}{\mathbf{C}}
\newcommand{\F}{\mathbf{F}}

\DeclarePairedDelimiter\abs{\lvert}{\rvert}
% Swap the definition of \abs* and \norm*, so that \abs
% and \norm resizes the size of the brackets, and the 
% starred version does not.
\makeatletter
\let\oldabs\abs
\def\abs{\@ifstar{\oldabs}{\oldabs*}}
%
\let\oldnorm\norm
\def\norm{\@ifstar{\oldnorm}{\oldnorm*}}
\makeatother

\setlist[enumerate,1]{label={(\alph*)}}

\begin{document}

\section{Section 1.C Exercises}

Exercises with solutions from Section 1.C of \hyperlink{ladr}{[LADR]}.

\begin{exercise}
\label{ex:1}
    For each of the following subsets of \( \F^3 \), determine whether it is a subspace of \( \F^3 \):
    \begin{enumerate}
        \item \( \{ (x_1, x_2, x_3) \in \F^3 : x_1 + 2 x_2 + 3 x_3 = 0 \} \);

        \item \( \{ (x_1, x_2, x_3) \in \F^3 : x_1 + 2 x_2 + 3 x_3 = 4 \} \);

        \item \( \{ (x_1, x_2, x_3) \in \F^3 : x_1 x_2 x_3 = 0 \} \);

        \item \( \{ (x_1, x_2, x_3) \in \F^3 : x_1 = 5 x_3 \} \).
    \end{enumerate}
\end{exercise}

\begin{solution}
    Let \( U \) denote the set in each part of this question.
    \begin{enumerate}
        \item This is a subspace of \( \F^3 \). Clearly the zero vector belongs to \( U \). Suppose that \( x = (x_1, x_2, x_3) \) and \( y = (y_1, y_2, y_3) \) belong to \( U \) and \( \alpha \in \F \). Then
        \begin{gather*}
            (x_1 + y_1) + 2(x_2 + y_2) + 3(x_3 + y_3) = (x_1 + 2 x_2 + 3 x_3) + (y_1 + 2 y_2 + 3 y_3) = 0 + 0 = 0, \\[2mm]
            \alpha x_1 + 2 (\alpha x_2) + 3 (\alpha x_3) = \alpha (x_1 + 2 x_2 + 3 x_3) = \alpha 0 = 0.
        \end{gather*}
        So \( x + y \) and \( \alpha x \) belong to \( U \) and hence \( U \) is a subspace of \( V \) by (1.34).

        \item This is not a subspace of \( \F^3 \) since it does not contain the zero vector.

        \item This is not a subspace of \( \F^3 \) since it is not closed under addition; \( (1, 1, 0) \) and \( (0, 0, 1) \) belong to \( U \) but \( (1, 1, 0) + (0, 0, 1) = (1, 1, 1) \) does not belong to \( U \).

        \item This is a subspace of \( \F^3 \); observe that \( U = \{ (x_1, x_2, x_3) \in \F^3 : x_1 - 5 x_3 = 0 \} \). Clearly the zero vector belongs to \( U \). Suppose that \( x = (x_1, x_2, x_3) \) and \( y = (y_1, y_2, y_3) \) belong to \( U \) and \( \alpha \in \F \). Then
        \begin{gather*}
            (x_1 + y_1) - 5(x_3 + y_3) = (x_1 - 5 x_3) + (y_1 - 5 y_3) = 0 + 0 = 0, \\[2mm]
            \alpha x_1 - 5 (\alpha x_3) = \alpha (x_1 - 5 x_3) = \alpha 0 = 0.
        \end{gather*}
        So \( x + y \) and \( \alpha x \) belong to \( U \) and hence \( U \) is a subspace of \( V \) by (1.34).
    \end{enumerate}
\end{solution}

\begin{exercise}
\label{ex:2}
    Verify all the assertions in Example 1.35.
\end{exercise}

\begin{solution}
    \begin{enumerate}
        \item The assertion is that if \( b \in \F \), then
        \[
            U = \{ (x_1, x_2, x_3, x_4) \in \F^4 : x_3 = 5 x_4 + b \} = \{ (x_1, x_2, x_3, x_4) \in \F^4 : x_3 - 5 x_4 = b \}
        \]
        is a subspace of \( \F^4 \) if and only if \( b = 0 \). If \( b \neq 0 \), then \( U \) is not a subspace of \( \F^4 \) because the zero vector does not belong to \( U \). If \( b = 0 \), then similar reasoning to \Cref{ex:1} (d) shows that \( U \) is a subspace of \( \F^4 \).

        \item The assertion is that the set of continuous real-valued functions on the interval \( [0, 1] \) is a subspace of \( \R^{[0, 1]} \), i.e.\
        \[
            U = \{ f : [0, 1] \to \R, f \text{ continuous} \}
        \]
        is a subspace of \( \R^{[0, 1]} \). Clearly the zero function \( 0 : [0, 1] \to \R, x \mapsto 0 \) is continuous and hence belongs to \( U \). Suppose that \( f, g \in U \) and \( \alpha \in \R \). Then it is well-known that \( f + g \) and \( \alpha f \) are continuous functions on \( [0, 1] \), i.e.\ \( f + g \in U \) and \( \alpha f \in U \). Hence \( U \) is a subspace of \( \R^{[0, 1]} \) by (1.34).

        \item The assertion is that the set of differentiable real-valued functions on \( \R \) is a subspace of \( \R^{\R} \). The function \( 0 : \R \to \R, x \mapsto 0 \) is certainly differentiable, and it is well-known that the sum of differentiable functions and constant multiples of differentiable functions are also differentiable. Hence the assertion holds by (1.34).

        \item The assertion is that the set \( U \) of differentiable real-valued functions \( f \) on the interval \( (0, 3) \) such that \( f'(2) = b \) is a subspace of \( \R^{(0, 3)} \) if and only if \( b = 0 \). First, suppose that \( b \neq 0 \). Then \( U \) is not a subspace since the zero function \( 0 : (0, 3) \to \R, x \mapsto 0 \) certainly does not satisfy \( 0'(2) = b \). Now suppose that \( b = 0 \); then the zero function satisfies \( 0'(2) = b \) and hence belongs to \( U \). If \( f, g \in U \) and \( \alpha \in \R \), then
        \begin{gather*}
            (f + g)'(2) = f'(2) + g'(2) = 0 + 0 = 0, \\[2mm]
            (\alpha f)'(2) = \alpha f'(2) = \alpha 0 = 0.
        \end{gather*}
        Hence \( f + g \) and \( \alpha f \) belong to \( U \) and so \( U \) is a subspace of \( \R^{(0, 3)} \) by (1.34).

        \item The assertion is that the set \( U \) of all sequences of complex numbers with limit 0 is a subspace of \( \C^{\infty} \). The zero sequence \( (0, 0, 0, \ldots) \) certainly has limit 0 and so belongs to \( U \). Suppose that \( x = (x_n) \) and \( y = (y_n) \) belong to \( U \) and \( \alpha \in \C \). Then
        \begin{gather*}
            \lim (x_n + y_n) = \lim x_n + \lim y_n = 0 + 0 = 0, \\[2mm]
            \lim (\alpha x_n) = \alpha \lim x_n = \alpha 0 = 0.
        \end{gather*}
        Hence \( x + y \) and \( \alpha x \) belong to \( U \) and so \( U \) is a subspace of \( \C^{\infty} \) by (1.34).
    \end{enumerate}
\end{solution}

\begin{exercise}
\label{ex:3}
    Show that the set of differentiable real-valued functions \( f \) on the interval \( (-4, 4) \) such that \( f'(-1) = 3 f(2) \) is a subspace of \( \R^{(-4, 4)} \).
\end{exercise}

\begin{solution}
    Let \( U \) be the set in question. The zero function \( 0 : (-4, 4) \to \R, x \mapsto 0 \) belongs to \( U \) since \( 0'(-1) = 0 \) and \( 0(2) = 0 \). Suppose that \( f, g \in U \) and \( \alpha \in \R \). Then
    \begin{gather*}
        (f + g)'(-1) = f'(-1) + g'(-1) = 3 f(2) + 3 g(2) = 3 (f(2) + g(2)) = 3 (f + g)(2), \\[2mm]
        (\alpha f)'(-1) = \alpha f'(-1) = \alpha (3 f(2)) = 3 (\alpha f(2)) = 3 (\alpha f)(2).
    \end{gather*}
    Hence \( f + g \) and \( \alpha f \) belong to \( U \) and so \( U \) is a subspace of \( \R^{(-4, 4)} \) by (1.34).
\end{solution}

\begin{exercise}
\label{ex:4}
    Suppose \( b \in \R \). Show that the set of continuous real-valued functions \( f \) on the interval \( [0, 1] \) such that \( \int_0^1 f = b \) is a subspace of \( \R^{[0, 1]} \) if and only if \( b = 0 \).
\end{exercise}

\begin{solution}
    Let \( U \) be the set in question and suppose that \( b \neq 0 \). Then \( U \) is not a subspace of \( \R^{[0, 1]} \) since the zero function \( 0 : [0, 1] \to \R, x \mapsto 0 \) does not belong to \( U \); \( \int_0^1 0 = 0 \neq b \). Now suppose that \( b = 0 \), so that the zero function belongs to \( U \). Suppose that \( f, g \in U \) and \( \alpha \in \R \). Then
    \begin{gather*}
        \int_0^1 (f + g) = \int_0^1 f + \int_0^1 g = 0 + 0 = 0, \\[2mm]
        \int_0^1 (\alpha f) = \alpha \int_0^1 f = \alpha 0 = 0.
    \end{gather*}
    Hence \( f + g \) and \( \alpha f \) belong to \( U \) and so \( U \) is a subspace of \( \R^{[0, 1]} \) by (1.34).
\end{solution}

\begin{exercise}
\label{ex:5}
    Is \( \R^2 \) a subspace of the complex vector space \( \C^2 \)?
\end{exercise}

\begin{solution}
    The question is whether the subset
    \[
        \R^2 = \{ (x, y) : x, y \in \R \} \subseteq \{ (z, w) : z, w \in \C \} = \C^2
    \]
    is a subspace, where we are taking complex scalars in \( \C^2 \). This is not a subspace since it is not closed under scalar multiplication; for example, \( (1, 0) \in \R^2 \) but \( i (1, 0) = (i, 0) \not\in \R^2 \).
\end{solution}

\begin{exercise}
\label{ex:6}
    \begin{enumerate}
        \item Is \( \{ (a, b, c) \in \R^3 : a^3 = b^3 \} \) a subspace of \( \R^3 \)?

        \item Is \( \{ (a, b, c) \in \C^3 : a^3 = b^3 \} \) a subspace of \( \C^3 \)?
    \end{enumerate}
\end{exercise}

\begin{solution}
    \begin{enumerate}
        \item Let \( U \) be the set in question. Since for real numbers \( a \) and \( b \) we have \( a^3 = b^3 \) if and only if \( a = b \), the set \( U \) can be written as
        \[
            U = \{ (a, a, c) \in \R^3 : a, c \in \R \}.
        \]
        Clearly, \( (0, 0, 0) \in U \). Suppose that \( u = (a, a, c) \) and \( v = (x, x, y) \) belong to \( U \) and \( \lambda \in \R \). Then
        \begin{gather*}
            u + v = (a, a, c) + (x, x, y) = (a + x, a + x, c + y) \in U, \\[2mm]
            \lambda u = \lambda (a, a, c) = (\lambda a, \lambda a, \lambda c) \in U.
        \end{gather*}
        Hence \( U \) is a subspace of \( \R^3 \) by (1.34).

        \item Let \( U \) be the set in question. Observe that
        \[
            \left( -\tfrac{1}{2} + \tfrac{\sqrt{3}}{2} i \right)^3 = \left( -\tfrac{1}{2} - \tfrac{\sqrt{3}}{2} i \right)^3 = 1.
        \]
        So if we let \( u = \left( -\tfrac{1}{2} + \tfrac{\sqrt{3}}{2} i, 1, 0 \right) \) and \( v = \left( -\tfrac{1}{2} - \tfrac{\sqrt{3}}{2} i, 1, 0 \right) \), then we have \( u, v \in U \). However,
        \[
            u + v = (-1, 2, 0) \not\in U.
        \]
        So \( U \) is not closed under addition and hence cannot be a subspace of \( \C^3 \).
    \end{enumerate}
\end{solution}

\begin{exercise}
\label{ex:7}
    Give an example of a nonempty subset \( U \) of \( \R^2 \) such that \( U \) is closed under addition and under taking additive inverses (meaning \( - u \in U \) whenever \( u \in U \)), but \( U \) is not a subspace of \( \R^2 \).    
\end{exercise}

\begin{solution}
    Let \( U = \{ (a, b) : a, b \in \Q \} \subseteq \R^2 \). Then \( U \) satisfies the required conditions since the sum of two rational numbers is a rational number and the additive inverse of a rational number is a rational number. However, \( U \) is not a subspace of \( \R^2 \) since it is not closed under scalar multiplication by arbitrary real numbers; for example, \( (1, 0) \in U \) but \( \sqrt{2} (1, 0) = \left( \sqrt{2}, 0 \right) \not\in U \).
\end{solution}

\begin{exercise}
\label{ex:8}
    Give an example of a nonempty subset \( U \) of \( \R^2 \) such that \( U \) is closed under scalar multiplication, but \( U \) is not a subspace of \( \R^2 \).
\end{exercise}

\begin{solution}
    Let \( U = \{ (x, 0) : x \in \R \} \cup \{ (0, y) : y \in \R \} \), the union of the \( x \)- and \( y \)-axes. Then \( U \) is closed under scalar multiplication (\( (\alpha x, 0) \) and \( (0, \alpha y) \) belong to \( U \) for any \( \alpha, x, y \in \R \)) but \( U \) is not closed under addition; for example, \( (1, 0) \) and \( (0, 1) \) belong to \( U \) but \( (1, 0) + (0, 1) = (1, 1) \) does not. It follows that \( U \) is not a subspace of \( \R^2 \).
\end{solution}

\begin{exercise}
\label{ex:9}
    A function \( f : \R \to \R \) is called \textit{\textbf{periodic}} if there exists a positive number \( p \) such that \( f(x) = f(x + p) \) for all \( x \in \R \). Is the set of periodic functions from \( \R \) to \( \R \) a subspace of \( \R^{\R} \)? Explain.
\end{exercise}

\begin{solution}
    This is not a subspace of \( \R^{\R} \) since it is not closed under addition. Consider the periodic functions \( \sin(x) \) and \( \sin \left( \sqrt{2} x \right) \) and let \( f(x) = \sin(x) + \sin \left( \sqrt{2} x \right) \). We will show that \( f(x) \) cannot be periodic. Seeking a contradiction, suppose there is a positive real number \( p \) such that \( f(x) = f(x + p) \) for all \( x \in \R \), i.e.\
    \begin{equation}
        \sin(x) + \sin \left( \sqrt{2} x \right) = \sin(x + p) + \sin \left( \sqrt{2} x + \sqrt{2} p \right) \text{ for all } x \in \R.
    \end{equation}
    By differentiating this equation twice, we see that
    \begin{equation}
        \sin(x) + 2 \sin \left( \sqrt{2} x \right) = \sin(x + p) + 2 \sin \left( \sqrt{2} x + \sqrt{2} p \right) \text{ for all } x \in \R.
    \end{equation}
    Subtracting equation (1) from equation (2) gives us
    \begin{equation}
        \sin \left( \sqrt{2} x \right) = \sin \left( \sqrt{2} x + \sqrt{2} p \right) \text{ for all } x \in \R,
    \end{equation}
    which together with equation (1) implies that
    \begin{equation}
        \sin(x) = \sin(x + p) \text{ for all } x \in \R.
    \end{equation}
    In particular, by taking \( x = 0 \) in equation (4) we must have \( 0 = \sin(p) \), which can be the case if and only if \( p = n \pi \) for some positive integer \( n \) (\( p \) was assumed to be positive). Substituting this value of \( p \) and \( x = 0 \) into equation (3) gives \( 0 = \sin \left( n \sqrt{2} \pi \right) \), which can be the case if and only if \( n \sqrt{2} \pi = m \pi \) for some integer \( m \). This implies that \( \sqrt{2} = \tfrac{m}{n} \), a rational number; but it is well-known that \( \sqrt{2} \) is not a rational number, so we have found our contradiction. It follows that \( f \) cannot be periodic and hence the set of all periodic functions from \( \R \) to \( \R \) is not a subspace since it is not closed under addition.
\end{solution}

\begin{exercise}
\label{ex:10}
    Suppose \( U_1 \) and \( U_2 \) are subspaces of \( V \). Prove that the intersection \( U_1 \cap U_2 \) is a subspace of \( V \).
\end{exercise}

\begin{solution}
    See \Cref{ex:11}.
\end{solution}

\begin{exercise}
\label{ex:11}
    Prove that the intersection of every collection of subspaces of \( V \) is a subspace of \( V \).
\end{exercise}

\begin{solution}
    Let \( \mathscr{U} \) be an arbitrary collection of subspaces of \( V \). We will show that \( U' = \bigcap_{U \in \mathscr{U}} U \) is a subspace of \( V \). Since \( 0 \in U \) for each \( U \in \mathscr{U} \), we have \( 0 \in U' \). Suppose that \( x, y \in U', \alpha \in \F \) and \( U \in \mathscr{U} \). Then \( x, y \in U \), so \( x + y \in U \) and \( \alpha x \in U \) since \( U \) is a subspace of \( V \). Since \( U \) was arbitrary, we have \( x + y \in U' \) and \( \alpha x \in U' \); it follows that \( U' \) is a subspace of \( V \) by (1.34).
\end{solution}

\begin{exercise}
\label{ex:12}
    Prove that the union of two subspaces of \( V \) is a subspace of \( V \) if and only if one of the subspaces is contained in the other.
\end{exercise}

\begin{solution}
    Suppose that \( U \) and \( W \) are subspaces of \( V \). We want to show that \( U \cup W \) is a subspace of \( V \) if and only if either \( U \subseteq W \) or \( W \subseteq U \). If either of \( U \) or \( W \) is contained in the other then either \( U \cup W = U \) or \( U \cup W = W \); in either case, \( U \cup W \) is a subspace of \( V \). Suppose that \( U \cup W \) is a subspace of \( V \) and \( U \not\subseteq W \); we need to show that \( W \subseteq U \). Since \( U \not\subseteq W \), there is a \( u \in U \) such that \( u \not\in W \). Let \( w \in W \) be given. By assumption, \( U \cup W \) is a subspace of \( V \), so \( u + w \in U \cup W \). It cannot be the case that \( u + w \in W \), since then \( u + w - w = u \in W \), so it must be the case that \( u + w \in U \). Then \( u + w - u = w \in U \) and hence \( W \subseteq U \) as desired.
\end{solution}

\begin{exercise}
\label{ex:13}
    Prove that the union of three subspaces of \( V \) is a subspace of \( V \) if and only if one of the subspaces contains the other two.

    \noindent [\textit{This exercise is surprisingly harder than the previous exercise, possibly because this exercise is not true if we replace} \( \F \) \textit{with a field containing only two elements.}]
\end{exercise}

\begin{solution}
    Let \( U_1, U_2, \) and \( U_3 \) be subspaces of \( V \). We want to show that \( U = U_1 \cup U_2 \cup U_3 \) is a subspace of \( V \) if and only if some \( U_i \) contains the other two. If some \( U_i \) contains the other two, say \( U_1 \) contains \( U_2 \) and \( U_3 \), then \( U = U_1 \) is a subspace of \( V \). Now suppose that \( U \) is a subspace of \( V \), and note that if any \( U_i \) is contained in the union of the other two, say \( U_1 \subseteq U_2 \cup U_3 \), then \( U = U_2 \cup U_3 \) and we may apply \Cref{ex:12} to say that either \( U_2 \subseteq U_3 \) or \( U_3 \subseteq U_2 \); in either case, one \( U_i \) contains the other two. Suppose therefore that no \( U_i \) is contained in the union of the other two, and seeking a contradiction suppose also that no \( U_i \) contains the other two. Hence
    \[
        U_1 \not\subseteq (U_2 \cup U_3) \quad \text{and} \quad (U_2 \cup U_3) \not\subseteq U_1.
    \]
    It follows that there exists a \( u \in U_1 \) such that \( u \not\in U_2 \cup U_3 \) and a \( v \in U_2 \cup U_3 \) such that \( v \not\in U_1 \). Let \( W = \{ v + \lambda u : \lambda \in \F \} \) and observe that no element of \( W \) belongs to \( U_1 \), for if \( v + \lambda u \in U_1 \) then \( v + \lambda u - \lambda u = v \in U_1 \); but \( v \not\in U_1 \). Then since \( W \subseteq U \) and \( W \cap U_1 = \emptyset \), it must be the case that \( W \subseteq U_2 \cup U_3 \). \( W \) is infinite since \( \F \) is infinite, so at least one of \( U_2 \) and \( U_3 \) must contain infinitely many members of \( W \); say \( U_j \). Then there exists \( \lambda \neq \mu \) in \( \F \) such that \( v + \lambda u \) and \( v + \mu u \) both belong to \( U_j \), which implies that \( (\lambda - \mu) u \in U_j \). Since \( \lambda - \mu \neq 0 \), this gives \( u \in U_j \), which contradicts \( u \not\in U_2 \cup U_3 \). We may conclude that one \( U_i \) contains the other two.
\end{solution}

\begin{exercise}
\label{ex:14}
    Verify the assertion in Example 1.38.
\end{exercise}

\begin{solution}
    Let
    \begin{gather*}
        U = \{ (x, x, y, y) \in \F^4 : x, y \in \F \}, \quad W = \{ (x, x, x, y) \in \F^4 : x, y \in \F \}, \\
        E = \{ (x, x, y, z) \in \F^4 : x, y, z \in \F \}.
    \end{gather*}
    The assertion is that \( U + W = E \). First, suppose that \( u + w \in U + W \) for some \( u = (x, x, y, y) \in U \) and \( w = (a, a, a, b) \in W \). Then
    \[
        u + w = (x, x, y, y) + (a, a, a, b) = (x + a, x + a, y + a, y + b),
    \]
    which has the form of an element of \( E \). It follows that \( U + W \subseteq E \). Now suppose that \( v = (x, x, y, z) \in E \). Then observe that
    \[
        v = (x, x, y, z) = (x, x, y, y) + (0, 0, 0, z - y)
    \]
    which has the form of an element of \( U + W \). It follows that \( E \subseteq U + W \) and we may conclude that \( U + W = E \).
\end{solution}

\begin{exercise}
\label{ex:15}
    Suppose \( U \) is a subspace of \( V \). What is \( U + U \)?
\end{exercise}

\begin{solution}
    We have \( U + U = U \). For \( u + v \in U + U \), we have \( u + v \in U \) since \( U \) is a subspace of \( V \); it follows that \( U + U \subseteq U \). For \( u \in U \), we have \( u = u + 0 \in U + U \), so that \( U \subseteq U + U \).
\end{solution}

\begin{exercise}
\label{ex:16}
    Is the operation of addition on the subspaces of \( V \) commutative? In other words, if \( U \) and \( W \) are subspaces of \( V \), is \( U + W = W + U \)?
\end{exercise}

\begin{solution}
    The operation is commutative, since addition of vectors in \( V \) is commutative. If \( u + w \in U + W \), then \( u + w = w + u \in W + U \), so that \( U + W \subseteq W + U \). Similarly, \( W + U \subseteq U + W \).
\end{solution}

\begin{exercise}
\label{ex:17}
    Is the operation of addition on the subspaces of \( V \) associative? In other words, if \( U_1, U_2, U_3 \) are subspaces of \( V \), is
    \[
        (U_1 + U_2) + U_3 = U_1 + (U_2 + U_3)?
    \]
\end{exercise}

\begin{solution}
    The operation is associative, since addition of vectors in \( V \) is associative. If \( (u_1 + u_2) + u_3 \in (U_1 + U_2) + U_3 \), then \( (u_1 + u_2) + u_3 = u_1 + (u_2 + u_3) \in U_1 + (U_2 + U_3) \), so that \( (U_1 + U_2) + U_3 \subseteq U_1 + (U_2 + U_3) \). Similarly, \( U_1 + (U_2 + U_3) \subseteq (U_1 + U_2) + U_3 \).
\end{solution}

\begin{exercise}
\label{ex:18}
    Does the operation of addition on the subspaces of \( V \) have an additive identity? Which subspaces have additive inverses?
\end{exercise}

\begin{solution}
    The subspace \( \{ 0 \} \) is the additive identity for the operation. If \( U \) is a subspace of \( V \) then \( u + 0 = u \) for any \( u \in U \); it follows that \( U + \{ 0 \} = U \).

    Since \( \{ 0 \} + \{ 0 \} = \{ 0 \} \), the subspace \( \{ 0 \} \) is its own additive inverse. We claim that no other subspace has an additive inverse, i.e.\ if \( U \) is a subspace of \( V \) with \( U \neq \{ 0 \} \), then there does not exist a subspace \( W \) satisfying \( U + W = \{ 0 \} \); indeed, simply observe that \( U \subseteq U + W \) for any subspace \( W \).
\end{solution}

\begin{exercise}
\label{ex:19}
    Prove or give a counterexample: if \( U_1, U_2, W \) are subspaces of \( V \) such that
    \[
        U_1 + W = U_2 + W,
    \]
    then \( U_1 = U_2 \).
\end{exercise}

\begin{solution}
    This is false. For a counterexample, consider the real vector space \( \R \). Then
    \[
        \{ 0 \} + \R = \R + \R = \R,
    \]
    but \( \{ 0 \} \neq \R \).
\end{solution}

\begin{exercise}
\label{ex:20}
    Suppose
    \[
        U = \{ (x, x, y, y) \in \F^4 : x, y \in \F \}.
    \]
    Find a subspace \( W \) of \( \F^4 \) such that \( \F^4 = U \oplus W \).
\end{exercise}

\begin{solution}
    Let \( W = \{ (0, a, 0, b) \in \F^4 : a, b \in \F \} \); it is easily verified that \( W \) is a subspace of \( \F^4 \). Suppose that \( v = (v_1, v_2, v_3, v_4) \in U \cap W \). Since \( v \in W \) we must have \( v_1 = v_3 = 0 \), then since \( v \in U \) we must have \( v_2 = v_1 = 0 \) and \( v_4 = v_3 = 0 \), i.e.\ \( v = 0 \). It follows that \( U \cap W = \{ 0 \} \) and hence by (1.45) the sum \( U + W \) is direct. Let \( v = (x, y, z, t) \in \F^4 \) be given. Then observe that
    \[
        v = (x, y, z, t) = (x, x, z, z) + (0, y - x, 0, t - z) \in U \oplus W.
    \]
    Hence \( \F^4 = U \oplus W \).
\end{solution}

\begin{exercise}
\label{ex:21}
    Suppose
    \[
        U = \{ (x, y, x + y, x - y, 2x) \in \F^5 : x, y \in \F \}.
    \]
    Find a subspace \( W \) of \( \F^5 \) such that \( \F^5 = U \oplus W \).
\end{exercise}

\begin{solution}
    Let \( W = \{ (0, 0, a, b, c) \in \F^5 : a, b, c \in \F \} \); it is easily verified that \( W \) is a subspace of \( \F^5 \). Suppose that \( v = (v_1, v_2, v_3, v_4, v_5) \in U \cap W \). Since \( v \in U \) we must have \( v = (v_1, v_2, v_1 + v_2, v_1 - v_2, 2 v_1) \), then since \( v \in W \) we must have \( v_1 = v_2 = 0 \) and and hence \( v = 0 \). It follows that \( U \cap W = \{ 0 \} \) and hence by (1.45) the sum \( U + W \) is direct. Let \( v = (v_1, v_2, v_3, v_4, v_5)\in \F^5 \) be given. Then observe that
    \begin{multline*}
        v = (v_1, v_2, v_3, v_4, v_5) = (v_1, v_2, v_1 + v_2, v_1 - v_2, 2 v_1) \\ + (0, 0, v_3 - (v_1 + v_2), v_4 - (v_1 - v_2), v_5 - 2 v_1) \in U \oplus W.
    \end{multline*}
    Hence \( \F^5 = U \oplus W \).
\end{solution}

\begin{exercise}
\label{ex:22}
    Suppose
    \[
        U = \{ (x, y, x + y, x - y, 2x) \in \F^5 : x, y \in \F \}.
    \]
    Find three subspaces \( W_1, W_2, W_3 \) of \( \F^5 \), none of which equals \( \{ 0 \} \), such that \( \F^5 = U \oplus W_1 \oplus W_2 \oplus W_3 \).
\end{exercise}

\begin{solution}
    Let
    \begin{gather*}
        W_1 = \{ (0, 0, a, 0, 0) \in \F^5 : a \in \F \}, \quad W_2 = \{ (0, 0, 0, b, 0) \in \F^5 : b \in \F \}, \\
        W_3 = \{ (0, 0, 0, 0, c) \in \F^5 : c \in \F \}.
    \end{gather*}
    Suppose that \( u = (x, y, x + y, x - y, 2x) \in U, w_1 = (0, 0, a, 0, 0) \in W_1, w_2 = (0, 0, 0, b, 0) \in W_2, \) and \( w_3 = (0, 0, 0, 0, c) \in W_3 \) are such that \( u + w_1 + w_2 + w_3 = 0 \), i.e.\
    \[
        (x, y, x + y + a, x - y + b, 2x + c) = (0, 0, 0, 0, 0).
    \]
    Then \( x = y = 0 \), which implies that \( a = b = c = 0 \). It follows that \( u + w_1 + w_2 + w_3 = 0 \) if and only if \( u = w_1 = w_2 = w_3 = 0 \), and so by (1.44) \( U + W_1 + W_2 + W_3 \) is a direct sum. Let \( v = (v_1, v_2, v_3, v_4, v_5)\in \F^5 \) be given. Then observe that
    \begin{multline*}
        v = (v_1, v_2, v_3, v_4, v_5) = (v_1, v_2, v_1 + v_2, v_1 - v_2, 2 v_1) + (0, 0, v_3 - (v_1 + v_2), 0, 0) \\ + (0, 0, 0, v_4 - (v_1 - v_2), 0) + (0, 0, 0, 0, v_5 - 2 v_1) \in U \oplus W_1 \oplus W_2 \oplus W_3.
    \end{multline*}
    Hence \( \F^5 = U \oplus W_1 \oplus W_2 \oplus W_3 \).
\end{solution}

\begin{exercise}
\label{ex:23}
    Prove or give a counterexample: if \( U_1, U_2, W \) are subspaces of \( V \) such that
    \[
        V = U_1 \oplus W \quad \text{and} \quad V = U_2 \oplus W,
    \]
    then \( U_1 = U_2 \).
\end{exercise}

\begin{solution}
    This is false. For a counterexample, consider \( V = \R^2 \),
    \[
        W = \{ (x, 0) \in \R^2 : x \in \R \}, U_1 = \{ (0, y) \in \R^2 : y \in \R \}, U_2 = \{ (y, y) \in \R^2 : y \in \R \}.
    \]
    It is easily verified that \( W \cap U_1 = W \cap U_2 = \{ 0 \} \), so that \( W + U_1 \) and \( W + U_2 \) are both direct sums, and that \( W \oplus U_1 = W \oplus U_2 = \R^2 \). However, \( U_1 \neq U_2 \), since \( (1, 1) \in U_2 \) but \( (1, 1) \not\in U_1 \).
\end{solution}

\begin{exercise}
\label{ex:24}
    A function \( f : \R \to \R \) is called \textit{\textbf{even}} if
    \[
        f(-x) = f(x)
    \]
    for all \( x \in \R \). A function \( f : \R \to \R \) is called \textit{\textbf{odd}} if
    \[
        f(-x) = -f(x)
    \]
    for all \( x \in \R \). Let \( U_{\text{e}} \) denote the set of real-valued even functions on \( \R \) and let \( U_{\text{o}} \) denote the set of real-valued odd functions on \( \R \). Show that \( \R^{\R} = U_{\text{e}} \oplus U_{\text{o}} \). 
\end{exercise}

\begin{solution}
    Suppose that \( f \in U_{\text{e}} \cap U_{\text{o}} \). Then \( f(x) = -f(x) \) for all \( x \in \R \), which implies that \( f(x) = 0 \) for all \( x \in \R \), i.e.\ \( f = 0 \). So \( U_{\text{e}} \cap U_{\text{o}} = \{ 0 \} \) and hence by (1.45) the sum \( U_{\text{e}} + U_{\text{o}} \) is direct. Let \( f : \R \to \R \) be given and define \( f_{\text{e}} : \R \to \R, f_{\text{o}} : \R \to \R \) by
    \[
        f_{\text{e}} (x) = \frac{f(x) + f(-x)}{2} \quad \text{and} \quad f_{\text{o}} (x) = \frac{f(x) - f(-x)}{2}.
    \]
    It is easily verified that \( f_{\text{e}} \) is an even function, that \( f_{\text{o}} \) is an odd function, and that \( f = f_{\text{e}} + f_{\text{o}} \). It follows that \( \R^{\R} = U_{\text{e}} \oplus U_{\text{o}} \).
\end{solution}

\noindent \hrulefill

\noindent \hypertarget{ladr}{\textcolor{blue}{[LADR]} Axler, S. (2015) \textit{Linear Algebra Done Right.} 3rd edn.}

\end{document}