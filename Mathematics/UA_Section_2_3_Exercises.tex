\documentclass[12pt]{article}
\usepackage[utf8]{inputenc}
\usepackage[utf8]{inputenc}
\usepackage{amsmath}
\usepackage{amsthm}
\usepackage{geometry}
\usepackage{amsfonts}
\usepackage{mathrsfs}
\usepackage{bm}
\usepackage{hyperref}
\usepackage[dvipsnames]{xcolor}
\usepackage{enumitem}
\usepackage{mathtools}
\usepackage{changepage}
\usepackage{lipsum}
\usepackage{tikz}
\usetikzlibrary{matrix}
\usepackage{tikz-cd}
\usepackage[nameinlink]{cleveref}
\geometry{
headheight=15pt,
left=60pt,
right=60pt
}
\usepackage{fancyhdr}
\pagestyle{fancy}
\fancyhf{}
\lhead{}
\chead{Section 2.3 Exercises}
\rhead{\thepage}
\hypersetup{
    colorlinks=true,
    linkcolor=blue,
    urlcolor=blue
}

\theoremstyle{definition}
\newtheorem*{remark}{Remark}

\newtheoremstyle{exercise}
    {}
    {}
    {}
    {}
    {\bfseries}
    {.}
    { }
    {\thmname{#1}\thmnumber{#2}\thmnote{ (#3)}}
\theoremstyle{exercise}
\newtheorem{exercise}{Exercise 2.3.}

\newtheoremstyle{solution}
    {}
    {}
    {}
    {}
    {\itshape\color{magenta}}
    {.}
    { }
    {\thmname{#1}\thmnote{ #3}}
\theoremstyle{solution}
\newtheorem*{solution}{Solution}

\Crefformat{exercise}{#2Exercise 2.3.#1#3}

\newcommand{\setcomp}[1]{#1^{\mathsf{c}}}
\newcommand{\N}{\mathbf{N}}
\newcommand{\Z}{\mathbf{Z}}
\newcommand{\Q}{\mathbf{Q}}
\newcommand{\R}{\mathbf{R}}
\newcommand{\C}{\mathbf{C}}

\DeclarePairedDelimiter\abs{\lvert}{\rvert}
% Swap the definition of \abs* and \norm*, so that \abs
% and \norm resizes the size of the brackets, and the 
% starred version does not.
\makeatletter
\let\oldabs\abs
\def\abs{\@ifstar{\oldabs}{\oldabs*}}
%
\let\oldnorm\norm
\def\norm{\@ifstar{\oldnorm}{\oldnorm*}}
\makeatother

\setlist[enumerate,1]{label={(\alph*)}}

\begin{document}

\section{Section 2.3 Exercises}

Exercises with solutions from Section 2.3 of \hyperlink{ua}{[UA]}.

\begin{exercise}
\label{ex:1}
    Let \( x_n \geq 0 \) for all \( n \in \N \).
    \begin{enumerate}
        \item If \( (x_n) \to 0 \), show that \( ( \sqrt{x_n} ) \to 0 \).

        \item If \( (x_n) \to x \), show that \( ( \sqrt{x_n} ) \to \sqrt{x} \).
    \end{enumerate}
\end{exercise}

\begin{solution}
    \begin{enumerate}
        \item Let \( \epsilon > 0 \) be given. Then there exists an \( N \in \N \) such that
        \[
            n \geq N \implies \abs{x_n - 0} = x_n < \epsilon^2 \iff \sqrt{x_n} < \epsilon.
        \]
        It follows that \( \lim (\sqrt{x_n}) = 0 \).

        \item By Theorem 2.3.4, we must have \( x \geq 0 \). The case \( x = 0 \) was handled in part (a), so suppose that \( x > 0 \iff \sqrt{x} > 0 \). Let \( \epsilon > 0 \) be given. Then there exists an \( N \in \N \) such that \( n \geq N \implies \abs{x_n - x} < \epsilon \sqrt{x} \). Observe that
        \[
            \abs{\sqrt{x_n} - \sqrt{x}} \abs{\sqrt{x_n} + \sqrt{x}} = \abs{x_n - x} \iff \abs{\sqrt{x_n} - \sqrt{x}} = \frac{\abs{x_n - x}}{\sqrt{x_n} + \sqrt{x}} \leq \frac{\abs{x_n - x}}{\sqrt{x}}.
        \]
        So if we take \( n \geq N \) we will have
        \[
            \abs{\sqrt{x_n} - \sqrt{x}} \leq \frac{\abs{x_n - x}}{\sqrt{x}} < \epsilon.
        \]
        It follows that \( \lim(\sqrt{x_n}) = \sqrt{x} \).
    \end{enumerate}
\end{solution}

\begin{exercise}
\label{ex:2}
    Using only Definition 2.2.3, prove that if \( (x_n) \to 2 \) then
    \begin{enumerate}
        \item \( \left( \frac{2 x_n - 1}{3} \right) \to 1 \);

        \item \( (1/x_n) \to 1/2 \).
    \end{enumerate}
\end{exercise}

\begin{solution}
    \begin{enumerate}
        \item Let \( \epsilon > 0 \) be given. There exists an \( N \in \N \) such that \( n \geq N \) implies that \( \abs{x_n - 2} < \tfrac{3 \epsilon}{2} \). Then for \( n \geq N \) we have
        \[
            \abs{\frac{2 x_n - 1}{3} - 1} = \abs{\frac{2 x_n - 4}{3}} = \tfrac{2}{3} \abs{x_n - 2} < \epsilon.
        \]
        It follows that \( \left( \frac{2 x_n - 1}{3} \right) \to 1 \).

        \item There is an \( N_1 \in \N \) such that \( n \geq N_1 \implies \abs{x_n - 2} < 1 \). Then for \( n \geq N_1 \) we have
        \[
            2 \leq \abs{x_n - 2} + \abs{x_n} < 1 + \abs{x_n} \implies 1 < \abs{x_n} \implies \frac{1}{\abs{x_n}} < 1.
        \]
        Let \( \epsilon > 0 \) be given. There is an \( N_2 \in \N \) such that \( n \geq N_2 \implies \abs{x_n - 2} < 2 \epsilon \). Set \( N = \max \{ N_1, N_2 \} \) and observe that for \( n \geq N \) we have
        \[
            \abs{\frac{1}{x_n} - \frac{1}{2}} = \abs{\frac{2 - x_n}{2 x_n}} = \frac{\abs{x_n - 2}}{2 \abs{x_n}} < \frac{\abs{x_n - 2}}{2} < \epsilon.
        \]
        It follows that \( (1/x_n) \to 1/2 \).
    \end{enumerate}
\end{solution}

\begin{exercise}[Squeeze Theorem]
\label{ex:3}
    Show that if \( x_n \leq y_n \leq z_n \) for all \( n \in \N \), and if \( \lim x_n = \lim z_n = l \), then \( \lim y_n = l \) as well.
\end{exercise}

\begin{solution}
    Let \( \epsilon > 0 \) be given. There are positive integers \( N_1 \) and \( N_2 \) such that
    \[
        n \geq N_1 \implies \abs{x_n - l} < \epsilon \iff -\epsilon < x_n - l < \epsilon,
    \]
    \[
        n \geq N_2 \implies \abs{z_n - l} < \epsilon \iff -\epsilon < z_n - l < \epsilon.
    \]
    Let \( N = \max \{ N_1, N_2 \} \). Then since \( x_n - l \leq y_n - l \leq z_n - l \) for all \( n \in \N \), for \( n \geq N \) we have
    \[
        -\epsilon < y_n - l < \epsilon \iff \abs{y_n - l} < \epsilon.
    \]
    It follows that \( \lim y_n = l \).
\end{solution}

\begin{exercise}
\label{ex:4}
    Let \( (a_n) \to 0 \), and use the Algebraic Limit Theorem to compute each of the following limits (assuming the fractions are always defined):
    \begin{enumerate}
        \item \( \lim \left( \frac{1 + 2 a_n}{1 + 3 a_n - 4 a_n^2} \right) \)

        \item \( \lim \left( \frac{(a_n + 2)^2 - 4}{a_n} \right) \)

        \item \( \lim \left( \frac{\frac{2}{a_n} + 3}{\frac{1}{a_n} + 5} \right) \).
    \end{enumerate}
\end{exercise}

\begin{solution}
    The manipulations of limits in these solutions are justified by the Algebraic Limit Theorem.
    \begin{enumerate}
        \item We have
        \[
            \lim \left( \frac{1 + 2 a_n}{1 + 3 a_n - 4 a_n^2} \right) = \frac{1 + 2 \lim a_n}{1 + 3 \lim a_n - 4 (\lim a_n)^2} = \frac{1}{1} = 1.
        \]

        \item We have
        \[
            \lim \left( \frac{(a_n + 2)^2 - 4}{a_n} \right) = \lim \left( \frac{a_n^2 + 4 a_n}{a_n} \right) = \lim (a_n + 4) = \lim a_n + 4 = 4.
        \]

        \item We have
        \[
            \lim \left( \frac{\frac{2}{a_n} + 3}{\frac{1}{a_n} + 5} \right) = \lim \left( \frac{2 + 3 a_n}{1 + 5 a_n} \right) = \frac{2 + 3 \lim a_n}{1 + 5 \lim a_n} = \frac{2}{1} = 2.
        \]
    \end{enumerate}
\end{solution}

\begin{exercise}
\label{ex:5}
    Let \( (x_n) \) and \( (y_n) \) be given, and define \( (z_n) \) to be the ``shuffled" sequence \( (x_1, y_1, x_2, y_2, x_3, y_3, \ldots, x_n, y_n, \ldots) \). Prove that \( (z_n) \) is convergent if and only if \( (x_n) \) and \( (y_n) \) are both convergent with \( \lim x_n = \lim y_n \).
\end{exercise}

\begin{solution}
    \( (z_n) \) is the sequence given by
    \[
        z_n = \begin{cases}
            x_{\frac{n + 1}{2}} & \text{if } n \text{ is odd}, \\
            y_{\frac{n}{2}} & \text{if } n \text{ is even}.
        \end{cases}
    \]
    Suppose that \( (x_n) \) and \( (y_n) \) are both convergent with \( \lim x_n = \lim y_n = l \) for some \( l \in \R \). Then there are positive integers \( N_1 \) and \( N_2 \) such that
    \[
        n \geq N_1 \implies \abs{x_n - l} < \epsilon \qquad \text{and} \qquad n \geq N_2 \implies \abs{y_n - l} < \epsilon.
    \]
    Let \( N = \max \{ N_1, N_2 \} \) and suppose \( n \in \N \) is such that \( n \geq 2N \). If \( n \) is odd then \( \frac{n + 1}{2} \in \N \) and
    \[
        n \geq 2N > 2N - 1 \implies \tfrac{n + 1}{2} > N \geq N_1 \implies \abs{x_{\frac{n + 1}{2}} - l} < \epsilon.
    \]
    Hence
    \[
        \abs{z_n - l} = \abs{x_{\frac{n + 1}{2}} - l} < \epsilon.
    \]
    If \( n \) is even then \( \frac{n}{2} \in \N \) and
    \[
        n \geq 2N \implies \tfrac{n}{2} \geq N \geq N_2 \implies \abs{y_{\frac{n}{2}} - l} < \epsilon.
    \]
    Hence
    \[
        \abs{z_n - l} = \abs{y_{\frac{n}{2}} - l} < \epsilon.
    \]
    In either case we have \( \abs{z_n - l} < \epsilon \), i.e.\
    \[
        n \geq 2N \implies \abs{z_n - l} < \epsilon.
    \]
    It follows that \( \lim z_n = l \).
    
    Now suppose that \( (z_n) \) is convergent with \( \lim z_n = l \) for some \( l \in \R \). Let \( \epsilon > 0 \) be given. Then there exists \( N \in \N \) such that \( n \geq N \implies \abs{z_n - l} < \epsilon \). Suppose \( n \in \N \) is such that \( n \geq N \). Then \( 2n > 2n - 1 \geq N \), so
    \[
        \abs{x_n - l} = \abs{z_{2n - 1} - l} < \epsilon \qquad \text{and} \qquad \abs{y_n - l} = \abs{z_{2n} - l} < \epsilon.
    \]
    It follows that \( \lim x_n = \lim y_n = l \).
\end{solution}

\begin{exercise}
\label{ex:6}
    Consider the sequence given by \( b_n = n - \sqrt{n^2 + 2n} \). Taking \( (1/n) \to 0 \) as given, and using both the Algebraic Limit Theorem and the result in \Cref{ex:1}, show \( \lim b_n \) exists and find the value of the limit.
\end{exercise}

\begin{solution}
    Observe that
    \[
        b_n = n - \sqrt{n^2 + 2n} = \frac{(n - \sqrt{n^2 + 2n})(n + \sqrt{n^2 + 2n})}{n + \sqrt{n^2 + 2n}} = \frac{-2n}{n + \sqrt{n^2 + 2n}} = \frac{-2}{1 + \sqrt{1 + (2/n)}}.
    \]
    Hence
    \[
        \lim b_n = \lim \left( \frac{-2}{1 + \sqrt{1 + (2/n)}} \right) = \frac{-2}{1 + \sqrt{1 + 2 \lim(1/n)}} = \frac{-2}{1 + \sqrt{1}} = -1.
    \]
\end{solution}

\begin{exercise}
\label{ex:7}
    Give an example of each of the following, or state that such a request is impossible by referencing the proper theorem(s):
    \begin{enumerate}
        \item sequences \( (x_n) \) and \( (y_n) \), which both diverge, but whose sum \( (x_n + y_n) \) converges;

        \item sequences \( (x_n) \) and \( (y_n) \), where \( (x_n) \) converges, \( (y_n) \) diverges, and \( (x_n + y_n) \) converges;

        \item a convergent sequence \( (b_n) \) with \( b_n \neq 0 \) for all \( n \) such that \( (1/b_n) \) diverges;

        \item an unbounded sequence \( (a_n) \) and a convergent sequence \( (b_n) \) with \( (a_n - b_n) \) bounded;

        \item two sequences \( (a_n) \) and \( (b_n) \), where \( (a_n b_n) \) and \( (a_n) \) converge but \( (b_n) \) does not.
    \end{enumerate}
\end{exercise}

\begin{solution}
    \begin{enumerate}
        \item Take \( x_n = n \) and \( y_n = -n \).

        \item This is impossible. If \( (x_n) \) and \( (x_n + y_n) \) both converge, then by the Algebraic Limit Theorem, \( (y_n) \) must be convergent with limit \( \lim y_n = \lim (x_n + y_n) - \lim x_n \).

        \item Take \( b_n = 1/n \).

        \item This is impossible; \( (a_n - b_n) \) must be unbounded. Since \( (b_n) \) is convergent, it must be bounded (Theorem 2.3.2), i.e.\ there exists some \( m \geq 0 \) such that \( \abs{b_n} \leq m \) for all \( n \in \N \). Let \( M \geq 0 \) be given. Since \( (a_n) \) is unbounded, there exists some \( N \in \N \) such that \( \abs{a_N} \geq M + m \). Then observe that
        \[
            \abs{a_N - b_N} \geq \abs{\abs{a_N} - \abs{b_N}} \geq \abs{a_N} - \abs{b_N} \geq M + m - m = M.
        \]

        \item Take \( a_n = 1/n^2 \) and \( b_n = n \).
    \end{enumerate}
\end{solution}

\begin{exercise}
\label{ex:8}
    Let \( (x_n) \to x \) and let \( p(x) \) be a polynomial.
    \begin{enumerate}
        \item Show \( p(x_n) \to p(x) \).

        \item Find an example of a function \( f(x) \) and a convergent sequence \( (x_n) \to x \) where the sequence \( f(x_n) \) converges, but not to \( f(x) \).
    \end{enumerate}
\end{exercise}

\begin{solution}
    \begin{enumerate}
        \item Suppose \( p(x) = a_m x^m + a_{m-1} x^{m-1} + \cdots + a_1 x + a_0 \). The Algebraic Limit Theorem and some simple induction arguments allow us to make the following manipulations:
        \begin{align*}
            \lim (p(x_n)) &= \lim (a_m x_n^m + a_{m-1} x_n^{m-1} + \cdots + a_1 x_n + a_0) \\
            &= a_m (\lim x_n)^m + a_{m-1} (\lim x_n)^{m-1} + \cdots + a_1 \lim x_n + a_0 \\
            &= a_m x^m + a_{m-1} x^{m-1} + \cdots + a_1 x + a_0 \\
            &= p(x).
        \end{align*}

        \item Consider the function \( f : \R \to \R \) given by
        \[
            f(x) = \begin{cases}
                0 & \text{if } x = 0, \\
                1 & \text{otherwise}
            \end{cases}
        \]
        and the convergent sequence \( (x_n) = (1/n) \to 0 \). Then the sequence \( (f(x_n)) = (1, 1, 1, \ldots) \) converges to \( 1 \neq 0 = f(0) \).
    \end{enumerate}
\end{solution}

\begin{exercise}
\label{ex:9}
    \begin{enumerate}
        \item Let \( (a_n) \) be a bounded (not necessarily convergent) sequence, and assume \( \lim b_n = 0 \). Show that \( \lim (a_n b_n) = 0 \). Why are we not allowed to use the Algebraic Limit Theorem to prove this?

        \item Can we conclude anything about the convergence of \( (a_n b_n) \) if we assume that \( (b_n) \) converges to some nonzero limit \( b \)?

        \item Use (a) to prove Theorem 2.3.3, part (iii), for the case when \( a = 0 \).
    \end{enumerate}
\end{exercise}

\begin{solution}
    \begin{enumerate}
        \item There is an \( M > 0 \) such that \( \abs{a_n} \leq M \) for all \( n \in \N \). Let \( \epsilon > 0 \) be given. Then there is an \( N \in \N \) such that
        \[
            n \geq N \implies \abs{b_n} < \frac{\epsilon}{M}.
        \]
        Observe that for \( n \geq N \) we have
        \[
            \abs{a_n b_n} = \abs{a_n} \abs{b_n} \leq M \abs{b_n} < \frac{M \epsilon}{M} = \epsilon.
        \]
        It follows that \( \lim (a_n b_n) = 0 \). We may not use the Algebraic Limit Theorem here since the sequence \( (a_n) \) is not necessarily convergent; the hypotheses of that theorem require both sequences \( (a_n) \) and \( (b_n) \) to be convergent.

        \item If the sequence \( (a_n) \) is convergent to some \( a \) then we may use the Algebraic Limit Theorem to conclude that \( \lim (a_n b_n) = ab \). If the sequence \( (a_n) \) is divergent, then \( (a_n b_n) \) must also be divergent. To see this, we will prove the contrapositive, i.e. if \( (a_n b_n) \) is convergent to some \( x \) then \( (a_n) \) is convergent. Indeed, since \( b \neq 0 \), the Algebraic Limit Theorem implies
        \[
            \lim a_n = \lim \left(\frac{a_n b_n}{b_n}\right) = \frac{x}{b}.
        \]

        \item Since \( (b_n) \) is convergent, it is bounded (Theorem 2.3.2). So we may apply part (a) (we have swapped the roles of \( (a_n) \) and \( (b_n) \)) to conclude that
        \[
            \lim (a_n b_n) = 0 = 0b = ab.
        \]
    \end{enumerate}
\end{solution}

\begin{exercise}
\label{ex:10}
    Consider the following list of conjectures. Provide a short proof for those that are true and a counterexample for any that are false.
    \begin{enumerate}
        \item If \( \lim (a_n - b_n) = 0 \), then \( \lim a_n = \lim b_n \).

        \item If \( (b_n) \to b \), then \( \abs{b_n} \to \abs{b} \).

        \item If \( (a_n) \to a \) and \( (b_n - a_n) \to 0 \), then \( (b_n) \to a \).

        \item If \( (a_n) \to 0 \) and \( \abs{b_n - b} \leq a_n \) for all \( n \in \N \), then \( (b_n) \to b \).
    \end{enumerate}
\end{exercise}

\begin{solution}
    \begin{enumerate}
        \item This is true if \( (a_n) \) and \( (b_n) \) are convergent sequences (Algebraic Limit Theorem), however it may be the case that they are both divergent and \( \lim (a_n - b_n) = 0 \); for example, \( a_n = b_n = n \). So the conjecture is false in general.

        \item This is true. Let \( \epsilon > 0 \) be given. Then there is an \( N \in \N \) such that \( n \geq N \implies \abs{b_n - b} < \epsilon \). For \( n \geq N \) the reverse triangle inequality gives
        \[
            \abs{\abs{b_n} - \abs{b}} \leq \abs{b_n - b} < \epsilon.
        \]
        It follows that \( \lim \abs{b_n} = \abs{b} \).

        \item This is true. Using the Algebraic Limit Theorem, we have
        \[
            \lim b_n = \lim (b_n - a_n + a_n) = \lim (b_n - a_n) + \lim a_n = 0 + a = a.
        \]

        \item This is true. Since \( 0 \leq \abs{b_n - b} \leq a_n \) for every \( n \in \N \), the Squeeze Theorem (\Cref{ex:3}) implies that \( \lim \abs{b_n - b} = 0 \), i.e.\ for every \( \epsilon > 0 \) there is an \( N \in \N \) such that
        \[
            n \geq N \implies \abs{\abs{b_n - b} - 0} = \abs{b_n - b} < \epsilon,
        \]
        which is exactly the statement \( \lim b_n = b \).
    \end{enumerate}
\end{solution}

\begin{exercise}[Cesaro Means]
\label{ex:11}
    \begin{enumerate}
        \item Show that if \( (x_n) \) is a convergent sequence, then the sequence given by the averages
        \[
            y_n = \frac{x_1 + x_2 + \cdots + x_n}{n}
        \]
        also converges to the same limit.

        \item Give an example to show that it is possible for the sequence \( (y_n) \) of averages to converge even if \( (x_n) \) does not.
    \end{enumerate}
\end{exercise}

\begin{solution}
    \begin{enumerate}
        \item Suppose \( \lim x_n = x \). Let \( \epsilon > 0 \) be given. There is a positive integer \( N_1 \geq 2 \) such that
        \[
            n \geq N_1 \implies \abs{x_n - x} < \frac{\epsilon}{2}.
        \]
        Given this \( N_1 \), there is an \( N_2 \in \N \) such that
        \[
            n \geq N_2 \implies \frac{\abs{x_1 - x} + \cdots + \abs{x_{N_1 - 1} - x}}{n} < \frac{\epsilon}{2}.
        \]
        Set \( N = \max \{ N_1, N_2 \} \) and observe that for \( n \geq N \) we have
        \begin{align*}
            \abs{y_n - x} &= \abs{\frac{x_1 + \cdots + x_n}{n} - x} \\
            &= \abs{\frac{x_1 + \cdots + x_n}{n} - \frac{nx}{n}} \\
            &= \abs{\frac{(x_1 - x) + \cdots + (x_n - x)}{n}} \\
            &\leq \frac{\abs{x_1 - x} + \cdots + \abs{x_{N_1 - 1} - x}}{n} + \frac{\abs{x_{N_1} - x} + \cdots + \abs{x_n - x}}{n} \\
            &< \frac{\epsilon}{2} + \frac{n - N_1 + 1}{n} \cdot \frac{\epsilon}{2} \\
            &< \frac{\epsilon}{2} + \frac{\epsilon}{2} \\
            &= \epsilon.
        \end{align*}
        It follows that \( \lim y_n = x \).

        \item Consider the divergent sequence \( x_n = (-1)^{n+1} \). The sequence of averages \( (y_n) \) is then
        \[
            y_n = \begin{cases}
                \frac{1}{n} & \text{if } n \text{ is odd}, \\
                0 & \text{if } n \text{ is even},
            \end{cases}
        \]
        which satisfies \( \lim y_n = 0 \).
    \end{enumerate}
\end{solution}

\begin{exercise}
\label{ex:12}
    A typical task in analysis is to decipher whether a property possessed by every term in a convergent sequence is necessarily  inherited by the limit. Assume \( (a_n) \to a \), and determine the validity of each claim. Try to produce a counterexample for any that are false.
    \begin{enumerate}
        \item If every \( a_n \) is an upper bound for a set \( B \), then \( a \) is also an upper bound for \( B \).

        \item If every \( a_n \) is in the complement of the interval \( (0, 1) \), then \( a \) is also in the complement of \( (0, 1) \).

        \item If every \( a_n \) is rational, then \( a \) is rational.
    \end{enumerate}
\end{exercise}

\begin{solution}
    \begin{enumerate}
        \item This is true. Let \( b \in B \) be given. Then \( b \leq a_n \) for all \( n \in \N \), so by the Order Limit Theorem we have \( b \leq a \). It follows that \( a \) is an upper bound for \( B \).

        \item This is true. Observe that for a real number \( x \) we have
        \[
            x \not\in (0, 1) \iff x \leq 0 \text{ or } x \geq 1 \iff \abs{x - \tfrac{1}{2}} \geq \tfrac{1}{2}.
        \]
        So for each \( n \in \N \) we have \( \abs{a_n - \tfrac{1}{2}} \geq \tfrac{1}{2} \). The Algebraic Limit Theorem and \Cref{ex:10} (b) imply that \( \lim \abs{a_n - \tfrac{1}{2}} = \abs{a - \tfrac{1}{2}} \), and hence the Order Limit Theorem implies that \( \abs{a - \tfrac{1}{2}} \geq \tfrac{1}{2} \). It follows that \( a \) belongs to the complement of \( (0, 1) \).

        \item This is false. By the density of \( \Q \) in \( \R \), for each \( n \in \N \) we may pick a rational number \( a_n \) satisfying \( \sqrt{2} < a_n < \sqrt{2} + \tfrac{1}{n} \). The Squeeze Theorem (\Cref{ex:3}) then implies that \( \lim a_n = \sqrt{2} \), an irrational number.
    \end{enumerate}
\end{solution}

\begin{exercise}[Iterated Limits]
\label{ex:13}
    Given a doubly indexed array \( a_{mn} \) where \( m, n \in \N \), what should \( \lim_{m, n \to \infty} a_{mn} \) represent?
    \begin{enumerate}
        \item Let \( a_{mn} = m/(m + n) \) and compute the \textit{iterated} limits
        \[
            \lim_{n \to \infty} \left( \lim_{m \to \infty} a_{mn} \right) \qquad \text{and} \qquad \lim_{m \to \infty} \left( \lim_{n \to \infty} a_{mn} \right).
        \]
        Define \( \lim_{m, n \to \infty} a_{mn} = a \) to mean that for all \( \epsilon > 0 \) there exists an \( N \in \N \) such that if both \( m, n \geq N \), then \( \abs{a_{mn} - a} < \epsilon \).

        \item Let \( a_{mn} = 1/(m + n) \). Does \( \lim_{m, n \to \infty} a_{mn} \) exist in this case? Do the two iterated limits exist? How do these three values compare? Answer these same questions for \( a_{mn} = mn/(m^2 + n^2) \).

        \item Produce an example where \( \lim_{m, n \to \infty} a_{mn} \) exists but where neither iterated limit can be computed.

        \item Assume \( \lim_{m, n \to \infty} a_{mn} = a \), and assume that for each fixed \( m \in \N \), \( \lim_{n \to \infty} (a_{mn}) = b_m \). Show \( \lim_{m \to \infty} b_m = a \).

        \item Prove that if \( \lim_{m, n \to \infty} a_{mn} \) exists and the iterated limits both exist, then all three limits must be equal.
    \end{enumerate}
\end{exercise}

\begin{solution}
    \begin{enumerate}
        \item We apply the Algebraic Limit Theorem.
        \[
            \lim_{m \to \infty} a_{mn} = \lim_{m \to \infty} \left( \frac{m}{m + n} \right) = \lim_{m \to \infty} \left( \frac{1}{1 + \frac{n}{m}} \right) = \frac{1}{1 + n \lim_{m \to \infty} \left( \frac{1}{m} \right)} = \frac{1}{1} = 1.
        \]
        Hence \( \lim_{n \to \infty} \left( \lim_{m \to \infty} a_{mn} \right) = \lim_{n \to \infty} (1) = 1 \). Similarly,
        \[
            \lim_{n \to \infty} a_{mn} = \lim_{n \to \infty} \left( \frac{m}{m + n} \right) = \lim_{n \to \infty} \left( \frac{\frac{m}{n}}{1 + \frac{m}{n}} \right) = \frac{m \lim_{n \to \infty} \left( \frac{1}{n} \right)}{1 + m \lim_{n \to \infty} \left( \frac{1}{n} \right)} = \frac{0}{1} = 0.
        \]
        Hence \( \lim_{m \to \infty} \left( \lim_{n \to \infty} a_{mn} \right) = \lim_{m \to \infty} (0) = 0 \).

        \item For \( a_{mn} = 1/(m + n) \), we have \( \lim_{m, n \to \infty} a_{mn} = 0 \). To see this, let \( \epsilon > 0 \) be given. There is an \( N \in \N \) such that \( n \geq N \implies \tfrac{1}{n} < \epsilon \). Then for \( m, n \geq N \) we have
        \[
            \abs{a_{mn} - 0} = \frac{1}{m + n} < \frac{1}{n} < \epsilon.
        \]
        The two iterated limits also exist and are equal to 0. To see this, observe that for all \( m \in \N \) we have \( 0 < \tfrac{1}{m + n} < \tfrac{1}{m} \). Then by the Squeeze Theorem, \( \lim_{m \to \infty} a_{mn} = \lim_{m \to \infty} \tfrac{1}{m + n} = 0 \). It follows that \( \lim_{n \to \infty} \left( \lim_{m \to \infty} a_{mn} \right) = \lim_{n \to \infty} (0) = 0 \). Similarly, \( \lim_{m \to \infty} \left( \lim_{n \to \infty} a_{mn} \right) = 0 \).

        Now let \( a_{mn} = mn/(m^2 + n^2) \). We claim that \( \lim_{m, n \to \infty} a_{mn} \) does not exist. To see this, suppose that \( \lim_{m, n \to \infty} a_{mn} = x \) for some \( x \in \R \). Then there exists some \( N \in \N \) such that \( m, n \geq N \implies \abs{a_{mn} - x} < \tfrac{1}{20} \). In particular, taking \( n = m \),
        \[
            m \geq N \implies \abs{\frac{m^2}{m^2 + m^2} - x} = \abs{\tfrac{1}{2} - x} < \tfrac{1}{20} \iff x \in \left( \tfrac{9}{20}, \tfrac{11}{20} \right).
        \]
        Furthermore, taking \( n = 2m \),
        \[
            m \geq N \implies \abs{\frac{2m^2}{m^2 + 4m^2} - x} = \abs{\tfrac{2}{5} - x} < \tfrac{1}{20} \iff x \in \left( \tfrac{7}{20}, \tfrac{9}{20} \right).
        \]
        So assuming that \( \lim_{m, n \to \infty} a_{mn} = x \) leads us to the contradiction that \( x < \tfrac{9}{20} \) and \( x > \tfrac{9}{20} \). It follows that \( \lim_{m, n \to \infty} a_{mn} \) does not exist. However, the two iterated limits do exist and are equal to 0. Using the Algebraic Limit Theorem, we have
        \[
            \lim_{m \to \infty} \left( \frac{mn}{m^2 + n^2} \right) = \lim_{m \to \infty} \left( \frac{\frac{n}{m}}{1 + \frac{n^2}{m^2}} \right) = \frac{n \lim_{m \to \infty} \left( \frac{1}{m} \right)}{1 + n^2 \lim_{m \to \infty} \left( \frac{1}{m^2} \right)} = \frac{0}{1} = 0.
        \]
        It follows that \( \lim_{n \to \infty} \left( \lim_{m \to \infty} a_{mn} \right) = 0 \) and similarly that \( \lim_{m \to \infty} \left( \lim_{n \to \infty} a_{mn} \right) = 0 \).

        \item Let \( a_{mn} = (-1)^{m + n} \left(\frac{1}{m} + \frac{1}{n} \right) \). We claim that \( \lim_{m, n \to \infty} a_{mn} = 0 \). To see this, let \( \epsilon > 0 \) be given. There is an \( N \in \N \) such that \( n \geq N \implies \tfrac{1}{n} < \tfrac{\epsilon}{2} \). Then for \( m, n \geq N \) we have
        \[
            \abs{a_{mn}} = \abs{(-1)^{m + n} \left(\frac{1}{m} + \frac{1}{n} \right)} = \frac{1}{m} + \frac{1}{n} < \frac{\epsilon}{2} + \frac{\epsilon}{2} = \epsilon.
        \]
        However, neither iterated limit exists. Fix \( n \in \N \) and observe that
        \begin{align*}
            \abs{a_{mn} - a_{m+1,n}} &= \abs{(-1)^{m + n} \left(\frac{1}{m} + \frac{1}{n} \right) - (-1)^{m + n + 1} \left(\frac{1}{m + 1} + \frac{1}{n} \right)} \\
            &= \abs{\frac{1}{m} + \frac{1}{n} + \frac{1}{m + 1} + \frac{1}{n}} \\
            &= \frac{1}{m} + \frac{1}{m + 1} + \frac{2}{n} \\
            &\geq \frac{2}{n}.
        \end{align*}
        Since \( n \in \N \) is fixed, this implies that the sequence \( (a_{mn} - a_{m+1,n})_{m \in \N} \) cannot converge to \( 0 \). Now observe that for any sequence \( (b_m) \), the Algebraic Limit Theorem implies that
        \[
            \lim_{m \to \infty} b_m = x \text{ for some } x \in \R \implies \lim_{m \to \infty} (b_m - b_{m+1}) = 0.
        \]
        The contrapositive of this statement then implies that the limit \( \lim_{m \to \infty} a_{mn} \) does not exist for any \( n \in \N \). It follows that the iterated limit \( \lim_{n \to \infty} \left( \lim_{m \to \infty} a_{mn} \right) \) does not exist. Swapping the roles of \( m \) and \( n \) in our argument shows that the iterated limit \( \lim_{m \to \infty} \left( \lim_{n \to \infty} a_{mn} \right) \) does not exist either.

        \item Seeking a contradiction, suppose that \( (b_m) \) does not converge to \( a \), i.e.\ there exists some \( \epsilon > 0 \) such that for all \( N \in \N \) there is an \( M \geq N \) such that \( \abs{b_M - a} \geq \epsilon \). Since \( \lim_{m, n \to \infty} a_{mn} = a \), there exists some \( N_1 \in \N \) such that
        \[
            m, n \geq N_1 \implies \abs{a_{mn} - a} < \tfrac{\epsilon}{2}. \tag{1}
        \]
        Then by the previous sentence, there exists \( M \geq N_1 \) such that \( \abs{b_M - a} \geq \epsilon \). By assumption, we have \( \lim_{n \to \infty} a_{Mn} = b_M \), so there is an \( N_2 \in \N \) such that \( n \geq N_2 \implies \abs{a_{Mn} - b_M} < \tfrac{\epsilon}{2} \). Let \( N = \max \{ N_1, N_2 \} \) and observe that \( \abs{a_{MN} - a} < \tfrac{\epsilon}{2} \) by (1). However, the reverse triangle inequality gives us
        \begin{align*}
            \abs{a_{MN} - a} &= \abs{a_{MN} - b_M + b_M - a} \\
            &\geq \abs{\abs{b_M - a} - \abs{a_{MN} - b_M}} \\
            &\geq \abs{b_M - a} - \abs{a_{MN} - b_M} \\
            &> \epsilon - \tfrac{\epsilon}{2} \\
            &= \tfrac{\epsilon}{2}.
        \end{align*}
        So assuming that \( (b_m) \) does not converge to \( a \) leads us the contradiction that there exist positive integers \( M \) and \( N \) such that \( \abs{a_{MN} - a} \) is both less than and greater than \( \tfrac{\epsilon}{2} \). Hence it must be the case that \( \lim_{m \to \infty} b_m = a \).

        \item If the iterated limit \( \lim_{m \to \infty} \left( \lim_{n \to \infty} a_{mn} \right) \) exists, then it must be the case that for each fixed \( m \in \N \), \( \lim_{n \to \infty} a_{mn} \) exists. Then by part (d), we must have
        \[
            \lim_{m \to \infty} \left( \lim_{n \to \infty} a_{mn} \right) = \lim_{m, n \to \infty} a_{mn}.
        \]
        Swapping the roles of \( m \) and \( n \) and repeating the above argument shows that
        \[
            \lim_{n \to \infty} \left( \lim_{m \to \infty} a_{mn} \right) = \lim_{m, n \to \infty} a_{mn}.
        \]
    \end{enumerate}
\end{solution}

\noindent \hrulefill

\noindent \hypertarget{ua}{\textcolor{blue}{[UA]} Abbott, S. (2015) \textit{Understanding Analysis.} 2nd edn.}

\end{document}
