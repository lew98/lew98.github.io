\documentclass[12pt]{article}
\usepackage[utf8]{inputenc}
\usepackage{amsmath}
\usepackage{amsthm}
\usepackage{geometry}
\usepackage{amsfonts}
\usepackage{bm}
\usepackage{hyperref}
\usepackage{xcolor}
\geometry{
headheight=15pt,
left=60pt,
right=60pt
}
\usepackage{fancyhdr}
\pagestyle{fancy}
\fancyhf{}
\lhead{}
\chead{Existence of \textit{n}th roots in \texorpdfstring{\(\mathbb{R}\)}{}}
\rhead{}

\pagenumbering{gobble}
\setlength{\parindent}{0pt}
\hypersetup{
    colorlinks=true,
    linkcolor=blue
}

\newcommand{\newp}{\vspace{5mm}}

\theoremstyle{definition}
\newtheorem{theorem}{Theorem}
\newtheorem{lemma}{Lemma}

\newtheorem*{remark}{Remark}

\begin{document}

The following is paraphrased from pages 10-11 of \hyperlink{pma}{[PMA]}.

\section{Existence of \textit{n}th roots in \texorpdfstring{\(\mathbb{R}\)}{}}

First, a useful inequality. Suppose \( n \) is a positive integer and \( a, b \) are real numbers such that \( 0 < a < b \). This implies that \( 0 < b^{n-2}a < b^{n-1} \). Furthermore, we have \( 0 < a^2 < b^2 \), which gives \( 0 < b^{n-3}a^2 < b^{n-1} \), and so on. Combining this with the equality
\[
b^n - a^n = (b - a)(b^{n-1} + b^{n-2}a + \cdots + a^{n-1})
\]
gives us the inequality
\[
b^n - a^n < (b - a)nb^{n-1}. \tag{1}
\]

\begin{theorem}

For every real \( x > 0 \) and every positive integer \( n \) there is exactly one positive real \( y \) such that \( y^n = x \).

\end{theorem}

\begin{proof}

Suppose \( y_1 \) and \( y_2 \) are positive real numbers such that \( y_1 \neq y_2 \). Without loss of generality, assume \( 0 < y_1 < y_2 \). Then \( 0 < y_1^n < y_2^n \), so that \( y_1^n \neq y_2^n \). Hence by the contrapositive, \( y_1^n = y_2^n \) implies that \( y_1 = y_2 \). This gives us the uniqueness of any such \( y \) in Theorem 1.

\newp

For existence, let \( E = \{ t \in \mathbb{R} : t > 0, t^n < x \} \). Observe that \( t = \frac{x}{1 + x} \) satisfies \( t < x \) and \( 0 < t < 1 \), which gives \( 0 < t^n < t < x \). Hence \( t \in E \) and so \( E \) is non-empty. Now suppose \( t \geq 1 + x > 1 \), so that \( t^n > t \geq 1 + x > x \). Then by the contrapositive, \( t^n < x \) implies that \( t < 1 + x \), and we see that \( E \) is bounded above by \( 1 + x \). We may now invoke the least-upper-bound property of \( \mathbb{R} \) and set \( y = \sup E \). Note that \( y \) must be positive, since \( \frac{x}{1 + x} \) belongs to \( E \). To show that \( y^n = x \), we will show that both of the assumptions \( y^n < x \) and \( y^n > x \) lead to contradictions.

\newp

First, assume that \( y^n < x \). Using the Archimedean property, choose \( h \) such that \( 0 < h < 1 \) and \( h < \frac{x - y^n}{n (y+1)^{n-1}} \). Now take \( a = y \) and \( b = y + h \) in inequality (1) to obtain
\[
(y + h)^n - y^n < h n (y + h)^{n-1} < h n (y + 1)^{n-1} < x - y^n,
\]
whence \( (y + h)^n < x \) and so \( y + h \in E \); but this contradicts the fact that \( y \) is the supremum of \( E \), since \( y + h > y \).

\newp

Next, assume that \( y^n > x \) and set \( k = \frac{y^n - x}{ny^{n-1}} < y \). Take \( a = y - k \) and \( b = y \) in inequality (1) to obtain
\[
y^n - (y - k)^n < kny^{n-1} = y^n - x,
\]
whence \( (y - k)^n \geq x \). Then \( t \geq y - k \) implies that \( t^n \geq x \); the contrapositive of this shows that \( y - k \) is an upper bound for \( E \). This contradicts the fact that \( y \) is the least upper bound of \( E \), since \( y - k < y \).
\end{proof}

\section{A corollary}

\begin{theorem}

Let \( a \) and \( b \) be positive real numbers and \( n \) a positive integer. Then
\[
\sqrt[n]{ab} = \sqrt[n]{a}\sqrt[n]{b}.
\]

\end{theorem}

\begin{proof}

Let \( \alpha = \sqrt[n]{a} \) and \( \beta = \sqrt[n]{b} \). Then by the commutativity of multiplication, we have
\[
(\alpha \beta)^n = \alpha^n \beta^n = ab.
\]
The uniqueness part of Theorem 1 then implies that \( \sqrt[n]{ab} = \alpha \beta = \sqrt[n]{a} \sqrt[n]{b} \).
\end{proof}

\hrulefill

\hypertarget{pma}{\textcolor{blue}{[PMA]} Rudin, W. (1976) \textit{Principles of Mathematical Analysis.} 3rd edn.}

\end{document}
